\documentclass[]{book}
\usepackage{lmodern}
\usepackage{amssymb,amsmath}
\usepackage{ifxetex,ifluatex}
\usepackage{fixltx2e} % provides \textsubscript
\ifnum 0\ifxetex 1\fi\ifluatex 1\fi=0 % if pdftex
  \usepackage[T1]{fontenc}
  \usepackage[utf8]{inputenc}
\else % if luatex or xelatex
  \ifxetex
    \usepackage{mathspec}
  \else
    \usepackage{fontspec}
  \fi
  \defaultfontfeatures{Ligatures=TeX,Scale=MatchLowercase}
\fi
% use upquote if available, for straight quotes in verbatim environments
\IfFileExists{upquote.sty}{\usepackage{upquote}}{}
% use microtype if available
\IfFileExists{microtype.sty}{%
\usepackage{microtype}
\UseMicrotypeSet[protrusion]{basicmath} % disable protrusion for tt fonts
}{}
\usepackage[margin=1in]{geometry}
\usepackage{hyperref}
\hypersetup{unicode=true,
            pdftitle={Data Science for Biological, Medical and Health Research: Notes for 432},
            pdfauthor={Thomas E. Love, Ph.D.},
            pdfborder={0 0 0},
            breaklinks=true}
\urlstyle{same}  % don't use monospace font for urls
\usepackage{natbib}
\bibliographystyle{apalike}
\usepackage{color}
\usepackage{fancyvrb}
\newcommand{\VerbBar}{|}
\newcommand{\VERB}{\Verb[commandchars=\\\{\}]}
\DefineVerbatimEnvironment{Highlighting}{Verbatim}{commandchars=\\\{\}}
% Add ',fontsize=\small' for more characters per line
\usepackage{framed}
\definecolor{shadecolor}{RGB}{248,248,248}
\newenvironment{Shaded}{\begin{snugshade}}{\end{snugshade}}
\newcommand{\KeywordTok}[1]{\textcolor[rgb]{0.13,0.29,0.53}{\textbf{#1}}}
\newcommand{\DataTypeTok}[1]{\textcolor[rgb]{0.13,0.29,0.53}{#1}}
\newcommand{\DecValTok}[1]{\textcolor[rgb]{0.00,0.00,0.81}{#1}}
\newcommand{\BaseNTok}[1]{\textcolor[rgb]{0.00,0.00,0.81}{#1}}
\newcommand{\FloatTok}[1]{\textcolor[rgb]{0.00,0.00,0.81}{#1}}
\newcommand{\ConstantTok}[1]{\textcolor[rgb]{0.00,0.00,0.00}{#1}}
\newcommand{\CharTok}[1]{\textcolor[rgb]{0.31,0.60,0.02}{#1}}
\newcommand{\SpecialCharTok}[1]{\textcolor[rgb]{0.00,0.00,0.00}{#1}}
\newcommand{\StringTok}[1]{\textcolor[rgb]{0.31,0.60,0.02}{#1}}
\newcommand{\VerbatimStringTok}[1]{\textcolor[rgb]{0.31,0.60,0.02}{#1}}
\newcommand{\SpecialStringTok}[1]{\textcolor[rgb]{0.31,0.60,0.02}{#1}}
\newcommand{\ImportTok}[1]{#1}
\newcommand{\CommentTok}[1]{\textcolor[rgb]{0.56,0.35,0.01}{\textit{#1}}}
\newcommand{\DocumentationTok}[1]{\textcolor[rgb]{0.56,0.35,0.01}{\textbf{\textit{#1}}}}
\newcommand{\AnnotationTok}[1]{\textcolor[rgb]{0.56,0.35,0.01}{\textbf{\textit{#1}}}}
\newcommand{\CommentVarTok}[1]{\textcolor[rgb]{0.56,0.35,0.01}{\textbf{\textit{#1}}}}
\newcommand{\OtherTok}[1]{\textcolor[rgb]{0.56,0.35,0.01}{#1}}
\newcommand{\FunctionTok}[1]{\textcolor[rgb]{0.00,0.00,0.00}{#1}}
\newcommand{\VariableTok}[1]{\textcolor[rgb]{0.00,0.00,0.00}{#1}}
\newcommand{\ControlFlowTok}[1]{\textcolor[rgb]{0.13,0.29,0.53}{\textbf{#1}}}
\newcommand{\OperatorTok}[1]{\textcolor[rgb]{0.81,0.36,0.00}{\textbf{#1}}}
\newcommand{\BuiltInTok}[1]{#1}
\newcommand{\ExtensionTok}[1]{#1}
\newcommand{\PreprocessorTok}[1]{\textcolor[rgb]{0.56,0.35,0.01}{\textit{#1}}}
\newcommand{\AttributeTok}[1]{\textcolor[rgb]{0.77,0.63,0.00}{#1}}
\newcommand{\RegionMarkerTok}[1]{#1}
\newcommand{\InformationTok}[1]{\textcolor[rgb]{0.56,0.35,0.01}{\textbf{\textit{#1}}}}
\newcommand{\WarningTok}[1]{\textcolor[rgb]{0.56,0.35,0.01}{\textbf{\textit{#1}}}}
\newcommand{\AlertTok}[1]{\textcolor[rgb]{0.94,0.16,0.16}{#1}}
\newcommand{\ErrorTok}[1]{\textcolor[rgb]{0.64,0.00,0.00}{\textbf{#1}}}
\newcommand{\NormalTok}[1]{#1}
\usepackage{longtable,booktabs}
\usepackage{graphicx,grffile}
\makeatletter
\def\maxwidth{\ifdim\Gin@nat@width>\linewidth\linewidth\else\Gin@nat@width\fi}
\def\maxheight{\ifdim\Gin@nat@height>\textheight\textheight\else\Gin@nat@height\fi}
\makeatother
% Scale images if necessary, so that they will not overflow the page
% margins by default, and it is still possible to overwrite the defaults
% using explicit options in \includegraphics[width, height, ...]{}
\setkeys{Gin}{width=\maxwidth,height=\maxheight,keepaspectratio}
\IfFileExists{parskip.sty}{%
\usepackage{parskip}
}{% else
\setlength{\parindent}{0pt}
\setlength{\parskip}{6pt plus 2pt minus 1pt}
}
\setlength{\emergencystretch}{3em}  % prevent overfull lines
\providecommand{\tightlist}{%
  \setlength{\itemsep}{0pt}\setlength{\parskip}{0pt}}
\setcounter{secnumdepth}{5}
% Redefines (sub)paragraphs to behave more like sections
\ifx\paragraph\undefined\else
\let\oldparagraph\paragraph
\renewcommand{\paragraph}[1]{\oldparagraph{#1}\mbox{}}
\fi
\ifx\subparagraph\undefined\else
\let\oldsubparagraph\subparagraph
\renewcommand{\subparagraph}[1]{\oldsubparagraph{#1}\mbox{}}
\fi

%%% Use protect on footnotes to avoid problems with footnotes in titles
\let\rmarkdownfootnote\footnote%
\def\footnote{\protect\rmarkdownfootnote}

%%% Change title format to be more compact
\usepackage{titling}

% Create subtitle command for use in maketitle
\newcommand{\subtitle}[1]{
  \posttitle{
    \begin{center}\large#1\end{center}
    }
}

\setlength{\droptitle}{-2em}
  \title{Data Science for Biological, Medical and Health Research: Notes for 432}
  \pretitle{\vspace{\droptitle}\centering\huge}
  \posttitle{\par}
  \author{Thomas E. Love, Ph.D.}
  \preauthor{\centering\large\emph}
  \postauthor{\par}
  \predate{\centering\large\emph}
  \postdate{\par}
  \date{Built 2018-02-05 09:17:03}

\usepackage{booktabs}
\usepackage{amsthm}
\makeatletter
\def\thm@space@setup{%
  \thm@preskip=8pt plus 2pt minus 4pt
  \thm@postskip=\thm@preskip
}
\makeatother

\usepackage{amsthm}
\newtheorem{theorem}{Theorem}[chapter]
\newtheorem{lemma}{Lemma}[chapter]
\theoremstyle{definition}
\newtheorem{definition}{Definition}[chapter]
\newtheorem{corollary}{Corollary}[chapter]
\newtheorem{proposition}{Proposition}[chapter]
\theoremstyle{definition}
\newtheorem{example}{Example}[chapter]
\theoremstyle{definition}
\newtheorem{exercise}{Exercise}[chapter]
\theoremstyle{remark}
\newtheorem*{remark}{Remark}
\newtheorem*{solution}{Solution}
\begin{document}
\maketitle

{
\setcounter{tocdepth}{1}
\tableofcontents
}
\chapter*{Introduction}\label{introduction}
\addcontentsline{toc}{chapter}{Introduction}

These Notes provide a series of examples using R to work through issues
that are likely to come up in PQHS/CRSP/MPHP 432.

While these Notes share some of the features of a textbook, they are
neither comprehensive nor completely original. The main purpose is to
give students in 432 a set of common materials on which to draw during
the course. In class, we will sometimes:

\begin{itemize}
\tightlist
\item
  reiterate points made in this document,
\item
  amplify what is here,
\item
  simplify the presentation of things done here,
\item
  use new examples to show some of the same techniques,
\item
  refer to issues not mentioned in this document,
\end{itemize}

but what we don't (always) do is follow these notes very precisely. We
assume instead that you will read the materials and try to learn from
them, just as you will attend classes and try to learn from them. We
welcome feedback of all kinds on this document or anything else. Just
email us at \texttt{431-help\ at\ case\ dot\ edu}, or submit a pull
request. Note that we still use \texttt{431-help} even though we're now
in 432.

What you will mostly find are brief explanations of a key idea or
summary, accompanied (most of the time) by R code and a demonstration of
the results of applying that code.

Everything you see here is available to you as HTML or PDF. You will
also have access to the R Markdown files, which contain the code which
generates everything in the document, including all of the R results. We
will demonstrate the use of R Markdown (this document is generated with
the additional help of an R package called bookdown) and R Studio (the
``program'' which we use to interface with the R language) in class.

To download the data and R code related to these notes, visit the Data
and Code section of \href{https://github.com/THOMASELOVE/432-2018}{the
432 course website}.

\chapter*{R Packages used in these
notes}\label{r-packages-used-in-these-notes}
\addcontentsline{toc}{chapter}{R Packages used in these notes}

Here, we'll load in the packages used in these notes.

\begin{Shaded}
\begin{Highlighting}[]
\KeywordTok{library}\NormalTok{(tableone)}
\KeywordTok{library}\NormalTok{(skimr)}
\KeywordTok{library}\NormalTok{(ggridges)}
\KeywordTok{library}\NormalTok{(magrittr)}
\KeywordTok{library}\NormalTok{(arm)}
\KeywordTok{library}\NormalTok{(rms)}
\KeywordTok{library}\NormalTok{(leaps)}
\KeywordTok{library}\NormalTok{(lars)}
\KeywordTok{library}\NormalTok{(Epi)}
\KeywordTok{library}\NormalTok{(pROC)}
\KeywordTok{library}\NormalTok{(ROCR)}
\KeywordTok{library}\NormalTok{(simputation)}
\KeywordTok{library}\NormalTok{(modelr)}
\KeywordTok{library}\NormalTok{(broom)}
\KeywordTok{library}\NormalTok{(tidyverse)}
\end{Highlighting}
\end{Shaded}

\chapter*{Data used in these notes}\label{data-used-in-these-notes}
\addcontentsline{toc}{chapter}{Data used in these notes}

Here, we'll load in the data sets used in these notes.

\begin{Shaded}
\begin{Highlighting}[]
\NormalTok{fakestroke <-}\StringTok{ }\KeywordTok{read.csv}\NormalTok{(}\StringTok{"data/fakestroke.csv"}\NormalTok{) }\OperatorTok\StringTok{ }\NormalTok{tbl_df}
\NormalTok{bloodbrain <-}\StringTok{ }\KeywordTok{read.csv}\NormalTok{(}\StringTok{"data/bloodbrain.csv"}\NormalTok{) }\OperatorTok\StringTok{ }\NormalTok{tbl_df}
\NormalTok{smartcle1 <-}\StringTok{ }\KeywordTok{read.csv}\NormalTok{(}\StringTok{"data/smartcle1.csv"}\NormalTok{) }\OperatorTok\StringTok{ }\NormalTok{tbl_df}
\NormalTok{bonding <-}\StringTok{ }\KeywordTok{read.csv}\NormalTok{(}\StringTok{"data/bonding.csv"}\NormalTok{) }\OperatorTok\StringTok{ }\NormalTok{tbl_df}
\NormalTok{cortisol <-}\StringTok{ }\KeywordTok{read.csv}\NormalTok{(}\StringTok{"data/cortisol.csv"}\NormalTok{) }\OperatorTok\StringTok{ }\NormalTok{tbl_df}
\NormalTok{emphysema <-}\StringTok{ }\KeywordTok{read.csv}\NormalTok{(}\StringTok{"data/emphysema.csv"}\NormalTok{) }\OperatorTok\StringTok{ }\NormalTok{tbl_df}
\NormalTok{prost <-}\StringTok{ }\KeywordTok{read.csv}\NormalTok{(}\StringTok{"data/prost.csv"}\NormalTok{) }\OperatorTok\StringTok{ }\NormalTok{tbl_df}
\NormalTok{pollution <-}\StringTok{ }\KeywordTok{read.csv}\NormalTok{(}\StringTok{"data/pollution.csv"}\NormalTok{) }\OperatorTok\StringTok{ }\NormalTok{tbl_df}
\NormalTok{resect <-}\StringTok{ }\KeywordTok{read.csv}\NormalTok{(}\StringTok{"data/resect.csv"}\NormalTok{) }\OperatorTok\StringTok{ }\NormalTok{tbl_df}
\end{Highlighting}
\end{Shaded}

\chapter*{Special Functions used in these
notes}\label{special-functions-used-in-these-notes}
\addcontentsline{toc}{chapter}{Special Functions used in these notes}

\begin{Shaded}
\begin{Highlighting}[]
\NormalTok{specify_decimal <-}\StringTok{ }\ControlFlowTok{function}\NormalTok{(x, k) }\KeywordTok{format}\NormalTok{(}\KeywordTok{round}\NormalTok{(x, k), }\DataTypeTok{nsmall=}\NormalTok{k)}
\KeywordTok{skim_with}\NormalTok{(}\DataTypeTok{numeric =} \KeywordTok{list}\NormalTok{(}\DataTypeTok{hist =} \OtherTok{NULL}\NormalTok{), }
          \DataTypeTok{integer =} \KeywordTok{list}\NormalTok{(}\DataTypeTok{hist =} \OtherTok{NULL}\NormalTok{), }
          \DataTypeTok{ts =} \KeywordTok{list}\NormalTok{(}\DataTypeTok{line_graph =} \OtherTok{NULL}\NormalTok{))}
\end{Highlighting}
\end{Shaded}

\chapter{Building Table 1}\label{building-table-1}

Many scientific articles involve direct comparison of results from
various exposures, perhaps treatments. In 431, we studied numerous
methods, including various sorts of hypothesis tests, confidence
intervals, and descriptive summaries, which can help us to understand
and compare outcomes in such a setting. One common approach is to
present what's often called Table 1. Table 1 provides a summary of the
characteristics of a sample, or of groups of samples, which is most
commonly used to help understand the nature of the data being compared.

\section{\texorpdfstring{Two examples from the \emph{New England Journal
of
Medicine}}{Two examples from the New England Journal of Medicine}}\label{two-examples-from-the-new-england-journal-of-medicine}

\subsection{A simple Table 1}\label{a-simple-table-1}

Table 1 is especially common in the context of clinical research.
Consider the excerpt below, from a January 2015 article in the \emph{New
England Journal of Medicine} \citep{Tolaney2015}.

\includegraphics[width=0.5\linewidth]{images/Tolaney-snip1}

This (partial) table reports baseline characteristics on age group, sex
and race, describing 406 patients with HER2-positive\footnote{HER2 =
  human epidermal growth factor receptor type 2. Over-expression of this
  occurs in 15-20\% of invasive breast cancers, and has been associated
  with poor outcomes.} invasive breast cancer that began the protocol
therapy. Age, sex and race (along with severity of illness) are the most
commonly identified characteristics in a Table 1.

In addition to the measures shown in this excerpt, the full Table also
includes detailed information on the primary tumor for each patient,
including its size, nodal status and histologic grade. Footnotes tell us
that the percentages shown are subject to rounding, and may not total
100, and that the race information was self-reported.

\subsection{A group comparison}\label{a-group-comparison}

A more typical Table 1 involves a group comparison, for example in this
excerpt from \citet{Roy2008}. This Table 1 describes a multi-center
randomized clinical trial comparing two different approaches to caring
for patients with heart failure and atrial fibrillation\footnote{The
  complete Table 1 appears on pages 2668-2669 of \citet{Roy2008}, but I
  have only reproduced the first page and the footnote in this excerpt.}.

\includegraphics[width=0.9\linewidth]{images/Roy-snip1}

The article provides percentages, means and standard deviations across
groups, but note that it does not provide p values for the comparison of
baseline characteristics. This is a common feature of NEJM reports on
randomized clinical trials, where we anticipate that the two groups will
be well matched at baseline. Note that the patients in this study were
\emph{randomly} assigned to either the rhythm-control group or to the
rate-control group, using blocked randomizations stratified by study
center.

\section{The MR CLEAN trial}\label{the-mr-clean-trial}

\citet{Berkhemer2015} reported on the MR CLEAN trial, involving 500
patients with acute ischemic stroke caused by a proximal intracranial
arterial occlusion. The trial was conducted at 16 medical centers in the
Netherlands, where 233 were randomly assigned to the intervention
(intraarterial treatment plus usual care) and 267 to control (usual care
alone.) The primary outcome was the modified Rankin scale score at 90
days; this categorical scale measures functional outcome, with scores
ranging from 0 (no symptoms) to 6 (death). The fundamental conclusion of
\citet{Berkhemer2015} was that in patients with acute ischemic stroke
caused by a proximal intracranial occlusion of the anterior circulation,
intraarterial treatment administered within 6 hours after stroke onset
was effective and safe.

Here's the Table 1 from \citet{Berkhemer2015}.

\includegraphics[width=0.9\linewidth]{images/Berkhemer-snip4complete}

The Table was accompanied by the following notes.

\includegraphics[width=0.9\linewidth]{images/Berkhemer-snip4notes}

\section{\texorpdfstring{Simulated \texttt{fakestroke}
data}{Simulated fakestroke data}}\label{simulated-fakestroke-data}

Consider the simulated data, available on the Data and Code page of
\href{https://github.com/THOMASELOVE/432-2018}{our course website} in
the \texttt{fakestroke.csv} file, which I built to let us mirror the
Table 1 for MR CLEAN \citep{Berkhemer2015}. The \texttt{fakestroke.csv}
file contains the following 18 variables for 500 patients.

\begin{longtable}[]{@{}rl@{}}
\toprule
\begin{minipage}[b]{0.16\columnwidth}\raggedleft\strut
Variable\strut
\end{minipage} & \begin{minipage}[b]{0.55\columnwidth}\raggedright\strut
Description\strut
\end{minipage}\tabularnewline
\midrule
\endhead
\begin{minipage}[t]{0.16\columnwidth}\raggedleft\strut
\texttt{studyid}\strut
\end{minipage} & \begin{minipage}[t]{0.55\columnwidth}\raggedright\strut
Study ID \# (z001 through z500)\strut
\end{minipage}\tabularnewline
\begin{minipage}[t]{0.16\columnwidth}\raggedleft\strut
\texttt{trt}\strut
\end{minipage} & \begin{minipage}[t]{0.55\columnwidth}\raggedright\strut
Treatment group (Intervention or Control)\strut
\end{minipage}\tabularnewline
\begin{minipage}[t]{0.16\columnwidth}\raggedleft\strut
\texttt{age}\strut
\end{minipage} & \begin{minipage}[t]{0.55\columnwidth}\raggedright\strut
Age in years\strut
\end{minipage}\tabularnewline
\begin{minipage}[t]{0.16\columnwidth}\raggedleft\strut
\texttt{sex}\strut
\end{minipage} & \begin{minipage}[t]{0.55\columnwidth}\raggedright\strut
Male or Female\strut
\end{minipage}\tabularnewline
\begin{minipage}[t]{0.16\columnwidth}\raggedleft\strut
\texttt{nihss}\strut
\end{minipage} & \begin{minipage}[t]{0.55\columnwidth}\raggedright\strut
NIH Stroke Scale Score (can range from 0-42; higher scores indicate more
severe neurological deficits)\strut
\end{minipage}\tabularnewline
\begin{minipage}[t]{0.16\columnwidth}\raggedleft\strut
\texttt{location}\strut
\end{minipage} & \begin{minipage}[t]{0.55\columnwidth}\raggedright\strut
Stroke Location - Left or Right Hemisphere\strut
\end{minipage}\tabularnewline
\begin{minipage}[t]{0.16\columnwidth}\raggedleft\strut
\texttt{hx.isch}\strut
\end{minipage} & \begin{minipage}[t]{0.55\columnwidth}\raggedright\strut
History of Ischemic Stroke (Yes/No)\strut
\end{minipage}\tabularnewline
\begin{minipage}[t]{0.16\columnwidth}\raggedleft\strut
\texttt{afib}\strut
\end{minipage} & \begin{minipage}[t]{0.55\columnwidth}\raggedright\strut
Atrial Fibrillation (1 = Yes, 0 = No)\strut
\end{minipage}\tabularnewline
\begin{minipage}[t]{0.16\columnwidth}\raggedleft\strut
\texttt{dm}\strut
\end{minipage} & \begin{minipage}[t]{0.55\columnwidth}\raggedright\strut
Diabetes Mellitus (1 = Yes, 0 = No)\strut
\end{minipage}\tabularnewline
\begin{minipage}[t]{0.16\columnwidth}\raggedleft\strut
\texttt{mrankin}\strut
\end{minipage} & \begin{minipage}[t]{0.55\columnwidth}\raggedright\strut
Pre-stroke modified Rankin scale score (0, 1, 2 or \textgreater{} 2)
indicating functional disability - complete range is 0 (no symptoms) to
6 (death)\strut
\end{minipage}\tabularnewline
\begin{minipage}[t]{0.16\columnwidth}\raggedleft\strut
\texttt{sbp}\strut
\end{minipage} & \begin{minipage}[t]{0.55\columnwidth}\raggedright\strut
Systolic blood pressure, in mm Hg\strut
\end{minipage}\tabularnewline
\begin{minipage}[t]{0.16\columnwidth}\raggedleft\strut
\texttt{iv.altep}\strut
\end{minipage} & \begin{minipage}[t]{0.55\columnwidth}\raggedright\strut
Treatment with IV alteplase (Yes/No)\strut
\end{minipage}\tabularnewline
\begin{minipage}[t]{0.16\columnwidth}\raggedleft\strut
\texttt{time.iv}\strut
\end{minipage} & \begin{minipage}[t]{0.55\columnwidth}\raggedright\strut
Time from stroke onset to start of IV alteplase (minutes) if
iv.altep=Yes\strut
\end{minipage}\tabularnewline
\begin{minipage}[t]{0.16\columnwidth}\raggedleft\strut
\texttt{aspects}\strut
\end{minipage} & \begin{minipage}[t]{0.55\columnwidth}\raggedright\strut
Alberta Stroke Program Early Computed Tomography score, which measures
extent of stroke from 0 - 10; higher scores indicate fewer early
ischemic changes\strut
\end{minipage}\tabularnewline
\begin{minipage}[t]{0.16\columnwidth}\raggedleft\strut
\texttt{ia.occlus}\strut
\end{minipage} & \begin{minipage}[t]{0.55\columnwidth}\raggedright\strut
Intracranial arterial occlusion, based on vessel imaging - five
categories\footnotemark{}\strut
\end{minipage}
\footnotetext{The five categories are Intracranial ICA, ICA with
  involvement of the M1 middle cerebral artery segment, M1 middle
  cerebral artery segment, M2 middle cerebral artery segment, A1 or A2
  anterior cerebral artery segment}\tabularnewline
\begin{minipage}[t]{0.16\columnwidth}\raggedleft\strut
\texttt{extra.ica}\strut
\end{minipage} & \begin{minipage}[t]{0.55\columnwidth}\raggedright\strut
Extracranial ICA occlusion (1 = Yes, 0 = No)\strut
\end{minipage}\tabularnewline
\begin{minipage}[t]{0.16\columnwidth}\raggedleft\strut
\texttt{time.rand}\strut
\end{minipage} & \begin{minipage}[t]{0.55\columnwidth}\raggedright\strut
Time from stroke onset to study randomization, in minutes\strut
\end{minipage}\tabularnewline
\begin{minipage}[t]{0.16\columnwidth}\raggedleft\strut
\texttt{time.punc}\strut
\end{minipage} & \begin{minipage}[t]{0.55\columnwidth}\raggedright\strut
Time from stroke onset to groin puncture, in minutes (only if
Intervention)\strut
\end{minipage}\tabularnewline
\bottomrule
\end{longtable}

Here's a quick look at the simulated data in \texttt{fakestroke}.

\begin{Shaded}
\begin{Highlighting}[]
\NormalTok{fakestroke}
\end{Highlighting}
\end{Shaded}

\begin{verbatim}
# A tibble: 500 x 18
   studyid trt        age sex   nihss location hx.isch  afib    dm mrankin
   <fct>   <fct>    <dbl> <fct> <int> <fct>    <fct>   <int> <int> <fct>  
 1 z001    Control   53.0 Male     21 Right    No          0     0 2      
 2 z002    Interve~  51.0 Male     23 Left     No          1     0 0      
 3 z003    Control   68.0 Fema~    11 Right    No          0     0 0      
 4 z004    Control   28.0 Male     22 Left     No          0     0 0      
 5 z005    Control   91.0 Male     24 Right    No          0     0 0      
 6 z006    Control   34.0 Fema~    18 Left     No          0     0 2      
 7 z007    Interve~  75.0 Male     25 Right    No          0     0 0      
 8 z008    Control   89.0 Fema~    18 Right    No          0     0 0      
 9 z009    Control   75.0 Male     25 Left     No          1     0 2      
10 z010    Interve~  26.0 Fema~    27 Right    No          0     0 0      
# ... with 490 more rows, and 8 more variables: sbp <int>, iv.altep <fct>,
#   time.iv <int>, aspects <int>, ia.occlus <fct>, extra.ica <int>,
#   time.rand <int>, time.punc <int>
\end{verbatim}

\section{\texorpdfstring{Building Table 1 for \texttt{fakestroke}:
Attempt
1}{Building Table 1 for fakestroke: Attempt 1}}\label{building-table-1-for-fakestroke-attempt-1}

Our goal, then, is to take the data in \texttt{fakestroke.csv} and use
it to generate a Table 1 for the study that compares the 233 patients in
the Intervention group to the 267 patients in the Control group, on all
of the other variables (except study ID \#) available. I'll use the
\texttt{tableone} package of functions available in R to help me
complete this task. We'll make a first attempt, using the
\texttt{CreateTableOne} function in the \texttt{tableone} package. To
use the function, we'll need to specify:

\begin{itemize}
\tightlist
\item
  the \texttt{vars} or variables we want to place in the rows of our
  Table 1 (which will include just about everything in the
  \texttt{fakestroke} data except the \texttt{studyid} code and the
  \texttt{trt} variable for which we have other plans, and the
  \texttt{time.punc} which applies only to subjects in the Intervention
  group.)

  \begin{itemize}
  \tightlist
  \item
    A useful trick here is to use the \texttt{dput} function,
    specifically something like \texttt{dput(names(fakestroke))} can be
    used to generate a list of all of the variables included in the
    \texttt{fakestroke} tibble, and then this can be copied and pasted
    into the \texttt{vars} specification, saving some typing.
  \end{itemize}
\item
  the \texttt{strata} which indicates the levels want to use in the
  columns of our Table 1 (for us, that's \texttt{trt})
\end{itemize}

\begin{Shaded}
\begin{Highlighting}[]
\NormalTok{fs.vars <-}\StringTok{ }\KeywordTok{c}\NormalTok{(}\StringTok{"age"}\NormalTok{, }\StringTok{"sex"}\NormalTok{, }\StringTok{"nihss"}\NormalTok{, }\StringTok{"location"}\NormalTok{, }
          \StringTok{"hx.isch"}\NormalTok{, }\StringTok{"afib"}\NormalTok{, }\StringTok{"dm"}\NormalTok{, }\StringTok{"mrankin"}\NormalTok{, }\StringTok{"sbp"}\NormalTok{,}
          \StringTok{"iv.altep"}\NormalTok{, }\StringTok{"time.iv"}\NormalTok{, }\StringTok{"aspects"}\NormalTok{, }
          \StringTok{"ia.occlus"}\NormalTok{, }\StringTok{"extra.ica"}\NormalTok{, }\StringTok{"time.rand"}\NormalTok{)}

\NormalTok{fs.trt <-}\StringTok{ }\KeywordTok{c}\NormalTok{(}\StringTok{"trt"}\NormalTok{)}

\NormalTok{att1 <-}\StringTok{ }\KeywordTok{CreateTableOne}\NormalTok{(}\DataTypeTok{data =}\NormalTok{ fakestroke, }
                       \DataTypeTok{vars =}\NormalTok{ fs.vars, }
                       \DataTypeTok{strata =}\NormalTok{ fs.trt)}
\KeywordTok{print}\NormalTok{(att1)}
\end{Highlighting}
\end{Shaded}

\begin{verbatim}
                       Stratified by trt
                        Control        Intervention   p      test
  n                        267            233                    
  age (mean (sd))        65.38 (16.10)  63.93 (18.09)  0.343     
  sex = Male (%)           157 (58.8)     135 (57.9)   0.917     
  nihss (mean (sd))      18.08 (4.32)   17.97 (5.04)   0.787     
  location = Right (%)     114 (42.7)     117 (50.2)   0.111     
  hx.isch = Yes (%)         25 ( 9.4)      29 (12.4)   0.335     
  afib (mean (sd))        0.26 (0.44)    0.28 (0.45)   0.534     
  dm (mean (sd))          0.13 (0.33)    0.12 (0.33)   0.923     
  mrankin (%)                                          0.922     
     > 2                    11 ( 4.1)      10 ( 4.3)             
     0                     214 (80.1)     190 (81.5)             
     1                      29 (10.9)      21 ( 9.0)             
     2                      13 ( 4.9)      12 ( 5.2)             
  sbp (mean (sd))       145.00 (24.40) 146.03 (26.00)  0.647     
  iv.altep = Yes (%)       242 (90.6)     203 (87.1)   0.267     
  time.iv (mean (sd))    87.96 (26.01)  98.22 (45.48)  0.003     
  aspects (mean (sd))     8.65 (1.47)    8.35 (1.64)   0.033     
  ia.occlus (%)                                        0.795     
     A1 or A2                2 ( 0.8)       1 ( 0.4)             
     ICA with M1            75 (28.2)      59 (25.3)             
     Intracranial ICA        3 ( 1.1)       1 ( 0.4)             
     M1                    165 (62.0)     154 (66.1)             
     M2                     21 ( 7.9)      18 ( 7.7)             
  extra.ica (mean (sd))   0.26 (0.44)    0.32 (0.47)   0.150     
  time.rand (mean (sd)) 213.88 (70.29) 202.51 (57.33)  0.051     
\end{verbatim}

\subsection{Some of this is very useful, and other parts need to be
fixed.}\label{some-of-this-is-very-useful-and-other-parts-need-to-be-fixed.}

\begin{enumerate}
\def\labelenumi{\arabic{enumi}.}
\tightlist
\item
  The 1/0 variables (\texttt{afib}, \texttt{dm}, \texttt{extra.ica})
  might be better if they were treated as the factors they are, and
  reported as the Yes/No variables are reported, with counts and
  percentages rather than with means and standard deviations.
\item
  In some cases, we may prefer to re-order the levels of the categorical
  (factor) variables, particularly the \texttt{mrankin} variable, but
  also the \texttt{ia.occlus} variable. It would also be more typical to
  put the Intervention group to the left and the Control group to the
  right, so we may need to adjust our \texttt{trt} variable's levels
  accordingly.
\item
  For each of the quantitative variables (\texttt{age}, \texttt{nihss},
  \texttt{sbp}, \texttt{time.iv}, \texttt{aspects}, \texttt{extra.ica},
  \texttt{time.rand} and \texttt{time.punc}) we should make a decision
  whether a summary with mean and standard deviation is appropriate, or
  whether we should instead summarize with, say, the median and
  quartiles. A mean and standard deviation really only yields an
  appropriate summary when the data are least approximately Normally
  distributed. This will make the \emph{p} values a bit more reasonable,
  too. The \texttt{test} column in the first attempt will soon have
  something useful to tell us.
\item
  If we'd left in the \texttt{time.punc} variable, we'd get some
  warnings, having to do with the fact that \texttt{time.punc} is only
  relevant to patients in the Intervention group.
\end{enumerate}

\subsection{\texorpdfstring{\texttt{fakestroke} Cleaning Up Categorical
Variables}{fakestroke Cleaning Up Categorical Variables}}\label{fakestroke-cleaning-up-categorical-variables}

Let's specify each of the categorical variables as categorical
explicitly. This helps the \texttt{CreateTableOne} function treat them
appropriately, and display them with counts and percentages. This
includes all of the 1/0, Yes/No and multi-categorical variables.

\begin{Shaded}
\begin{Highlighting}[]
\NormalTok{fs.factorvars <-}\StringTok{ }\KeywordTok{c}\NormalTok{(}\StringTok{"sex"}\NormalTok{, }\StringTok{"location"}\NormalTok{, }\StringTok{"hx.isch"}\NormalTok{, }\StringTok{"afib"}\NormalTok{, }\StringTok{"dm"}\NormalTok{, }
                   \StringTok{"mrankin"}\NormalTok{, }\StringTok{"iv.altep"}\NormalTok{, }\StringTok{"ia.occlus"}\NormalTok{, }\StringTok{"extra.ica"}\NormalTok{)}
\end{Highlighting}
\end{Shaded}

Then we simply add a \texttt{factorVars\ =\ fs.factorvars} call to the
\texttt{CreateTableOne} function.

We also want to re-order some of those categorical variables, so that
the levels are more useful to us. Specifically, we want to:

\begin{itemize}
\tightlist
\item
  place Intervention before Control in the \texttt{trt} variable,
\item
  reorder the \texttt{mrankin} scale as 0, 1, 2, \textgreater{} 2, and
\item
  rearrange the \texttt{ia.occlus} variable to the order\footnote{We
    might also have considered reordering the \texttt{ia.occlus} factor
    by its frequency, using the \texttt{fct\_infreq} function} presented
  in \citet{Berkhemer2015}.
\end{itemize}

To accomplish this, we'll use the \texttt{fct\_relevel} function from
the \texttt{forcats} package (loaded with the rest of the core
\texttt{tidyverse} packages) to reorder our levels manually.

\begin{Shaded}
\begin{Highlighting}[]
\NormalTok{fakestroke <-}\StringTok{ }\NormalTok{fakestroke }\OperatorTok
\StringTok{    }\KeywordTok{mutate}\NormalTok{(}\DataTypeTok{trt =} \KeywordTok{fct_relevel}\NormalTok{(trt, }\StringTok{"Intervention"}\NormalTok{, }\StringTok{"Control"}\NormalTok{),}
           \DataTypeTok{mrankin =} \KeywordTok{fct_relevel}\NormalTok{(mrankin, }\StringTok{"0"}\NormalTok{, }\StringTok{"1"}\NormalTok{, }\StringTok{"2"}\NormalTok{, }\StringTok{"> 2"}\NormalTok{),}
           \DataTypeTok{ia.occlus =} \KeywordTok{fct_relevel}\NormalTok{(ia.occlus, }\StringTok{"Intracranial ICA"}\NormalTok{, }
                                   \StringTok{"ICA with M1"}\NormalTok{, }\StringTok{"M1"}\NormalTok{, }\StringTok{"M2"}\NormalTok{, }
                                   \StringTok{"A1 or A2"}\NormalTok{)}
\NormalTok{           ) }
\end{Highlighting}
\end{Shaded}

\section{\texorpdfstring{\texttt{fakestroke} Table 1: Attempt
2}{fakestroke Table 1: Attempt 2}}\label{fakestroke-table-1-attempt-2}

\begin{Shaded}
\begin{Highlighting}[]
\NormalTok{att2 <-}\StringTok{ }\KeywordTok{CreateTableOne}\NormalTok{(}\DataTypeTok{data =}\NormalTok{ fakestroke, }
                       \DataTypeTok{vars =}\NormalTok{ fs.vars,}
                       \DataTypeTok{factorVars =}\NormalTok{ fs.factorvars,}
                       \DataTypeTok{strata =}\NormalTok{ fs.trt)}
\KeywordTok{print}\NormalTok{(att2)}
\end{Highlighting}
\end{Shaded}

\begin{verbatim}
                       Stratified by trt
                        Intervention   Control        p      test
  n                        233            267                    
  age (mean (sd))        63.93 (18.09)  65.38 (16.10)  0.343     
  sex = Male (%)           135 (57.9)     157 (58.8)   0.917     
  nihss (mean (sd))      17.97 (5.04)   18.08 (4.32)   0.787     
  location = Right (%)     117 (50.2)     114 (42.7)   0.111     
  hx.isch = Yes (%)         29 (12.4)      25 ( 9.4)   0.335     
  afib = 1 (%)              66 (28.3)      69 (25.8)   0.601     
  dm = 1 (%)                29 (12.4)      34 (12.7)   1.000     
  mrankin (%)                                          0.922     
     0                     190 (81.5)     214 (80.1)             
     1                      21 ( 9.0)      29 (10.9)             
     2                      12 ( 5.2)      13 ( 4.9)             
     > 2                    10 ( 4.3)      11 ( 4.1)             
  sbp (mean (sd))       146.03 (26.00) 145.00 (24.40)  0.647     
  iv.altep = Yes (%)       203 (87.1)     242 (90.6)   0.267     
  time.iv (mean (sd))    98.22 (45.48)  87.96 (26.01)  0.003     
  aspects (mean (sd))     8.35 (1.64)    8.65 (1.47)   0.033     
  ia.occlus (%)                                        0.795     
     Intracranial ICA        1 ( 0.4)       3 ( 1.1)             
     ICA with M1            59 (25.3)      75 (28.2)             
     M1                    154 (66.1)     165 (62.0)             
     M2                     18 ( 7.7)      21 ( 7.9)             
     A1 or A2                1 ( 0.4)       2 ( 0.8)             
  extra.ica = 1 (%)         75 (32.2)      70 (26.3)   0.179     
  time.rand (mean (sd)) 202.51 (57.33) 213.88 (70.29)  0.051     
\end{verbatim}

The categorical data presentation looks much improved.

\subsection{What summaries should we
show?}\label{what-summaries-should-we-show}

Now, we'll move on to the issue of making a decision about what type of
summary to show for the quantitative variables. Since the
\texttt{fakestroke} data are just simulated and only match the summary
statistics of the original results, not the details, we'll adopt the
decisions made by \citet{Berkhemer2015}, which were to use medians and
interquartile ranges to summarize the distributions of all of the
continuous variables \textbf{except} systolic blood pressure.

\begin{itemize}
\tightlist
\item
  Specifying certain quantitative variables as \emph{non-normal} causes
  R to show them with medians and the 25th and 75th percentiles, rather
  than means and standard deviations, and also causes those variables to
  be tested using non-parametric tests, like the Wilcoxon signed rank
  test, rather than the t test. The \texttt{test} column indicates this
  with the word \texttt{nonnorm}.

  \begin{itemize}
  \tightlist
  \item
    In real data situations, what should we do? The answer is to look at
    the data. I would not make the decision as to which approach to take
    without first plotting (perhaps in a histogram or a Normal Q-Q plot)
    the observed distributions in each of the two samples, so that I
    could make a sound decision about whether Normality was a reasonable
    assumption. If the means and medians are meaningfully different from
    each other, this is especially important.
  \item
    To be honest, though, if the variable in question is a relatively
    unimportant covariate and the \emph{p} values for the two approaches
    are nearly the same, I'd say that further investigation is rarely
    important,
  \end{itemize}
\item
  Specifying \emph{exact} tests for certain categorical variables (we'll
  try this for the \texttt{location} and \texttt{mrankin} variables) can
  be done, and these changes will be noted in the \texttt{test} column,
  as well.

  \begin{itemize}
  \tightlist
  \item
    In real data situations, I would rarely be concerned about this
    issue, and often choose Pearson (approximate) options across the
    board. This is reasonable so long as the number of subjects falling
    in each category is reasonably large, say above 10. If not, then an
    exact test may be a tiny improvement.
  \item
    Paraphrasing \citet{Rosenbaum2017}, having an exact rather than an
    approximate test result is about as valuable as having a nice crease
    in your trousers.
  \end{itemize}
\end{itemize}

To finish our Table 1, then, we need to specify which variables should
be treated as non-Normal in the \texttt{print} statement - notice that
we don't need to redo the \texttt{CreateTableOne} for this change.

\begin{Shaded}
\begin{Highlighting}[]
\KeywordTok{print}\NormalTok{(att2, }
      \DataTypeTok{nonnormal =} \KeywordTok{c}\NormalTok{(}\StringTok{"age"}\NormalTok{, }\StringTok{"nihss"}\NormalTok{, }\StringTok{"time.iv"}\NormalTok{, }\StringTok{"aspects"}\NormalTok{, }\StringTok{"time.rand"}\NormalTok{),}
      \DataTypeTok{exact =} \KeywordTok{c}\NormalTok{(}\StringTok{"location"}\NormalTok{, }\StringTok{"mrankin"}\NormalTok{))}
\end{Highlighting}
\end{Shaded}

\begin{verbatim}
                          Stratified by trt
                           Intervention            Control                
  n                           233                     267                 
  age (median [IQR])        65.80 [54.50, 76.00]    65.70 [55.75, 76.20]  
  sex = Male (%)              135 (57.9)              157 (58.8)          
  nihss (median [IQR])      17.00 [14.00, 21.00]    18.00 [14.00, 22.00]  
  location = Right (%)        117 (50.2)              114 (42.7)          
  hx.isch = Yes (%)            29 (12.4)               25 ( 9.4)          
  afib = 1 (%)                 66 (28.3)               69 (25.8)          
  dm = 1 (%)                   29 (12.4)               34 (12.7)          
  mrankin (%)                                                             
     0                        190 (81.5)              214 (80.1)          
     1                         21 ( 9.0)               29 (10.9)          
     2                         12 ( 5.2)               13 ( 4.9)          
     > 2                       10 ( 4.3)               11 ( 4.1)          
  sbp (mean (sd))          146.03 (26.00)          145.00 (24.40)         
  iv.altep = Yes (%)          203 (87.1)              242 (90.6)          
  time.iv (median [IQR])    85.00 [67.00, 110.00]   87.00 [65.00, 116.00] 
  aspects (median [IQR])     9.00 [7.00, 10.00]      9.00 [8.00, 10.00]   
  ia.occlus (%)                                                           
     Intracranial ICA           1 ( 0.4)                3 ( 1.1)          
     ICA with M1               59 (25.3)               75 (28.2)          
     M1                       154 (66.1)              165 (62.0)          
     M2                        18 ( 7.7)               21 ( 7.9)          
     A1 or A2                   1 ( 0.4)                2 ( 0.8)          
  extra.ica = 1 (%)            75 (32.2)               70 (26.3)          
  time.rand (median [IQR]) 204.00 [152.00, 249.50] 196.00 [149.00, 266.00]
                          Stratified by trt
                           p      test   
  n                                      
  age (median [IQR])        0.579 nonnorm
  sex = Male (%)            0.917        
  nihss (median [IQR])      0.453 nonnorm
  location = Right (%)      0.106 exact  
  hx.isch = Yes (%)         0.335        
  afib = 1 (%)              0.601        
  dm = 1 (%)                1.000        
  mrankin (%)               0.917 exact  
     0                                   
     1                                   
     2                                   
     > 2                                 
  sbp (mean (sd))           0.647        
  iv.altep = Yes (%)        0.267        
  time.iv (median [IQR])    0.596 nonnorm
  aspects (median [IQR])    0.075 nonnorm
  ia.occlus (%)             0.795        
     Intracranial ICA                    
     ICA with M1                         
     M1                                  
     M2                                  
     A1 or A2                            
  extra.ica = 1 (%)         0.179        
  time.rand (median [IQR])  0.251 nonnorm
\end{verbatim}

\section{Obtaining a more detailed
Summary}\label{obtaining-a-more-detailed-summary}

If this was a real data set, we'd want to get a more detailed
description of the data to make decisions about things like potentially
collapsing categories of a variable, or whether or not a normal
distribution was useful for a particular continuous variable, etc. You
can do this with the \texttt{summary} command applied to a created Table
1, which shows, among other things, the effect of changing from normal
to non-normal \emph{p} values for continuous variables, and from
approximate to ``exact'' \emph{p} values for categorical factors.

Again, as noted above, in a real data situation, we'd want to plot the
quantitative variables (within each group) to make a smart decision
about whether a t test or Wilcoxon approach is more appropriate.

Note in the summary below that we have some missing values here. Often,
we'll present this information within the Table 1, as well.

\begin{Shaded}
\begin{Highlighting}[]
\KeywordTok{summary}\NormalTok{(att2)}
\end{Highlighting}
\end{Shaded}

\begin{verbatim}

     ### Summary of continuous variables ###

trt: Intervention
            n miss p.miss mean sd median p25 p75 min max  skew  kurt
age       233    0    0.0   64 18     66  54  76  23  96 -0.34 -0.52
nihss     233    0    0.0   18  5     17  14  21  10  28  0.48 -0.74
sbp       233    0    0.0  146 26    146 129 164  78 214 -0.07 -0.22
time.iv   233   30   12.9   98 45     85  67 110  42 218  1.03  0.08
aspects   233    0    0.0    8  2      9   7  10   5  10 -0.56 -0.98
time.rand 233    2    0.9  203 57    204 152 250 100 300  0.01 -1.16
-------------------------------------------------------- 
trt: Control
            n miss p.miss mean sd median p25 p75 min max   skew  kurt
age       267    0    0.0   65 16     66  56  76  24  94 -0.296 -0.28
nihss     267    0    0.0   18  4     18  14  22  11  25  0.017 -1.24
sbp       267    1    0.4  145 24    145 128 161  82 231  0.156  0.08
time.iv   267   25    9.4   88 26     87  65 116  44 130  0.001 -1.32
aspects   267    4    1.5    9  1      9   8  10   5  10 -1.071  0.36
time.rand 267    0    0.0  214 70    196 149 266 120 360  0.508 -0.93

p-values
              pNormal pNonNormal
age       0.342813660 0.57856976
nihss     0.787487252 0.45311695
sbp       0.647157646 0.51346132
time.iv   0.003073372 0.59641104
aspects   0.032662901 0.07464683
time.rand 0.050803672 0.25134327

Standardize mean differences
              1 vs 2
age       0.08478764
nihss     0.02405390
sbp       0.04100833
time.iv   0.27691223
aspects   0.19210662
time.rand 0.17720957

=======================================================================================

     ### Summary of categorical variables ### 

trt: Intervention
       var   n miss p.miss            level freq percent cum.percent
       sex 233    0    0.0           Female   98    42.1        42.1
                                       Male  135    57.9       100.0
                                                                    
  location 233    0    0.0             Left  116    49.8        49.8
                                      Right  117    50.2       100.0
                                                                    
   hx.isch 233    0    0.0               No  204    87.6        87.6
                                        Yes   29    12.4       100.0
                                                                    
      afib 233    0    0.0                0  167    71.7        71.7
                                          1   66    28.3       100.0
                                                                    
        dm 233    0    0.0                0  204    87.6        87.6
                                          1   29    12.4       100.0
                                                                    
   mrankin 233    0    0.0                0  190    81.5        81.5
                                          1   21     9.0        90.6
                                          2   12     5.2        95.7
                                        > 2   10     4.3       100.0
                                                                    
  iv.altep 233    0    0.0               No   30    12.9        12.9
                                        Yes  203    87.1       100.0
                                                                    
 ia.occlus 233    0    0.0 Intracranial ICA    1     0.4         0.4
                                ICA with M1   59    25.3        25.8
                                         M1  154    66.1        91.8
                                         M2   18     7.7        99.6
                                   A1 or A2    1     0.4       100.0
                                                                    
 extra.ica 233    0    0.0                0  158    67.8        67.8
                                          1   75    32.2       100.0
                                                                    
-------------------------------------------------------- 
trt: Control
       var   n miss p.miss            level freq percent cum.percent
       sex 267    0    0.0           Female  110    41.2        41.2
                                       Male  157    58.8       100.0
                                                                    
  location 267    0    0.0             Left  153    57.3        57.3
                                      Right  114    42.7       100.0
                                                                    
   hx.isch 267    0    0.0               No  242    90.6        90.6
                                        Yes   25     9.4       100.0
                                                                    
      afib 267    0    0.0                0  198    74.2        74.2
                                          1   69    25.8       100.0
                                                                    
        dm 267    0    0.0                0  233    87.3        87.3
                                          1   34    12.7       100.0
                                                                    
   mrankin 267    0    0.0                0  214    80.1        80.1
                                          1   29    10.9        91.0
                                          2   13     4.9        95.9
                                        > 2   11     4.1       100.0
                                                                    
  iv.altep 267    0    0.0               No   25     9.4         9.4
                                        Yes  242    90.6       100.0
                                                                    
 ia.occlus 267    1    0.4 Intracranial ICA    3     1.1         1.1
                                ICA with M1   75    28.2        29.3
                                         M1  165    62.0        91.4
                                         M2   21     7.9        99.2
                                   A1 or A2    2     0.8       100.0
                                                                    
 extra.ica 267    1    0.4                0  196    73.7        73.7
                                          1   70    26.3       100.0
                                                                    

p-values
            pApprox    pExact
sex       0.9171387 0.8561188
location  0.1113553 0.1056020
hx.isch   0.3352617 0.3124683
afib      0.6009691 0.5460206
dm        1.0000000 1.0000000
mrankin   0.9224798 0.9173657
iv.altep  0.2674968 0.2518374
ia.occlus 0.7945580 0.8189090
extra.ica 0.1793385 0.1667574

Standardize mean differences
               1 vs 2
sex       0.017479025
location  0.151168444
hx.isch   0.099032275
afib      0.055906317
dm        0.008673478
mrankin   0.062543164
iv.altep  0.111897009
ia.occlus 0.117394890
extra.ica 0.129370206
\end{verbatim}

In this case, I have simulated the data to mirror the results in the
published Table 1 for this study. In no way have I captured the full
range of the real data, or any of the relationships in that data, so
it's more important here to see what's available in the analysis, rather
than to interpret it closely in the clinical context.

\section{Exporting the Completed Table 1 from R to Excel or
Word}\label{exporting-the-completed-table-1-from-r-to-excel-or-word}

Once you've built the table and are generally satisfied with it, you'll
probably want to be able to drop it into Excel or Word for final
cleanup.

\subsection{Approach A: Save and open in
Excel}\label{approach-a-save-and-open-in-excel}

One option is to \textbf{save the Table 1} to a \texttt{.csv} file
within our \texttt{data} subfolder (note that the \texttt{data} folder
must already exist), which you can then open directly in Excel. This is
the approach I generally use. Note the addition of some \texttt{quote},
\texttt{noSpaces} and \texttt{printToggle} selections here.

\begin{Shaded}
\begin{Highlighting}[]
\NormalTok{fs.table1save <-}\StringTok{ }\KeywordTok{print}\NormalTok{(att2, }
      \DataTypeTok{nonnormal =} \KeywordTok{c}\NormalTok{(}\StringTok{"age"}\NormalTok{, }\StringTok{"nihss"}\NormalTok{, }\StringTok{"time.iv"}\NormalTok{, }\StringTok{"aspects"}\NormalTok{, }\StringTok{"time.rand"}\NormalTok{),}
      \DataTypeTok{exact =} \KeywordTok{c}\NormalTok{(}\StringTok{"location"}\NormalTok{, }\StringTok{"mrankin"}\NormalTok{),}
      \DataTypeTok{quote =} \OtherTok{FALSE}\NormalTok{, }\DataTypeTok{noSpaces =} \OtherTok{TRUE}\NormalTok{, }\DataTypeTok{printToggle =} \OtherTok{FALSE}\NormalTok{)}

\KeywordTok{write.csv}\NormalTok{(fs.table1save, }\DataTypeTok{file =} \StringTok{"data/fs-table1.csv"}\NormalTok{)}
\end{Highlighting}
\end{Shaded}

When I then open the \texttt{fs-table1.csv} file in Excel, it looks like
this:

\includegraphics[width=0.9\linewidth]{images/fs-table1inExcel}

And from here, I can either drop it directly into Word, or present it as
is, or start tweaking it to meet formatting needs.

\subsection{Approach B: Produce the Table so you can cut and paste
it}\label{approach-b-produce-the-table-so-you-can-cut-and-paste-it}

\begin{Shaded}
\begin{Highlighting}[]
\KeywordTok{print}\NormalTok{(att2, }
      \DataTypeTok{nonnormal =} \KeywordTok{c}\NormalTok{(}\StringTok{"age"}\NormalTok{, }\StringTok{"nihss"}\NormalTok{, }\StringTok{"time.iv"}\NormalTok{, }\StringTok{"aspects"}\NormalTok{, }\StringTok{"time.rand"}\NormalTok{),}
      \DataTypeTok{exact =} \KeywordTok{c}\NormalTok{(}\StringTok{"location"}\NormalTok{, }\StringTok{"mrankin"}\NormalTok{),}
      \DataTypeTok{quote =} \OtherTok{TRUE}\NormalTok{, }\DataTypeTok{noSpaces =} \OtherTok{TRUE}\NormalTok{)}
\end{Highlighting}
\end{Shaded}

This will look like a mess by itself, but if you:

\begin{enumerate}
\def\labelenumi{\arabic{enumi}.}
\tightlist
\item
  copy and paste that mess into Excel
\item
  select Text to Columns from the Data menu
\item
  select Delimited, then Space and select Treat consecutive delimiters
  as one
\end{enumerate}

you should get something usable again.

Or, in Word,

\begin{enumerate}
\def\labelenumi{\arabic{enumi}.}
\tightlist
\item
  insert the text
\item
  select the text with your mouse
\item
  select Insert \ldots{} Table \ldots{} Convert Text to Table
\item
  place a quotation mark in the ``Other'' area under Separate text at
  \ldots{}
\end{enumerate}

After dropping blank columns, the result looks pretty good.

\section{A Controlled Biological Experiment - The Blood-Brain
Barrier}\label{a-controlled-biological-experiment---the-blood-brain-barrier}

My source for the data and the following explanatory paragraph is page
307 from \citet{RamseySchafer2002}. The original data come from
\citet{Barnett1995}.

\begin{quote}
The human brain (and that of rats, coincidentally) is protected from the
bacteria and toxins that course through the bloodstream by something
called the blood-brain barrier. After a method of disrupting the barrier
was developed, researchers tested this new mechanism, as follows. A
series of 34 rats were inoculated with human lung cancer cells to induce
brain tumors. After 9-11 days they were infused with either the barrier
disruption (BD) solution or, as a control, a normal saline (NS)
solution. Fifteen minutes later, the rats received a standard dose of a
particular therapeutic antibody (L6-F(ab')2. The key measure of the
effectiveness of transmission across the brain-blood barrier is the
ratio of the antibody concentration in the brain tumor to the antibody
concentration in normal tissue outside the brain. The rats were then
sacrificed, and the amounts of antibody in the brain tumor and in normal
tissue from the liver were measured. The study's primary objective is to
determine whether the antibody concentration in the tumor increased when
the blood-barrier disruption infusion was given, and if so, by how much?
\end{quote}

\section{\texorpdfstring{The \texttt{bloodbrain.csv}
file}{The bloodbrain.csv file}}\label{the-bloodbrain.csv-file}

Consider the data, available on the Data and Code page of
\href{https://github.com/THOMASELOVE/432-2018}{our course website} in
the \texttt{bloodbrain.csv} file, which includes the following
variables:

\begin{longtable}[]{@{}rl@{}}
\toprule
\begin{minipage}[b]{0.12\columnwidth}\raggedleft\strut
Variable\strut
\end{minipage} & \begin{minipage}[b]{0.59\columnwidth}\raggedright\strut
Description\strut
\end{minipage}\tabularnewline
\midrule
\endhead
\begin{minipage}[t]{0.12\columnwidth}\raggedleft\strut
\texttt{case}\strut
\end{minipage} & \begin{minipage}[t]{0.59\columnwidth}\raggedright\strut
identification number for the rat (1 - 34)\strut
\end{minipage}\tabularnewline
\begin{minipage}[t]{0.12\columnwidth}\raggedleft\strut
\texttt{brain}\strut
\end{minipage} & \begin{minipage}[t]{0.59\columnwidth}\raggedright\strut
an outcome: Brain tumor antibody count (per gram)\strut
\end{minipage}\tabularnewline
\begin{minipage}[t]{0.12\columnwidth}\raggedleft\strut
\texttt{liver}\strut
\end{minipage} & \begin{minipage}[t]{0.59\columnwidth}\raggedright\strut
an outcome: Liver antibody count (per gram)\strut
\end{minipage}\tabularnewline
\begin{minipage}[t]{0.12\columnwidth}\raggedleft\strut
\texttt{tlratio}\strut
\end{minipage} & \begin{minipage}[t]{0.59\columnwidth}\raggedright\strut
an outcome: tumor / liver concentration ratio\strut
\end{minipage}\tabularnewline
\begin{minipage}[t]{0.12\columnwidth}\raggedleft\strut
\texttt{solution}\strut
\end{minipage} & \begin{minipage}[t]{0.59\columnwidth}\raggedright\strut
the treatment: BD (barrier disruption) or NS (normal saline)\strut
\end{minipage}\tabularnewline
\begin{minipage}[t]{0.12\columnwidth}\raggedleft\strut
\texttt{sactime}\strut
\end{minipage} & \begin{minipage}[t]{0.59\columnwidth}\raggedright\strut
a design variable: Sacrifice time (hours; either 0.5, 3, 24 or 72)\strut
\end{minipage}\tabularnewline
\begin{minipage}[t]{0.12\columnwidth}\raggedleft\strut
\texttt{postin}\strut
\end{minipage} & \begin{minipage}[t]{0.59\columnwidth}\raggedright\strut
covariate: Days post-inoculation of lung cancer cells (9, 10 or
11)\strut
\end{minipage}\tabularnewline
\begin{minipage}[t]{0.12\columnwidth}\raggedleft\strut
\texttt{sex}\strut
\end{minipage} & \begin{minipage}[t]{0.59\columnwidth}\raggedright\strut
covariate: M or F\strut
\end{minipage}\tabularnewline
\begin{minipage}[t]{0.12\columnwidth}\raggedleft\strut
\texttt{wt.init}\strut
\end{minipage} & \begin{minipage}[t]{0.59\columnwidth}\raggedright\strut
covariate: Initial weight (grams)\strut
\end{minipage}\tabularnewline
\begin{minipage}[t]{0.12\columnwidth}\raggedleft\strut
\texttt{wt.loss}\strut
\end{minipage} & \begin{minipage}[t]{0.59\columnwidth}\raggedright\strut
covariate: Weight loss (grams)\strut
\end{minipage}\tabularnewline
\begin{minipage}[t]{0.12\columnwidth}\raggedleft\strut
\texttt{wt.tumor}\strut
\end{minipage} & \begin{minipage}[t]{0.59\columnwidth}\raggedright\strut
covariate: Tumor weight (10\textsuperscript{-4} grams)\strut
\end{minipage}\tabularnewline
\bottomrule
\end{longtable}

And here's what the data look like in R.

\begin{Shaded}
\begin{Highlighting}[]
\NormalTok{bloodbrain}
\end{Highlighting}
\end{Shaded}

\begin{verbatim}
# A tibble: 34 x 11
    case  brain   liver tlratio solution sactime postin sex   wt.init
   <int>  <int>   <int>   <dbl> <fct>      <dbl>  <int> <fct>   <int>
 1     1  41081 1456164  0.0282 BD         0.500     10 F         239
 2     2  44286 1602171  0.0276 BD         0.500     10 F         225
 3     3 102926 1601936  0.0642 BD         0.500     10 F         224
 4     4  25927 1776411  0.0146 BD         0.500     10 F         184
 5     5  42643 1351184  0.0316 BD         0.500     10 F         250
 6     6  31342 1790863  0.0175 NS         0.500     10 F         196
 7     7  22815 1633386  0.0140 NS         0.500     10 F         200
 8     8  16629 1618757  0.0103 NS         0.500     10 F         273
 9     9  22315 1567602  0.0142 NS         0.500     10 F         216
10    10  77961 1060057  0.0735 BD         3.00      10 F         267
# ... with 24 more rows, and 2 more variables: wt.loss <dbl>,
#   wt.tumor <int>
\end{verbatim}

\section{\texorpdfstring{A Table 1 for
\texttt{bloodbrain}}{A Table 1 for bloodbrain}}\label{a-table-1-for-bloodbrain}

\citet{Barnett1995} did not provide a Table 1 for these data, so let's
build one to compare the two \texttt{solutions} (\texttt{BD} vs.
\texttt{NS}) on the covariates and outcomes, plus the natural logarithm
of the tumor/liver concentration ratio (\texttt{tlratio}). We'll opt to
treat the sacrifice time (\texttt{sactime}) and the days
post-inoculation of lung cancer cells (\texttt{postin}) as categorical
rather than quantitative variables.

\begin{Shaded}
\begin{Highlighting}[]
\NormalTok{bloodbrain <-}\StringTok{ }\NormalTok{bloodbrain }\OperatorTok
\StringTok{    }\KeywordTok{mutate}\NormalTok{(}\DataTypeTok{logTL =} \KeywordTok{log}\NormalTok{(tlratio))}

\KeywordTok{dput}\NormalTok{(}\KeywordTok{names}\NormalTok{(bloodbrain))}
\end{Highlighting}
\end{Shaded}

\begin{verbatim}
c("case", "brain", "liver", "tlratio", "solution", "sactime", 
"postin", "sex", "wt.init", "wt.loss", "wt.tumor", "logTL")
\end{verbatim}

OK - there's the list of variables we'll need. I'll put the outcomes at
the bottom of the table.

\begin{Shaded}
\begin{Highlighting}[]
\NormalTok{bb.vars <-}\StringTok{ }\KeywordTok{c}\NormalTok{(}\StringTok{"sactime"}\NormalTok{, }\StringTok{"postin"}\NormalTok{, }\StringTok{"sex"}\NormalTok{, }\StringTok{"wt.init"}\NormalTok{, }\StringTok{"wt.loss"}\NormalTok{, }
             \StringTok{"wt.tumor"}\NormalTok{, }\StringTok{"brain"}\NormalTok{, }\StringTok{"liver"}\NormalTok{, }\StringTok{"tlratio"}\NormalTok{, }\StringTok{"logTL"}\NormalTok{)}

\NormalTok{bb.factors <-}\StringTok{ }\KeywordTok{c}\NormalTok{(}\StringTok{"sactime"}\NormalTok{, }\StringTok{"sex"}\NormalTok{, }\StringTok{"postin"}\NormalTok{)}

\NormalTok{bb.att1 <-}\StringTok{ }\KeywordTok{CreateTableOne}\NormalTok{(}\DataTypeTok{data =}\NormalTok{ bloodbrain,}
                          \DataTypeTok{vars =}\NormalTok{ bb.vars,}
                          \DataTypeTok{factorVars =}\NormalTok{ bb.factors,}
                          \DataTypeTok{strata =} \KeywordTok{c}\NormalTok{(}\StringTok{"solution"}\NormalTok{))}
\KeywordTok{summary}\NormalTok{(bb.att1)}
\end{Highlighting}
\end{Shaded}

\begin{verbatim}

     ### Summary of continuous variables ###

solution: BD
          n miss p.miss   mean    sd median    p25   p75    min   max
wt.init  17    0      0    243 3e+01  2e+02  2e+02 3e+02  2e+02 3e+02
wt.loss  17    0      0      3 5e+00  4e+00  1e+00 6e+00 -5e+00 1e+01
wt.tumor 17    0      0    157 8e+01  2e+02  1e+02 2e+02  2e+01 4e+02
brain    17    0      0  56043 3e+04  5e+04  4e+04 8e+04  6e+03 1e+05
liver    17    0      0 672577 7e+05  6e+05  2e+04 1e+06  2e+03 2e+06
tlratio  17    0      0      2 3e+00  1e-01  6e-02 3e+00  1e-02 9e+00
logTL    17    0      0     -1 2e+00 -2e+00 -3e+00 1e+00 -4e+00 2e+00
          skew kurt
wt.init  -0.39  0.7
wt.loss  -0.10  0.2
wt.tumor  0.53  1.0
brain     0.29 -0.6
liver     0.35 -1.7
tlratio   1.58  1.7
logTL     0.08 -1.7
-------------------------------------------------------- 
solution: NS
          n miss p.miss   mean    sd median    p25    p75    min   max
wt.init  17    0      0    240 3e+01  2e+02  2e+02  3e+02  2e+02 3e+02
wt.loss  17    0      0      4 4e+00  3e+00  2e+00  7e+00 -4e+00 1e+01
wt.tumor 17    0      0    209 1e+02  2e+02  2e+02  3e+02  3e+01 5e+02
brain    17    0      0  23887 1e+04  2e+04  1e+04  3e+04  1e+03 5e+04
liver    17    0      0 664975 7e+05  7e+05  2e+04  1e+06  9e+02 2e+06
tlratio  17    0      0      1 2e+00  5e-02  3e-02  9e-01  1e-02 7e+00
logTL    17    0      0     -2 2e+00 -3e+00 -3e+00 -7e-02 -5e+00 2e+00
          skew  kurt
wt.init   0.33 -0.48
wt.loss  -0.09  0.08
wt.tumor  0.63  0.77
brain     0.30 -0.35
liver     0.40 -1.56
tlratio   2.27  4.84
logTL     0.27 -1.61

p-values
             pNormal  pNonNormal
wt.init  0.807308940 0.641940278
wt.loss  0.683756156 0.876749808
wt.tumor 0.151510151 0.190482094
brain    0.001027678 0.002579901
liver    0.974853609 0.904045603
tlratio  0.320501715 0.221425879
logTL    0.351633525 0.221425879

Standardize mean differences
             1 vs 2
wt.init  0.08435244
wt.loss  0.14099823
wt.tumor 0.50397184
brain    1.23884159
liver    0.01089667
tlratio  0.34611465
logTL    0.32420504

=======================================================================================

     ### Summary of categorical variables ### 

solution: BD
     var  n miss p.miss level freq percent cum.percent
 sactime 17    0    0.0   0.5    5    29.4        29.4
                            3    4    23.5        52.9
                           24    4    23.5        76.5
                           72    4    23.5       100.0
                                                      
  postin 17    0    0.0     9    1     5.9         5.9
                           10   14    82.4        88.2
                           11    2    11.8       100.0
                                                      
     sex 17    0    0.0     F   13    76.5        76.5
                            M    4    23.5       100.0
                                                      
-------------------------------------------------------- 
solution: NS
     var  n miss p.miss level freq percent cum.percent
 sactime 17    0    0.0   0.5    4    23.5        23.5
                            3    5    29.4        52.9
                           24    4    23.5        76.5
                           72    4    23.5       100.0
                                                      
  postin 17    0    0.0     9    2    11.8        11.8
                           10   13    76.5        88.2
                           11    2    11.8       100.0
                                                      
     sex 17    0    0.0     F   13    76.5        76.5
                            M    4    23.5       100.0
                                                      

p-values
          pApprox pExact
sactime 0.9739246      1
postin  0.8309504      1
sex     1.0000000      1

Standardize mean differences
           1 vs 2
sactime 0.1622214
postin  0.2098877
sex     0.0000000
\end{verbatim}

Note that, in this particular case, the decisions we make about
normality vs.~non-normality (for quantitative variables) and the
decisions we make about approximate vs.~exact testing (for categorical
variables) won't actually change the implications of the \emph{p}
values. Each approach gives similar results for each variable. Of
course, that's not always true.

\subsection{\texorpdfstring{Generate final Table 1 for
\texttt{bloodbrain}}{Generate final Table 1 for bloodbrain}}\label{generate-final-table-1-for-bloodbrain}

I'll choose to treat \texttt{tlratio} and its logarithm as non-Normal,
but otherwise, use t tests, but admittedly, that's an arbitrary
decision, really.

\begin{Shaded}
\begin{Highlighting}[]
\KeywordTok{print}\NormalTok{(bb.att1, }\DataTypeTok{nonnormal =} \KeywordTok{c}\NormalTok{(}\StringTok{"tlratio"}\NormalTok{, }\StringTok{"logTL"}\NormalTok{))}
\end{Highlighting}
\end{Shaded}

\begin{verbatim}
                        Stratified by solution
                         BD                      NS                      
  n                             17                      17               
  sactime (%)                                                            
     0.5                         5 (29.4)                4 (23.5)        
     3                           4 (23.5)                5 (29.4)        
     24                          4 (23.5)                4 (23.5)        
     72                          4 (23.5)                4 (23.5)        
  postin (%)                                                             
     9                           1 ( 5.9)                2 (11.8)        
     10                         14 (82.4)               13 (76.5)        
     11                          2 (11.8)                2 (11.8)        
  sex = M (%)                    4 (23.5)                4 (23.5)        
  wt.init (mean (sd))       242.82 (27.23)          240.47 (28.54)       
  wt.loss (mean (sd))         3.34 (4.68)             3.94 (3.88)        
  wt.tumor (mean (sd))      157.29 (84.00)          208.53 (116.68)      
  brain (mean (sd))       56043.41 (33675.40)     23887.18 (14610.53)    
  liver (mean (sd))      672577.35 (694479.58)   664975.47 (700773.13)   
  tlratio (median [IQR])      0.12 [0.06, 2.84]       0.05 [0.03, 0.94]  
  logTL (median [IQR])       -2.10 [-2.74, 1.04]     -2.95 [-3.41, -0.07]
                        Stratified by solution
                         p      test   
  n                                    
  sactime (%)             0.974        
     0.5                               
     3                                 
     24                                
     72                                
  postin (%)              0.831        
     9                                 
     10                                
     11                                
  sex = M (%)             1.000        
  wt.init (mean (sd))     0.807        
  wt.loss (mean (sd))     0.684        
  wt.tumor (mean (sd))    0.152        
  brain (mean (sd))       0.001        
  liver (mean (sd))       0.975        
  tlratio (median [IQR])  0.221 nonnorm
  logTL (median [IQR])    0.221 nonnorm
\end{verbatim}

Or, we can get an Excel-readable version placed in a \texttt{data}
subfolder, using

\begin{Shaded}
\begin{Highlighting}[]
\NormalTok{bb.t1 <-}\StringTok{ }\KeywordTok{print}\NormalTok{(bb.att1, }\DataTypeTok{nonnormal =} \KeywordTok{c}\NormalTok{(}\StringTok{"tlratio"}\NormalTok{, }\StringTok{"logTL"}\NormalTok{), }\DataTypeTok{quote =} \OtherTok{FALSE}\NormalTok{,}
               \DataTypeTok{noSpaces =} \OtherTok{TRUE}\NormalTok{, }\DataTypeTok{printToggle =} \OtherTok{FALSE}\NormalTok{)}

\KeywordTok{write.csv}\NormalTok{(bb.t1, }\DataTypeTok{file =} \StringTok{"data/bb-table1.csv"}\NormalTok{)}
\end{Highlighting}
\end{Shaded}

which, when dropped into Excel, will look like this:

\includegraphics[width=0.9\linewidth]{images/bb-table1inExcel}

One thing I would definitely clean up here, in practice, is to change
the presentation of the \emph{p} value for \texttt{sex} from 1 to
\textgreater{} 0.99, or just omit it altogether. I'd also drop the
\texttt{computer-ese} where possible, add units for the measures, round
\textbf{a lot}, identify the outcomes carefully, and use notes to
indicate deviations from the main approach.

\subsection{A More Finished Version (after Cleanup in
Word)}\label{a-more-finished-version-after-cleanup-in-word}

\includegraphics[width=0.95\linewidth]{images/bb-table1inWord}

\chapter{Linear Regression on a small SMART data
set}\label{linear-regression-on-a-small-smart-data-set}

\section{BRFSS and SMART}\label{brfss-and-smart}

The Centers for Disease Control analyzes Behavioral Risk Factor
Surveillance System (BRFSS) survey data for specific metropolitan and
micropolitan statistical areas (MMSAs) in a program called the
\href{https://www.cdc.gov/brfss/smart/Smart_data.htm}{Selected
Metropolitan/Micropolitan Area Risk Trends of BRFSS} (SMART BRFSS.)

In this work, we will focus on
\href{https://www.cdc.gov/brfss/smart/smart_2016.html}{data from the
2016 SMART}, and in particular on data from the Cleveland-Elyria, OH,
Metropolitan Statistical Area. The purpose of this survey is to provide
localized health information that can help public health practitioners
identify local emerging health problems, plan and evaluate local
responses, and efficiently allocate resources to specific needs.

\subsection{Key resources}\label{key-resources}

\begin{itemize}
\tightlist
\item
  the full data are available in the form of the 2016 SMART BRFSS MMSA
  Data, found in a zipped
  \href{https://www.cdc.gov/brfss/smart/2016/MMSA2016_XPT.zip}{SAS
  Transport Format} file. The data were released in August 2017.
\item
  the
  \href{https://www.cdc.gov/brfss/smart/2016/mmsa_varlayout_16.pdf}{MMSA
  Variable Layout PDF} which simply lists the variables included in the
  data file
\item
  the
  \href{https://www.cdc.gov/brfss/annual_data/2016/pdf/2016_calculated_variables_version4.pdf}{Calculated
  Variables PDF} which describes the risk factors by data variable names
  - there is also an
  \href{https://www.cdc.gov/brfss/annual_data/2016/Summary_Matrix_16.html}{online
  summary matrix of these calculated variables}, as well.
\item
  the lengthy
  \href{https://www.cdc.gov/brfss/questionnaires/pdf-ques/2016_BRFSS_Questionnaire_FINAL.pdf}{2016
  Survey Questions PDF} which lists all questions asked as part of the
  BRFSS in 2016
\item
  the enormous
  \href{https://www.cdc.gov/brfss/annual_data/2016/pdf/codebook16_llcp.pdf}{Codebook
  for the 2016 BRFSS Survey PDF} which identifies the variables by name
  for us.
\end{itemize}

Later this term, we'll use all of those resources to help construct a
more complete data set than we'll study today. I'll also demonstrate how
I built the \texttt{smartcle1} data set that we'll use in this Chapter.

\section{\texorpdfstring{The \texttt{smartcle1} data:
Cookbook}{The smartcle1 data: Cookbook}}\label{the-smartcle1-data-cookbook}

The \texttt{smartcle1.csv} data file available on the Data and Code page
of \href{https://github.com/THOMASELOVE/432-2018}{our website} describes
information on 11 variables for 1036 respondents to the BRFSS 2016, who
live in the Cleveland-Elyria, OH, Metropolitan Statistical Area. The
variables in the \texttt{smartcle1.csv} file are listed below, along
with (in some cases) the BRFSS items that generate these responses.

\begin{longtable}[]{@{}rl@{}}
\toprule
\begin{minipage}[b]{0.14\columnwidth}\raggedleft\strut
Variable\strut
\end{minipage} & \begin{minipage}[b]{0.74\columnwidth}\raggedright\strut
Description\strut
\end{minipage}\tabularnewline
\midrule
\endhead
\begin{minipage}[t]{0.14\columnwidth}\raggedleft\strut
\texttt{SEQNO}\strut
\end{minipage} & \begin{minipage}[t]{0.74\columnwidth}\raggedright\strut
respondent identification number (all begin with 2016)\strut
\end{minipage}\tabularnewline
\begin{minipage}[t]{0.14\columnwidth}\raggedleft\strut
\texttt{physhealth}\strut
\end{minipage} & \begin{minipage}[t]{0.74\columnwidth}\raggedright\strut
Now thinking about your physical health, which includes physical illness
and injury, for how many days during the past 30 days was your physical
health not good?\strut
\end{minipage}\tabularnewline
\begin{minipage}[t]{0.14\columnwidth}\raggedleft\strut
\texttt{menthealth}\strut
\end{minipage} & \begin{minipage}[t]{0.74\columnwidth}\raggedright\strut
Now thinking about your mental health, which includes stress,
depression, and problems with emotions, for how many days during the
past 30 days was your mental health not good?\strut
\end{minipage}\tabularnewline
\begin{minipage}[t]{0.14\columnwidth}\raggedleft\strut
\texttt{poorhealth}\strut
\end{minipage} & \begin{minipage}[t]{0.74\columnwidth}\raggedright\strut
During the past 30 days, for about how many days did poor physical or
mental health keep you from doing your usual activities, such as
self-care, work, or recreation?\strut
\end{minipage}\tabularnewline
\begin{minipage}[t]{0.14\columnwidth}\raggedleft\strut
\texttt{genhealth}\strut
\end{minipage} & \begin{minipage}[t]{0.74\columnwidth}\raggedright\strut
Would you say that in general, your health is \ldots{} (five categories:
Excellent, Very Good, Good, Fair or Poor)\strut
\end{minipage}\tabularnewline
\begin{minipage}[t]{0.14\columnwidth}\raggedleft\strut
\texttt{bmi}\strut
\end{minipage} & \begin{minipage}[t]{0.74\columnwidth}\raggedright\strut
Body mass index, in kg/m\textsuperscript{2}\strut
\end{minipage}\tabularnewline
\begin{minipage}[t]{0.14\columnwidth}\raggedleft\strut
\texttt{female}\strut
\end{minipage} & \begin{minipage}[t]{0.74\columnwidth}\raggedright\strut
Sex, 1 = female, 0 = male\strut
\end{minipage}\tabularnewline
\begin{minipage}[t]{0.14\columnwidth}\raggedleft\strut
\texttt{internet30}\strut
\end{minipage} & \begin{minipage}[t]{0.74\columnwidth}\raggedright\strut
Have you used the internet in the past 30 days? (1 = yes, 0 = no)\strut
\end{minipage}\tabularnewline
\begin{minipage}[t]{0.14\columnwidth}\raggedleft\strut
\texttt{exerany}\strut
\end{minipage} & \begin{minipage}[t]{0.74\columnwidth}\raggedright\strut
During the past month, other than your regular job, did you participate
in any physical activities or exercises such as running, calisthenics,
golf, gardening, or walking for exercise? (1 = yes, 0 = no)\strut
\end{minipage}\tabularnewline
\begin{minipage}[t]{0.14\columnwidth}\raggedleft\strut
\texttt{sleephrs}\strut
\end{minipage} & \begin{minipage}[t]{0.74\columnwidth}\raggedright\strut
On average, how many hours of sleep do you get in a 24-hour
period?\strut
\end{minipage}\tabularnewline
\begin{minipage}[t]{0.14\columnwidth}\raggedleft\strut
\texttt{alcdays}\strut
\end{minipage} & \begin{minipage}[t]{0.74\columnwidth}\raggedright\strut
How many days during the past 30 days did you have at least one drink of
any alcoholic beverage such as beer, wine, a malt beverage or
liquor?\strut
\end{minipage}\tabularnewline
\bottomrule
\end{longtable}

\begin{Shaded}
\begin{Highlighting}[]
\KeywordTok{str}\NormalTok{(smartcle1)}
\end{Highlighting}
\end{Shaded}

\begin{verbatim}
Classes 'tbl_df', 'tbl' and 'data.frame':   1036 obs. of  11 variables:
 $ SEQNO     : num  2.02e+09 2.02e+09 2.02e+09 2.02e+09 2.02e+09 ...
 $ physhealth: int  0 0 1 0 5 4 2 2 0 0 ...
 $ menthealth: int  0 0 5 0 0 18 0 3 0 0 ...
 $ poorhealth: int  NA NA 0 NA 0 6 0 0 NA NA ...
 $ genhealth : Factor w/ 5 levels "1_Excellent",..: 2 1 2 3 1 2 3 3 2 3 ...
 $ bmi       : num  26.7 23.7 26.9 21.7 24.1 ...
 $ female    : int  1 0 0 1 0 0 1 1 0 0 ...
 $ internet30: int  1 1 1 1 1 1 1 1 1 1 ...
 $ exerany   : int  1 1 0 1 1 1 1 1 1 0 ...
 $ sleephrs  : int  6 6 8 9 7 5 9 7 7 7 ...
 $ alcdays   : int  1 4 4 3 2 28 4 2 4 25 ...
\end{verbatim}

\section{\texorpdfstring{\texttt{smartcle2}: Omitting Missing
Observations: Complete-Case
Analyses}{smartcle2: Omitting Missing Observations: Complete-Case Analyses}}\label{smartcle2-omitting-missing-observations-complete-case-analyses}

For the purpose of fitting our first few models, we will eliminate the
missingness problem, and look only at the \emph{complete cases} in our
\texttt{smartcle1} data. We will discuss methods for imputing missing
data later in these Notes.

To inspect the missingness in our data, we might consider using the
\texttt{skim} function from the \texttt{skimr} package. We'll exclude
the respondent identifier code (\texttt{SEQNO}) from this summary as
uninteresting.

\begin{Shaded}
\begin{Highlighting}[]
\KeywordTok{skim_with}\NormalTok{(}\DataTypeTok{numeric =} \KeywordTok{list}\NormalTok{(}\DataTypeTok{hist =} \OtherTok{NULL}\NormalTok{), }\DataTypeTok{integer =} \KeywordTok{list}\NormalTok{(}\DataTypeTok{hist =} \OtherTok{NULL}\NormalTok{))}
\NormalTok{## above line eliminates the sparkline histograms}
\NormalTok{## it can be commented out when working in the console,}
\NormalTok{## but I need it to produce the Notes without errors right now}

\NormalTok{smartcle1 }\OperatorTok\StringTok{ }
\StringTok{    }\KeywordTok{skim}\NormalTok{(}\OperatorTok{-}\NormalTok{SEQNO)}
\end{Highlighting}
\end{Shaded}

\begin{verbatim}
Skim summary statistics
 n obs: 1036 
 n variables: 11 

Variable type: factor 
  variable missing complete    n n_unique
 genhealth       3     1033 1036        5
                             top_counts ordered
 2_V: 350, 3_G: 344, 1_E: 173, 4_F: 122   FALSE

Variable type: integer 
   variable missing complete    n mean   sd p0 p25 median p75 p100
    alcdays      46      990 1036 4.65 8.05  0   0      1   4   30
    exerany       3     1033 1036 0.76 0.43  0   1      1   1    1
     female       0     1036 1036 0.6  0.49  0   0      1   1    1
 internet30       6     1030 1036 0.81 0.39  0   1      1   1    1
 menthealth      11     1025 1036 2.72 6.82  0   0      0   2   30
 physhealth      17     1019 1036 3.97 8.67  0   0      0   2   30
 poorhealth     543      493 1036 4.07 8.09  0   0      0   3   30
   sleephrs       8     1028 1036 7.02 1.53  1   6      7   8   20

Variable type: numeric 
 variable missing complete    n  mean   sd    p0  p25 median   p75  p100
      bmi      84      952 1036 27.89 6.47 12.71 23.7  26.68 30.53 66.06
\end{verbatim}

Now, we'll create a new tibble called \texttt{smartcle2} which contains
every variable except \texttt{poorhealth}, and which includes all
respondents with complete data on the variables (other than
\texttt{poorhealth}). We'll store those observations with complete data
in the \texttt{smartcle2} tibble.

\begin{Shaded}
\begin{Highlighting}[]
\NormalTok{smartcle2 <-}\StringTok{ }\NormalTok{smartcle1 }\OperatorTok\StringTok{ }
\StringTok{    }\KeywordTok{select}\NormalTok{(}\OperatorTok{-}\NormalTok{poorhealth) }\OperatorTok
\StringTok{    }\KeywordTok{filter}\NormalTok{(}\KeywordTok{complete.cases}\NormalTok{(.))}

\NormalTok{smartcle2}
\end{Highlighting}
\end{Shaded}

\begin{verbatim}
# A tibble: 896 x 10
     SEQNO physhealth menthealth genhealth   bmi female internet30 exerany
     <dbl>      <int>      <int> <fct>     <dbl>  <int>      <int>   <int>
 1  2.02e9          0          0 2_VeryGo~  26.7      1          1       1
 2  2.02e9          0          0 1_Excell~  23.7      0          1       1
 3  2.02e9          1          5 2_VeryGo~  26.9      0          1       0
 4  2.02e9          0          0 3_Good     21.7      1          1       1
 5  2.02e9          5          0 1_Excell~  24.1      0          1       1
 6  2.02e9          4         18 2_VeryGo~  27.6      0          1       1
 7  2.02e9          2          0 3_Good     25.7      1          1       1
 8  2.02e9          2          3 3_Good     28.5      1          1       1
 9  2.02e9          0          0 2_VeryGo~  28.6      0          1       1
10  2.02e9          0          0 3_Good     23.1      0          1       0
# ... with 886 more rows, and 2 more variables: sleephrs <int>,
#   alcdays <int>
\end{verbatim}

Note that there are only 896 respondents with \textbf{complete} data on
the 10 variables (excluding \texttt{poorhealth}) in the
\texttt{smartcle2} tibble, as compared to our original
\texttt{smartcle1} data which described 1036 respondents and 11
variables, but with lots of missing data.

\section{\texorpdfstring{Summarizing the \texttt{smartcle2} data
numerically}{Summarizing the smartcle2 data numerically}}\label{summarizing-the-smartcle2-data-numerically}

\subsection{\texorpdfstring{The New Toy: The \texttt{skim}
function}{The New Toy: The skim function}}\label{the-new-toy-the-skim-function}

\begin{Shaded}
\begin{Highlighting}[]
\KeywordTok{skim}\NormalTok{(smartcle2, }\OperatorTok{-}\NormalTok{SEQNO)}
\end{Highlighting}
\end{Shaded}

\begin{verbatim}
Skim summary statistics
 n obs: 896 
 n variables: 10 

Variable type: factor 
  variable missing complete   n n_unique
 genhealth       0      896 896        5
                             top_counts ordered
 2_V: 306, 3_G: 295, 1_E: 155, 4_F: 102   FALSE

Variable type: integer 
   variable missing complete   n mean   sd p0 p25 median p75 p100
    alcdays       0      896 896 4.83 8.14  0   0      1   5   30
    exerany       0      896 896 0.77 0.42  0   1      1   1    1
     female       0      896 896 0.58 0.49  0   0      1   1    1
 internet30       0      896 896 0.81 0.39  0   1      1   1    1
 menthealth       0      896 896 2.69 6.72  0   0      0   2   30
 physhealth       0      896 896 3.99 8.64  0   0      0   2   30
   sleephrs       0      896 896 7.02 1.48  1   6      7   8   20

Variable type: numeric 
 variable missing complete   n  mean   sd    p0  p25 median   p75  p100
      bmi       0      896 896 27.87 6.33 12.71 23.7   26.8 30.53 66.06
\end{verbatim}

\subsection{\texorpdfstring{The usual \texttt{summary} for a data
frame}{The usual summary for a data frame}}\label{the-usual-summary-for-a-data-frame}

Of course, we can use the usual \texttt{summary} to get some basic
information about the data.

\begin{Shaded}
\begin{Highlighting}[]
\KeywordTok{summary}\NormalTok{(smartcle2)}
\end{Highlighting}
\end{Shaded}

\begin{verbatim}
     SEQNO             physhealth      menthealth           genhealth  
 Min.   :2.016e+09   Min.   : 0.00   Min.   : 0.000   1_Excellent:155  
 1st Qu.:2.016e+09   1st Qu.: 0.00   1st Qu.: 0.000   2_VeryGood :306  
 Median :2.016e+09   Median : 0.00   Median : 0.000   3_Good     :295  
 Mean   :2.016e+09   Mean   : 3.99   Mean   : 2.693   4_Fair     :102  
 3rd Qu.:2.016e+09   3rd Qu.: 2.00   3rd Qu.: 2.000   5_Poor     : 38  
 Max.   :2.016e+09   Max.   :30.00   Max.   :30.000                    
      bmi            female         internet30        exerany      
 Min.   :12.71   Min.   :0.0000   Min.   :0.0000   Min.   :0.0000  
 1st Qu.:23.70   1st Qu.:0.0000   1st Qu.:1.0000   1st Qu.:1.0000  
 Median :26.80   Median :1.0000   Median :1.0000   Median :1.0000  
 Mean   :27.87   Mean   :0.5848   Mean   :0.8147   Mean   :0.7667  
 3rd Qu.:30.53   3rd Qu.:1.0000   3rd Qu.:1.0000   3rd Qu.:1.0000  
 Max.   :66.06   Max.   :1.0000   Max.   :1.0000   Max.   :1.0000  
    sleephrs         alcdays      
 Min.   : 1.000   Min.   : 0.000  
 1st Qu.: 6.000   1st Qu.: 0.000  
 Median : 7.000   Median : 1.000  
 Mean   : 7.022   Mean   : 4.834  
 3rd Qu.: 8.000   3rd Qu.: 5.000  
 Max.   :20.000   Max.   :30.000  
\end{verbatim}

\subsection{\texorpdfstring{The \texttt{describe} function in
\texttt{Hmisc}}{The describe function in Hmisc}}\label{the-describe-function-in-hmisc}

Or we can use the \texttt{describe} function from the \texttt{Hmisc}
package.

\begin{Shaded}
\begin{Highlighting}[]
\NormalTok{Hmisc}\OperatorTok{::}\KeywordTok{describe}\NormalTok{(}\KeywordTok{select}\NormalTok{(smartcle2, bmi, genhealth, female))}
\end{Highlighting}
\end{Shaded}

\begin{verbatim}
select(smartcle2, bmi, genhealth, female) 

 3  Variables      896  Observations
---------------------------------------------------------------------------
bmi 
       n  missing distinct     Info     Mean      Gmd      .05      .10 
     896        0      467        1    27.87    6.572    20.06    21.23 
     .25      .50      .75      .90      .95 
   23.70    26.80    30.53    35.36    39.30 

lowest : 12.71 13.34 14.72 16.22 17.30, highest: 56.89 57.04 60.95 61.84 66.06
---------------------------------------------------------------------------
genhealth 
       n  missing distinct 
     896        0        5 
                                                                      
Value      1_Excellent  2_VeryGood      3_Good      4_Fair      5_Poor
Frequency          155         306         295         102          38
Proportion       0.173       0.342       0.329       0.114       0.042
---------------------------------------------------------------------------
female 
       n  missing distinct     Info      Sum     Mean      Gmd 
     896        0        2    0.728      524   0.5848   0.4862 

---------------------------------------------------------------------------
\end{verbatim}

\section{Counting as exploratory data
analysis}\label{counting-as-exploratory-data-analysis}

Counting things can be amazingly useful.

\subsection{How many respondents had exercised in the past 30 days? Did
this vary by
sex?}\label{how-many-respondents-had-exercised-in-the-past-30-days-did-this-vary-by-sex}

\begin{Shaded}
\begin{Highlighting}[]
\NormalTok{smartcle2 }\OperatorTok\StringTok{ }\KeywordTok{count}\NormalTok{(female, exerany) }\OperatorTok\StringTok{ }\KeywordTok{mutate}\NormalTok{(}\DataTypeTok{percent =} \DecValTok{100}\OperatorTok{*}\NormalTok{n }\OperatorTok{/}\StringTok{ }\KeywordTok{sum}\NormalTok{(n))}
\end{Highlighting}
\end{Shaded}

\begin{verbatim}
# A tibble: 4 x 4
  female exerany     n percent
   <int>   <int> <int>   <dbl>
1      0       0    64    7.14
2      0       1   308   34.4 
3      1       0   145   16.2 
4      1       1   379   42.3 
\end{verbatim}

so we know now that 42.3\% of the subjects in our data were women who
exercised. Suppose that instead we want to find the percentage of
exercisers within each sex\ldots{}

\begin{Shaded}
\begin{Highlighting}[]
\NormalTok{smartcle2 }\OperatorTok
\StringTok{    }\KeywordTok{count}\NormalTok{(female, exerany) }\OperatorTok
\StringTok{    }\KeywordTok{group_by}\NormalTok{(female) }\OperatorTok
\StringTok{    }\KeywordTok{mutate}\NormalTok{(}\DataTypeTok{prob =} \DecValTok{100}\OperatorTok{*}\NormalTok{n }\OperatorTok{/}\StringTok{ }\KeywordTok{sum}\NormalTok{(n)) }
\end{Highlighting}
\end{Shaded}

\begin{verbatim}
# A tibble: 4 x 4
# Groups:   female [2]
  female exerany     n  prob
   <int>   <int> <int> <dbl>
1      0       0    64  17.2
2      0       1   308  82.8
3      1       0   145  27.7
4      1       1   379  72.3
\end{verbatim}

and now we know that 82.8\% of the males exercised at least once in the
last 30 days, as compared to 72.3\% of the females.

\subsection{\texorpdfstring{What's the distribution of
\texttt{sleephrs}?}{What's the distribution of sleephrs?}}\label{whats-the-distribution-of-sleephrs}

We can count quantitative variables with discrete sets of possible
values, like \texttt{sleephrs}, which is captured as an integer (that
must fall between 0 and 24.)

\begin{Shaded}
\begin{Highlighting}[]
\NormalTok{smartcle2 }\OperatorTok\StringTok{ }\KeywordTok{count}\NormalTok{(sleephrs)}
\end{Highlighting}
\end{Shaded}

\begin{verbatim}
# A tibble: 14 x 2
   sleephrs     n
      <int> <int>
 1        1     5
 2        2     1
 3        3     6
 4        4    20
 5        5    63
 6        6   192
 7        7   276
 8        8   266
 9        9    38
10       10    22
11       11     2
12       12     2
13       16     2
14       20     1
\end{verbatim}

Of course, a natural summary of a quantitative variable like this would
be graphical.

\begin{Shaded}
\begin{Highlighting}[]
\KeywordTok{ggplot}\NormalTok{(smartcle2, }\KeywordTok{aes}\NormalTok{(sleephrs)) }\OperatorTok{+}
\StringTok{    }\KeywordTok{geom_histogram}\NormalTok{(}\DataTypeTok{binwidth =} \DecValTok{1}\NormalTok{, }\DataTypeTok{fill =} \StringTok{"dodgerblue"}\NormalTok{, }\DataTypeTok{col =} \StringTok{"darkred"}\NormalTok{)}
\end{Highlighting}
\end{Shaded}

\includegraphics{bookdown-demo_files/figure-latex/c2_histogram_sleephrs_smartcle2-1.pdf}

\subsection{\texorpdfstring{What's the distribution of
\texttt{BMI}?}{What's the distribution of BMI?}}\label{whats-the-distribution-of-bmi}

\begin{Shaded}
\begin{Highlighting}[]
\KeywordTok{ggplot}\NormalTok{(smartcle2, }\KeywordTok{aes}\NormalTok{(bmi)) }\OperatorTok{+}
\StringTok{    }\KeywordTok{geom_histogram}\NormalTok{(}\DataTypeTok{bins =} \DecValTok{30}\NormalTok{, }\DataTypeTok{col =} \StringTok{"white"}\NormalTok{)}
\end{Highlighting}
\end{Shaded}

\includegraphics{bookdown-demo_files/figure-latex/c2_histogram_bmi_smartcle2-1.pdf}

\subsection{How many of the respondents have a BMI below
30?}\label{how-many-of-the-respondents-have-a-bmi-below-30}

\begin{Shaded}
\begin{Highlighting}[]
\NormalTok{smartcle2 }\OperatorTok\StringTok{ }\KeywordTok{count}\NormalTok{(bmi }\OperatorTok{<}\StringTok{ }\DecValTok{30}\NormalTok{) }\OperatorTok\StringTok{ }\KeywordTok{mutate}\NormalTok{(}\DataTypeTok{proportion =}\NormalTok{ n }\OperatorTok{/}\StringTok{ }\KeywordTok{sum}\NormalTok{(n))}
\end{Highlighting}
\end{Shaded}

\begin{verbatim}
# A tibble: 2 x 3
  `bmi < 30`     n proportion
  <lgl>      <int>      <dbl>
1 F            253      0.282
2 T            643      0.718
\end{verbatim}

\subsection{How many of the respondents who have a BMI \textless{} 30
exercised?}\label{how-many-of-the-respondents-who-have-a-bmi-30-exercised}

\begin{Shaded}
\begin{Highlighting}[]
\NormalTok{smartcle2 }\OperatorTok\StringTok{ }\KeywordTok{count}\NormalTok{(exerany, bmi }\OperatorTok{<}\StringTok{ }\DecValTok{30}\NormalTok{) }\OperatorTok
\StringTok{    }\KeywordTok{group_by}\NormalTok{(exerany) }\OperatorTok
\StringTok{    }\KeywordTok{mutate}\NormalTok{(}\DataTypeTok{percent =} \DecValTok{100}\OperatorTok{*}\NormalTok{n}\OperatorTok{/}\KeywordTok{sum}\NormalTok{(n))}
\end{Highlighting}
\end{Shaded}

\begin{verbatim}
# A tibble: 4 x 4
# Groups:   exerany [2]
  exerany `bmi < 30`     n percent
    <int> <lgl>      <int>   <dbl>
1       0 F             88    42.1
2       0 T            121    57.9
3       1 F            165    24.0
4       1 T            522    76.0
\end{verbatim}

\subsection{Is obesity associated with sex, in these
data?}\label{is-obesity-associated-with-sex-in-these-data}

\begin{Shaded}
\begin{Highlighting}[]
\NormalTok{smartcle2 }\OperatorTok\StringTok{ }\KeywordTok{count}\NormalTok{(female, bmi }\OperatorTok{<}\StringTok{ }\DecValTok{30}\NormalTok{) }\OperatorTok
\StringTok{    }\KeywordTok{group_by}\NormalTok{(female) }\OperatorTok
\StringTok{    }\KeywordTok{mutate}\NormalTok{(}\DataTypeTok{percent =} \DecValTok{100}\OperatorTok{*}\NormalTok{n}\OperatorTok{/}\KeywordTok{sum}\NormalTok{(n))}
\end{Highlighting}
\end{Shaded}

\begin{verbatim}
# A tibble: 4 x 4
# Groups:   female [2]
  female `bmi < 30`     n percent
   <int> <lgl>      <int>   <dbl>
1      0 F            105    28.2
2      0 T            267    71.8
3      1 F            148    28.2
4      1 T            376    71.8
\end{verbatim}

\subsection{\texorpdfstring{Comparing \texttt{sleephrs} summaries by
obesity
status}{Comparing sleephrs summaries by obesity status}}\label{comparing-sleephrs-summaries-by-obesity-status}

Can we compare the \texttt{sleephrs} means, medians and
75\textsuperscript{th} percentiles for respondents whose BMI is below 30
to the respondents whose BMI is not?

\begin{Shaded}
\begin{Highlighting}[]
\NormalTok{smartcle2 }\OperatorTok
\StringTok{    }\KeywordTok{group_by}\NormalTok{(bmi }\OperatorTok{<}\StringTok{ }\DecValTok{30}\NormalTok{) }\OperatorTok
\StringTok{    }\KeywordTok{summarize}\NormalTok{(}\KeywordTok{mean}\NormalTok{(sleephrs), }\KeywordTok{median}\NormalTok{(sleephrs), }
              \DataTypeTok{q75 =} \KeywordTok{quantile}\NormalTok{(sleephrs, }\FloatTok{0.75}\NormalTok{))}
\end{Highlighting}
\end{Shaded}

\begin{verbatim}
# A tibble: 2 x 4
  `bmi < 30` `mean(sleephrs)` `median(sleephrs)`   q75
  <lgl>                 <dbl>              <int> <dbl>
1 F                      6.93                  7  8.00
2 T                      7.06                  7  8.00
\end{verbatim}

\subsection{\texorpdfstring{The \texttt{skim} function within a
pipe}{The skim function within a pipe}}\label{the-skim-function-within-a-pipe}

The \textbf{skim} function works within pipes and with the other
\texttt{tidyverse} functions.

\begin{Shaded}
\begin{Highlighting}[]
\NormalTok{smartcle2 }\OperatorTok
\StringTok{    }\KeywordTok{group_by}\NormalTok{(exerany) }\OperatorTok
\StringTok{    }\KeywordTok{skim}\NormalTok{(bmi, sleephrs)}
\end{Highlighting}
\end{Shaded}

\begin{verbatim}
Skim summary statistics
 n obs: 896 
 n variables: 10 
 group variables: exerany 

Variable type: integer 
 exerany variable missing complete   n mean   sd p0 p25 median p75 p100
       0 sleephrs       0      209 209 7    1.85  1   6      7   8   20
       1 sleephrs       0      687 687 7.03 1.34  1   6      7   8   16

Variable type: numeric 
 exerany variable missing complete   n  mean   sd    p0   p25 median   p75
       0      bmi       0      209 209 29.57 7.46 18    24.11  28.49 33.13
       1      bmi       0      687 687 27.35 5.84 12.71 23.7   26.52 29.81
  p100
 66.06
 60.95
\end{verbatim}

\section{\texorpdfstring{First Modeling Attempt: Can \texttt{bmi}
predict
\texttt{physhealth}?}{First Modeling Attempt: Can bmi predict physhealth?}}\label{first-modeling-attempt-can-bmi-predict-physhealth}

We'll start with an effort to predict \texttt{physhealth} using
\texttt{bmi}. A natural graph would be a scatterplot.

\begin{Shaded}
\begin{Highlighting}[]
\KeywordTok{ggplot}\NormalTok{(}\DataTypeTok{data =}\NormalTok{ smartcle2, }\KeywordTok{aes}\NormalTok{(}\DataTypeTok{x =}\NormalTok{ bmi, }\DataTypeTok{y =}\NormalTok{ physhealth)) }\OperatorTok{+}
\StringTok{    }\KeywordTok{geom_point}\NormalTok{()}
\end{Highlighting}
\end{Shaded}

\includegraphics{bookdown-demo_files/figure-latex/scatter_physhealth_bmi_1-1.pdf}

A good question to ask ourselves here might be: ``In what BMI range can
we make a reasonable prediction of \texttt{physhealth}?''

Now, we might take the plot above and add a simple linear model \ldots{}

\begin{Shaded}
\begin{Highlighting}[]
\KeywordTok{ggplot}\NormalTok{(}\DataTypeTok{data =}\NormalTok{ smartcle2, }\KeywordTok{aes}\NormalTok{(}\DataTypeTok{x =}\NormalTok{ bmi, }\DataTypeTok{y =}\NormalTok{ physhealth)) }\OperatorTok{+}
\StringTok{    }\KeywordTok{geom_point}\NormalTok{() }\OperatorTok{+}
\StringTok{    }\KeywordTok{geom_smooth}\NormalTok{(}\DataTypeTok{method =} \StringTok{"lm"}\NormalTok{, }\DataTypeTok{se =} \OtherTok{FALSE}\NormalTok{)}
\end{Highlighting}
\end{Shaded}

\includegraphics{bookdown-demo_files/figure-latex/c2_scatter_physhealth_bmi_2-1.pdf}

which shows the same least squares regression model that we can fit with
the \texttt{lm} command.

\subsection{Fitting a Simple Regression
Model}\label{fitting-a-simple-regression-model}

\begin{Shaded}
\begin{Highlighting}[]
\NormalTok{model_A <-}\StringTok{ }\KeywordTok{lm}\NormalTok{(physhealth }\OperatorTok{~}\StringTok{ }\NormalTok{bmi, }\DataTypeTok{data =}\NormalTok{ smartcle2)}

\NormalTok{model_A}
\end{Highlighting}
\end{Shaded}

\begin{verbatim}

Call:
lm(formula = physhealth ~ bmi, data = smartcle2)

Coefficients:
(Intercept)          bmi  
    -1.4514       0.1953  
\end{verbatim}

\begin{Shaded}
\begin{Highlighting}[]
\KeywordTok{summary}\NormalTok{(model_A)}
\end{Highlighting}
\end{Shaded}

\begin{verbatim}

Call:
lm(formula = physhealth ~ bmi, data = smartcle2)

Residuals:
   Min     1Q Median     3Q    Max 
-9.171 -4.057 -3.193 -1.576 28.073 

Coefficients:
            Estimate Std. Error t value Pr(>|t|)    
(Intercept) -1.45143    1.29185  -1.124    0.262    
bmi          0.19527    0.04521   4.319 1.74e-05 ***
---
Signif. codes:  0 '***' 0.001 '**' 0.01 '*' 0.05 '.' 0.1 ' ' 1

Residual standard error: 8.556 on 894 degrees of freedom
Multiple R-squared:  0.02044,   Adjusted R-squared:  0.01934 
F-statistic: 18.65 on 1 and 894 DF,  p-value: 1.742e-05
\end{verbatim}

\begin{Shaded}
\begin{Highlighting}[]
\KeywordTok{confint}\NormalTok{(model_A, }\DataTypeTok{level =} \FloatTok{0.95}\NormalTok{)}
\end{Highlighting}
\end{Shaded}

\begin{verbatim}
                 2.5 %    97.5 %
(Intercept) -3.9868457 1.0839862
bmi          0.1065409 0.2840068
\end{verbatim}

The model coefficients can be obtained by printing the model object, and
the \texttt{summary} function provides several useful descriptions of
the model's residuals, its statistical significance, and quality of fit.

\subsection{Model Summary for a Simple (One-Predictor)
Regression}\label{model-summary-for-a-simple-one-predictor-regression}

The fitted model predicts \texttt{physhealth} with the equation -1.45 +
0.195*\texttt{bmi}, as we can read off from the model coefficients.

Each of the 896 respondents included in the \texttt{smartcle2} data
makes a contribution to this model.

\subsubsection{Residuals}\label{residuals}

Suppose Harry is one of the people in that group, and Harry's data is
\texttt{bmi} = 20, and \texttt{physhealth} = 3.

\begin{itemize}
\tightlist
\item
  Harry's \emph{observed} value of \texttt{physhealth} is just the value
  we have in the data for them, in this case, observed
  \texttt{physhealth} = 3 for Harry.
\item
  Harry's \emph{fitted} or \emph{predicted} \texttt{physhealth} value is
  the result of calculating -1.45 + 0.195*\texttt{bmi} for Harry. So, if
  Harry's BMI was 20, then Harry's predicted \texttt{physhealth} value
  is -1.45 + (0.195)(20) = 2.45.
\item
  The \emph{residual} for Harry is then his \emph{observed} outcome
  minus his \emph{fitted} outcome, so Harry has a residual of 3 - 2.45 =
  0.55.
\item
  Graphically, a residual represents vertical distance between the
  observed point and the fitted regression line.
\item
  Points above the regression line will have positive residuals, and
  points below the regression line will have negative residuals. Points
  on the line have zero residuals.
\end{itemize}

The residuals are summarized at the top of the \texttt{summary} output
for linear model.

\begin{itemize}
\tightlist
\item
  The mean residual will always be zero in an ordinary least squares
  model, but a five number summary of the residuals is provided by the
  summary, as is an estimated standard deviation of the residuals
  (called here the Residual standard error.)
\item
  In the \texttt{smartcle2} data, the minimum residual was -9.17, so for
  one subject, the observed value was 9.17 days smaller than the
  predicted value. This means that the prediction was 9.17 days too
  large for that subject.
\item
  Similarly, the maximum residual was 28.07 days, so for one subject the
  prediction was 28.07 days too small. Not a strong performance.
\item
  In a least squares model, the residuals are assumed to follow a Normal
  distribution, with mean zero, and standard deviation (for the
  \texttt{smartcle2} data) of about 8.6 days. Thus, by the definition of
  a Normal distribution, we'd expect
\item
  about 68\% of the residuals to be between -8.6 and +8.6 days,
\item
  about 95\% of the residuals to be between -17.2 and +17.2 days,
\item
  about all (99.7\%) of the residuals to be between -25.8 and +25.8
  days.
\end{itemize}

\subsubsection{Coefficients section}\label{coefficients-section}

The \texttt{summary} for a linear model shows Estimates, Standard
Errors, t values and \emph{p} values for each coefficient fit.

\begin{itemize}
\tightlist
\item
  The Estimates are the point estimates of the intercept and slope of
  \texttt{bmi} in our model.
\item
  In this case, our estimated slope is 0.195, which implies that if
  Harry's BMI is 20 and Sally's BMI is 21, we predict that Sally's
  \texttt{physhealth} will be 0.195 days larger than Harry's.
\item
  The Standard Errors are also provided for each estimate. We can create
  rough 95\% confidence intervals by adding and subtracting two standard
  errors from each coefficient, or we can get a slightly more accurate
  answer with the \texttt{confint} function.
\item
  Here, the 95\% confidence interval for the slope of \texttt{bmi} is
  estimated to be (0.11, 0.28). This is a good measure of the
  uncertainty in the slope that is captured by our model. We are 95\%
  confident in the process of building this interval, but this doesn't
  mean we're 95\% sure that the true slope is actually in that interval.
\end{itemize}

Also available are a \emph{t} value (just the Estimate divided by the
Standard Error) and the appropriate \emph{p} value for testing the null
hypothesis that the true value of the coefficient is 0 against a
two-tailed alternative.

\begin{itemize}
\tightlist
\item
  If a slope coefficient is statistically significantly different from
  0, this implies that 0 will not be part of the uncertainty interval
  obtained through \texttt{confint}.
\item
  If the slope was zero, it would suggest that \texttt{bmi} would add no
  predictive value to the model. But that's unlikely here.
\end{itemize}

If the \texttt{bmi} slope coefficient is associated with a small
\emph{p} value, as in the case of our \texttt{model\_A}, it suggests
that the model including \texttt{bmi} is statistically significantly
better at predicting \texttt{physhealth} than the model without
\texttt{bmi}.

\begin{itemize}
\tightlist
\item
  Without \texttt{bmi} our \texttt{model\_A} would become an
  \emph{intercept-only} model, in this case, which would predict the
  mean \texttt{physhealth} for everyone, regardless of any other
  information.
\end{itemize}

\subsubsection{Model Fit Summaries}\label{model-fit-summaries}

The \texttt{summary} of a linear model also displays:

\begin{itemize}
\tightlist
\item
  The residual standard error and associated degrees of freedom for the
  residuals.
\item
  For a simple (one-predictor) least regression like this, the residual
  degrees of freedom will be the sample size minus 2.
\item
  The multiple R-squared (or coefficient of determination)
\item
  This is interpreted as the proportion of variation in the outcome
  (\texttt{physhealth}) accounted for by the model, and will always fall
  between 0 and 1 as a result.
\item
  Our model\_A accounts for a mere 2\% of the variation in
  \texttt{physhealth}.
\item
  The Adjusted R-squared value ``adjusts'' for the size of our model in
  terms of the number of coefficients included in the model.
\item
  The adjusted R-squared will always be less than the Multiple
  R-squared.
\item
  We still hope to find models with relatively large adjusted
  R\textsuperscript{2} values.
\item
  In particular, we hope to find models where the adjusted
  R\textsuperscript{2} isn't substantially less than the Multiple
  R-squared.
\item
  The adjusted R-squared is usually a better estimate of likely
  performance of our model in new data than is the Multiple R-squared.
\item
  The adjusted R-squared result is no longer interpretable as a
  proportion of anything - in fact, it can fall below 0.
\item
  We can obtain the adjusted R\textsuperscript{2} from the raw
  R\textsuperscript{2}, the number of observations \emph{N} and the
  number of predictors \emph{p} included in the model, as follows:
\end{itemize}

\[
R^2_{adj} = 1 - \frac{(1 - R^2)(N - 1)}{N - p - 1},
\]

\begin{itemize}
\tightlist
\item
  The F statistic and \emph{p} value from a global ANOVA test of the
  model.

  \begin{itemize}
  \tightlist
  \item
    Obtaining a statistically significant result here is usually pretty
    straightforward, since the comparison is between our model, and a
    model which simply predicts the mean value of the outcome for
    everyone.
  \item
    In a simple (one-predictor) linear regression like this, the t
    statistic for the slope is just the square root of the F statistic,
    and the resulting \emph{p} values for the slope's t test and for the
    global F test will be identical.
  \end{itemize}
\item
  To see the complete ANOVA F test for this model, we can run
  \texttt{anova(model\_A)}.
\end{itemize}

\begin{Shaded}
\begin{Highlighting}[]
\KeywordTok{anova}\NormalTok{(model_A)}
\end{Highlighting}
\end{Shaded}

\begin{verbatim}
Analysis of Variance Table

Response: physhealth
           Df Sum Sq Mean Sq F value    Pr(>F)    
bmi         1   1366  1365.5  18.655 1.742e-05 ***
Residuals 894  65441    73.2                      
---
Signif. codes:  0 '***' 0.001 '**' 0.01 '*' 0.05 '.' 0.1 ' ' 1
\end{verbatim}

\subsection{\texorpdfstring{Using the \texttt{broom}
package}{Using the broom package}}\label{using-the-broom-package}

The \texttt{broom} package has three functions of particular use in a
linear regression model:

\subsubsection{\texorpdfstring{The \texttt{tidy}
function}{The tidy function}}\label{the-tidy-function}

\texttt{tidy} builds a data frame/tibble containing information about
the coefficients in the model, their standard errors, t statistics and
\emph{p} values.

\begin{Shaded}
\begin{Highlighting}[]
\KeywordTok{tidy}\NormalTok{(model_A)}
\end{Highlighting}
\end{Shaded}

\begin{verbatim}
         term   estimate  std.error statistic      p.value
1 (Intercept) -1.4514298 1.29185199 -1.123526 2.615156e-01
2         bmi  0.1952739 0.04521145  4.319125 1.741859e-05
\end{verbatim}

\subsubsection{\texorpdfstring{The \texttt{glance}
function}{The glance function}}\label{the-glance-function}

glance` builds a data frame/tibble containing summary statistics about
the model, including

\begin{itemize}
\tightlist
\item
  the (raw) multiple R\textsuperscript{2} and adjusted R\^{}2
\item
  \texttt{sigma} which is the residual standard error
\item
  the F \texttt{statistic}, \texttt{p.value} model \texttt{df} and
  \texttt{df.residual} associated with the global ANOVA test, plus
\item
  several statistics that will be useful in comparing models down the
  line:
\item
  the model's log likelihood function value, \texttt{logLik}
\item
  the model's Akaike's Information Criterion value, \texttt{AIC}
\item
  the model's Bayesian Information Criterion value, \texttt{BIC}
\item
  and the model's \texttt{deviance} statistic
\end{itemize}

\begin{Shaded}
\begin{Highlighting}[]
\KeywordTok{glance}\NormalTok{(model_A)}
\end{Highlighting}
\end{Shaded}

\begin{verbatim}
   r.squared adj.r.squared    sigma statistic      p.value df    logLik
1 0.02044019    0.01934449 8.555737  18.65484 1.741859e-05  2 -3193.723
       AIC     BIC deviance df.residual
1 6393.446 6407.84 65441.36         894
\end{verbatim}

\subsubsection{\texorpdfstring{The \texttt{augment}
function}{The augment function}}\label{the-augment-function}

\texttt{augment} builds a data frame/tibble which adds fitted values,
residuals and other diagnostic summaries that describe each observation
to the original data used to fit the model, and this includes

\begin{itemize}
\tightlist
\item
  \texttt{.fitted} and \texttt{.resid}, the fitted and residual values,
  in addition to
\item
  \texttt{.hat}, the leverage value for this observation
\item
  \texttt{.cooksd}, the Cook's distance measure of \emph{influence} for
  this observation
\item
  \texttt{.stdresid}, the standardized residual (think of this as a
  z-score - a measure of the residual divided by its associated standard
  deviation \texttt{.sigma})
\item
  and \texttt{se.fit} which will help us generate prediction intervals
  for the model downstream
\end{itemize}

Note that each of the new columns begins with \texttt{.} to avoid
overwriting any data.

\begin{Shaded}
\begin{Highlighting}[]
\KeywordTok{head}\NormalTok{(}\KeywordTok{augment}\NormalTok{(model_A))}
\end{Highlighting}
\end{Shaded}

\begin{verbatim}
  physhealth   bmi  .fitted   .se.fit      .resid        .hat   .sigma
1          0 26.69 3.760430 0.2907252 -3.76043009 0.001154651 8.559600
2          0 23.70 3.176561 0.3422908 -3.17656119 0.001600574 8.559865
3          1 26.92 3.805343 0.2890054 -2.80534308 0.001141030 8.560010
4          0 21.66 2.778202 0.4005101 -2.77820248 0.002191352 8.560020
5          5 24.09 3.252718 0.3329154  1.74728200 0.001514095 8.560326
6          4 27.64 3.945940 0.2860087  0.05405972 0.001117490 8.560526
       .cooksd   .std.resid
1 1.117852e-04 -0.439775451
2 1.106717e-04 -0.371575999
3 6.147744e-05 -0.328077528
4 1.160381e-04 -0.325074461
5 3.167016e-05  0.204378225
6 2.235722e-08  0.006322069
\end{verbatim}

For more on the \texttt{broom} package, you may want to look at
\href{https://cran.r-project.org/web/packages/broom/vignettes/broom.html}{this
vignette}.

\subsection{How does the model do? (Residuals vs.~Fitted
Values)}\label{how-does-the-model-do-residuals-vs.fitted-values}

\begin{itemize}
\tightlist
\item
  Remember that the R\textsuperscript{2} value was about 2\%.
\end{itemize}

\begin{Shaded}
\begin{Highlighting}[]
\KeywordTok{plot}\NormalTok{(model_A, }\DataTypeTok{which =} \DecValTok{1}\NormalTok{)}
\end{Highlighting}
\end{Shaded}

\includegraphics{bookdown-demo_files/figure-latex/chapter2_first_resid_plot_model_A-1.pdf}

This is a plot of residuals vs.~fitted values. The goal here is for this
plot to look like a random scatter of points, perhaps like a ``fuzzy
football'', and that's \textbf{not} what we have. Why?

If you prefer, here's a \texttt{ggplot2} version of a similar plot, now
looking at standardized residuals instead of raw residuals, and adding a
loess smooth and a linear fit to the result.

\begin{Shaded}
\begin{Highlighting}[]
\KeywordTok{ggplot}\NormalTok{(}\KeywordTok{augment}\NormalTok{(model_A), }\KeywordTok{aes}\NormalTok{(}\DataTypeTok{x =}\NormalTok{ .fitted, }\DataTypeTok{y =}\NormalTok{ .std.resid)) }\OperatorTok{+}
\StringTok{    }\KeywordTok{geom_point}\NormalTok{() }\OperatorTok{+}
\StringTok{    }\KeywordTok{geom_smooth}\NormalTok{(}\DataTypeTok{method =} \StringTok{"lm"}\NormalTok{, }\DataTypeTok{se =} \OtherTok{FALSE}\NormalTok{, }\DataTypeTok{col =} \StringTok{"red"}\NormalTok{, }\DataTypeTok{linetype =} \StringTok{"dashed"}\NormalTok{) }\OperatorTok{+}
\StringTok{    }\KeywordTok{geom_smooth}\NormalTok{(}\DataTypeTok{method =} \StringTok{"loess"}\NormalTok{, }\DataTypeTok{se =} \OtherTok{FALSE}\NormalTok{, }\DataTypeTok{col =} \StringTok{"navy"}\NormalTok{) }\OperatorTok{+}
\StringTok{    }\KeywordTok{theme_bw}\NormalTok{()}
\end{Highlighting}
\end{Shaded}

\includegraphics{bookdown-demo_files/figure-latex/chapter2_ggplot_first_resid_plot_model_A-1.pdf}

The problem we're having here becomes, I think, a little more obvious if
we look at what we're predicting. Does \texttt{physhealth} look like a
good candidate for a linear model?

\begin{Shaded}
\begin{Highlighting}[]
\KeywordTok{ggplot}\NormalTok{(smartcle2, }\KeywordTok{aes}\NormalTok{(}\DataTypeTok{x =}\NormalTok{ physhealth)) }\OperatorTok{+}
\KeywordTok{geom_histogram}\NormalTok{(}\DataTypeTok{bins =} \DecValTok{30}\NormalTok{, }\DataTypeTok{fill =} \StringTok{"dodgerblue"}\NormalTok{, }\DataTypeTok{color =} \StringTok{"royalblue"}\NormalTok{)}
\end{Highlighting}
\end{Shaded}

\includegraphics{bookdown-demo_files/figure-latex/histogram_of_physhealth_smartcle2-1.pdf}

\begin{Shaded}
\begin{Highlighting}[]
\NormalTok{smartcle2 }\OperatorTok\StringTok{ }\KeywordTok{count}\NormalTok{(physhealth }\OperatorTok{==}\StringTok{ }\DecValTok{0}\NormalTok{, physhealth }\OperatorTok{==}\StringTok{ }\DecValTok{30}\NormalTok{)}
\end{Highlighting}
\end{Shaded}

\begin{verbatim}
# A tibble: 3 x 3
  `physhealth == 0` `physhealth == 30`     n
  <lgl>             <lgl>              <int>
1 F                 F                    231
2 F                 T                     74
3 T                 F                    591
\end{verbatim}

No matter what model we fit, if we are predicting \texttt{physhealth},
and most of the data are values of 0 and 30, we have limited variation
in our outcome, and so our linear model will be somewhat questionable
just on that basis.

A normal Q-Q plot of the standardized residuals for our
\texttt{model\_A} shows this problem, too.

\begin{Shaded}
\begin{Highlighting}[]
\KeywordTok{plot}\NormalTok{(model_A, }\DataTypeTok{which =} \DecValTok{2}\NormalTok{)}
\end{Highlighting}
\end{Shaded}

\includegraphics{bookdown-demo_files/figure-latex/chapter2_second_resid_plot_model_A-1.pdf}

We're going to need a method to deal with this sort of outcome, that has
both a floor and a ceiling. We'll get there eventually, but linear
regression alone doesn't look promising.

All right, so that didn't go anywhere great. Let's try again, with a new
outcome.

\section{A New Small Study: Predicting
BMI}\label{a-new-small-study-predicting-bmi}

We'll begin by investigating the problem of predicting \texttt{bmi}, at
first with just three regression inputs: \texttt{sex}, \texttt{exerany}
and \texttt{sleephrs}, in our new \texttt{smartcle2} data set.

\begin{itemize}
\tightlist
\item
  The outcome of interest is \texttt{bmi}.
\item
  Inputs to the regression model are:

  \begin{itemize}
  \tightlist
  \item
    \texttt{female} = 1 if the subject is female, and 0 if they are male
  \item
    \texttt{exerany} = 1 if the subject exercised in the past 30 days,
    and 0 if they didn't
  \item
    \texttt{sleephrs} = hours slept in a typical 24-hour period (treated
    as quantitative)
  \end{itemize}
\end{itemize}

\subsection{\texorpdfstring{Does \texttt{female} predict \texttt{bmi}
well?}{Does female predict bmi well?}}\label{does-female-predict-bmi-well}

\subsubsection{Graphical Assessment}\label{graphical-assessment}

\begin{Shaded}
\begin{Highlighting}[]
\KeywordTok{ggplot}\NormalTok{(smartcle2, }\KeywordTok{aes}\NormalTok{(}\DataTypeTok{x =}\NormalTok{ female, }\DataTypeTok{y =}\NormalTok{ bmi)) }\OperatorTok{+}
\StringTok{    }\KeywordTok{geom_point}\NormalTok{()}
\end{Highlighting}
\end{Shaded}

\includegraphics{bookdown-demo_files/figure-latex/c2_sex_bmi_plot1-1.pdf}

Not so helpful. We should probably specify that \texttt{female} is a
factor, and try another plotting approach.

\begin{Shaded}
\begin{Highlighting}[]
\KeywordTok{ggplot}\NormalTok{(smartcle2, }\KeywordTok{aes}\NormalTok{(}\DataTypeTok{x =} \KeywordTok{factor}\NormalTok{(female), }\DataTypeTok{y =}\NormalTok{ bmi)) }\OperatorTok{+}
\StringTok{    }\KeywordTok{geom_boxplot}\NormalTok{()}
\end{Highlighting}
\end{Shaded}

\includegraphics{bookdown-demo_files/figure-latex/c2_sex_bmi_plot2-1.pdf}

The median BMI looks a little higher for males. Let's see if a model
reflects that.

\section{\texorpdfstring{\texttt{c2\_m1}: A simple t-test
model}{c2\_m1: A simple t-test model}}\label{c2_m1-a-simple-t-test-model}

\begin{Shaded}
\begin{Highlighting}[]
\NormalTok{c2_m1 <-}\StringTok{ }\KeywordTok{lm}\NormalTok{(bmi }\OperatorTok{~}\StringTok{ }\NormalTok{female, }\DataTypeTok{data =}\NormalTok{ smartcle2)}
\NormalTok{c2_m1}
\end{Highlighting}
\end{Shaded}

\begin{verbatim}

Call:
lm(formula = bmi ~ female, data = smartcle2)

Coefficients:
(Intercept)       female  
    28.3600      -0.8457  
\end{verbatim}

\begin{Shaded}
\begin{Highlighting}[]
\KeywordTok{summary}\NormalTok{(c2_m1)}
\end{Highlighting}
\end{Shaded}

\begin{verbatim}

Call:
lm(formula = bmi ~ female, data = smartcle2)

Residuals:
    Min      1Q  Median      3Q     Max 
-15.650  -4.129  -1.080   2.727  38.546 

Coefficients:
            Estimate Std. Error t value Pr(>|t|)    
(Intercept)  28.3600     0.3274  86.613   <2e-16 ***
female       -0.8457     0.4282  -1.975   0.0485 *  
---
Signif. codes:  0 '***' 0.001 '**' 0.01 '*' 0.05 '.' 0.1 ' ' 1

Residual standard error: 6.315 on 894 degrees of freedom
Multiple R-squared:  0.004345,  Adjusted R-squared:  0.003231 
F-statistic: 3.902 on 1 and 894 DF,  p-value: 0.04855
\end{verbatim}

\begin{Shaded}
\begin{Highlighting}[]
\KeywordTok{confint}\NormalTok{(c2_m1)}
\end{Highlighting}
\end{Shaded}

\begin{verbatim}
                2.5 %      97.5 %
(Intercept) 27.717372 29.00262801
female      -1.686052 -0.00539878
\end{verbatim}

The model suggests, based on these 896 subjects, that

\begin{itemize}
\tightlist
\item
  our best prediction for males is BMI = 28.36 kg/m\textsuperscript{2},
  and
\item
  our best prediction for females is BMI = 28.36 - 0.85 = 27.51
  kg/m\textsuperscript{2}.
\item
  the mean difference between females and males is -0.85
  kg/m\textsuperscript{2} in BMI
\item
  a 95\% confidence (uncertainty) interval for that mean female - male
  difference in BMI ranges from -1.69 to -0.01
\item
  the model accounts for 0.4\% of the variation in BMI, so that knowing
  the respondent's sex does very little to reduce the size of the
  prediction errors as compared to an intercept only model that would
  predict the overall mean (regardless of sex) for all subjects.
\item
  the model makes some enormous errors, with one subject being predicted
  to have a BMI 38 points lower than his/her actual BMI.
\end{itemize}

Note that this simple regression model just gives us the t-test.

\begin{Shaded}
\begin{Highlighting}[]
\KeywordTok{t.test}\NormalTok{(bmi }\OperatorTok{~}\StringTok{ }\NormalTok{female, }\DataTypeTok{var.equal =} \OtherTok{TRUE}\NormalTok{, }\DataTypeTok{data =}\NormalTok{ smartcle2)}
\end{Highlighting}
\end{Shaded}

\begin{verbatim}

    Two Sample t-test

data:  bmi by female
t = 1.9752, df = 894, p-value = 0.04855
alternative hypothesis: true difference in means is not equal to 0
95 percent confidence interval:
 0.00539878 1.68605160
sample estimates:
mean in group 0 mean in group 1 
       28.36000        27.51427 
\end{verbatim}

\section{\texorpdfstring{\texttt{c2\_m2}: Adding another predictor
(two-way ANOVA without
interaction)}{c2\_m2: Adding another predictor (two-way ANOVA without interaction)}}\label{c2_m2-adding-another-predictor-two-way-anova-without-interaction}

When we add in the information about \texttt{exerany} to our original
model, we might first picture the data. We could look at separate
histograms,

\begin{Shaded}
\begin{Highlighting}[]
\KeywordTok{ggplot}\NormalTok{(smartcle2, }\KeywordTok{aes}\NormalTok{(}\DataTypeTok{x =}\NormalTok{ bmi)) }\OperatorTok{+}
\StringTok{    }\KeywordTok{geom_histogram}\NormalTok{(}\DataTypeTok{bins =} \DecValTok{30}\NormalTok{) }\OperatorTok{+}
\StringTok{    }\KeywordTok{facet_grid}\NormalTok{(female }\OperatorTok{~}\StringTok{ }\NormalTok{exerany, }\DataTypeTok{labeller =}\NormalTok{ label_both)}
\end{Highlighting}
\end{Shaded}

\includegraphics{bookdown-demo_files/figure-latex/c2_smartcle2_plot_bmi_hist_by_female_exerany-1.pdf}

or maybe boxplots?

\begin{Shaded}
\begin{Highlighting}[]
\KeywordTok{ggplot}\NormalTok{(smartcle2, }\KeywordTok{aes}\NormalTok{(}\DataTypeTok{x =} \KeywordTok{factor}\NormalTok{(female), }\DataTypeTok{y =}\NormalTok{ bmi)) }\OperatorTok{+}
\StringTok{    }\KeywordTok{geom_boxplot}\NormalTok{() }\OperatorTok{+}
\StringTok{    }\KeywordTok{facet_wrap}\NormalTok{(}\OperatorTok{~}\StringTok{ }\NormalTok{exerany, }\DataTypeTok{labeller =}\NormalTok{ label_both)}
\end{Highlighting}
\end{Shaded}

\includegraphics{bookdown-demo_files/figure-latex/c2_smartcle2_plot_bmi_box_by_female_exerany-1.pdf}

\begin{Shaded}
\begin{Highlighting}[]
\KeywordTok{ggplot}\NormalTok{(smartcle2, }\KeywordTok{aes}\NormalTok{(}\DataTypeTok{x =}\NormalTok{ female, }\DataTypeTok{y =}\NormalTok{ bmi))}\OperatorTok{+}
\StringTok{    }\KeywordTok{geom_point}\NormalTok{(}\DataTypeTok{size =} \DecValTok{3}\NormalTok{, }\DataTypeTok{alpha =} \FloatTok{0.2}\NormalTok{) }\OperatorTok{+}
\StringTok{    }\KeywordTok{theme_bw}\NormalTok{() }\OperatorTok{+}
\StringTok{    }\KeywordTok{facet_wrap}\NormalTok{(}\OperatorTok{~}\StringTok{ }\NormalTok{exerany, }\DataTypeTok{labeller =}\NormalTok{ label_both)}
\end{Highlighting}
\end{Shaded}

\includegraphics{bookdown-demo_files/figure-latex/c2_smartcle2_plot_bmi_points_by_female_exerany-1.pdf}

OK. Let's try fitting a model.

\begin{Shaded}
\begin{Highlighting}[]
\NormalTok{c2_m2 <-}\StringTok{ }\KeywordTok{lm}\NormalTok{(bmi }\OperatorTok{~}\StringTok{ }\NormalTok{female }\OperatorTok{+}\StringTok{ }\NormalTok{exerany, }\DataTypeTok{data =}\NormalTok{ smartcle2)}
\NormalTok{c2_m2}
\end{Highlighting}
\end{Shaded}

\begin{verbatim}

Call:
lm(formula = bmi ~ female + exerany, data = smartcle2)

Coefficients:
(Intercept)       female      exerany  
     30.334       -1.095       -2.384  
\end{verbatim}

This new model predicts only four predicted values:

\begin{itemize}
\tightlist
\item
  \texttt{bmi} = 30.334 if the subject is male and did not exercise (so
  \texttt{female} = 0 and \texttt{exerany} = 0)
\item
  \texttt{bmi} = 30.334 - 1.095 = 29.239 if the subject is female and
  did not exercise (\texttt{female} = 1 and \texttt{exerany} = 0)
\item
  \texttt{bmi} = 30.334 - 2.384 = 27.950 if the subject is male and
  exercised (so \texttt{female} = 0 and \texttt{exerany} = 1), and,
  finally
\item
  \texttt{bmi} = 30.334 - 1.095 - 2.384 = 26.855 if the subject is
  female and exercised (so both \texttt{female} and \texttt{exerany} =
  1).
\end{itemize}

For those who did not exercise, the model is:

\begin{itemize}
\tightlist
\item
  \texttt{bmi} = 30.334 - 1.095 \texttt{female}
\end{itemize}

and for those who did exercise, the model is:

\begin{itemize}
\tightlist
\item
  \texttt{bmi} = 27.95 - 1.095 \texttt{female}
\end{itemize}

Only the intercept of the \texttt{bmi-female} model changes depending on
\texttt{exerany}.

\begin{Shaded}
\begin{Highlighting}[]
\KeywordTok{summary}\NormalTok{(c2_m2)}
\end{Highlighting}
\end{Shaded}

\begin{verbatim}

Call:
lm(formula = bmi ~ female + exerany, data = smartcle2)

Residuals:
    Min      1Q  Median      3Q     Max 
-15.240  -4.091  -1.095   2.602  36.822 

Coefficients:
            Estimate Std. Error t value Pr(>|t|)    
(Intercept)  30.3335     0.5231   57.99  < 2e-16 ***
female       -1.0952     0.4262   -2.57   0.0103 *  
exerany      -2.3836     0.4965   -4.80 1.86e-06 ***
---
Signif. codes:  0 '***' 0.001 '**' 0.01 '*' 0.05 '.' 0.1 ' ' 1

Residual standard error: 6.239 on 893 degrees of freedom
Multiple R-squared:  0.02939,   Adjusted R-squared:  0.02722 
F-statistic: 13.52 on 2 and 893 DF,  p-value: 1.641e-06
\end{verbatim}

\begin{Shaded}
\begin{Highlighting}[]
\KeywordTok{confint}\NormalTok{(c2_m2)}
\end{Highlighting}
\end{Shaded}

\begin{verbatim}
                2.5 %     97.5 %
(Intercept) 29.306846 31.3602182
female      -1.931629 -0.2588299
exerany     -3.358156 -1.4090777
\end{verbatim}

The slopes of both \texttt{female} and \texttt{exerany} have confidence
intervals that are completely below zero, indicating that both
\texttt{female} sex and \texttt{exerany} appear to be associated with
reductions in \texttt{bmi}.

The R\textsuperscript{2} value suggests that just under 3\% of the
variation in \texttt{bmi} is accounted for by this ANOVA model.

In fact, this regression (on two binary indicator variables) is simply a
two-way ANOVA model without an interaction term.

\begin{Shaded}
\begin{Highlighting}[]
\KeywordTok{anova}\NormalTok{(c2_m2)}
\end{Highlighting}
\end{Shaded}

\begin{verbatim}
Analysis of Variance Table

Response: bmi
           Df Sum Sq Mean Sq F value    Pr(>F)    
female      1    156  155.61  3.9977   0.04586 *  
exerany     1    897  896.93 23.0435 1.856e-06 ***
Residuals 893  34759   38.92                      
---
Signif. codes:  0 '***' 0.001 '**' 0.01 '*' 0.05 '.' 0.1 ' ' 1
\end{verbatim}

\section{\texorpdfstring{\texttt{c2\_m3}: Adding the interaction term
(Two-way ANOVA with
interaction)}{c2\_m3: Adding the interaction term (Two-way ANOVA with interaction)}}\label{c2_m3-adding-the-interaction-term-two-way-anova-with-interaction}

Suppose we want to let the effect of \texttt{female} vary depending on
the \texttt{exerany} status. Then we need to incorporate an interaction
term in our model.

\begin{Shaded}
\begin{Highlighting}[]
\NormalTok{c2_m3 <-}\StringTok{ }\KeywordTok{lm}\NormalTok{(bmi }\OperatorTok{~}\StringTok{ }\NormalTok{female }\OperatorTok{*}\StringTok{ }\NormalTok{exerany, }\DataTypeTok{data =}\NormalTok{ smartcle2)}
\NormalTok{c2_m3}
\end{Highlighting}
\end{Shaded}

\begin{verbatim}

Call:
lm(formula = bmi ~ female * exerany, data = smartcle2)

Coefficients:
   (Intercept)          female         exerany  female:exerany  
       30.1359         -0.8104         -2.1450         -0.3592  
\end{verbatim}

So, for example, for a male who exercises, this model predicts

\begin{itemize}
\tightlist
\item
  \texttt{bmi} = 30.136 - 0.810 (0) - 2.145 (1) - 0.359 (0)(1) = 30.136
  - 2.145 = 27.991
\end{itemize}

And for a female who exercises, the model predicts

\begin{itemize}
\tightlist
\item
  \texttt{bmi} = 30.136 - 0.810 (1) - 2.145 (1) - 0.359 (1)(1) = 30.136
  - 0.810 - 2.145 - 0.359 = 26.822
\end{itemize}

For those who did not exercise, the model is:

\begin{itemize}
\tightlist
\item
  \texttt{bmi} = 30.136 - 0.81 \texttt{female}
\end{itemize}

But for those who did exercise, the model is:

\begin{itemize}
\tightlist
\item
  \texttt{bmi} = (30.136 - 2.145) + (-0.810 + (-0.359)) \texttt{female},
  or ,,,
\item
  \texttt{bmi} = 27.991 - 1.169 \texttt{female}
\end{itemize}

Now, both the slope and the intercept of the \texttt{bmi-female} model
change depending on \texttt{exerany}.

\begin{Shaded}
\begin{Highlighting}[]
\KeywordTok{summary}\NormalTok{(c2_m3)}
\end{Highlighting}
\end{Shaded}

\begin{verbatim}

Call:
lm(formula = bmi ~ female * exerany, data = smartcle2)

Residuals:
    Min      1Q  Median      3Q     Max 
-15.281  -4.101  -1.061   2.566  36.734 

Coefficients:
               Estimate Std. Error t value Pr(>|t|)    
(Intercept)     30.1359     0.7802  38.624   <2e-16 ***
female          -0.8104     0.9367  -0.865   0.3872    
exerany         -2.1450     0.8575  -2.501   0.0125 *  
female:exerany  -0.3592     1.0520  -0.341   0.7328    
---
Signif. codes:  0 '***' 0.001 '**' 0.01 '*' 0.05 '.' 0.1 ' ' 1

Residual standard error: 6.242 on 892 degrees of freedom
Multiple R-squared:  0.02952,   Adjusted R-squared:  0.02625 
F-statistic: 9.044 on 3 and 892 DF,  p-value: 6.669e-06
\end{verbatim}

\begin{Shaded}
\begin{Highlighting}[]
\KeywordTok{confint}\NormalTok{(c2_m3)}
\end{Highlighting}
\end{Shaded}

\begin{verbatim}
                   2.5 %     97.5 %
(Intercept)    28.604610 31.6672650
female         -2.648893  1.0280526
exerany        -3.827886 -0.4620407
female:exerany -2.423994  1.7055248
\end{verbatim}

In fact, this regression (on two binary indicator variables and a
product term) is simply a two-way ANOVA model with an interaction term.

\begin{Shaded}
\begin{Highlighting}[]
\KeywordTok{anova}\NormalTok{(c2_m3)}
\end{Highlighting}
\end{Shaded}

\begin{verbatim}
Analysis of Variance Table

Response: bmi
                Df Sum Sq Mean Sq F value    Pr(>F)    
female           1    156  155.61  3.9938   0.04597 *  
exerany          1    897  896.93 23.0207 1.878e-06 ***
female:exerany   1      5    4.54  0.1166   0.73283    
Residuals      892  34754   38.96                      
---
Signif. codes:  0 '***' 0.001 '**' 0.01 '*' 0.05 '.' 0.1 ' ' 1
\end{verbatim}

The interaction term doesn't change very much here. Its uncertainty
interval includes zero, and the overall model still accounts for just
under 3\% of the variation in \texttt{bmi}.

\section{\texorpdfstring{\texttt{c2\_m4}: Using \texttt{female} and
\texttt{sleephrs} in a model for
\texttt{bmi}}{c2\_m4: Using female and sleephrs in a model for bmi}}\label{c2_m4-using-female-and-sleephrs-in-a-model-for-bmi}

\begin{Shaded}
\begin{Highlighting}[]
\KeywordTok{ggplot}\NormalTok{(smartcle2, }\KeywordTok{aes}\NormalTok{(}\DataTypeTok{x =}\NormalTok{ sleephrs, }\DataTypeTok{y =}\NormalTok{ bmi, }\DataTypeTok{color =} \KeywordTok{factor}\NormalTok{(female))) }\OperatorTok{+}
\StringTok{    }\KeywordTok{geom_point}\NormalTok{() }\OperatorTok{+}\StringTok{ }
\StringTok{    }\KeywordTok{guides}\NormalTok{(}\DataTypeTok{col =} \OtherTok{FALSE}\NormalTok{) }\OperatorTok{+}
\StringTok{    }\KeywordTok{geom_smooth}\NormalTok{(}\DataTypeTok{method =} \StringTok{"lm"}\NormalTok{, }\DataTypeTok{se =} \OtherTok{FALSE}\NormalTok{) }\OperatorTok{+}
\StringTok{    }\KeywordTok{facet_wrap}\NormalTok{(}\OperatorTok{~}\StringTok{ }\NormalTok{female, }\DataTypeTok{labeller =}\NormalTok{ label_both) }
\end{Highlighting}
\end{Shaded}

\includegraphics{bookdown-demo_files/figure-latex/graph_to_set_up_c2_m4-1.pdf}

Does the difference in slopes of \texttt{bmi} and \texttt{sleephrs} for
males and females appear to be substantial and important?

\begin{Shaded}
\begin{Highlighting}[]
\NormalTok{c2_m4 <-}\StringTok{ }\KeywordTok{lm}\NormalTok{(bmi }\OperatorTok{~}\StringTok{ }\NormalTok{female }\OperatorTok{*}\StringTok{ }\NormalTok{sleephrs, }\DataTypeTok{data =}\NormalTok{ smartcle2)}

\KeywordTok{summary}\NormalTok{(c2_m4)}
\end{Highlighting}
\end{Shaded}

\begin{verbatim}

Call:
lm(formula = bmi ~ female * sleephrs, data = smartcle2)

Residuals:
    Min      1Q  Median      3Q     Max 
-15.498  -4.179  -1.035   2.830  38.204 

Coefficients:
                Estimate Std. Error t value Pr(>|t|)    
(Intercept)      27.2661     1.6320  16.707   <2e-16 ***
female            2.5263     2.0975   1.204    0.229    
sleephrs          0.1569     0.2294   0.684    0.494    
female:sleephrs  -0.4797     0.2931  -1.636    0.102    
---
Signif. codes:  0 '***' 0.001 '**' 0.01 '*' 0.05 '.' 0.1 ' ' 1

Residual standard error: 6.31 on 892 degrees of freedom
Multiple R-squared:  0.008341,  Adjusted R-squared:  0.005006 
F-statistic: 2.501 on 3 and 892 DF,  p-value: 0.05818
\end{verbatim}

Does it seem as though the addition of \texttt{sleephrs} has improved
our model substantially over a model with \texttt{female} alone (which,
you recall, was \texttt{c2\_m1})?

Since the \texttt{c2\_m4} model contains the \texttt{c2\_m1} model's
predictors as a subset and the outcome is the same for each model, we
consider the models \emph{nested} and have some extra tools available to
compare them.

\begin{itemize}
\tightlist
\item
  I might start by looking at the basic summaries for each model.
\end{itemize}

\begin{Shaded}
\begin{Highlighting}[]
\KeywordTok{glance}\NormalTok{(c2_m4)}
\end{Highlighting}
\end{Shaded}

\begin{verbatim}
    r.squared adj.r.squared    sigma statistic    p.value df    logLik
1 0.008341404   0.005006229 6.309685   2.50104 0.05818038  4 -2919.873
       AIC      BIC deviance df.residual
1 5849.747 5873.736 35512.42         892
\end{verbatim}

\begin{Shaded}
\begin{Highlighting}[]
\KeywordTok{glance}\NormalTok{(c2_m1)}
\end{Highlighting}
\end{Shaded}

\begin{verbatim}
    r.squared adj.r.squared   sigma statistic    p.value df    logLik
1 0.004345169   0.003231461 6.31531  3.901534 0.04854928  2 -2921.675
      AIC      BIC deviance df.residual
1 5849.35 5863.744 35655.53         894
\end{verbatim}

\begin{itemize}
\tightlist
\item
  The R\textsuperscript{2} is twice as large for the model with
  \texttt{sleephrs}, but still very tiny.
\item
  The \emph{p} value for the global ANOVA test is actually less
  significant in \texttt{c2\_m4} than in \texttt{c2\_m1}.
\item
  Smaller AIC and smaller BIC statistics are more desirable. Here,
  there's little to choose from, but \texttt{c2\_m1} is a little better
  on each standard.
\item
  We might also consider a significance test by looking at an ANOVA
  model comparison. This is only appropriate because \texttt{c2\_m1} is
  nested in \texttt{c2\_m4}.
\end{itemize}

\begin{Shaded}
\begin{Highlighting}[]
\KeywordTok{anova}\NormalTok{(c2_m4, c2_m1)}
\end{Highlighting}
\end{Shaded}

\begin{verbatim}
Analysis of Variance Table

Model 1: bmi ~ female * sleephrs
Model 2: bmi ~ female
  Res.Df   RSS Df Sum of Sq      F Pr(>F)
1    892 35512                           
2    894 35656 -2   -143.11 1.7973 0.1663
\end{verbatim}

The addition of the \texttt{sleephrs} term picked up 143 in the sum of
squares column, at a cost of two degrees of freedom, yielding a \emph{p}
value of 0.166, suggesting that this isn't a significant improvement
over the model that just did a t-test on \texttt{female}.

\section{Making Predictions with a Linear Regression
Model}\label{making-predictions-with-a-linear-regression-model}

Recall model 4, which yields predictions for body mass index on the
basis of the main effects of sex (\texttt{female}) and hours of sleep
(\texttt{sleephrs}) and their interaction.

\begin{Shaded}
\begin{Highlighting}[]
\NormalTok{c2_m4}
\end{Highlighting}
\end{Shaded}

\begin{verbatim}

Call:
lm(formula = bmi ~ female * sleephrs, data = smartcle2)

Coefficients:
    (Intercept)           female         sleephrs  female:sleephrs  
        27.2661           2.5263           0.1569          -0.4797  
\end{verbatim}

\subsection{Fitting an Individual Prediction and 95\% Prediction
Interval}\label{fitting-an-individual-prediction-and-95-prediction-interval}

What do we predict for the \texttt{bmi} of a subject who is
\texttt{female} and gets 8 hours of sleep per night?

\begin{Shaded}
\begin{Highlighting}[]
\NormalTok{c2_new1 <-}\StringTok{ }\KeywordTok{data_frame}\NormalTok{(}\DataTypeTok{female =} \DecValTok{1}\NormalTok{, }\DataTypeTok{sleephrs =} \DecValTok{8}\NormalTok{)}
\KeywordTok{predict}\NormalTok{(c2_m4, }\DataTypeTok{newdata =}\NormalTok{ c2_new1, }\DataTypeTok{interval =} \StringTok{"prediction"}\NormalTok{, }\DataTypeTok{level =} \FloatTok{0.95}\NormalTok{)}
\end{Highlighting}
\end{Shaded}

\begin{verbatim}
       fit     lwr     upr
1 27.21065 14.8107 39.6106
\end{verbatim}

The predicted \texttt{bmi} for this new subject is 27.61. The prediction
interval shows the bounds of a 95\% uncertainty interval for a predicted
\texttt{bmi} for an individual female subject who gets 8 hours of sleep
on average per evening. From the \texttt{predict} function applied to a
linear model, we can get the prediction intervals for any new data
points in this manner.

\subsection{Confidence Interval for an Average
Prediction}\label{confidence-interval-for-an-average-prediction}

\begin{itemize}
\tightlist
\item
  What do we predict for the \textbf{average body mass index of a
  population of subjects} who are female and sleep for 8 hours?
\end{itemize}

\begin{Shaded}
\begin{Highlighting}[]
\KeywordTok{predict}\NormalTok{(c2_m4, }\DataTypeTok{newdata =}\NormalTok{ c2_new1, }\DataTypeTok{interval =} \StringTok{"confidence"}\NormalTok{, }\DataTypeTok{level =} \FloatTok{0.95}\NormalTok{)}
\end{Highlighting}
\end{Shaded}

\begin{verbatim}
       fit      lwr      upr
1 27.21065 26.57328 27.84801
\end{verbatim}

\begin{itemize}
\tightlist
\item
  How does this result compare to the prediction interval?
\end{itemize}

\subsection{Fitting Multiple Individual Predictions to New
Data}\label{fitting-multiple-individual-predictions-to-new-data}

\begin{itemize}
\tightlist
\item
  How does our prediction change for a respondent if they instead get 7,
  or 9 hours of sleep? What if they are male, instead of female?
\end{itemize}

\begin{Shaded}
\begin{Highlighting}[]
\NormalTok{c2_new2 <-}\StringTok{ }\KeywordTok{data_frame}\NormalTok{(}\DataTypeTok{subjectid =} \DecValTok{1001}\OperatorTok{:}\DecValTok{1006}\NormalTok{, }\DataTypeTok{female =} \KeywordTok{c}\NormalTok{(}\DecValTok{1}\NormalTok{, }\DecValTok{1}\NormalTok{, }\DecValTok{1}\NormalTok{, }\DecValTok{0}\NormalTok{, }\DecValTok{0}\NormalTok{, }\DecValTok{0}\NormalTok{), }\DataTypeTok{sleephrs =} \KeywordTok{c}\NormalTok{(}\DecValTok{7}\NormalTok{, }\DecValTok{8}\NormalTok{, }\DecValTok{9}\NormalTok{, }\DecValTok{7}\NormalTok{, }\DecValTok{8}\NormalTok{, }\DecValTok{9}\NormalTok{))}
\NormalTok{pred2 <-}\StringTok{ }\KeywordTok{predict}\NormalTok{(c2_m4, }\DataTypeTok{newdata =}\NormalTok{ c2_new2, }\DataTypeTok{interval =} \StringTok{"prediction"}\NormalTok{, }\DataTypeTok{level =} \FloatTok{0.95}\NormalTok{) }\OperatorTok\StringTok{ }\NormalTok{tbl_df}

\NormalTok{result2 <-}\StringTok{ }\KeywordTok{bind_cols}\NormalTok{(c2_new2, pred2)}
\NormalTok{result2}
\end{Highlighting}
\end{Shaded}

\begin{verbatim}
# A tibble: 6 x 6
  subjectid female sleephrs   fit   lwr   upr
      <int>  <dbl>    <dbl> <dbl> <dbl> <dbl>
1      1001   1.00     7.00  27.5  15.1  39.9
2      1002   1.00     8.00  27.2  14.8  39.6
3      1003   1.00     9.00  26.9  14.5  39.3
4      1004   0        7.00  28.4  16.0  40.8
5      1005   0        8.00  28.5  16.1  40.9
6      1006   0        9.00  28.7  16.2  41.1
\end{verbatim}

The \texttt{result2} tibble contains predictions for each scenario.

\begin{itemize}
\tightlist
\item
  Which has a bigger impact on these predictions and prediction
  intervals? A one category change in \texttt{female} or a one hour
  change in \texttt{sleephrs}?
\end{itemize}

\subsection{Simulation to represent predictive uncertainty in Model
4}\label{simulation-to-represent-predictive-uncertainty-in-model-4}

Suppose we want to predict the \texttt{bmi} of a female subject who
sleeps for eight hours per night. As we have seen, we can do this
automatically for a linear model like this one, using the
\texttt{predict} function applied to the linear model, but a simulation
prediction can also be done. Recall the detail of \texttt{c2\_m4}:

\begin{Shaded}
\begin{Highlighting}[]
\NormalTok{c2_m4}
\end{Highlighting}
\end{Shaded}

\begin{verbatim}

Call:
lm(formula = bmi ~ female * sleephrs, data = smartcle2)

Coefficients:
    (Intercept)           female         sleephrs  female:sleephrs  
        27.2661           2.5263           0.1569          -0.4797  
\end{verbatim}

\begin{Shaded}
\begin{Highlighting}[]
\KeywordTok{glance}\NormalTok{(c2_m4)}
\end{Highlighting}
\end{Shaded}

\begin{verbatim}
    r.squared adj.r.squared    sigma statistic    p.value df    logLik
1 0.008341404   0.005006229 6.309685   2.50104 0.05818038  4 -2919.873
       AIC      BIC deviance df.residual
1 5849.747 5873.736 35512.42         892
\end{verbatim}

We see that the residual standard error for our \texttt{bmi} predictions
with this model is 6.31.

For a female respondent sleeping eight hours, recall that our point
estimate (predicted value) of \texttt{bmi} is 27.21

\begin{Shaded}
\begin{Highlighting}[]
\KeywordTok{predict}\NormalTok{(c2_m4, }\DataTypeTok{newdata =}\NormalTok{ c2_new1, }\DataTypeTok{interval =} \StringTok{"prediction"}\NormalTok{, }\DataTypeTok{level =} \FloatTok{0.95}\NormalTok{)}
\end{Highlighting}
\end{Shaded}

\begin{verbatim}
       fit     lwr     upr
1 27.21065 14.8107 39.6106
\end{verbatim}

The standard deviation is 6.31, so we could summarize the predictive
distribution with a command that tells R to draw 1000 random numbers
from a normal distribution with mean 27.21 and standard deviation 6.31.
Let's summarize that and get a quick picture.

\begin{Shaded}
\begin{Highlighting}[]
\KeywordTok{set.seed}\NormalTok{(}\DecValTok{432094}\NormalTok{)}
\NormalTok{pred.sim <-}\StringTok{ }\KeywordTok{rnorm}\NormalTok{(}\DecValTok{1000}\NormalTok{, }\FloatTok{27.21}\NormalTok{, }\FloatTok{6.31}\NormalTok{)}
\KeywordTok{hist}\NormalTok{(pred.sim, }\DataTypeTok{col =} \StringTok{"royalblue"}\NormalTok{)}
\end{Highlighting}
\end{Shaded}

\includegraphics{bookdown-demo_files/figure-latex/unnamed-chunk-10-1.pdf}

\begin{Shaded}
\begin{Highlighting}[]
\KeywordTok{mean}\NormalTok{(pred.sim)}
\end{Highlighting}
\end{Shaded}

\begin{verbatim}
[1] 27.41856
\end{verbatim}

\begin{Shaded}
\begin{Highlighting}[]
\KeywordTok{quantile}\NormalTok{(pred.sim, }\KeywordTok{c}\NormalTok{(}\FloatTok{0.025}\NormalTok{, }\FloatTok{0.975}\NormalTok{))}
\end{Highlighting}
\end{Shaded}

\begin{verbatim}
    2.5%    97.5% 
14.48487 40.16778 
\end{verbatim}

How do these results compare to the prediction interval of (14.81,
39.61) that we generated earlier?

\section{Centering the model}\label{centering-the-model}

Our model \texttt{c2\_m4} has four predictors (the constant,
\texttt{sleephrs}, \texttt{female} and their interaction) but just two
inputs (\texttt{female} and \texttt{sleephrs}.) If we \textbf{center}
the quantitative input \texttt{sleephrs} before building the model, we
get a more interpretable interaction term.

\begin{Shaded}
\begin{Highlighting}[]
\NormalTok{smartcle2_c <-}\StringTok{ }\NormalTok{smartcle2 }\OperatorTok
\StringTok{    }\KeywordTok{mutate}\NormalTok{(}\DataTypeTok{sleephrs_c =}\NormalTok{ sleephrs }\OperatorTok{-}\StringTok{ }\KeywordTok{mean}\NormalTok{(sleephrs))}

\NormalTok{c2_m4_c <-}\StringTok{ }\KeywordTok{lm}\NormalTok{(bmi }\OperatorTok{~}\StringTok{ }\NormalTok{female }\OperatorTok{*}\StringTok{ }\NormalTok{sleephrs_c, }\DataTypeTok{data =}\NormalTok{ smartcle2_c)}

\KeywordTok{summary}\NormalTok{(c2_m4_c)}
\end{Highlighting}
\end{Shaded}

\begin{verbatim}

Call:
lm(formula = bmi ~ female * sleephrs_c, data = smartcle2_c)

Residuals:
    Min      1Q  Median      3Q     Max 
-15.498  -4.179  -1.035   2.830  38.204 

Coefficients:
                  Estimate Std. Error t value Pr(>|t|)    
(Intercept)        28.3681     0.3274  86.658   <2e-16 ***
female             -0.8420     0.4280  -1.967   0.0495 *  
sleephrs_c          0.1569     0.2294   0.684   0.4940    
female:sleephrs_c  -0.4797     0.2931  -1.636   0.1021    
---
Signif. codes:  0 '***' 0.001 '**' 0.01 '*' 0.05 '.' 0.1 ' ' 1

Residual standard error: 6.31 on 892 degrees of freedom
Multiple R-squared:  0.008341,  Adjusted R-squared:  0.005006 
F-statistic: 2.501 on 3 and 892 DF,  p-value: 0.05818
\end{verbatim}

What has changed as compared to the original \texttt{c2\_m4}?

\begin{itemize}
\tightlist
\item
  Our original model was \texttt{bmi} = 27.26 + 2.53 \texttt{female} +
  0.16 \texttt{sleephrs} - 0.48 \texttt{female} x \texttt{sleephrs}
\item
  Our new model is \texttt{bmi} = 28.37 - 0.84 \texttt{female} + 0.16
  centered \texttt{sleephrs} - 0.48 \texttt{female} x centered
  \texttt{sleephrs}.
\end{itemize}

So our new model on centered data is:

\begin{itemize}
\tightlist
\item
  28.37 + 0.16 centered \texttt{sleephrs\_c} for male subjects, and
\item
  (28.37 - 0.84) + (0.16 - 0.48) centered \texttt{sleephrs\_c}, or 27.53
  - 0.32 centered \texttt{sleephrs\_c} for female subjects.
\end{itemize}

In our new (centered \texttt{sleephrs\_c}) model,

\begin{itemize}
\tightlist
\item
  the main effect of \texttt{female} now corresponds to a predictive
  difference (female - male) in \texttt{bmi} with \texttt{sleephrs} at
  its mean value, 7.02 hours,
\item
  the intercept term is now the predicted \texttt{bmi} for a male
  respondent who sleeps an average number of hours, and
\item
  the product term corresponds to the change in the slope of centered
  \texttt{sleephrs\_c} on \texttt{bmi} for a female rather than a male
  subject, while
\item
  the residual standard deviation and the R-squared values remain
  unchanged from the model before centering.
\end{itemize}

\subsection{\texorpdfstring{Plot of Model 4 on Centered
\texttt{sleephrs}:
\texttt{c2\_m4\_c}}{Plot of Model 4 on Centered sleephrs: c2\_m4\_c}}\label{plot-of-model-4-on-centered-sleephrs-c2_m4_c}

\begin{Shaded}
\begin{Highlighting}[]
\KeywordTok{ggplot}\NormalTok{(smartcle2_c, }\KeywordTok{aes}\NormalTok{(}\DataTypeTok{x =}\NormalTok{ sleephrs_c, }\DataTypeTok{y =}\NormalTok{ bmi, }\DataTypeTok{group =}\NormalTok{ female, }\DataTypeTok{col =} \KeywordTok{factor}\NormalTok{(female))) }\OperatorTok{+}
\StringTok{    }\KeywordTok{geom_point}\NormalTok{(}\DataTypeTok{alpha =} \FloatTok{0.5}\NormalTok{, }\DataTypeTok{size =} \DecValTok{2}\NormalTok{) }\OperatorTok{+}
\StringTok{    }\KeywordTok{geom_smooth}\NormalTok{(}\DataTypeTok{method =} \StringTok{"lm"}\NormalTok{, }\DataTypeTok{se =} \OtherTok{FALSE}\NormalTok{) }\OperatorTok{+}
\StringTok{    }\KeywordTok{guides}\NormalTok{(}\DataTypeTok{color =} \OtherTok{FALSE}\NormalTok{) }\OperatorTok{+}
\StringTok{    }\KeywordTok{labs}\NormalTok{(}\DataTypeTok{x =} \StringTok{"Sleep Hours, centered"}\NormalTok{, }\DataTypeTok{y =} \StringTok{"Body Mass Index"}\NormalTok{,}
         \DataTypeTok{title =} \StringTok{"Model `c2_m4` on centered data"}\NormalTok{) }\OperatorTok{+}
\StringTok{    }\KeywordTok{facet_wrap}\NormalTok{(}\OperatorTok{~}\StringTok{ }\NormalTok{female, }\DataTypeTok{labeller =}\NormalTok{ label_both)}
\end{Highlighting}
\end{Shaded}

\includegraphics{bookdown-demo_files/figure-latex/unnamed-chunk-12-1.pdf}

\section{Rescaling an input by subtracting the mean and dividing by 2
standard
deviations}\label{rescaling-an-input-by-subtracting-the-mean-and-dividing-by-2-standard-deviations}

Centering helped us interpret the main effects in the regression, but it
still leaves a scaling problem.

\begin{itemize}
\tightlist
\item
  The \texttt{female} coefficient estimate is much larger than that of
  \texttt{sleephrs}, but this is misleading, considering that we are
  comparing the complete change in one variable (sex = female or not) to
  a 1-hour change in average sleep.
\item
  \citet{GelmanHill2007} recommend all continuous predictors be scaled
  by dividing by 2 standard deviations, so that:

  \begin{itemize}
  \tightlist
  \item
    a 1-unit change in the rescaled predictor corresponds to a change
    from 1 standard deviation below the mean, to 1 standard deviation
    above.
  \item
    an unscaled binary (1/0) predictor with 50\% probability of
    occurring will be exactly comparable to a rescaled continuous
    predictor done in this way.
  \end{itemize}
\end{itemize}

\begin{Shaded}
\begin{Highlighting}[]
\NormalTok{smartcle2_rescale <-}\StringTok{ }\NormalTok{smartcle2 }\OperatorTok
\StringTok{    }\KeywordTok{mutate}\NormalTok{(}\DataTypeTok{sleephrs_z =}\NormalTok{ (sleephrs }\OperatorTok{-}\StringTok{ }\KeywordTok{mean}\NormalTok{(sleephrs))}\OperatorTok{/}\NormalTok{(}\DecValTok{2}\OperatorTok{*}\KeywordTok{sd}\NormalTok{(sleephrs)))}
\end{Highlighting}
\end{Shaded}

\subsection{\texorpdfstring{Refitting model \texttt{c2\_m4} to the
rescaled
data}{Refitting model c2\_m4 to the rescaled data}}\label{refitting-model-c2_m4-to-the-rescaled-data}

\begin{Shaded}
\begin{Highlighting}[]
\NormalTok{c2_m4_z <-}\StringTok{ }\KeywordTok{lm}\NormalTok{(bmi }\OperatorTok{~}\StringTok{ }\NormalTok{female }\OperatorTok{*}\StringTok{ }\NormalTok{sleephrs_z, }\DataTypeTok{data =}\NormalTok{ smartcle2_rescale)}

\KeywordTok{summary}\NormalTok{(c2_m4_z)}
\end{Highlighting}
\end{Shaded}

\begin{verbatim}

Call:
lm(formula = bmi ~ female * sleephrs_z, data = smartcle2_rescale)

Residuals:
    Min      1Q  Median      3Q     Max 
-15.498  -4.179  -1.035   2.830  38.204 

Coefficients:
                  Estimate Std. Error t value Pr(>|t|)    
(Intercept)        28.3681     0.3274  86.658   <2e-16 ***
female             -0.8420     0.4280  -1.967   0.0495 *  
sleephrs_z          0.4637     0.6778   0.684   0.4940    
female:sleephrs_z  -1.4173     0.8661  -1.636   0.1021    
---
Signif. codes:  0 '***' 0.001 '**' 0.01 '*' 0.05 '.' 0.1 ' ' 1

Residual standard error: 6.31 on 892 degrees of freedom
Multiple R-squared:  0.008341,  Adjusted R-squared:  0.005006 
F-statistic: 2.501 on 3 and 892 DF,  p-value: 0.05818
\end{verbatim}

\subsection{Interpreting the model on rescaled
data}\label{interpreting-the-model-on-rescaled-data}

What has changed as compared to the original \texttt{c2\_m4}?

\begin{itemize}
\tightlist
\item
  Our original model was \texttt{bmi} = 27.26 + 2.53 \texttt{female} +
  0.16 \texttt{sleephrs} - 0.48 \texttt{female} x \texttt{sleephrs}
\item
  Our model on centered \texttt{sleephrs} was \texttt{bmi} = 28.37 -
  0.84 \texttt{female} + 0.16 centered \texttt{sleephrs\_c} - 0.48
  \texttt{female} x centered \texttt{sleephrs\_c}.
\item
  Our new model on rescaled \texttt{sleephrs} is \texttt{bmi} = 28.37 -
  0.84 \texttt{female} + 0.46 rescaled \texttt{sleephrs\_z} - 1.42
  \texttt{female} x rescaled \texttt{sleephrs\_z}.
\end{itemize}

So our rescaled model is:

\begin{itemize}
\tightlist
\item
  28.37 + 0.46 rescaled \texttt{sleephrs\_z} for male subjects, and
\item
  (28.37 - 0.84) + (0.46 - 1.42) rescaled \texttt{sleephrs\_z}, or 27.53
  - 0.96 rescaled \texttt{sleephrs\_z} for female subjects.
\end{itemize}

In this new rescaled (\texttt{sleephrs\_z}) model, then,

\begin{itemize}
\tightlist
\item
  the main effect of \texttt{female}, -0.84, still corresponds to a
  predictive difference (female - male) in \texttt{bmi} with
  \texttt{sleephrs} at its mean value, 7.02 hours,
\item
  the intercept term is still the predicted \texttt{bmi} for a male
  respondent who sleeps an average number of hours, and
\item
  the residual standard deviation and the R-squared values remain
  unchanged,
\end{itemize}

as before, but now we also have that:

\begin{itemize}
\tightlist
\item
  the coefficient of \texttt{sleephrs\_z} indicates the predictive
  difference in \texttt{bmi} associated with a change in
  \texttt{sleephrs} of 2 standard deviations (from one standard
  deviation below the mean of 7.02 to one standard deviation above
  7.02.)

  \begin{itemize}
  \tightlist
  \item
    Since the standard deviation of \texttt{sleephrs} is 1.48, this
    corresponds to a change from 5.54 hours per night to 8.50 hours per
    night.
  \end{itemize}
\item
  the coefficient of the product term (-1.42) corresponds to the change
  in the coefficient of \texttt{sleephrs\_z} for females as compared to
  males.
\end{itemize}

\subsection{Plot of model on rescaled
data}\label{plot-of-model-on-rescaled-data}

\begin{Shaded}
\begin{Highlighting}[]
\KeywordTok{ggplot}\NormalTok{(smartcle2_rescale, }\KeywordTok{aes}\NormalTok{(}\DataTypeTok{x =}\NormalTok{ sleephrs_z, }\DataTypeTok{y =}\NormalTok{ bmi, }
                              \DataTypeTok{group =}\NormalTok{ female, }\DataTypeTok{col =} \KeywordTok{factor}\NormalTok{(female))) }\OperatorTok{+}
\StringTok{    }\KeywordTok{geom_point}\NormalTok{(}\DataTypeTok{alpha =} \FloatTok{0.5}\NormalTok{) }\OperatorTok{+}
\StringTok{    }\KeywordTok{geom_smooth}\NormalTok{(}\DataTypeTok{method =} \StringTok{"lm"}\NormalTok{, }\DataTypeTok{size =} \FloatTok{1.5}\NormalTok{) }\OperatorTok{+}
\StringTok{    }\KeywordTok{scale_color_discrete}\NormalTok{(}\DataTypeTok{name =} \StringTok{"Is subject female?"}\NormalTok{) }\OperatorTok{+}
\StringTok{    }\KeywordTok{labs}\NormalTok{(}\DataTypeTok{x =} \StringTok{"Sleep Hours, standardized (2 sd)"}\NormalTok{, }\DataTypeTok{y =} \StringTok{"Body Mass Index"}\NormalTok{,}
         \DataTypeTok{title =} \StringTok{"Model `c2_m4_z` on rescaled data"}\NormalTok{)}
\end{Highlighting}
\end{Shaded}

\includegraphics{bookdown-demo_files/figure-latex/unnamed-chunk-14-1.pdf}

\section{\texorpdfstring{\texttt{c2\_m5}: What if we add more
variables?}{c2\_m5: What if we add more variables?}}\label{c2_m5-what-if-we-add-more-variables}

We can boost our R\textsuperscript{2} a bit, to over 5\%, by adding in
two new variables, related to whether or not the subject (in the past 30
days) used the internet, and on how many days the subject drank
alcoholic beverages.

\begin{Shaded}
\begin{Highlighting}[]
\NormalTok{c2_m5 <-}\StringTok{ }\KeywordTok{lm}\NormalTok{(bmi }\OperatorTok{~}\StringTok{ }\NormalTok{female }\OperatorTok{+}\StringTok{ }\NormalTok{exerany }\OperatorTok{+}\StringTok{ }\NormalTok{sleephrs }\OperatorTok{+}\StringTok{ }\NormalTok{internet30 }\OperatorTok{+}\StringTok{ }\NormalTok{alcdays,}
         \DataTypeTok{data =}\NormalTok{ smartcle2)}
\KeywordTok{summary}\NormalTok{(c2_m5)}
\end{Highlighting}
\end{Shaded}

\begin{verbatim}

Call:
lm(formula = bmi ~ female + exerany + sleephrs + internet30 + 
    alcdays, data = smartcle2)

Residuals:
    Min      1Q  Median      3Q     Max 
-16.147  -3.997  -0.856   2.487  35.965 

Coefficients:
            Estimate Std. Error t value Pr(>|t|)    
(Intercept) 30.84066    1.18458  26.035  < 2e-16 ***
female      -1.28801    0.42805  -3.009   0.0027 ** 
exerany     -2.42161    0.49853  -4.858 1.40e-06 ***
sleephrs    -0.14118    0.13988  -1.009   0.3131    
internet30   1.38916    0.54252   2.561   0.0106 *  
alcdays     -0.10460    0.02595  -4.030 6.04e-05 ***
---
Signif. codes:  0 '***' 0.001 '**' 0.01 '*' 0.05 '.' 0.1 ' ' 1

Residual standard error: 6.174 on 890 degrees of freedom
Multiple R-squared:  0.05258,   Adjusted R-squared:  0.04726 
F-statistic: 9.879 on 5 and 890 DF,  p-value: 3.304e-09
\end{verbatim}

\begin{enumerate}
\def\labelenumi{\arabic{enumi}.}
\tightlist
\item
  Here's the ANOVA for this model. What can we study with this?
\end{enumerate}

\begin{Shaded}
\begin{Highlighting}[]
\KeywordTok{anova}\NormalTok{(c2_m5)}
\end{Highlighting}
\end{Shaded}

\begin{verbatim}
Analysis of Variance Table

Response: bmi
            Df Sum Sq Mean Sq F value    Pr(>F)    
female       1    156  155.61  4.0818   0.04365 *  
exerany      1    897  896.93 23.5283 1.453e-06 ***
sleephrs     1     33   32.90  0.8631   0.35313    
internet30   1    178  178.33  4.6779   0.03082 *  
alcdays      1    619  619.26 16.2443 6.044e-05 ***
Residuals  890  33928   38.12                      
---
Signif. codes:  0 '***' 0.001 '**' 0.01 '*' 0.05 '.' 0.1 ' ' 1
\end{verbatim}

\begin{enumerate}
\def\labelenumi{\arabic{enumi}.}
\setcounter{enumi}{1}
\tightlist
\item
  Consider the revised output below. Now what can we study?
\end{enumerate}

\begin{Shaded}
\begin{Highlighting}[]
\KeywordTok{anova}\NormalTok{(}\KeywordTok{lm}\NormalTok{(bmi }\OperatorTok{~}\StringTok{ }\NormalTok{exerany }\OperatorTok{+}\StringTok{ }\NormalTok{internet30 }\OperatorTok{+}\StringTok{ }\NormalTok{alcdays }\OperatorTok{+}\StringTok{ }\NormalTok{female }\OperatorTok{+}\StringTok{ }\NormalTok{sleephrs,}
         \DataTypeTok{data =}\NormalTok{ smartcle2))}
\end{Highlighting}
\end{Shaded}

\begin{verbatim}
Analysis of Variance Table

Response: bmi
            Df Sum Sq Mean Sq F value    Pr(>F)    
exerany      1    795  795.46 20.8664 5.618e-06 ***
internet30   1    212  211.95  5.5599 0.0185925 *  
alcdays      1    486  486.03 12.7496 0.0003752 ***
female       1    351  350.75  9.2010 0.0024891 ** 
sleephrs     1     39   38.83  1.0186 0.3131176    
Residuals  890  33928   38.12                      
---
Signif. codes:  0 '***' 0.001 '**' 0.01 '*' 0.05 '.' 0.1 ' ' 1
\end{verbatim}

\begin{enumerate}
\def\labelenumi{\arabic{enumi}.}
\setcounter{enumi}{2}
\tightlist
\item
  What does the output below let us conclude?
\end{enumerate}

\begin{Shaded}
\begin{Highlighting}[]
\KeywordTok{anova}\NormalTok{(}\KeywordTok{lm}\NormalTok{(bmi }\OperatorTok{~}\StringTok{ }\NormalTok{exerany }\OperatorTok{+}\StringTok{ }\NormalTok{internet30 }\OperatorTok{+}\StringTok{ }\NormalTok{alcdays }\OperatorTok{+}\StringTok{ }\NormalTok{female }\OperatorTok{+}\StringTok{ }\NormalTok{sleephrs, }
         \DataTypeTok{data =}\NormalTok{ smartcle2),}
      \KeywordTok{lm}\NormalTok{(bmi }\OperatorTok{~}\StringTok{ }\NormalTok{exerany }\OperatorTok{+}\StringTok{ }\NormalTok{female }\OperatorTok{+}\StringTok{ }\NormalTok{alcdays, }
         \DataTypeTok{data =}\NormalTok{ smartcle2))}
\end{Highlighting}
\end{Shaded}

\begin{verbatim}
Analysis of Variance Table

Model 1: bmi ~ exerany + internet30 + alcdays + female + sleephrs
Model 2: bmi ~ exerany + female + alcdays
  Res.Df   RSS Df Sum of Sq      F  Pr(>F)  
1    890 33928                              
2    892 34221 -2    -293.2 3.8456 0.02173 *
---
Signif. codes:  0 '***' 0.001 '**' 0.01 '*' 0.05 '.' 0.1 ' ' 1
\end{verbatim}

\begin{enumerate}
\def\labelenumi{\arabic{enumi}.}
\setcounter{enumi}{3}
\tightlist
\item
  What does it mean for the models to be ``nested''?
\end{enumerate}

\section{\texorpdfstring{\texttt{c2\_m6}: Would adding self-reported
health
help?}{c2\_m6: Would adding self-reported health help?}}\label{c2_m6-would-adding-self-reported-health-help}

And we can do even a bit better than that by adding in a
multi-categorical measure: self-reported general health.

\begin{Shaded}
\begin{Highlighting}[]
\NormalTok{c2_m6 <-}\StringTok{ }\KeywordTok{lm}\NormalTok{(bmi }\OperatorTok{~}\StringTok{ }\NormalTok{female }\OperatorTok{+}\StringTok{ }\NormalTok{exerany }\OperatorTok{+}\StringTok{ }\NormalTok{sleephrs }\OperatorTok{+}\StringTok{ }\NormalTok{internet30 }\OperatorTok{+}\StringTok{ }\NormalTok{alcdays }\OperatorTok{+}\StringTok{ }\NormalTok{genhealth,}
         \DataTypeTok{data =}\NormalTok{ smartcle2)}
\KeywordTok{summary}\NormalTok{(c2_m6)}
\end{Highlighting}
\end{Shaded}

\begin{verbatim}

Call:
lm(formula = bmi ~ female + exerany + sleephrs + internet30 + 
    alcdays + genhealth, data = smartcle2)

Residuals:
    Min      1Q  Median      3Q     Max 
-16.331  -3.813  -0.838   2.679  34.166 

Coefficients:
                    Estimate Std. Error t value Pr(>|t|)    
(Intercept)         26.49498    1.31121  20.206  < 2e-16 ***
female              -0.85520    0.41969  -2.038 0.041879 *  
exerany             -1.61968    0.50541  -3.205 0.001400 ** 
sleephrs            -0.12719    0.13613  -0.934 0.350368    
internet30           2.02498    0.53898   3.757 0.000183 ***
alcdays             -0.08431    0.02537  -3.324 0.000925 ***
genhealth2_VeryGood  2.10537    0.59408   3.544 0.000415 ***
genhealth3_Good      4.08245    0.60739   6.721 3.22e-11 ***
genhealth4_Fair      4.99213    0.80178   6.226 7.37e-10 ***
genhealth5_Poor      3.11025    1.12614   2.762 0.005866 ** 
---
Signif. codes:  0 '***' 0.001 '**' 0.01 '*' 0.05 '.' 0.1 ' ' 1

Residual standard error: 5.993 on 886 degrees of freedom
Multiple R-squared:  0.1115,    Adjusted R-squared:  0.1024 
F-statistic: 12.35 on 9 and 886 DF,  p-value: < 2.2e-16
\end{verbatim}

\begin{enumerate}
\def\labelenumi{\arabic{enumi}.}
\item
  If Harry and Marty have the same values of \texttt{female},
  \texttt{exerany}, \texttt{sleephrs}, \texttt{internet30} and
  \texttt{alcdays}, but Harry rates his health as Good, and Marty rates
  his as Fair, then what is the difference in the predictions? Who is
  predicted to have a larger BMI, and by how much?
\item
  What does this normal probability plot of the residuals suggest?
\end{enumerate}

\begin{Shaded}
\begin{Highlighting}[]
\KeywordTok{plot}\NormalTok{(c2_m6, }\DataTypeTok{which =} \DecValTok{2}\NormalTok{)}
\end{Highlighting}
\end{Shaded}

\includegraphics{bookdown-demo_files/figure-latex/c2_m6_residuals_normality-1.pdf}

\section{\texorpdfstring{\texttt{c2\_m7}: What if we added the
\texttt{menthealth}
variable?}{c2\_m7: What if we added the menthealth variable?}}\label{c2_m7-what-if-we-added-the-menthealth-variable}

\begin{Shaded}
\begin{Highlighting}[]
\NormalTok{c2_m7 <-}\StringTok{ }\KeywordTok{lm}\NormalTok{(bmi }\OperatorTok{~}\StringTok{ }\NormalTok{female }\OperatorTok{+}\StringTok{ }\NormalTok{exerany }\OperatorTok{+}\StringTok{ }\NormalTok{sleephrs }\OperatorTok{+}\StringTok{ }\NormalTok{internet30 }\OperatorTok{+}\StringTok{ }\NormalTok{alcdays }\OperatorTok{+}\StringTok{ }
\StringTok{                }\NormalTok{genhealth }\OperatorTok{+}\StringTok{ }\NormalTok{physhealth }\OperatorTok{+}\StringTok{ }\NormalTok{menthealth,}
         \DataTypeTok{data =}\NormalTok{ smartcle2)}

\KeywordTok{summary}\NormalTok{(c2_m7)}
\end{Highlighting}
\end{Shaded}

\begin{verbatim}

Call:
lm(formula = bmi ~ female + exerany + sleephrs + internet30 + 
    alcdays + genhealth + physhealth + menthealth, data = smartcle2)

Residuals:
    Min      1Q  Median      3Q     Max 
-16.060  -3.804  -0.890   2.794  33.972 

Coefficients:
                    Estimate Std. Error t value Pr(>|t|)    
(Intercept)         25.88208    1.31854  19.629  < 2e-16 ***
female              -0.96435    0.41908  -2.301 0.021616 *  
exerany             -1.43171    0.50635  -2.828 0.004797 ** 
sleephrs            -0.08033    0.13624  -0.590 0.555583    
internet30           2.00267    0.53759   3.725 0.000207 ***
alcdays             -0.07997    0.02528  -3.163 0.001614 ** 
genhealth2_VeryGood  2.09533    0.59238   3.537 0.000425 ***
genhealth3_Good      3.90949    0.60788   6.431 2.07e-10 ***
genhealth4_Fair      4.27152    0.83986   5.086 4.47e-07 ***
genhealth5_Poor      1.26021    1.31556   0.958 0.338361    
physhealth           0.06088    0.03005   2.026 0.043064 *  
menthealth           0.06636    0.03177   2.089 0.037021 *  
---
Signif. codes:  0 '***' 0.001 '**' 0.01 '*' 0.05 '.' 0.1 ' ' 1

Residual standard error: 5.964 on 884 degrees of freedom
Multiple R-squared:  0.1219,    Adjusted R-squared:  0.111 
F-statistic: 11.16 on 11 and 884 DF,  p-value: < 2.2e-16
\end{verbatim}

\section{Key Regression Assumptions for Building Effective Prediction
Models}\label{key-regression-assumptions-for-building-effective-prediction-models}

\begin{enumerate}
\def\labelenumi{\arabic{enumi}.}
\tightlist
\item
  Validity - the data you are analyzing should map to the research
  question you are trying to answer.

  \begin{itemize}
  \tightlist
  \item
    The outcome should accurately reflect the phenomenon of interest.
  \item
    The model should include all relevant predictors. (It can be
    difficult to decide which predictors are necessary, and what to do
    with predictors that have large standard errors.)
  \item
    The model should generalize to all of the cases to which it will be
    applied.
  \item
    Can the available data answer our question reliably?
  \end{itemize}
\item
  Additivity and linearity - most important assumption of a regression
  model is that its deterministic component is a linear function of the
  predictors. We often think about transformations in this setting.
\item
  Independence of errors - errors from the prediction line are
  independent of each other
\item
  Equal variance of errors - if this is violated, we can more
  efficiently estimate paramaters using \emph{weighted least squares}
  approaches, where each point is weighted inversely proportional to its
  variance, but this doesn't affect the coefficients much, if at all.
\item
  Normality of errors - not generally important for estimating the
  regression line
\end{enumerate}

\subsection{\texorpdfstring{Checking Assumptions in model
\texttt{c2\_m7}}{Checking Assumptions in model c2\_m7}}\label{checking-assumptions-in-model-c2_m7}

\begin{enumerate}
\def\labelenumi{\arabic{enumi}.}
\tightlist
\item
  How does the assumption of linearity behind this model look?
\end{enumerate}

\begin{Shaded}
\begin{Highlighting}[]
\KeywordTok{plot}\NormalTok{(c2_m7, }\DataTypeTok{which =} \DecValTok{1}\NormalTok{)}
\end{Highlighting}
\end{Shaded}

\includegraphics{bookdown-demo_files/figure-latex/residual_plot1_c2_m7-1.pdf}

We see no strong signs of serious non-linearity here. There's no obvious
curve in the plot, for example.

\begin{enumerate}
\def\labelenumi{\arabic{enumi}.}
\setcounter{enumi}{1}
\tightlist
\item
  What can we conclude from the plot below?
\end{enumerate}

\begin{Shaded}
\begin{Highlighting}[]
\KeywordTok{plot}\NormalTok{(c2_m7, }\DataTypeTok{which =} \DecValTok{5}\NormalTok{)}
\end{Highlighting}
\end{Shaded}

\includegraphics{bookdown-demo_files/figure-latex/residual_plot5_c2_m7-1.pdf}

This plot can help us identify points with large standardized residuals,
large leverage values, and large influence on the model (as indicated by
large values of Cook's distance.) In this case, I see no signs of any
points used in the model with especially large influence, although there
are some poorly fitted points (with especially large standardized
residuals.)

\chapter{Analysis of Variance}\label{analysis-of-variance}

\section{\texorpdfstring{The \texttt{bonding} data: A Designed Dental
Experiment}{The bonding data: A Designed Dental Experiment}}\label{the-bonding-data-a-designed-dental-experiment}

The \texttt{bonding} data describe a designed experiment into the
properties of four different resin types (\texttt{resin} = A, B, C, D)
and two different curing light sources (\texttt{light} = Halogen, LED)
as they relate to the resulting bonding strength (measured in
MPa\footnote{The MPa is defined as the failure load (in Newtons) divided
  by the entire bonded area, in mm\textsuperscript{2}.}) on the surface
of teeth. The source is \citet{Kim2014}.

The experiment involved making measurements of bonding strength under a
total of 80 experimental setups, or runs, with 10 runs completed at each
of the eight combinations of a light source and a resin type. The data
are gathered in the \texttt{bonding.csv} file.

\begin{Shaded}
\begin{Highlighting}[]
\NormalTok{bonding}
\end{Highlighting}
\end{Shaded}

\begin{verbatim}
# A tibble: 80 x 4
   run_ID light   resin strength
   <fct>  <fct>   <fct>    <dbl>
 1 R101   LED     B         12.8
 2 R102   Halogen B         22.2
 3 R103   Halogen B         24.6
 4 R104   LED     A         17.0
 5 R105   LED     C         32.2
 6 R106   Halogen B         27.1
 7 R107   LED     A         23.4
 8 R108   Halogen A         23.5
 9 R109   Halogen D         37.3
10 R110   Halogen A         19.7
# ... with 70 more rows
\end{verbatim}

\section{A One-Factor Analysis of
Variance}\label{a-one-factor-analysis-of-variance}

Suppose we are interested in the distribution of the \texttt{strength}
values for the four different types of \texttt{resin}.

\begin{Shaded}
\begin{Highlighting}[]
\NormalTok{bonding }\OperatorTok\StringTok{ }\KeywordTok{group_by}\NormalTok{(resin) }\OperatorTok\StringTok{ }\KeywordTok{summarize}\NormalTok{(}\DataTypeTok{n =} \KeywordTok{n}\NormalTok{(), }\KeywordTok{mean}\NormalTok{(strength), }\KeywordTok{median}\NormalTok{(strength))}
\end{Highlighting}
\end{Shaded}

\begin{verbatim}
# A tibble: 4 x 4
  resin     n `mean(strength)` `median(strength)`
  <fct> <int>            <dbl>              <dbl>
1 A        20             18.4               18.0
2 B        20             22.2               22.7
3 C        20             25.2               25.7
4 D        20             32.1               35.3
\end{verbatim}

I'd begin serious work with a plot.

\subsection{Look at the Data!}\label{look-at-the-data}

\begin{Shaded}
\begin{Highlighting}[]
\KeywordTok{ggplot}\NormalTok{(bonding, }\KeywordTok{aes}\NormalTok{(}\DataTypeTok{x =}\NormalTok{ resin, }\DataTypeTok{y =}\NormalTok{ strength)) }\OperatorTok{+}
\StringTok{    }\KeywordTok{geom_boxplot}\NormalTok{()}
\end{Highlighting}
\end{Shaded}

\includegraphics{bookdown-demo_files/figure-latex/c3_oneway_bonding_resin_boxplot-1.pdf}

Another good plot for this purpose is a ridgeline plot.

\begin{Shaded}
\begin{Highlighting}[]
\KeywordTok{ggplot}\NormalTok{(bonding, }\KeywordTok{aes}\NormalTok{(}\DataTypeTok{x =}\NormalTok{ strength, }\DataTypeTok{y =}\NormalTok{ resin, }\DataTypeTok{fill =}\NormalTok{ resin)) }\OperatorTok{+}
\StringTok{    }\KeywordTok{geom_density_ridges2}\NormalTok{() }\OperatorTok{+}
\StringTok{    }\KeywordTok{guides}\NormalTok{(}\DataTypeTok{fill =} \OtherTok{FALSE}\NormalTok{)}
\end{Highlighting}
\end{Shaded}

\begin{verbatim}
Picking joint bandwidth of 3.09
\end{verbatim}

\includegraphics{bookdown-demo_files/figure-latex/c3_oneway_bonding_resin_ridgelineplot-1.pdf}

\subsection{Table of Summary
Statistics}\label{table-of-summary-statistics}

With the small size of this experiment (\emph{n} = 20 for each
\texttt{resin} type), graphical summaries may not perform as well as
they often do. We'll also produce a quick table of summary statistics
for \texttt{strength} within each \texttt{resin} type, with the
\texttt{skim()} function.

\begin{Shaded}
\begin{Highlighting}[]
\NormalTok{bonding }\OperatorTok\StringTok{ }\KeywordTok{group_by}\NormalTok{(resin) }\OperatorTok\StringTok{ }\KeywordTok{skim}\NormalTok{(strength)}
\end{Highlighting}
\end{Shaded}

\begin{verbatim}
Skim summary statistics
 n obs: 80 
 n variables: 4 
 group variables: resin 

Variable type: numeric 
 resin variable missing complete  n  mean   sd   p0   p25 median   p75
     A strength       0       20 20 18.41 4.81  9.3 15.73  17.95 20.4 
     B strength       0       20 20 22.23 6.75 11.8 18.45  22.7  25.75
     C strength       0       20 20 25.16 6.33 14.5 20.65  25.7  30.7 
     D strength       0       20 20 32.08 9.74 17.3 21.82  35.3  40.15
 p100
 28  
 35.2
 34.5
 47.2
\end{verbatim}

Since the means and medians within each group are fairly close, and the
distributions (with the possible exception of \texttt{resin} D) are
reasonably well approximated by the Normal, I'll fit an ANOVA
model\footnote{If the data weren't approximately Normally distributed,
  we might instead consider a rank-based alternative to ANOVA, like the
  Kruskal-Wallis test.}.

\begin{Shaded}
\begin{Highlighting}[]
\KeywordTok{anova}\NormalTok{(}\KeywordTok{lm}\NormalTok{(strength }\OperatorTok{~}\StringTok{ }\NormalTok{resin, }\DataTypeTok{data =}\NormalTok{ bonding))}
\end{Highlighting}
\end{Shaded}

\begin{verbatim}
Analysis of Variance Table

Response: strength
          Df Sum Sq Mean Sq F value   Pr(>F)    
resin      3 1999.7  666.57  13.107 5.52e-07 ***
Residuals 76 3865.2   50.86                     
---
Signif. codes:  0 '***' 0.001 '**' 0.01 '*' 0.05 '.' 0.1 ' ' 1
\end{verbatim}

It appears that the \texttt{resin} types have a significant association
with mean \texttt{strength} of the bonds. Can we identify which
\texttt{resin} types have generally higher or lower \texttt{strength}?

\begin{Shaded}
\begin{Highlighting}[]
\KeywordTok{TukeyHSD}\NormalTok{(}\KeywordTok{aov}\NormalTok{(}\KeywordTok{lm}\NormalTok{(strength }\OperatorTok{~}\StringTok{ }\NormalTok{resin, }\DataTypeTok{data =}\NormalTok{ bonding)))}
\end{Highlighting}
\end{Shaded}

\begin{verbatim}
  Tukey multiple comparisons of means
    95% family-wise confidence level

Fit: aov(formula = lm(strength ~ resin, data = bonding))

$resin
      diff        lwr       upr     p adj
B-A  3.815 -2.1088676  9.738868 0.3351635
C-A  6.740  0.8161324 12.663868 0.0193344
D-A 13.660  7.7361324 19.583868 0.0000003
C-B  2.925 -2.9988676  8.848868 0.5676635
D-B  9.845  3.9211324 15.768868 0.0002276
D-C  6.920  0.9961324 12.843868 0.0154615
\end{verbatim}

Based on these confidence intervals (which have a family-wise 95\%
confidence level), we see that D is associated with significantly larger
mean \texttt{strength} than A or B or C, and that C is also associated
with significantly larger mean \texttt{strength} than A. This may be
easier to see in a plot of these confidence intervals.

\begin{Shaded}
\begin{Highlighting}[]
\KeywordTok{plot}\NormalTok{(}\KeywordTok{TukeyHSD}\NormalTok{(}\KeywordTok{aov}\NormalTok{(}\KeywordTok{lm}\NormalTok{(strength }\OperatorTok{~}\StringTok{ }\NormalTok{resin, }\DataTypeTok{data =}\NormalTok{ bonding))))}
\end{Highlighting}
\end{Shaded}

\includegraphics{bookdown-demo_files/figure-latex/unnamed-chunk-19-1.pdf}

\section{A Two-Way ANOVA: Looking at Two
Factors}\label{a-two-way-anova-looking-at-two-factors}

Now, we'll now add consideration of the \texttt{light} source into our
study. We can look at the distribution of the \texttt{strength} values
at the combinations of both \texttt{light} and \texttt{resin}, with a
plot like this one\ldots{}

\begin{Shaded}
\begin{Highlighting}[]
\KeywordTok{ggplot}\NormalTok{(bonding, }\KeywordTok{aes}\NormalTok{(}\DataTypeTok{x =}\NormalTok{ resin, }\DataTypeTok{y =}\NormalTok{ strength, }\DataTypeTok{color =}\NormalTok{ light)) }\OperatorTok{+}
\StringTok{    }\KeywordTok{geom_point}\NormalTok{(}\DataTypeTok{size =} \DecValTok{2}\NormalTok{, }\DataTypeTok{alpha =} \FloatTok{0.5}\NormalTok{) }\OperatorTok{+}
\StringTok{    }\KeywordTok{facet_wrap}\NormalTok{(}\OperatorTok{~}\StringTok{ }\NormalTok{light) }\OperatorTok{+}
\StringTok{    }\KeywordTok{guides}\NormalTok{(}\DataTypeTok{color =} \OtherTok{FALSE}\NormalTok{) }\OperatorTok{+}
\StringTok{    }\KeywordTok{scale_color_manual}\NormalTok{(}\DataTypeTok{values =} \KeywordTok{c}\NormalTok{(}\StringTok{"purple"}\NormalTok{, }\StringTok{"darkorange"}\NormalTok{)) }\OperatorTok{+}
\StringTok{    }\KeywordTok{theme_bw}\NormalTok{() }
\end{Highlighting}
\end{Shaded}

\includegraphics{bookdown-demo_files/figure-latex/c3_bonding_points_plot-1.pdf}

\section{A Means Plot (with standard deviations) to check for
interaction}\label{a-means-plot-with-standard-deviations-to-check-for-interaction}

Sometimes, we'll instead look at a plot simply of the means (and, often,
the standard deviations) of \texttt{strength} at each combination of
\texttt{light} and \texttt{resin}. We'll start by building up a data set
with the summaries we want to plot.

\begin{Shaded}
\begin{Highlighting}[]
\NormalTok{bond.sum <-}\StringTok{ }\NormalTok{bonding }\OperatorTok\StringTok{ }
\StringTok{    }\KeywordTok{group_by}\NormalTok{(resin, light) }\OperatorTok
\StringTok{    }\KeywordTok{summarize}\NormalTok{(}\DataTypeTok{mean.str =} \KeywordTok{mean}\NormalTok{(strength), }\DataTypeTok{sd.str =} \KeywordTok{sd}\NormalTok{(strength))}

\NormalTok{bond.sum}
\end{Highlighting}
\end{Shaded}

\begin{verbatim}
# A tibble: 8 x 4
# Groups:   resin [?]
  resin light   mean.str sd.str
  <fct> <fct>      <dbl>  <dbl>
1 A     Halogen     17.8   4.02
2 A     LED         19.1   5.63
3 B     Halogen     19.9   5.62
4 B     LED         24.6   7.25
5 C     Halogen     22.5   6.19
6 C     LED         27.8   5.56
7 D     Halogen     40.3   4.15
8 D     LED         23.8   5.70
\end{verbatim}

Now, we'll use this new data set to plot the means and standard
deviations of \texttt{strength} at each combination of \texttt{resin}
and \texttt{light}.

\begin{Shaded}
\begin{Highlighting}[]
\NormalTok{## The error bars will overlap unless we adjust the position.}
\NormalTok{pd <-}\StringTok{ }\KeywordTok{position_dodge}\NormalTok{(}\FloatTok{0.2}\NormalTok{) }\CommentTok{# move them .1 to the left and right}

\KeywordTok{ggplot}\NormalTok{(bond.sum, }\KeywordTok{aes}\NormalTok{(}\DataTypeTok{x =}\NormalTok{ resin, }\DataTypeTok{y =}\NormalTok{ mean.str, }\DataTypeTok{col =}\NormalTok{ light)) }\OperatorTok{+}
\StringTok{    }\KeywordTok{geom_errorbar}\NormalTok{(}\KeywordTok{aes}\NormalTok{(}\DataTypeTok{ymin =}\NormalTok{ mean.str }\OperatorTok{-}\StringTok{ }\NormalTok{sd.str, }
                      \DataTypeTok{ymax =}\NormalTok{ mean.str }\OperatorTok{+}\StringTok{ }\NormalTok{sd.str),}
                  \DataTypeTok{width =} \FloatTok{0.2}\NormalTok{, }\DataTypeTok{position =}\NormalTok{ pd) }\OperatorTok{+}
\StringTok{    }\KeywordTok{geom_point}\NormalTok{(}\DataTypeTok{size =} \DecValTok{2}\NormalTok{, }\DataTypeTok{position =}\NormalTok{ pd) }\OperatorTok{+}\StringTok{ }
\StringTok{    }\KeywordTok{geom_line}\NormalTok{(}\KeywordTok{aes}\NormalTok{(}\DataTypeTok{group =}\NormalTok{ light), }\DataTypeTok{position =}\NormalTok{ pd) }\OperatorTok{+}
\StringTok{    }\KeywordTok{scale_color_manual}\NormalTok{(}\DataTypeTok{values =} \KeywordTok{c}\NormalTok{(}\StringTok{"purple"}\NormalTok{, }\StringTok{"darkorange"}\NormalTok{)) }\OperatorTok{+}
\StringTok{    }\KeywordTok{theme_bw}\NormalTok{() }\OperatorTok{+}
\StringTok{    }\KeywordTok{labs}\NormalTok{(}\DataTypeTok{y =} \StringTok{"Bonding Strength (MPa)"}\NormalTok{, }\DataTypeTok{x =} \StringTok{"Resin Type"}\NormalTok{,}
         \DataTypeTok{title =} \StringTok{"Observed Means (+/- SD) of Bonding Strength"}\NormalTok{)}
\end{Highlighting}
\end{Shaded}

\includegraphics{bookdown-demo_files/figure-latex/c3_ggplot_means_plot_bonding-1.pdf}

Is there evidence of a meaningful interaction between the resin type and
the \texttt{light} source on the bonding strength in this plot?

\begin{itemize}
\tightlist
\item
  Sure. A meaningful interaction just means that the strength associated
  with different \texttt{resin} types depends on the \texttt{light}
  source.

  \begin{itemize}
  \tightlist
  \item
    With LED \texttt{light}, it appears that \texttt{resin} C leads to
    the strongest bonding strength.
  \item
    With Halogen \texttt{light}, though, it seems that \texttt{resin} D
    is substantially stronger.
  \end{itemize}
\item
  Note that the lines we see here connecting the \texttt{light} sources
  aren't in parallel (as they would be if we had zero interaction
  between \texttt{resin} and \texttt{light}), but rather, they cross.
\end{itemize}

\subsection{\texorpdfstring{Skimming the data after grouping by
\texttt{resin} and
\texttt{light}}{Skimming the data after grouping by resin and light}}\label{skimming-the-data-after-grouping-by-resin-and-light}

We might want to look at a numerical summary of the \texttt{strengths}
within these groups, too.

\begin{Shaded}
\begin{Highlighting}[]
\NormalTok{bonding }\OperatorTok
\StringTok{    }\KeywordTok{group_by}\NormalTok{(resin, light) }\OperatorTok
\StringTok{    }\KeywordTok{skim}\NormalTok{(strength) }
\end{Highlighting}
\end{Shaded}

\begin{verbatim}
Skim summary statistics
 n obs: 80 
 n variables: 4 
 group variables: resin, light 

Variable type: numeric 
 resin   light variable missing complete  n  mean   sd   p0   p25 median
     A Halogen strength       0       10 10 17.77 4.02  9.3 15.75  18.35
     A     LED strength       0       10 10 19.06 5.63 11.6 16.18  17.8 
     B Halogen strength       0       10 10 19.9  5.62 11.8 14.78  21.75
     B     LED strength       0       10 10 24.56 7.25 12.8 20.45  24.45
     C Halogen strength       0       10 10 22.54 6.19 14.5 18.85  21.3 
     C     LED strength       0       10 10 27.77 5.56 16.5 24.7   28.45
     D Halogen strength       0       10 10 40.3  4.15 35.5 36.55  40.4 
     D     LED strength       0       10 10 23.85 5.7  17.3 19.75  21.45
   p75 p100
 20    23.5
 22.5  28  
 24.12 27.1
 27.87 35.2
 25.8  33  
 31.83 34.5
 43.62 47.2
 28.2  35.1
\end{verbatim}

\section{Fitting the Two-Way ANOVA model with
Interaction}\label{fitting-the-two-way-anova-model-with-interaction}

\begin{Shaded}
\begin{Highlighting}[]
\NormalTok{c3_m1 <-}\StringTok{ }\KeywordTok{lm}\NormalTok{(strength }\OperatorTok{~}\StringTok{ }\NormalTok{resin }\OperatorTok{*}\StringTok{ }\NormalTok{light, }\DataTypeTok{data =}\NormalTok{ bonding)}

\KeywordTok{summary}\NormalTok{(c3_m1)}
\end{Highlighting}
\end{Shaded}

\begin{verbatim}

Call:
lm(formula = strength ~ resin * light, data = bonding)

Residuals:
    Min      1Q  Median      3Q     Max 
-11.760  -3.663  -0.320   3.697  11.250 

Coefficients:
                Estimate Std. Error t value Pr(>|t|)    
(Intercept)       17.770      1.771  10.033 2.57e-15 ***
resinB             2.130      2.505   0.850   0.3979    
resinC             4.770      2.505   1.904   0.0609 .  
resinD            22.530      2.505   8.995 2.13e-13 ***
lightLED           1.290      2.505   0.515   0.6081    
resinB:lightLED    3.370      3.542   0.951   0.3446    
resinC:lightLED    3.940      3.542   1.112   0.2697    
resinD:lightLED  -17.740      3.542  -5.008 3.78e-06 ***
---
Signif. codes:  0 '***' 0.001 '**' 0.01 '*' 0.05 '.' 0.1 ' ' 1

Residual standard error: 5.601 on 72 degrees of freedom
Multiple R-squared:  0.6149,    Adjusted R-squared:  0.5775 
F-statistic: 16.42 on 7 and 72 DF,  p-value: 9.801e-13
\end{verbatim}

\subsection{The ANOVA table for our
model}\label{the-anova-table-for-our-model}

In a two-way ANOVA model, we begin by assessing the interaction term. If
it's important, then our best model is the model including the
interaction. If it's not important, we will often move on to consider a
new model, fit without an interaction.

The ANOVA table is especially helpful in this case, because it lets us
look specifically at the interaction effect.

\begin{Shaded}
\begin{Highlighting}[]
\KeywordTok{anova}\NormalTok{(c3_m1)}
\end{Highlighting}
\end{Shaded}

\begin{verbatim}
Analysis of Variance Table

Response: strength
            Df  Sum Sq Mean Sq F value    Pr(>F)    
resin        3 1999.72  666.57 21.2499 5.792e-10 ***
light        1   34.72   34.72  1.1067    0.2963    
resin:light  3 1571.96  523.99 16.7043 2.457e-08 ***
Residuals   72 2258.52   31.37                      
---
Signif. codes:  0 '***' 0.001 '**' 0.01 '*' 0.05 '.' 0.1 ' ' 1
\end{verbatim}

\subsection{Is the interaction
important?}\label{is-the-interaction-important}

In this case, the interaction:

\begin{itemize}
\tightlist
\item
  is evident in the means plot, and
\item
  is highly statistically significant, and
\item
  accounts for a sizeable fraction (27\%) of the overall variation
\end{itemize}

\[ 
\eta^2_{interaction} = \frac{\mbox{SS(resin:light)}}{SS(Total)}
= \frac{1571.96}{1999.72 + 34.72 + 1571.96 + 2258.52} = 0.268
\]

If the interaction were \emph{either} large or significant we would be
inclined to keep it in the model. In this case, it's both, so there's no
real reason to remove it.

\subsection{Interpreting the
Interaction}\label{interpreting-the-interaction}

Recall the model equation, which is:

\begin{Shaded}
\begin{Highlighting}[]
\NormalTok{c3_m1}
\end{Highlighting}
\end{Shaded}

\begin{verbatim}

Call:
lm(formula = strength ~ resin * light, data = bonding)

Coefficients:
    (Intercept)           resinB           resinC           resinD  
          17.77             2.13             4.77            22.53  
       lightLED  resinB:lightLED  resinC:lightLED  resinD:lightLED  
           1.29             3.37             3.94           -17.74  
\end{verbatim}

so we have:

\[
strength = 17.77 + 2.13 resinB + 4.77 resinC + 22.53 resinD \\
+ 1.29 lightLED + 3.37 resinB*lightLED \\
+ 3.94 resinC*lightLED - 17.74 resinD*lightLED
\]

So, if \texttt{light} = Halogen, our equation is:

\[
strength = 17.77 + 2.13 resinB + 4.77 resinC + 22.53 resinD 
\]

And if \texttt{light} = LED, our equation is:

\[
strength = 19.06 + 5.50 resinB + 8.71 resinC + 4.79 resinD 
\]

Note that both the intercept and the slopes change as a result of the
interaction. The model yields a different prediction for every possible
combination of a \texttt{resin} type and a \texttt{light} source.

\section{\texorpdfstring{Comparing Individual Combinations of
\texttt{resin} and
\texttt{light}}{Comparing Individual Combinations of resin and light}}\label{comparing-individual-combinations-of-resin-and-light}

To make comparisons between individual combinations of a \texttt{resin}
type and a \texttt{light} source, using something like Tukey's HSD
approach for multiple comparisons, we first refit the model using the
\texttt{aov} structure, rather than \texttt{lm}.

\begin{Shaded}
\begin{Highlighting}[]
\NormalTok{c3m1_aov <-}\StringTok{ }\KeywordTok{aov}\NormalTok{(strength }\OperatorTok{~}\StringTok{ }\NormalTok{resin }\OperatorTok{*}\StringTok{ }\NormalTok{light, }\DataTypeTok{data =}\NormalTok{ bonding)}

\KeywordTok{summary}\NormalTok{(c3m1_aov)}
\end{Highlighting}
\end{Shaded}

\begin{verbatim}
            Df Sum Sq Mean Sq F value   Pr(>F)    
resin        3 1999.7   666.6  21.250 5.79e-10 ***
light        1   34.7    34.7   1.107    0.296    
resin:light  3 1572.0   524.0  16.704 2.46e-08 ***
Residuals   72 2258.5    31.4                     
---
Signif. codes:  0 '***' 0.001 '**' 0.01 '*' 0.05 '.' 0.1 ' ' 1
\end{verbatim}

And now, we can obtain Tukey HSD comparisons (which will maintain an
overall 95\% family-wise confidence level) across the \texttt{resin}
types, the \texttt{light} sources, and the combinations, with the
TukeyHSD command. This approach is only completely appropriate if these
comparisons are pre-planned, and if the design is balanced (as this is,
with the same sample size for each combination of a \texttt{light}
source and \texttt{resin} type.)

\begin{Shaded}
\begin{Highlighting}[]
\KeywordTok{TukeyHSD}\NormalTok{(c3m1_aov)}
\end{Highlighting}
\end{Shaded}

\begin{verbatim}
  Tukey multiple comparisons of means
    95% family-wise confidence level

Fit: aov(formula = strength ~ resin * light, data = bonding)

$resin
      diff       lwr       upr     p adj
B-A  3.815 -0.843129  8.473129 0.1461960
C-A  6.740  2.081871 11.398129 0.0016436
D-A 13.660  9.001871 18.318129 0.0000000
C-B  2.925 -1.733129  7.583129 0.3568373
D-B  9.845  5.186871 14.503129 0.0000026
D-C  6.920  2.261871 11.578129 0.0011731

$light
               diff       lwr      upr     p adj
LED-Halogen -1.3175 -3.814042 1.179042 0.2963128

$`resin:light`
                      diff          lwr        upr     p adj
B:Halogen-A:Halogen   2.13  -5.68928258   9.949283 0.9893515
C:Halogen-A:Halogen   4.77  -3.04928258  12.589283 0.5525230
D:Halogen-A:Halogen  22.53  14.71071742  30.349283 0.0000000
A:LED-A:Halogen       1.29  -6.52928258   9.109283 0.9995485
B:LED-A:Halogen       6.79  -1.02928258  14.609283 0.1361092
C:LED-A:Halogen      10.00   2.18071742  17.819283 0.0037074
D:LED-A:Halogen       6.08  -1.73928258  13.899283 0.2443200
C:Halogen-B:Halogen   2.64  -5.17928258  10.459283 0.9640100
D:Halogen-B:Halogen  20.40  12.58071742  28.219283 0.0000000
A:LED-B:Halogen      -0.84  -8.65928258   6.979283 0.9999747
B:LED-B:Halogen       4.66  -3.15928258  12.479283 0.5818695
C:LED-B:Halogen       7.87   0.05071742  15.689283 0.0473914
D:LED-B:Halogen       3.95  -3.86928258  11.769283 0.7621860
D:Halogen-C:Halogen  17.76   9.94071742  25.579283 0.0000000
A:LED-C:Halogen      -3.48 -11.29928258   4.339283 0.8591455
B:LED-C:Halogen       2.02  -5.79928258   9.839283 0.9922412
C:LED-C:Halogen       5.23  -2.58928258  13.049283 0.4323859
D:LED-C:Halogen       1.31  -6.50928258   9.129283 0.9995004
A:LED-D:Halogen     -21.24 -29.05928258 -13.420717 0.0000000
B:LED-D:Halogen     -15.74 -23.55928258  -7.920717 0.0000006
C:LED-D:Halogen     -12.53 -20.34928258  -4.710717 0.0001014
D:LED-D:Halogen     -16.45 -24.26928258  -8.630717 0.0000002
B:LED-A:LED           5.50  -2.31928258  13.319283 0.3665620
C:LED-A:LED           8.71   0.89071742  16.529283 0.0185285
D:LED-A:LED           4.79  -3.02928258  12.609283 0.5471915
C:LED-B:LED           3.21  -4.60928258  11.029283 0.9027236
D:LED-B:LED          -0.71  -8.52928258   7.109283 0.9999920
D:LED-C:LED          -3.92 -11.73928258   3.899283 0.7690762
\end{verbatim}

One conclusion from this is that the combination of D and Halogen is
significantly stronger than each of the other seven combinations.

\section{\texorpdfstring{The \texttt{bonding} model without
Interaction}{The bonding model without Interaction}}\label{the-bonding-model-without-interaction}

It seems incorrect in this situation to fit a model without the
interaction term, but we'll do so just so you can see what's involved.

\begin{Shaded}
\begin{Highlighting}[]
\NormalTok{c3_m2 <-}\StringTok{ }\KeywordTok{lm}\NormalTok{(strength }\OperatorTok{~}\StringTok{ }\NormalTok{resin }\OperatorTok{+}\StringTok{ }\NormalTok{light, }\DataTypeTok{data =}\NormalTok{ bonding)}

\KeywordTok{summary}\NormalTok{(c3_m2)}
\end{Highlighting}
\end{Shaded}

\begin{verbatim}

Call:
lm(formula = strength ~ resin + light, data = bonding)

Residuals:
     Min       1Q   Median       3Q      Max 
-14.1163  -4.9531   0.1187   4.4613  14.4663 

Coefficients:
            Estimate Std. Error t value Pr(>|t|)    
(Intercept)   19.074      1.787  10.676  < 2e-16 ***
resinB         3.815      2.260   1.688  0.09555 .  
resinC         6.740      2.260   2.982  0.00386 ** 
resinD        13.660      2.260   6.044 5.39e-08 ***
lightLED      -1.317      1.598  -0.824  0.41229    
---
Signif. codes:  0 '***' 0.001 '**' 0.01 '*' 0.05 '.' 0.1 ' ' 1

Residual standard error: 7.147 on 75 degrees of freedom
Multiple R-squared:  0.3469,    Adjusted R-squared:  0.312 
F-statistic: 9.958 on 4 and 75 DF,  p-value: 1.616e-06
\end{verbatim}

In the no-interaction model, if \texttt{light} = Halogen, our equation
is:

\[
strength = 19.07 + 3.82 resinB + 6.74 resinC + 13.66 resinD
\]

And if \texttt{light} = LED, our equation is:

\[
strength = 17.75 + 3.82 resinB + 6.74 resinC + 13.66 resinD
\]

So, in the no-interaction model, only the intercept changes.

\begin{Shaded}
\begin{Highlighting}[]
\KeywordTok{anova}\NormalTok{(c3_m2)}
\end{Highlighting}
\end{Shaded}

\begin{verbatim}
Analysis of Variance Table

Response: strength
          Df Sum Sq Mean Sq F value    Pr(>F)    
resin      3 1999.7  666.57 13.0514 6.036e-07 ***
light      1   34.7   34.72  0.6797    0.4123    
Residuals 75 3830.5   51.07                      
---
Signif. codes:  0 '***' 0.001 '**' 0.01 '*' 0.05 '.' 0.1 ' ' 1
\end{verbatim}

And, it appears, if we ignore the interaction, then \texttt{resin} type
has a significant impact on \texttt{strength} but \texttt{light} source
doesn't. This is clearer when we look at boxplots of the separated
\texttt{light} and \texttt{resin} groups.

\begin{Shaded}
\begin{Highlighting}[]
\NormalTok{p1 <-}\StringTok{ }\KeywordTok{ggplot}\NormalTok{(bonding, }\KeywordTok{aes}\NormalTok{(}\DataTypeTok{x =}\NormalTok{ light, }\DataTypeTok{y =}\NormalTok{ strength)) }\OperatorTok{+}\StringTok{ }
\StringTok{    }\KeywordTok{geom_boxplot}\NormalTok{()}
\NormalTok{p2 <-}\StringTok{ }\KeywordTok{ggplot}\NormalTok{(bonding, }\KeywordTok{aes}\NormalTok{(}\DataTypeTok{x =}\NormalTok{ resin, }\DataTypeTok{y =}\NormalTok{ strength)) }\OperatorTok{+}
\StringTok{    }\KeywordTok{geom_boxplot}\NormalTok{()}

\NormalTok{gridExtra}\OperatorTok{::}\KeywordTok{grid.arrange}\NormalTok{(p1, p2, }\DataTypeTok{nrow =} \DecValTok{1}\NormalTok{)}
\end{Highlighting}
\end{Shaded}

\includegraphics{bookdown-demo_files/figure-latex/boxplots_c3_bonding_without_interaction-1.pdf}

\section{\texorpdfstring{\texttt{cortisol}: A Hypothetical Clinical
Trial}{cortisol: A Hypothetical Clinical Trial}}\label{cortisol-a-hypothetical-clinical-trial}

156 adults who complained of problems with a high-stress lifestyle were
enrolled in a hypothetical clinical trial of the effectiveness of a
behavioral intervention designed to help reduce stress levels, as
measured by salivary cortisol.

The subjects were randomly assigned to one of three intervention groups
(usual care, low dose, and high dose.) The ``low dose'' subjects
received a one-week intervention with a follow-up at week 5. The ``high
dose'' subjects received a more intensive three-week intervention, with
follow up at week 5.

Since cortisol levels rise and fall with circadian rhythms, the cortisol
measurements were taken just after rising for all subjects. These
measurements were taken at baseline, and again at five weeks. The
difference (baseline - week 5) in cortisol level (in micrograms / l)
serves as the primary outcome.

\subsection{\texorpdfstring{Codebook and Raw Data for
\texttt{cortisol}}{Codebook and Raw Data for cortisol}}\label{codebook-and-raw-data-for-cortisol}

The data are gathered in the \texttt{cortisol} data set. Included are:

\begin{longtable}[]{@{}rl@{}}
\toprule
Variable & Description\tabularnewline
\midrule
\endhead
\texttt{subject} & subject identification code\tabularnewline
\texttt{interv} & intervention group (UC = usual care, Low,
High)\tabularnewline
\texttt{waist} & waist circumference at baseline (in
inches)\tabularnewline
\texttt{sex} & male or female\tabularnewline
\texttt{cort.1} & salivary cortisol level (microg/l) week
1\tabularnewline
\texttt{cort.5} & salivary cortisol level (microg/l) week
5\tabularnewline
\bottomrule
\end{longtable}

\begin{Shaded}
\begin{Highlighting}[]
\NormalTok{cortisol}
\end{Highlighting}
\end{Shaded}

\begin{verbatim}
# A tibble: 156 x 6
   subject interv waist sex   cort.1 cort.5
     <int> <fct>  <dbl> <fct>  <dbl>  <dbl>
 1    1001 UC      48.3 M      13.4   13.3 
 2    1002 Low     58.3 M      17.8   16.6 
 3    1003 High    43.0 M      14.4   12.7 
 4    1004 Low     44.9 M       9.00   9.80
 5    1005 High    46.1 M      14.2   14.2 
 6    1006 UC      41.3 M      14.8   15.1 
 7    1007 Low     51.0 F      13.7   16.0 
 8    1008 UC      42.0 F      17.3   18.7 
 9    1009 Low     24.7 F      15.3   15.8 
10    1010 Low     59.4 M      12.4   11.7 
# ... with 146 more rows
\end{verbatim}

\section{Creating a factor combining sex and
waist}\label{creating-a-factor-combining-sex-and-waist}

Next, we'll put the \texttt{waist} and \texttt{sex} data in the
\texttt{cortisol} example together. We want to build a second
categorical variable (called \texttt{fat\_est}) combining this
information, to indicate ``healthy'' vs. ``unhealthy'' levels of fat
around the waist.

\begin{itemize}
\tightlist
\item
  Male subjects whose waist circumference is 40 inches or more, and
\item
  Female subjects whose waist circumference is 35 inches or more, will
  fall in the ``unhealthy'' group.
\end{itemize}

\begin{Shaded}
\begin{Highlighting}[]
\NormalTok{cortisol <-}\StringTok{ }\NormalTok{cortisol }\OperatorTok
\StringTok{    }\KeywordTok{mutate}\NormalTok{(}
        \DataTypeTok{fat_est =} \KeywordTok{factor}\NormalTok{(}\KeywordTok{case_when}\NormalTok{(}
\NormalTok{            sex }\OperatorTok{==}\StringTok{ "M"} \OperatorTok{&}\StringTok{ }\NormalTok{waist }\OperatorTok{>=}\StringTok{ }\DecValTok{40} \OperatorTok{~}\StringTok{ "unhealthy"}\NormalTok{,}
\NormalTok{            sex }\OperatorTok{==}\StringTok{ "F"} \OperatorTok{&}\StringTok{ }\NormalTok{waist }\OperatorTok{>=}\StringTok{ }\DecValTok{35} \OperatorTok{~}\StringTok{ "unhealthy"}\NormalTok{,}
            \OtherTok{TRUE}                     \OperatorTok{~}\StringTok{ "healthy"}\NormalTok{)),}
        \DataTypeTok{cort_diff =}\NormalTok{ cort.}\DecValTok{1} \OperatorTok{-}\StringTok{ }\NormalTok{cort.}\DecValTok{5}\NormalTok{)}

\KeywordTok{summary}\NormalTok{(cortisol)}
\end{Highlighting}
\end{Shaded}

\begin{verbatim}
    subject      interv       waist       sex        cort.1      
 Min.   :1001   High:53   Min.   :20.80   F:83   Min.   : 6.000  
 1st Qu.:1040   Low :52   1st Qu.:33.27   M:73   1st Qu.: 9.675  
 Median :1078   UC  :51   Median :40.35          Median :12.400  
 Mean   :1078             Mean   :40.42          Mean   :12.686  
 3rd Qu.:1117             3rd Qu.:47.77          3rd Qu.:16.025  
 Max.   :1156             Max.   :59.90          Max.   :19.000  
     cort.5          fat_est      cort_diff      
 Min.   : 4.2   healthy  : 56   Min.   :-2.3000  
 1st Qu.: 9.6   unhealthy:100   1st Qu.:-0.5000  
 Median :12.6                   Median : 0.2000  
 Mean   :12.4                   Mean   : 0.2821  
 3rd Qu.:15.7                   3rd Qu.: 1.2000  
 Max.   :19.7                   Max.   : 2.0000  
\end{verbatim}

\section{\texorpdfstring{A Means Plot for the \texttt{cortisol} trial
(with standard
errors)}{A Means Plot for the cortisol trial (with standard errors)}}\label{a-means-plot-for-the-cortisol-trial-with-standard-errors}

Again, we'll start by building up a data set with the summaries we want
to plot.

\begin{Shaded}
\begin{Highlighting}[]
\NormalTok{cort.sum <-}\StringTok{ }\NormalTok{cortisol }\OperatorTok\StringTok{ }
\StringTok{    }\KeywordTok{group_by}\NormalTok{(interv, fat_est) }\OperatorTok
\StringTok{    }\KeywordTok{summarize}\NormalTok{(}\DataTypeTok{mean.cort =} \KeywordTok{mean}\NormalTok{(cort_diff), }
              \DataTypeTok{se.cort =} \KeywordTok{sd}\NormalTok{(cort_diff)}\OperatorTok{/}\KeywordTok{sqrt}\NormalTok{(}\KeywordTok{n}\NormalTok{()))}

\NormalTok{cort.sum}
\end{Highlighting}
\end{Shaded}

\begin{verbatim}
# A tibble: 6 x 4
# Groups:   interv [?]
  interv fat_est   mean.cort se.cort
  <fct>  <fct>         <dbl>   <dbl>
1 High   healthy       0.695   0.217
2 High   unhealthy     0.352   0.150
3 Low    healthy       0.500   0.182
4 Low    unhealthy     0.327   0.190
5 UC     healthy       0.347   0.225
6 UC     unhealthy    -0.226   0.155
\end{verbatim}

Now, we'll use this new data set to plot the means and standard errors.

\begin{Shaded}
\begin{Highlighting}[]
\NormalTok{## The error bars will overlap unless we adjust the position.}
\NormalTok{pd <-}\StringTok{ }\KeywordTok{position_dodge}\NormalTok{(}\FloatTok{0.2}\NormalTok{) }\CommentTok{# move them .1 to the left and right}

\KeywordTok{ggplot}\NormalTok{(cort.sum, }\KeywordTok{aes}\NormalTok{(}\DataTypeTok{x =}\NormalTok{ interv, }\DataTypeTok{y =}\NormalTok{ mean.cort, }\DataTypeTok{col =}\NormalTok{ fat_est)) }\OperatorTok{+}
\StringTok{    }\KeywordTok{geom_errorbar}\NormalTok{(}\KeywordTok{aes}\NormalTok{(}\DataTypeTok{ymin =}\NormalTok{ mean.cort }\OperatorTok{-}\StringTok{ }\NormalTok{se.cort, }
                      \DataTypeTok{ymax =}\NormalTok{ mean.cort }\OperatorTok{+}\StringTok{ }\NormalTok{se.cort),}
                  \DataTypeTok{width =} \FloatTok{0.2}\NormalTok{, }\DataTypeTok{position =}\NormalTok{ pd) }\OperatorTok{+}
\StringTok{    }\KeywordTok{geom_point}\NormalTok{(}\DataTypeTok{size =} \DecValTok{2}\NormalTok{, }\DataTypeTok{position =}\NormalTok{ pd) }\OperatorTok{+}\StringTok{ }
\StringTok{    }\KeywordTok{geom_line}\NormalTok{(}\KeywordTok{aes}\NormalTok{(}\DataTypeTok{group =}\NormalTok{ fat_est), }\DataTypeTok{position =}\NormalTok{ pd) }\OperatorTok{+}
\StringTok{    }\KeywordTok{scale_color_manual}\NormalTok{(}\DataTypeTok{values =} \KeywordTok{c}\NormalTok{(}\StringTok{"royalblue"}\NormalTok{, }\StringTok{"darkred"}\NormalTok{)) }\OperatorTok{+}
\StringTok{    }\KeywordTok{theme_bw}\NormalTok{() }\OperatorTok{+}
\StringTok{    }\KeywordTok{labs}\NormalTok{(}\DataTypeTok{y =} \StringTok{"Salivary Cortisol Level"}\NormalTok{, }\DataTypeTok{x =} \StringTok{"Intervention Group"}\NormalTok{,}
         \DataTypeTok{title =} \StringTok{"Observed Means (+/- SE) of Salivary Cortisol"}\NormalTok{)}
\end{Highlighting}
\end{Shaded}

\includegraphics{bookdown-demo_files/figure-latex/c3_ggplot_means_plot_cortisol-1.pdf}

\section{\texorpdfstring{A Two-Way ANOVA model for \texttt{cortisol}
with
Interaction}{A Two-Way ANOVA model for cortisol with Interaction}}\label{a-two-way-anova-model-for-cortisol-with-interaction}

\begin{Shaded}
\begin{Highlighting}[]
\NormalTok{c3_m3 <-}\StringTok{ }\KeywordTok{lm}\NormalTok{(cort_diff }\OperatorTok{~}\StringTok{ }\NormalTok{interv }\OperatorTok{*}\StringTok{ }\NormalTok{fat_est, }\DataTypeTok{data =}\NormalTok{ cortisol)}

\KeywordTok{anova}\NormalTok{(c3_m3)}
\end{Highlighting}
\end{Shaded}

\begin{verbatim}
Analysis of Variance Table

Response: cort_diff
                Df  Sum Sq Mean Sq F value  Pr(>F)  
interv           2   7.847  3.9235  4.4698 0.01301 *
fat_est          1   4.614  4.6139  5.2564 0.02326 *
interv:fat_est   2   0.943  0.4715  0.5371 0.58554  
Residuals      150 131.666  0.8778                  
---
Signif. codes:  0 '***' 0.001 '**' 0.01 '*' 0.05 '.' 0.1 ' ' 1
\end{verbatim}

Does it seem like we need the interaction term in this case?

\begin{Shaded}
\begin{Highlighting}[]
\KeywordTok{summary}\NormalTok{(c3_m3)}
\end{Highlighting}
\end{Shaded}

\begin{verbatim}

Call:
lm(formula = cort_diff ~ interv * fat_est, data = cortisol)

Residuals:
     Min       1Q   Median       3Q      Max 
-2.62727 -0.75702  0.08636  0.84848  2.12647 

Coefficients:
                           Estimate Std. Error t value Pr(>|t|)   
(Intercept)                  0.6950     0.2095   3.317  0.00114 **
intervLow                   -0.1950     0.3001  -0.650  0.51689   
intervUC                    -0.3479     0.3091  -1.126  0.26206   
fat_estunhealthy            -0.3435     0.2655  -1.294  0.19774   
intervLow:fat_estunhealthy   0.1708     0.3785   0.451  0.65256   
intervUC:fat_estunhealthy   -0.2300     0.3846  -0.598  0.55068   
---
Signif. codes:  0 '***' 0.001 '**' 0.01 '*' 0.05 '.' 0.1 ' ' 1

Residual standard error: 0.9369 on 150 degrees of freedom
Multiple R-squared:  0.0924,    Adjusted R-squared:  0.06214 
F-statistic: 3.054 on 5 and 150 DF,  p-value: 0.01179
\end{verbatim}

How do you reconcile the apparent difference in significance levels
between this regression summary and the ANOVA table above?

\section{\texorpdfstring{A Two-Way ANOVA model for \texttt{cortisol}
without
Interaction}{A Two-Way ANOVA model for cortisol without Interaction}}\label{a-two-way-anova-model-for-cortisol-without-interaction}

\subsection{The Graph}\label{the-graph}

\begin{Shaded}
\begin{Highlighting}[]
\NormalTok{p1 <-}\StringTok{ }\KeywordTok{ggplot}\NormalTok{(cortisol, }\KeywordTok{aes}\NormalTok{(}\DataTypeTok{x =}\NormalTok{ interv, }\DataTypeTok{y =}\NormalTok{ cort_diff)) }\OperatorTok{+}\StringTok{ }
\StringTok{    }\KeywordTok{geom_boxplot}\NormalTok{()}
\NormalTok{p2 <-}\StringTok{ }\KeywordTok{ggplot}\NormalTok{(cortisol, }\KeywordTok{aes}\NormalTok{(}\DataTypeTok{x =}\NormalTok{ fat_est, }\DataTypeTok{y =}\NormalTok{ cort_diff)) }\OperatorTok{+}
\StringTok{    }\KeywordTok{geom_boxplot}\NormalTok{()}

\NormalTok{gridExtra}\OperatorTok{::}\KeywordTok{grid.arrange}\NormalTok{(p1, p2, }\DataTypeTok{nrow =} \DecValTok{1}\NormalTok{)}
\end{Highlighting}
\end{Shaded}

\includegraphics{bookdown-demo_files/figure-latex/boxplots_c3_cortisol_without_interaction-1.pdf}

\subsection{The ANOVA Model}\label{the-anova-model}

\begin{Shaded}
\begin{Highlighting}[]
\NormalTok{c3_m4 <-}\StringTok{ }\KeywordTok{lm}\NormalTok{(cort_diff }\OperatorTok{~}\StringTok{ }\NormalTok{interv }\OperatorTok{+}\StringTok{ }\NormalTok{fat_est, }\DataTypeTok{data =}\NormalTok{ cortisol)}

\KeywordTok{anova}\NormalTok{(c3_m4)}
\end{Highlighting}
\end{Shaded}

\begin{verbatim}
Analysis of Variance Table

Response: cort_diff
           Df  Sum Sq Mean Sq F value  Pr(>F)  
interv      2   7.847  3.9235  4.4972 0.01266 *
fat_est     1   4.614  4.6139  5.2886 0.02283 *
Residuals 152 132.609  0.8724                  
---
Signif. codes:  0 '***' 0.001 '**' 0.01 '*' 0.05 '.' 0.1 ' ' 1
\end{verbatim}

How do these results compare to those we saw in the model with
interaction?

\subsection{The Regression Summary}\label{the-regression-summary}

\begin{Shaded}
\begin{Highlighting}[]
\KeywordTok{summary}\NormalTok{(c3_m4)}
\end{Highlighting}
\end{Shaded}

\begin{verbatim}

Call:
lm(formula = cort_diff ~ interv + fat_est, data = cortisol)

Residuals:
     Min       1Q   Median       3Q      Max 
-2.55929 -0.74527  0.05457  0.86456  2.05489 

Coefficients:
                 Estimate Std. Error t value Pr(>|t|)    
(Intercept)       0.70452    0.16093   4.378 2.22e-05 ***
intervLow        -0.08645    0.18232  -0.474  0.63606    
intervUC         -0.50063    0.18334  -2.731  0.00707 ** 
fat_estunhealthy -0.35878    0.15601  -2.300  0.02283 *  
---
Signif. codes:  0 '***' 0.001 '**' 0.01 '*' 0.05 '.' 0.1 ' ' 1

Residual standard error: 0.934 on 152 degrees of freedom
Multiple R-squared:  0.0859,    Adjusted R-squared:  0.06785 
F-statistic: 4.761 on 3 and 152 DF,  p-value: 0.00335
\end{verbatim}

\subsection{Tukey HSD Comparisons}\label{tukey-hsd-comparisons}

Without the interaction term, we can make direct comparisons between
levels of the intervention, and between levels of the \texttt{fat\_est}
variable. This is probably best done here in a Tukey HSD comparison.

\begin{Shaded}
\begin{Highlighting}[]
\KeywordTok{TukeyHSD}\NormalTok{(}\KeywordTok{aov}\NormalTok{(cort_diff }\OperatorTok{~}\StringTok{ }\NormalTok{interv }\OperatorTok{+}\StringTok{ }\NormalTok{fat_est, }\DataTypeTok{data =}\NormalTok{ cortisol))}
\end{Highlighting}
\end{Shaded}

\begin{verbatim}
  Tukey multiple comparisons of means
    95% family-wise confidence level

Fit: aov(formula = cort_diff ~ interv + fat_est, data = cortisol)

$interv
                diff        lwr         upr     p adj
Low-High -0.09074746 -0.5222655  0.34077063 0.8724916
UC-High  -0.51642619 -0.9500745 -0.08277793 0.0150150
UC-Low   -0.42567873 -0.8613670  0.01000948 0.0570728

$fat_est
                        diff        lwr         upr     p adj
unhealthy-healthy -0.3582443 -0.6662455 -0.05024305 0.0229266
\end{verbatim}

What conclusions can we draw, at a 5\% significance level?

\chapter{Analysis of Covariance}\label{analysis-of-covariance}

\section{An Emphysema Study}\label{an-emphysema-study}

My source for this example is \citet{Riffenburgh2006}, section 18.4.
Serum theophylline levels (in mg/dl) were measured in 16 patients with
emphysema at baseline, then 5 days later (at the end of a course of
antibiotics) and then at 10 days after baseline. Clinicians anticipate
that the antibiotic will increase the theophylline level. The data are
stored in the \texttt{emphysema.csv} data file, and note that the age
for patient 5 is not available.

\subsection{Codebook}\label{codebook}

\begin{longtable}[]{@{}rl@{}}
\toprule
Variable & Description\tabularnewline
\midrule
\endhead
\texttt{patient} & ID code\tabularnewline
\texttt{age} & patient's age in years\tabularnewline
\texttt{sex} & patient's sex (F or M)\tabularnewline
\texttt{st\_base} & patient's serum theophylline at baseline
(mg/dl)\tabularnewline
\texttt{st\_day5} & patient's serum theophylline at day 5
(mg/dl)\tabularnewline
\texttt{st\_day10} & patient's serum theophylline at day 10
(mg/dl)\tabularnewline
\bottomrule
\end{longtable}

We're going to look at the change from baseline to day 5 as our outcome
of interest, since the clinical expectation is that the antibiotic
(azithromycin) will increase theophylline levels.

\begin{Shaded}
\begin{Highlighting}[]
\NormalTok{emphysema <-}\StringTok{ }\NormalTok{emphysema }\OperatorTok\StringTok{ }
\StringTok{    }\KeywordTok{mutate}\NormalTok{(}\DataTypeTok{st_delta =}\NormalTok{ st_day5 }\OperatorTok{-}\StringTok{ }\NormalTok{st_base)}

\NormalTok{emphysema}
\end{Highlighting}
\end{Shaded}

\begin{verbatim}
# A tibble: 16 x 7
   patient   age sex   st_base st_day5 st_day10 st_delta
     <int> <int> <fct>   <dbl>   <dbl>    <dbl>    <dbl>
 1       1    61 F       14.1     2.30    10.3   -11.8  
 2       2    70 F        7.20    5.40     7.30  - 1.80 
 3       3    65 M       14.2    11.9     11.3   - 2.30 
 4       4    65 M       10.3    10.7     13.8     0.400
 5       5    NA M        9.90   10.7     11.7     0.800
 6       6    76 M        5.20    6.80     4.20    1.60 
 7       7    72 M       10.4    14.6     14.1     4.20 
 8       8    69 F       10.5     7.20     5.40  - 3.30 
 9       9    66 M        5.00    5.00     5.10    0    
10      10    62 M        8.60    8.10     7.40  - 0.500
11      11    65 F       16.6    14.9     13.0   - 1.70 
12      12    71 M       16.4    18.6     17.1     2.20 
13      13    51 F       12.2    11.0     12.3   - 1.20 
14      14    71 M        6.60    3.70     4.50  - 2.90 
15      15    64 F       15.4    15.2     13.6   - 0.200
16      16    50 M       10.2    10.8     11.2     0.600
\end{verbatim}

\section{\texorpdfstring{Does \texttt{sex} affect the mean change in
theophylline?}{Does sex affect the mean change in theophylline?}}\label{does-sex-affect-the-mean-change-in-theophylline}

\begin{Shaded}
\begin{Highlighting}[]
\NormalTok{emphysema }\OperatorTok\StringTok{ }\KeywordTok{skim}\NormalTok{(st_delta)}
\end{Highlighting}
\end{Shaded}

\begin{verbatim}
Skim summary statistics
 n obs: 16 
 n variables: 7 

Variable type: numeric 
 variable missing complete  n  mean   sd    p0   p25 median  p75 p100
 st_delta       0       16 16 -0.99 3.48 -11.8 -1.92  -0.35 0.65  4.2
\end{verbatim}

\begin{Shaded}
\begin{Highlighting}[]
\NormalTok{emphysema }\OperatorTok\StringTok{ }\KeywordTok{group_by}\NormalTok{(sex) }\OperatorTok\StringTok{ }\KeywordTok{skim}\NormalTok{(st_delta)}
\end{Highlighting}
\end{Shaded}

\begin{verbatim}
Skim summary statistics
 n obs: 16 
 n variables: 7 
 group variables: sex 

Variable type: numeric 
 sex variable missing complete  n  mean   sd    p0   p25 median   p75 p100
   F st_delta       0        6  6 -3.33 4.27 -11.8 -2.92  -1.75 -1.32 -0.2
   M st_delta       0       10 10  0.41 2.07  -2.9 -0.38   0.5   1.4   4.2
\end{verbatim}

Overall, the mean change in theophylline during the course of the
antibiotic is -0.99, but this is -3.33 for female patients and 0.41 for
male patients.

A one-way ANOVA model looks like this:

\begin{Shaded}
\begin{Highlighting}[]
\KeywordTok{anova}\NormalTok{(}\KeywordTok{lm}\NormalTok{(st_delta }\OperatorTok{~}\StringTok{ }\NormalTok{sex, }\DataTypeTok{data =}\NormalTok{ emphysema))}
\end{Highlighting}
\end{Shaded}

\begin{verbatim}
Analysis of Variance Table

Response: st_delta
          Df  Sum Sq Mean Sq F value  Pr(>F)  
sex        1  52.547  52.547  5.6789 0.03189 *
Residuals 14 129.542   9.253                  
---
Signif. codes:  0 '***' 0.001 '**' 0.01 '*' 0.05 '.' 0.1 ' ' 1
\end{verbatim}

The ANOVA F test finds a statistically significant difference between
the mean \texttt{st\_delta} among males and the mean \texttt{st\_delta}
among females. But is there more to the story?

\section{\texorpdfstring{Is there an association between \texttt{age}
and \texttt{sex} in this
study?}{Is there an association between age and sex in this study?}}\label{is-there-an-association-between-age-and-sex-in-this-study}

\begin{Shaded}
\begin{Highlighting}[]
\NormalTok{emphysema }\OperatorTok\StringTok{ }\KeywordTok{group_by}\NormalTok{(sex) }\OperatorTok\StringTok{ }\KeywordTok{skim}\NormalTok{(age)}
\end{Highlighting}
\end{Shaded}

\begin{verbatim}
Skim summary statistics
 n obs: 16 
 n variables: 7 
 group variables: sex 

Variable type: integer 
 sex variable missing complete  n  mean   sd p0   p25 median p75 p100
   F      age       0        6  6 63.33 6.89 51 61.75   64.5  68   70
   M      age       1        9 10 66.44 7.57 50 65      66    71   76
\end{verbatim}

But we note that the male patients are also older than the female
patients, on average (mean age for males is 66.4, for females 63.3)

\begin{itemize}
\tightlist
\item
  Does the fact that male patients are older affect change in
  theophylline level?
\item
  And how should we deal with the one missing \texttt{age} value (in a
  male patient)?
\end{itemize}

\section{\texorpdfstring{Adding a quantitative covariate, \texttt{age},
to the
model}{Adding a quantitative covariate, age, to the model}}\label{adding-a-quantitative-covariate-age-to-the-model}

We could fit an ANOVA model to predict \texttt{st\_delta} using
\texttt{sex} and \texttt{age} directly, but only if we categorized
\texttt{age} into two or more groups. Because \texttt{age} is not
categorical, we cannot include it in an ANOVA. But if age is an
influence, and we don't adjust for it, it may well bias the outcome of
our initial ANOVA. With a quantitative variable like \texttt{age}, we
will need a method called ANCOVA, for \textbf{analysis of covariance}.

\subsection{The ANCOVA model}\label{the-ancova-model}

ANCOVA in this case is just an ANOVA model with our outcome
(\texttt{st\_delta}) adjusted for a continuous covariate, called
\texttt{age}. For the moment, we'll ignore the one subject with missing
\texttt{age} and simply fit the regression model with \texttt{sex} and
\texttt{age}.

\begin{Shaded}
\begin{Highlighting}[]
\KeywordTok{summary}\NormalTok{(}\KeywordTok{lm}\NormalTok{(st_delta }\OperatorTok{~}\StringTok{ }\NormalTok{sex }\OperatorTok{+}\StringTok{ }\NormalTok{age, }\DataTypeTok{data =}\NormalTok{ emphysema))}
\end{Highlighting}
\end{Shaded}

\begin{verbatim}

Call:
lm(formula = st_delta ~ sex + age, data = emphysema)

Residuals:
    Min      1Q  Median      3Q     Max 
-8.3352 -0.4789  0.6948  1.5580  3.5202 

Coefficients:
            Estimate Std. Error t value Pr(>|t|)  
(Intercept) -6.90266    7.92948  -0.871   0.4011  
sexM         3.52466    1.75815   2.005   0.0681 .
age          0.05636    0.12343   0.457   0.6561  
---
Signif. codes:  0 '***' 0.001 '**' 0.01 '*' 0.05 '.' 0.1 ' ' 1

Residual standard error: 3.255 on 12 degrees of freedom
  (1 observation deleted due to missingness)
Multiple R-squared:  0.2882,    Adjusted R-squared:  0.1696 
F-statistic:  2.43 on 2 and 12 DF,  p-value: 0.13
\end{verbatim}

This model assumes that the slope of the regression line between
\texttt{st\_delta} and \texttt{age} is the same for both sexes.

Note that the model yields \texttt{st\_delta} = -6.9 + 3.52
(\texttt{sex} = male) + 0.056 \texttt{age}, or

\begin{itemize}
\tightlist
\item
  \texttt{st\_delta} = -6.9 + 0.056 \texttt{age} for female patients,
  and
\item
  \texttt{st\_delta} = (-6.9 + 3.52) + 0.056 \texttt{age} = -3.38 +
  0.056 \texttt{age} for male patients.
\end{itemize}

Note that we can test this assumption of equal slopes by fitting an
alternative model (with a product term between \texttt{sex} and
\texttt{age}) that doesn't require the assumption, and we'll do that
later.

\subsection{The ANCOVA Table}\label{the-ancova-table}

First, though, we'll look at the ANCOVA table.

\begin{Shaded}
\begin{Highlighting}[]
\KeywordTok{anova}\NormalTok{(}\KeywordTok{lm}\NormalTok{(st_delta }\OperatorTok{~}\StringTok{ }\NormalTok{sex }\OperatorTok{+}\StringTok{ }\NormalTok{age, }\DataTypeTok{data =}\NormalTok{ emphysema))}
\end{Highlighting}
\end{Shaded}

\begin{verbatim}
Analysis of Variance Table

Response: st_delta
          Df  Sum Sq Mean Sq F value  Pr(>F)  
sex        1  49.284  49.284  4.6507 0.05203 .
age        1   2.209   2.209  0.2085 0.65612  
Residuals 12 127.164  10.597                  
---
Signif. codes:  0 '***' 0.001 '**' 0.01 '*' 0.05 '.' 0.1 ' ' 1
\end{verbatim}

When we tested \texttt{sex} without accounting for \texttt{age}, we
found a \emph{p} value of 0.032, which is less than our usual cutpoint
of 0.05. But when we adjusted for \texttt{age}, we find that
\texttt{sex} loses significance, even though \texttt{age} is not a
significant influence on \texttt{st\_delta} by itself, according to the
ANCOVA table.

\section{Rerunning the ANCOVA model after simple
imputation}\label{rerunning-the-ancova-model-after-simple-imputation}

We could have \emph{imputed} the missing \texttt{age} value for patient
5, rather than just deleting that patient. Suppose we do the simplest
potentially reasonable thing to do: insert the mean \texttt{age} in
where the NA value currently exists.

\begin{Shaded}
\begin{Highlighting}[]
\NormalTok{emph_imp <-}\StringTok{ }\KeywordTok{replace_na}\NormalTok{(emphysema, }\KeywordTok{list}\NormalTok{(}\DataTypeTok{age =} \KeywordTok{mean}\NormalTok{(emphysema}\OperatorTok{$}\NormalTok{age, }\DataTypeTok{na.rm =} \OtherTok{TRUE}\NormalTok{)))}

\NormalTok{emph_imp}
\end{Highlighting}
\end{Shaded}

\begin{verbatim}
# A tibble: 16 x 7
   patient   age sex   st_base st_day5 st_day10 st_delta
     <int> <dbl> <fct>   <dbl>   <dbl>    <dbl>    <dbl>
 1       1  61.0 F       14.1     2.30    10.3   -11.8  
 2       2  70.0 F        7.20    5.40     7.30  - 1.80 
 3       3  65.0 M       14.2    11.9     11.3   - 2.30 
 4       4  65.0 M       10.3    10.7     13.8     0.400
 5       5  65.2 M        9.90   10.7     11.7     0.800
 6       6  76.0 M        5.20    6.80     4.20    1.60 
 7       7  72.0 M       10.4    14.6     14.1     4.20 
 8       8  69.0 F       10.5     7.20     5.40  - 3.30 
 9       9  66.0 M        5.00    5.00     5.10    0    
10      10  62.0 M        8.60    8.10     7.40  - 0.500
11      11  65.0 F       16.6    14.9     13.0   - 1.70 
12      12  71.0 M       16.4    18.6     17.1     2.20 
13      13  51.0 F       12.2    11.0     12.3   - 1.20 
14      14  71.0 M        6.60    3.70     4.50  - 2.90 
15      15  64.0 F       15.4    15.2     13.6   - 0.200
16      16  50.0 M       10.2    10.8     11.2     0.600
\end{verbatim}

More on simple imputation and missing data is coming soon.

For now, we can rerun the ANCOVA model on this new data set, after
imputation\ldots{}

\begin{Shaded}
\begin{Highlighting}[]
\KeywordTok{anova}\NormalTok{(}\KeywordTok{lm}\NormalTok{(st_delta }\OperatorTok{~}\StringTok{ }\NormalTok{sex }\OperatorTok{+}\StringTok{ }\NormalTok{age, }\DataTypeTok{data =}\NormalTok{ emph_imp))}
\end{Highlighting}
\end{Shaded}

\begin{verbatim}
Analysis of Variance Table

Response: st_delta
          Df  Sum Sq Mean Sq F value  Pr(>F)  
sex        1  52.547  52.547  5.3623 0.03755 *
age        1   2.151   2.151  0.2195 0.64721  
Residuals 13 127.392   9.799                  
---
Signif. codes:  0 '***' 0.001 '**' 0.01 '*' 0.05 '.' 0.1 ' ' 1
\end{verbatim}

When we do this, we see that now the \texttt{sex} variable returns to a
\emph{p} value below 0.05. Our complete case analysis (which omitted
patient 5) gives us a different result than the ANCOVA based on the data
after mean imputation.

\section{Looking at a factor-covariate
interaction}\label{looking-at-a-factor-covariate-interaction}

Let's run a model including the interaction (product) term between
\texttt{age} and \texttt{sex}, which implies that the slope of
\texttt{age} on our outcome (\texttt{st\_delta}) depends on the
patient's sex. We'll use the imputed data again. Here is the new ANCOVA
table, which suggests that the interaction of \texttt{age} and
\texttt{sex} is small (because it accounts for only a small amount of
the total Sum of Squares) and not significant (p = 0.91).

\begin{Shaded}
\begin{Highlighting}[]
\KeywordTok{anova}\NormalTok{(}\KeywordTok{lm}\NormalTok{(st_delta }\OperatorTok{~}\StringTok{ }\NormalTok{sex }\OperatorTok{*}\StringTok{ }\NormalTok{age, }\DataTypeTok{data =}\NormalTok{ emph_imp))}
\end{Highlighting}
\end{Shaded}

\begin{verbatim}
Analysis of Variance Table

Response: st_delta
          Df  Sum Sq Mean Sq F value  Pr(>F)  
sex        1  52.547  52.547  4.9549 0.04594 *
age        1   2.151   2.151  0.2028 0.66051  
sex:age    1   0.130   0.130  0.0123 0.91355  
Residuals 12 127.261  10.605                  
---
Signif. codes:  0 '***' 0.001 '**' 0.01 '*' 0.05 '.' 0.1 ' ' 1
\end{verbatim}

Since the interaction term is neither substantial nor significant, we
probably don't need it here. But let's look at its interpretation
anyway, just to fix ideas. To do that, we'll need the coefficients from
the underlying regression model.

\begin{Shaded}
\begin{Highlighting}[]
\KeywordTok{tidy}\NormalTok{(}\KeywordTok{lm}\NormalTok{(st_delta }\OperatorTok{~}\StringTok{ }\NormalTok{sex }\OperatorTok{*}\StringTok{ }\NormalTok{age, }\DataTypeTok{data =}\NormalTok{ emph_imp))}
\end{Highlighting}
\end{Shaded}

\begin{verbatim}
         term    estimate  std.error  statistic   p.value
1 (Intercept) -5.64606742 13.4536974 -0.4196666 0.6821446
2        sexM  1.72031026 16.8389209  0.1021627 0.9203148
3         age  0.03651685  0.2113871  0.1727488 0.8657284
4    sexM:age  0.02885946  0.2603044  0.1108681 0.9135536
\end{verbatim}

Our ANCOVA model for \texttt{st\_delta} incorporating the \texttt{age} x
\texttt{sex} product term is -5.65 + 1.72 (sex = M) + 0.037 age + 0.029
(sex = M)(age). So that means:

\begin{itemize}
\tightlist
\item
  our model for females is \texttt{st\_delta} = -5.65 + 0.037
  \texttt{age}
\item
  our model for males is \texttt{st\_delta} = (-5.65 + 1.72) + (0.037 +
  0.029) \texttt{age}, or -3.93 + 0.066 \texttt{age}
\end{itemize}

but, again, our conclusion from the ANCOVA table is that this increase
in complexity (letting both the slope and intercept vary by
\texttt{sex}) doesn't add much in the way of predictive value for our
\texttt{st\_delta} outcome.

\section{Centering the Covariate to Facilitate ANCOVA
Interpretation}\label{centering-the-covariate-to-facilitate-ancova-interpretation}

When developing an ANCOVA model, we will often \textbf{center} or even
\textbf{center and rescale} the covariate to facilitate interpretation
of the product term. In this case, let's center \texttt{age} and rescale
it by dividing by two standard deviations.

\begin{Shaded}
\begin{Highlighting}[]
\NormalTok{emph_imp }\OperatorTok\StringTok{ }\KeywordTok{skim}\NormalTok{(age)}
\end{Highlighting}
\end{Shaded}

\begin{verbatim}
Skim summary statistics
 n obs: 16 
 n variables: 7 

Variable type: numeric 
 variable missing complete  n mean   sd p0  p25 median   p75 p100
      age       0       16 16 65.2 6.98 50 63.5   65.1 70.25   76
\end{verbatim}

Note that in our imputed data, the mean \texttt{age} is 65.2 and the
standard deviation of \texttt{age} is 7 years.

So we build the rescaled \texttt{age} variable that I'll call
\texttt{age\_z}, and then use it to refit our model.

\begin{Shaded}
\begin{Highlighting}[]
\NormalTok{emph_imp <-}\StringTok{ }\NormalTok{emph_imp }\OperatorTok
\StringTok{    }\KeywordTok{mutate}\NormalTok{(}\DataTypeTok{age_z =}\NormalTok{ (age }\OperatorTok{-}\StringTok{ }\KeywordTok{mean}\NormalTok{(age))}\OperatorTok{/}\StringTok{ }\NormalTok{(}\DecValTok{2} \OperatorTok{*}\StringTok{ }\KeywordTok{sd}\NormalTok{(age)))}

\KeywordTok{anova}\NormalTok{(}\KeywordTok{lm}\NormalTok{(st_delta }\OperatorTok{~}\StringTok{ }\NormalTok{sex }\OperatorTok{*}\StringTok{ }\NormalTok{age_z, }\DataTypeTok{data =}\NormalTok{ emph_imp))}
\end{Highlighting}
\end{Shaded}

\begin{verbatim}
Analysis of Variance Table

Response: st_delta
          Df  Sum Sq Mean Sq F value  Pr(>F)  
sex        1  52.547  52.547  4.9549 0.04594 *
age_z      1   2.151   2.151  0.2028 0.66051  
sex:age_z  1   0.130   0.130  0.0123 0.91355  
Residuals 12 127.261  10.605                  
---
Signif. codes:  0 '***' 0.001 '**' 0.01 '*' 0.05 '.' 0.1 ' ' 1
\end{verbatim}

\begin{Shaded}
\begin{Highlighting}[]
\KeywordTok{tidy}\NormalTok{(}\KeywordTok{lm}\NormalTok{(st_delta }\OperatorTok{~}\StringTok{ }\NormalTok{sex }\OperatorTok{*}\StringTok{ }\NormalTok{age_z, }\DataTypeTok{data =}\NormalTok{ emph_imp))}
\end{Highlighting}
\end{Shaded}

\begin{verbatim}
         term   estimate std.error  statistic    p.value
1 (Intercept) -3.2651685  1.386802 -2.3544587 0.03641637
2        sexM  3.6019471  1.735706  2.0752055 0.06013138
3       age_z  0.5096337  2.950144  0.1727488 0.86572835
4  sexM:age_z  0.4027661  3.632839  0.1108681 0.91355364
\end{verbatim}

Comparing the two models, we have:

\begin{itemize}
\tightlist
\item
  (unscaled): \texttt{st\_delta} = -5.65 + 1.72 (\texttt{sex} = M) +
  0.037 \texttt{age} + 0.029 (\texttt{sex} = M) x (\texttt{age})
\item
  (rescaled): \texttt{st\_delta} = -3.27 + 3.60 (\texttt{sex} = M) +
  0.510 rescaled \texttt{age\_z} + 0.402 (\texttt{sex} = M) x (rescaled
  \texttt{age\_z})
\end{itemize}

In essence, the rescaled model on \texttt{age\_z} is:

\begin{itemize}
\tightlist
\item
  \texttt{st\_delta} = -3.27 + 0.510 \texttt{age\_z} for female
  subjects, and
\item
  \texttt{st\_delta} = (-3.27 + 3.60) + (0.510 + 0.402) \texttt{age\_z}
  = 0.33 + 0.912 \texttt{age\_z} for male subjects
\end{itemize}

Interpreting the centered, rescaled model, we have:

\begin{itemize}
\tightlist
\item
  no change in the ANOVA results or R-squared or residual standard
  deviation compared to the uncentered, unscaled model, but
\item
  the intercept (-3.27) now represents the \texttt{st\_delta} for a
  female of average age,
\item
  the \texttt{sex} slope (3.60) represents the (male - female)
  difference in predicted \texttt{st\_delta} for a person of average
  age,
\item
  the \texttt{age\_z} slope (0.510) represents the difference in
  predicted \texttt{st\_delta} for a female one standard deviation older
  than the mean age as compared to a female one standard deviation
  younger than the mean age, and
\item
  the product term's slope (0.402) represents the male - female
  difference in the slope of \texttt{age\_z}, so that if you add the
  \texttt{age\_z} slope (0.510) and the interaction slope (0.402) you
  see the difference in predicted \texttt{st\_delta} for a male one
  standard deviation older than the mean age as compared to a male one
  standard deviation younger than the mean age.
\end{itemize}

\chapter{Missing Data Mechanisms and Single
Imputation}\label{missing-data-mechanisms-and-single-imputation}

Almost all serious statistical analyses have to deal with missing data.
Data values that are missing are indicated in R, and to R, by the symbol
\texttt{NA}.

\section{A Toy Example}\label{a-toy-example}

In the following tiny data set called \texttt{sbp\_example}, we have
four variables for a set of 15 subjects. In addition to a subject id, we
have:

\begin{itemize}
\tightlist
\item
  the treatment this subject received (A, B or C are the treatments),
\item
  an indicator (1 = yes, 0 = no) of whether the subject has diabetes,
\item
  the subject's systolic blood pressure at baseline
\item
  the subject's systolic blood pressure after the application of the
  treatment
\end{itemize}

\begin{Shaded}
\begin{Highlighting}[]
\CommentTok{# create some temporary variables}

\NormalTok{subject <-}\StringTok{ }\DecValTok{101}\OperatorTok{:}\DecValTok{115}
\NormalTok{x1 <-}\StringTok{ }\KeywordTok{c}\NormalTok{(}\StringTok{"A"}\NormalTok{, }\StringTok{"B"}\NormalTok{, }\StringTok{"C"}\NormalTok{, }\StringTok{"A"}\NormalTok{, }\StringTok{"C"}\NormalTok{, }\StringTok{"A"}\NormalTok{, }\StringTok{"A"}\NormalTok{, }\OtherTok{NA}\NormalTok{, }\StringTok{"B"}\NormalTok{, }\StringTok{"C"}\NormalTok{, }\StringTok{"A"}\NormalTok{, }\StringTok{"B"}\NormalTok{, }\StringTok{"C"}\NormalTok{, }\StringTok{"A"}\NormalTok{, }\StringTok{"B"}\NormalTok{)}
\NormalTok{x2 <-}\StringTok{ }\KeywordTok{c}\NormalTok{(}\DecValTok{1}\NormalTok{, }\DecValTok{0}\NormalTok{, }\DecValTok{0}\NormalTok{, }\DecValTok{1}\NormalTok{, }\OtherTok{NA}\NormalTok{, }\DecValTok{1}\NormalTok{, }\DecValTok{0}\NormalTok{, }\DecValTok{1}\NormalTok{, }\OtherTok{NA}\NormalTok{, }\DecValTok{1}\NormalTok{, }\DecValTok{0}\NormalTok{, }\DecValTok{0}\NormalTok{, }\DecValTok{1}\NormalTok{, }\DecValTok{1}\NormalTok{, }\OtherTok{NA}\NormalTok{)}
\NormalTok{x3 <-}\StringTok{ }\KeywordTok{c}\NormalTok{(}\DecValTok{120}\NormalTok{, }\DecValTok{145}\NormalTok{, }\DecValTok{150}\NormalTok{, }\OtherTok{NA}\NormalTok{, }\DecValTok{155}\NormalTok{, }\OtherTok{NA}\NormalTok{, }\DecValTok{135}\NormalTok{, }\OtherTok{NA}\NormalTok{, }\DecValTok{115}\NormalTok{, }\DecValTok{170}\NormalTok{, }\DecValTok{150}\NormalTok{, }\DecValTok{145}\NormalTok{, }\DecValTok{140}\NormalTok{, }\DecValTok{160}\NormalTok{, }\DecValTok{135}\NormalTok{)}
\NormalTok{x4 <-}\StringTok{ }\KeywordTok{c}\NormalTok{(}\DecValTok{105}\NormalTok{, }\DecValTok{135}\NormalTok{, }\DecValTok{150}\NormalTok{, }\DecValTok{120}\NormalTok{, }\DecValTok{135}\NormalTok{, }\DecValTok{115}\NormalTok{, }\DecValTok{160}\NormalTok{, }\DecValTok{150}\NormalTok{, }\DecValTok{130}\NormalTok{, }\DecValTok{155}\NormalTok{, }\DecValTok{140}\NormalTok{, }\DecValTok{140}\NormalTok{, }\DecValTok{150}\NormalTok{, }\DecValTok{135}\NormalTok{, }\DecValTok{120}\NormalTok{)}

\NormalTok{sbp_example <-}\StringTok{ }
\StringTok{  }\KeywordTok{data.frame}\NormalTok{(subject, }\DataTypeTok{treat =}\NormalTok{ x1, }\DataTypeTok{diabetes =}\NormalTok{ x2, }
             \DataTypeTok{sbp.before =}\NormalTok{ x3, }\DataTypeTok{sbp.after =}\NormalTok{ x4) }\OperatorTok
\StringTok{  }\NormalTok{tbl_df}

\KeywordTok{rm}\NormalTok{(subject, x1, x2, x3, x4) }\CommentTok{# just cleaning up}

\NormalTok{sbp_example}
\end{Highlighting}
\end{Shaded}

\begin{verbatim}
# A tibble: 15 x 5
   subject treat diabetes sbp.before sbp.after
     <int> <fct>    <dbl>      <dbl>     <dbl>
 1     101 A         1.00        120       105
 2     102 B         0           145       135
 3     103 C         0           150       150
 4     104 A         1.00         NA       120
 5     105 C        NA           155       135
 6     106 A         1.00         NA       115
 7     107 A         0           135       160
 8     108 <NA>      1.00         NA       150
 9     109 B        NA           115       130
10     110 C         1.00        170       155
11     111 A         0           150       140
12     112 B         0           145       140
13     113 C         1.00        140       150
14     114 A         1.00        160       135
15     115 B        NA           135       120
\end{verbatim}

\subsection{How many missing values do we have in each
column?}\label{how-many-missing-values-do-we-have-in-each-column}

\begin{Shaded}
\begin{Highlighting}[]
\KeywordTok{colSums}\NormalTok{(}\KeywordTok{is.na}\NormalTok{(sbp_example))}
\end{Highlighting}
\end{Shaded}

\begin{verbatim}
   subject      treat   diabetes sbp.before  sbp.after 
         0          1          3          3          0 
\end{verbatim}

We are missing one \texttt{treat}, 3 \texttt{diabetes} and 3
\texttt{sbp.before} values.

\subsection{What is the pattern of missing
data?}\label{what-is-the-pattern-of-missing-data}

\begin{Shaded}
\begin{Highlighting}[]
\NormalTok{mice}\OperatorTok{::}\KeywordTok{md.pattern}\NormalTok{(sbp_example)}
\end{Highlighting}
\end{Shaded}

\begin{verbatim}
  subject sbp.after treat diabetes sbp.before  
9       1         1     1        1          1 0
3       1         1     1        0          1 1
2       1         1     1        1          0 1
1       1         1     0        1          0 2
        0         0     1        3          3 7
\end{verbatim}

We have nine subjects with complete data, three subjects with missing
\texttt{diabetes} (only), two subjects with missing \texttt{sbp.before}
(only), and 1 subject with missing \texttt{treat} and
\texttt{sbp.before}.

\subsection{How can we identify the subjects with missing
data?}\label{how-can-we-identify-the-subjects-with-missing-data}

\begin{Shaded}
\begin{Highlighting}[]
\NormalTok{sbp_example }\OperatorTok\StringTok{ }\KeywordTok{filter}\NormalTok{(}\OperatorTok{!}\KeywordTok{complete.cases}\NormalTok{(.))}
\end{Highlighting}
\end{Shaded}

\begin{verbatim}
# A tibble: 6 x 5
  subject treat diabetes sbp.before sbp.after
    <int> <fct>    <dbl>      <dbl>     <dbl>
1     104 A         1.00         NA       120
2     105 C        NA           155       135
3     106 A         1.00         NA       115
4     108 <NA>      1.00         NA       150
5     109 B        NA           115       130
6     115 B        NA           135       120
\end{verbatim}

\section{Missing-data mechanisms}\label{missing-data-mechanisms}

My source for this description of mechanisms is Chapter 25 of
\citet{GelmanHill2007}, and that chapter is
\href{http://www.stat.columbia.edu/~gelman/arm/missing.pdf}{available at
this link}.

\begin{enumerate}
\def\labelenumi{\arabic{enumi}.}
\tightlist
\item
  \textbf{MCAR = Missingness completely at random}. A variable is
  missing completely at random if the probability of missingness is the
  same for all units, for example, if for each subject, we decide
  whether to collect the \texttt{diabetes} status by rolling a die and
  refusing to answer if a ``6'' shows up. If data are missing completely
  at random, then throwing out cases with missing data does not bias
  your inferences.
\item
  \textbf{Missingness that depends only on observed predictors}. A more
  general assumption, called \textbf{missing at random} or \textbf{MAR},
  is that the probability a variable is missing depends only on
  available information. Here, we would have to be willing to assume
  that the probability of nonresponse to \texttt{diabetes} depends only
  on the other, fully recorded variables in the data. It is often
  reasonable to model this process as a logistic regression, where the
  outcome variable equals 1 for observed cases and 0 for missing. When
  an outcome variable is missing at random, it is acceptable to exclude
  the missing cases (that is, to treat them as NA), as long as the
  regression controls for all the variables that affect the probability
  of missingness.
\item
  \textbf{Missingness that depends on unobserved predictors}.
  Missingness is no longer ``at random'' if it depends on information
  that has not been recorded and this information also predicts the
  missing values. If a particular treatment causes discomfort, a patient
  is more likely to drop out of the study. This missingness is not at
  random (unless ``discomfort'' is measured and observed for all
  patients). If missingness is not at random, it must be explicitly
  modeled, or else you must accept some bias in your inferences.
\item
  \textbf{Missingness that depends on the missing value itself.}
  Finally, a particularly difficult situation arises when the
  probability of missingness depends on the (potentially missing)
  variable itself. For example, suppose that people with higher earnings
  are less likely to reveal them.
\end{enumerate}

Essentially, situations 3 and 4 are referred to collectively as
\textbf{non-random missingness}, and cause more trouble for us than 1
and 2.

\section{Options for Dealing with
Missingness}\label{options-for-dealing-with-missingness}

There are several available methods for dealing with missing data that
are MCAR or MAR, but they basically boil down to:

\begin{itemize}
\tightlist
\item
  Complete Case (or Available Case) analyses
\item
  Single Imputation
\item
  Multiple Imputation
\end{itemize}

\section{Complete Case (and Available Case)
analyses}\label{complete-case-and-available-case-analyses}

In \textbf{Complete Case} analyses, rows containing NA values are
omitted from the data before analyses commence. This is the default
approach for many statistical software packages, and may introduce
unpredictable bias and fail to include some useful, often hard-won
information.

\begin{itemize}
\tightlist
\item
  A complete case analysis can be appropriate when the number of missing
  observations is not large, and the missing pattern is either MCAR
  (missing completely at random) or MAR (missing at random.)
\item
  Two problems arise with complete-case analysis:

  \begin{enumerate}
  \def\labelenumi{\arabic{enumi}.}
  \tightlist
  \item
    If the units with missing values differ systematically from the
    completely observed cases, this could bias the complete-case
    analysis.
  \item
    If many variables are included in a model, there may be very few
    complete cases, so that most of the data would be discarded for the
    sake of a straightforward analysis.
  \end{enumerate}
\item
  A related approach is \emph{available-case} analysis where different
  aspects of a problem are studied with different subsets of the data,
  perhaps identified on the basis of what is missing in them.
\end{itemize}

\section{Single Imputation}\label{single-imputation}

In \textbf{single imputation} analyses, NA values are estimated/replaced
\emph{one time} with \emph{one particular data value} for the purpose of
obtaining more complete samples, at the expense of creating some
potential bias in the eventual conclusions or obtaining slightly
\emph{less} accurate estimates than would be available if there were no
missing values in the data.

\begin{itemize}
\tightlist
\item
  A single imputation can be just a replacement with the mean or median
  (for a quantity) or the mode (for a categorical variable.) However,
  such an approach, though easy to understand, underestimates variance
  and ignores the relationship of missing values to other variables.
\item
  Single imputation can also be done using a variety of models to try to
  capture information about the NA values that are available in other
  variables within the data set.
\item
  The \texttt{simputation} package can help us execute single
  imputations using a wide variety of techniques, within the pipe
  approach used by the \texttt{tidyverse}. Another approach I have used
  in the past is the \texttt{mice} package, which can also perform
  single imputations.
\end{itemize}

\section{Multiple Imputation}\label{multiple-imputation}

\textbf{Multiple imputation}, where NA values are repeatedly
estimated/replaced with multiple data values, for the purpose of
obtaining mode complete samples \emph{and} capturing details of the
variation inherent in the fact that the data have missingness, so as to
obtain \emph{more} accurate estimates than are possible with single
imputation.

\begin{itemize}
\tightlist
\item
  We'll postpone the discussion of multiple imputation for a while.
\end{itemize}

\section{Building a Complete Case
Analysis}\label{building-a-complete-case-analysis}

We can drop all of the missing values from a data set with
\texttt{drop\_na} or with \texttt{na.omit} or by filtering for
\texttt{complete.cases}. Any of these approaches produces the same
result - a new data set with 9 rows (after dropping the six subjects
with any NA values) and 5 columns.

\begin{Shaded}
\begin{Highlighting}[]
\NormalTok{cc.}\DecValTok{1}\NormalTok{ <-}\StringTok{ }\KeywordTok{na.omit}\NormalTok{(sbp_example)}
\NormalTok{cc.}\DecValTok{2}\NormalTok{ <-}\StringTok{ }\NormalTok{sbp_example }\OperatorTok\StringTok{ }\NormalTok{drop_na}
\NormalTok{cc.}\DecValTok{3}\NormalTok{ <-}\StringTok{ }\NormalTok{sbp_example }\OperatorTok\StringTok{ }\KeywordTok{filter}\NormalTok{(}\KeywordTok{complete.cases}\NormalTok{(.))}
\end{Highlighting}
\end{Shaded}

\section{Single Imputation with the Mean or
Mode}\label{single-imputation-with-the-mean-or-mode}

The most straightforward approach to single imputation is to impute a
single summary of the variable, such as the mean, median or mode.

\begin{Shaded}
\begin{Highlighting}[]
\KeywordTok{skim}\NormalTok{(sbp_example)}
\end{Highlighting}
\end{Shaded}

\begin{verbatim}
Skim summary statistics
 n obs: 15 
 n variables: 5 

Variable type: factor 
 variable missing complete  n n_unique              top_counts ordered
    treat       1       14 15        3 A: 6, B: 4, C: 4, NA: 1   FALSE

Variable type: integer 
 variable missing complete  n mean   sd  p0   p25 median   p75 p100
  subject       0       15 15  108 4.47 101 104.5    108 111.5  115

Variable type: numeric 
   variable missing complete  n   mean    sd  p0 p25 median    p75 p100
   diabetes       3       12 15   0.58  0.51   0   0      1   1       1
  sbp.after       0       15 15 136    15.83 105 125    135 150     160
 sbp.before       3       12 15 143.33 15.72 115 135    145 151.25  170
\end{verbatim}

Here, suppose we decide to impute

\begin{itemize}
\tightlist
\item
  \texttt{sbp.before} with the mean (143.33) among non-missing values,
\item
  \texttt{diabetes} with its median (1) among non-missing values, and
\item
  \texttt{treat} with its most common value, or mode (A)
\end{itemize}

\begin{Shaded}
\begin{Highlighting}[]
\NormalTok{si.}\DecValTok{1}\NormalTok{ <-}\StringTok{ }\NormalTok{sbp_example }\OperatorTok
\StringTok{    }\KeywordTok{replace_na}\NormalTok{(}\KeywordTok{list}\NormalTok{(}\DataTypeTok{sbp.before =} \FloatTok{143.33}\NormalTok{,}
                    \DataTypeTok{diabetes =} \DecValTok{1}\NormalTok{,}
                    \DataTypeTok{treat =} \StringTok{"A"}\NormalTok{))}
\NormalTok{si.}\DecValTok{1}
\end{Highlighting}
\end{Shaded}

\begin{verbatim}
# A tibble: 15 x 5
   subject treat diabetes sbp.before sbp.after
     <int> <fct>    <dbl>      <dbl>     <dbl>
 1     101 A         1.00        120       105
 2     102 B         0           145       135
 3     103 C         0           150       150
 4     104 A         1.00        143       120
 5     105 C         1.00        155       135
 6     106 A         1.00        143       115
 7     107 A         0           135       160
 8     108 A         1.00        143       150
 9     109 B         1.00        115       130
10     110 C         1.00        170       155
11     111 A         0           150       140
12     112 B         0           145       140
13     113 C         1.00        140       150
14     114 A         1.00        160       135
15     115 B         1.00        135       120
\end{verbatim}

We could accomplish the same thing with, for example:

\begin{Shaded}
\begin{Highlighting}[]
\NormalTok{si.}\DecValTok{2}\NormalTok{ <-}\StringTok{ }\NormalTok{sbp_example }\OperatorTok
\StringTok{    }\KeywordTok{replace_na}\NormalTok{(}\KeywordTok{list}\NormalTok{(}\DataTypeTok{sbp.before =} \KeywordTok{mean}\NormalTok{(sbp_example}\OperatorTok{$}\NormalTok{sbp.before, }\DataTypeTok{na.rm =} \OtherTok{TRUE}\NormalTok{),}
                    \DataTypeTok{diabetes =} \KeywordTok{median}\NormalTok{(sbp_example}\OperatorTok{$}\NormalTok{diabetes, }\DataTypeTok{na.rm =} \OtherTok{TRUE}\NormalTok{),}
                    \DataTypeTok{treat =} \StringTok{"A"}\NormalTok{))}
\end{Highlighting}
\end{Shaded}

\section{\texorpdfstring{Doing Single Imputation with
\texttt{simputation}}{Doing Single Imputation with simputation}}\label{doing-single-imputation-with-simputation}

Single imputation is a potentially appropriate method when missingness
can be assumed to be either completely at random (MCAR) or dependent
only on observed predictors (MAR). We'll use the \texttt{simputation}
package to accomplish it.

\begin{itemize}
\tightlist
\item
  The \texttt{simputation} vignette is available at
  \url{https://cran.r-project.org/web/packages/simputation/vignettes/intro.html}
\item
  The \texttt{simputation} reference manual is available at
  \url{https://cran.r-project.org/web/packages/simputation/simputation.pdf}
\end{itemize}

\subsection{Mirroring Our Prior Approach (imputing
means/medians/modes)}\label{mirroring-our-prior-approach-imputing-meansmediansmodes}

Suppose we want to mirror what we did above, simply impute the mean for
\texttt{sbp.before} and the median for \texttt{diabetes} again.

\begin{Shaded}
\begin{Highlighting}[]
\NormalTok{si.}\DecValTok{3}\NormalTok{ <-}\StringTok{ }\NormalTok{sbp_example }\OperatorTok
\StringTok{    }\KeywordTok{impute_lm}\NormalTok{(sbp.before }\OperatorTok{~}\StringTok{ }\DecValTok{1}\NormalTok{) }\OperatorTok
\StringTok{    }\KeywordTok{impute_median}\NormalTok{(diabetes }\OperatorTok{~}\StringTok{ }\DecValTok{1}\NormalTok{) }\OperatorTok
\StringTok{    }\KeywordTok{replace_na}\NormalTok{(}\KeywordTok{list}\NormalTok{(}\DataTypeTok{treat =} \StringTok{"A"}\NormalTok{))}

\NormalTok{si.}\DecValTok{3}
\end{Highlighting}
\end{Shaded}

\begin{verbatim}
# A tibble: 15 x 5
   subject treat diabetes sbp.before sbp.after
 *   <int> <fct>    <dbl>      <dbl>     <dbl>
 1     101 A         1.00        120       105
 2     102 B         0           145       135
 3     103 C         0           150       150
 4     104 A         1.00        143       120
 5     105 C         1.00        155       135
 6     106 A         1.00        143       115
 7     107 A         0           135       160
 8     108 A         1.00        143       150
 9     109 B         1.00        115       130
10     110 C         1.00        170       155
11     111 A         0           150       140
12     112 B         0           145       140
13     113 C         1.00        140       150
14     114 A         1.00        160       135
15     115 B         1.00        135       120
\end{verbatim}

\subsection{\texorpdfstring{Using a model to impute \texttt{sbp.before}
and
\texttt{diabetes}}{Using a model to impute sbp.before and diabetes}}\label{using-a-model-to-impute-sbp.before-and-diabetes}

Suppose we wanted to use:

\begin{itemize}
\tightlist
\item
  a robust linear model to predict \texttt{sbp.before} missing values,
  on the basis of \texttt{sbp.after} and \texttt{diabetes} status, and
\item
  a predictive mean matching approach to predict \texttt{diabetes}
  status, on the basis of \texttt{sbp.after}, and
\item
  a decision tree approach to predict \texttt{treat} status, using all
  other variables in the data
\end{itemize}

\begin{Shaded}
\begin{Highlighting}[]
\KeywordTok{set.seed}\NormalTok{(}\DecValTok{50001}\NormalTok{)}

\NormalTok{imp.}\DecValTok{4}\NormalTok{ <-}\StringTok{ }\NormalTok{sbp_example }\OperatorTok
\StringTok{    }\KeywordTok{impute_rlm}\NormalTok{(sbp.before }\OperatorTok{~}\StringTok{ }\NormalTok{sbp.after }\OperatorTok{+}\StringTok{ }\NormalTok{diabetes) }\OperatorTok
\StringTok{    }\KeywordTok{impute_pmm}\NormalTok{(diabetes }\OperatorTok{~}\StringTok{ }\NormalTok{sbp.after) }\OperatorTok
\StringTok{    }\KeywordTok{impute_cart}\NormalTok{(treat }\OperatorTok{~}\StringTok{ }\NormalTok{.)}

\NormalTok{imp.}\DecValTok{4}
\end{Highlighting}
\end{Shaded}

\begin{verbatim}
# A tibble: 15 x 5
   subject treat diabetes sbp.before sbp.after
 *   <int> <fct>    <dbl>      <dbl>     <dbl>
 1     101 A         1.00        120       105
 2     102 B         0           145       135
 3     103 C         0           150       150
 4     104 A         1.00        139       120
 5     105 C         1.00        155       135
 6     106 A         1.00        136       115
 7     107 A         0           135       160
 8     108 A         1.00        155       150
 9     109 B         1.00        115       130
10     110 C         1.00        170       155
11     111 A         0           150       140
12     112 B         0           145       140
13     113 C         1.00        140       150
14     114 A         1.00        160       135
15     115 B         1.00        135       120
\end{verbatim}

Details on the many available methods in \texttt{simputation} are
provided
\href{https://cran.r-project.org/web/packages/simputation/simputation.pdf}{in
its manual}. These include:

\begin{itemize}
\tightlist
\item
  \texttt{impute\_cart} uses a Classification and Regression Tree
  approach for numerical or categorical data. There is also an
  \texttt{impute\_rf} command which uses Random Forests for imputation.
\item
  \texttt{impute\_pmm} is one of several ``hot deck'' options for
  imputation, this one is predictive mean matching, which can be used
  with numeric data (only). Missing values are first imputed using a
  predictive model. Next, these predictions are replaced with the
  observed values which are nearest to the prediction. Other imputation
  options in this group include random hot deck, sequential hot deck and
  k-nearest neighbor imputation.
\item
  \texttt{impute\_rlm} is one of several regression imputation methods,
  including linear models, robust linear models (which use what is
  called M-estimation to impute numerical variables) and lasso/elastic
  net/ridge regression models.
\end{itemize}

The \texttt{simputation} package can also do EM-based multivariate
imputation, and multivariate random forest imputation, and several other
approaches.

\chapter{A Study of Prostate Cancer}\label{a-study-of-prostate-cancer}

\section{Data Load and Background}\label{data-load-and-background}

The data in \texttt{prost.csv} is derived from \citet{Stamey1989} who
examined the relationship between the level of prostate-specific antigen
and a number of clinical measures in 97 men who were about to receive a
radical prostatectomy. The \texttt{prost} data, as I'll name it in R,
contains 97 rows and 11 columns.

\begin{Shaded}
\begin{Highlighting}[]
\NormalTok{prost}
\end{Highlighting}
\end{Shaded}

\begin{verbatim}
# A tibble: 97 x 10
   subject   lpsa lcavol lweight   age bph      svi   lcp gleason pgg45
     <int>  <dbl>  <dbl>   <dbl> <int> <fct>  <int> <dbl> <fct>   <int>
 1       1 -0.431 -0.580    2.77    50 Low        0 -1.39 6           0
 2       2 -0.163 -0.994    3.32    58 Low        0 -1.39 6           0
 3       3 -0.163 -0.511    2.69    74 Low        0 -1.39 7          20
 4       4 -0.163 -1.20     3.28    58 Low        0 -1.39 6           0
 5       5  0.372  0.751    3.43    62 Low        0 -1.39 6           0
 6       6  0.765 -1.05     3.23    50 Low        0 -1.39 6           0
 7       7  0.765  0.737    3.47    64 Medium     0 -1.39 6           0
 8       8  0.854  0.693    3.54    58 High       0 -1.39 6           0
 9       9  1.05  -0.777    3.54    47 Low        0 -1.39 6           0
10      10  1.05   0.223    3.24    63 Low        0 -1.39 6           0
# ... with 87 more rows
\end{verbatim}

Note that a related \texttt{prost} data frame is also available as part
of several R packages, including the \texttt{faraway} package, but there
is an error in the \texttt{lweight} data for subject 32 in those
presentations. The value of \texttt{lweight} for subject 32 should not
be 6.1, corresponding to a prostate that is 449 grams in size, but
instead the \texttt{lweight} value should be 3.804438, corresponding to
a 44.9 gram prostate\footnote{\url{https://statweb.stanford.edu/~tibs/ElemStatLearn/}
  attributes the correction to Professor Stephen W. Link.}.

I've also changed the \texttt{gleason} and \texttt{bph} variables from
their presentation in other settings, to let me teach some additional
details.

\section{Code Book}\label{code-book}

\begin{longtable}[]{@{}rl@{}}
\toprule
Variable & Description\tabularnewline
\midrule
\endhead
\texttt{subject} & subject number (1 to 97)\tabularnewline
\texttt{lpsa} & log(prostate specific antigen in ng/ml), our
\textbf{outcome}\tabularnewline
\texttt{lcavol} & log(cancer volume in
cm\textsuperscript{3})\tabularnewline
\texttt{lweight} & log(prostate weight, in g)\tabularnewline
\texttt{age} & age\tabularnewline
\texttt{bph} & benign prostatic hyperplasia amount (Low, Medium, or
High)\tabularnewline
\texttt{svi} & seminal vesicle invasion (1 = yes, 0 = no)\tabularnewline
\texttt{lcp} & log(capsular penetration, in cm)\tabularnewline
\texttt{gleason} & combined Gleason score (6, 7, or \textgreater{} 7
here)\tabularnewline
\texttt{pgg45} & percentage Gleason scores 4 or 5\tabularnewline
\bottomrule
\end{longtable}

Notes:

\begin{itemize}
\tightlist
\item
  in general, higher levels of PSA are stronger indicators of prostate
  cancer. An old standard (established almost exclusively with testing
  in white males, and definitely flawed) suggested that values below 4
  were normal, and above 4 needed further testing. A PSA of 4
  corresponds to an \texttt{lpsa} of 1.39.
\item
  all logarithms are natural (base \emph{e}) logarithms, obtained in R
  with the function \texttt{log()}
\item
  all variables other than \texttt{subject} and \texttt{lpsa} are
  candidate predictors
\item
  the \texttt{gleason} variable captures the highest combined Gleason
  score{[}\^{}Scores range (in these data) from 6 (a
  well-differentiated, or low-grade cancer) to 9 (a high-grade cancer),
  although the maximum possible score is 10. 6 is the lowest score used
  for cancerous prostates. As this combination value increases, the rate
  at which the cancer grows and spreads should increase. This score
  refers to the combined Gleason grade, which is based on the sum of two
  areas (each scored 1-5) that make up most of the cancer.{]} in a
  biopsy, and higher scores indicate more aggressive cancer cells. It's
  stored here as 6, 7, or \textgreater{} 7.
\item
  the \texttt{pgg45} variable captures the percentage of individual
  Gleason scores{[}\^{}The 1-5 scale for individual biopsies are defined
  so that 1 indicates something that looks like normal prostate tissue,
  and 5 indicates that the cells and their growth patterns look very
  abnormal. In this study, the percentage of 4s and 5s shown in the data
  appears to be based on 5-20 individual scores in most subjects.{]}
  that are 4 or 5, on a 1-5 scale, where higher scores indicate more
  abnormal cells.
\end{itemize}

\section{Additions for Later Use}\label{additions-for-later-use}

The code below adds to the \texttt{prost} tibble:

\begin{itemize}
\tightlist
\item
  a factor version of the \texttt{svi} variable, called \texttt{svi\_f},
  with levels No and Yes,
\item
  a factor version of \texttt{gleason} called \texttt{gleason\_f}, with
  the levels ordered \textgreater{} 7, 7, and finally 6,
\item
  a factor version of \texttt{bph} called \texttt{bph\_f}, with levels
  ordered Low, Medium, High,
\item
  a centered version of \texttt{lcavol} called \texttt{lcavol\_c},
\item
  exponentiated \texttt{cavol} and \texttt{psa} results derived from the
  natural logarithms \texttt{lcavol} and \texttt{lpsa}.
\end{itemize}

\begin{Shaded}
\begin{Highlighting}[]
\NormalTok{prost <-}\StringTok{ }\NormalTok{prost }\OperatorTok
\StringTok{    }\KeywordTok{mutate}\NormalTok{(}\DataTypeTok{svi_f =} \KeywordTok{fct_recode}\NormalTok{(}\KeywordTok{factor}\NormalTok{(svi), }\StringTok{"No"}\NormalTok{ =}\StringTok{ "0"}\NormalTok{, }\StringTok{"Yes"}\NormalTok{ =}\StringTok{ "1"}\NormalTok{),}
           \DataTypeTok{gleason_f =} \KeywordTok{fct_relevel}\NormalTok{(gleason, }\KeywordTok{c}\NormalTok{(}\StringTok{"> 7"}\NormalTok{, }\StringTok{"7"}\NormalTok{, }\StringTok{"6"}\NormalTok{)),}
           \DataTypeTok{bph_f =} \KeywordTok{fct_relevel}\NormalTok{(bph, }\KeywordTok{c}\NormalTok{(}\StringTok{"Low"}\NormalTok{, }\StringTok{"Medium"}\NormalTok{, }\StringTok{"High"}\NormalTok{)),}
           \DataTypeTok{lcavol_c =}\NormalTok{ lcavol }\OperatorTok{-}\StringTok{ }\KeywordTok{mean}\NormalTok{(lcavol),}
           \DataTypeTok{cavol =} \KeywordTok{exp}\NormalTok{(lcavol),}
           \DataTypeTok{psa =} \KeywordTok{exp}\NormalTok{(lpsa))}

\KeywordTok{glimpse}\NormalTok{(prost)}
\end{Highlighting}
\end{Shaded}

\begin{verbatim}
Observations: 97
Variables: 16
$ subject   <int> 1, 2, 3, 4, 5, 6, 7, 8, 9, 10, 11, 12, 13, 14, 15, 1...
$ lpsa      <dbl> -0.4307829, -0.1625189, -0.1625189, -0.1625189, 0.37...
$ lcavol    <dbl> -0.5798185, -0.9942523, -0.5108256, -1.2039728, 0.75...
$ lweight   <dbl> 2.769459, 3.319626, 2.691243, 3.282789, 3.432373, 3....
$ age       <int> 50, 58, 74, 58, 62, 50, 64, 58, 47, 63, 65, 63, 63, ...
$ bph       <fct> Low, Low, Low, Low, Low, Low, Medium, High, Low, Low...
$ svi       <int> 0, 0, 0, 0, 0, 0, 0, 0, 0, 0, 0, 0, 0, 0, 0, 0, 0, 0...
$ lcp       <dbl> -1.3862944, -1.3862944, -1.3862944, -1.3862944, -1.3...
$ gleason   <fct> 6, 6, 7, 6, 6, 6, 6, 6, 6, 6, 6, 6, 7, 7, 7, 6, 7, 6...
$ pgg45     <int> 0, 0, 20, 0, 0, 0, 0, 0, 0, 0, 0, 0, 30, 5, 5, 0, 30...
$ svi_f     <fct> No, No, No, No, No, No, No, No, No, No, No, No, No, ...
$ gleason_f <fct> 6, 6, 7, 6, 6, 6, 6, 6, 6, 6, 6, 6, 7, 7, 7, 6, 7, 6...
$ bph_f     <fct> Low, Low, Low, Low, Low, Low, Medium, High, Low, Low...
$ lcavol_c  <dbl> -1.9298281, -2.3442619, -1.8608352, -2.5539824, -0.5...
$ cavol     <dbl> 0.56, 0.37, 0.60, 0.30, 2.12, 0.35, 2.09, 2.00, 0.46...
$ psa       <dbl> 0.65, 0.85, 0.85, 0.85, 1.45, 2.15, 2.15, 2.35, 2.85...
\end{verbatim}

\section{Fitting and Evaluating a Two-Predictor
Model}\label{fitting-and-evaluating-a-two-predictor-model}

To begin, let's use two predictors (\texttt{lcavol} and \texttt{svi})
and their interaction in a linear regression model that predicts
\texttt{lpsa}. I'll call this model \texttt{c5\_prost\_A}

Earlier, we centered the \texttt{lcavol} values to facilitate
interpretation of the terms. I'll use that centered version (called
\texttt{lcavol\_c}) of the quantitative predictor, and the 1/0 version
of the \texttt{svi} variable{[}\^{}We could certainly use the factor
version of \texttt{svi} here, but it won't change the model in any
meaningful way. There's no distinction in model \emph{fitting} via
\texttt{lm} between a 0/1 numeric variable and a No/Yes factor variable.
The factor version of this information will be useful elsewhere, for
instance in plotting the model.{]}.

\begin{Shaded}
\begin{Highlighting}[]
\NormalTok{c5_prost_A <-}\StringTok{ }\KeywordTok{lm}\NormalTok{(lpsa }\OperatorTok{~}\StringTok{ }\NormalTok{lcavol_c }\OperatorTok{*}\StringTok{ }\NormalTok{svi, }\DataTypeTok{data =}\NormalTok{ prost)}
\KeywordTok{summary}\NormalTok{(c5_prost_A)}
\end{Highlighting}
\end{Shaded}

\begin{verbatim}

Call:
lm(formula = lpsa ~ lcavol_c * svi, data = prost)

Residuals:
    Min      1Q  Median      3Q     Max 
-1.6305 -0.5007  0.1266  0.4886  1.6847 

Coefficients:
             Estimate Std. Error t value Pr(>|t|)    
(Intercept)   2.33134    0.09128  25.540  < 2e-16 ***
lcavol_c      0.58640    0.08207   7.145 1.98e-10 ***
svi           0.60132    0.35833   1.678   0.0967 .  
lcavol_c:svi  0.06479    0.26614   0.243   0.8082    
---
Signif. codes:  0 '***' 0.001 '**' 0.01 '*' 0.05 '.' 0.1 ' ' 1

Residual standard error: 0.7595 on 93 degrees of freedom
Multiple R-squared:  0.5806,    Adjusted R-squared:  0.5671 
F-statistic: 42.92 on 3 and 93 DF,  p-value: < 2.2e-16
\end{verbatim}

\subsection{\texorpdfstring{Using
\texttt{tidy}}{Using tidy}}\label{using-tidy}

It can be very useful to build a data frame of the model's results. We
can use the \texttt{tidy} function in the \texttt{broom} package to do
so.

\begin{Shaded}
\begin{Highlighting}[]
\KeywordTok{tidy}\NormalTok{(c5_prost_A)}
\end{Highlighting}
\end{Shaded}

\begin{verbatim}
          term   estimate  std.error  statistic      p.value
1  (Intercept) 2.33134409 0.09128253 25.5398727 8.246849e-44
2     lcavol_c 0.58639599 0.08206929  7.1451331 1.981492e-10
3          svi 0.60131973 0.35832695  1.6781314 9.667899e-02
4 lcavol_c:svi 0.06479298 0.26614194  0.2434527 8.081909e-01
\end{verbatim}

This makes it much easier to pull out individual elements of the model
fit.

For example, to specify the coefficient for \texttt{svi}, rounded to
three decimal places, I could use
\texttt{tidy(c5\_prost\_A)\ \%\textgreater{}\%\ filter(term\ ==\ "svi")\ \%\textgreater{}\%\ select(estimate)\ \%\textgreater{}\%\ round(.,\ 3)}

\begin{itemize}
\tightlist
\item
  The result is 0.601.
\item
  If you look at the Markdown file, you'll see that the number shown in
  the bullet point above this one was generated using inline R code, and
  the function specified above.
\end{itemize}

\subsection{Interpretation}\label{interpretation}

\begin{enumerate}
\def\labelenumi{\arabic{enumi}.}
\tightlist
\item
  The intercept, 2.33, for the model is the predicted value of
  \texttt{lpsa} when \texttt{lcavol} is at its average and there is no
  seminal vesicle invasion (e.g. \texttt{svi} = 0).
\item
  The coefficient for \texttt{lcavol\_c}, 0.59, is the predicted change
  in \texttt{lpsa} associated with a one unit increase in
  \texttt{lcavol} (or \texttt{lcavol\_c}) when there is no seminal
  vesicle invasion.
\item
  The coefficient for \texttt{svi}, 0.60, is the predicted change in
  \texttt{lpsa} associated with having no \texttt{svi} to having an
  \texttt{svi} while the \texttt{lcavol} remains at its average.
\item
  The coefficient for \texttt{lcavol\_c:svi}, the product term, which is
  0.06, is the difference in the slope of \texttt{lcavol\_c} for a
  subject with \texttt{svi} as compared to one with no \texttt{svi}.
\end{enumerate}

\emph{Note}: If you look at the R Markdown, you'll notice that in bullet
point 3, I didn't use \texttt{round} to round off the estimate (as I did
in the other three bullets), but instead a special function I specified
at the start of the R Markdown file called \texttt{specify\_decimal()}
which uses the \texttt{format} function. This forces, in this case, the
trailing zero in the two decimal representation of the \texttt{svi}
coefficient to be shown. The special function, again, is:

\texttt{specify\_decimal\ \textless{}-\ function(x,\ k)\ format(round(x,\ k),\ nsmall=k)}

\section{\texorpdfstring{Exploring Model
\texttt{c5\_prost\_A}}{Exploring Model c5\_prost\_A}}\label{exploring-model-c5_prost_a}

The \texttt{glance} function from the \texttt{broom} package builds a
nice one-row summary for the model.

\begin{Shaded}
\begin{Highlighting}[]
\KeywordTok{glance}\NormalTok{(c5_prost_A)}
\end{Highlighting}
\end{Shaded}

\begin{verbatim}
  r.squared adj.r.squared     sigma statistic      p.value df    logLik
1 0.5806435     0.5671158 0.7594785  42.92278 1.678836e-17  4 -108.9077
       AIC      BIC deviance df.residual
1 227.8153 240.6889 53.64311          93
\end{verbatim}

This summary includes, in order,

\begin{itemize}
\tightlist
\item
  the model \(R^2\), adjusted \(R^2\) and \(\hat{\sigma}\), the residual
  standard deviation,
\item
  the ANOVA F statistic and associated \emph{p} value,
\item
  the number of degrees of freedom used by the model, and its
  log-likelihood ratio
\item
  the model's AIC (Akaike Information Criterion) and BIC (Bayesian
  Information Criterion)
\item
  the model's deviance statistic and residual degrees of freedom
\end{itemize}

\subsection{\texorpdfstring{\texttt{summary} for Model
\texttt{c5\_prost\_A}}{summary for Model c5\_prost\_A}}\label{summary-for-model-c5_prost_a}

If necessary, we can also run \texttt{summary} on this
\texttt{c5\_prost\_A} object to pick up some additional summaries. Since
the \texttt{svi} variable is binary, the interaction term is, too, so
the \emph{t} test here and the \emph{F} test in the ANOVA yield the same
result.

\begin{Shaded}
\begin{Highlighting}[]
\KeywordTok{summary}\NormalTok{(c5_prost_A)}
\end{Highlighting}
\end{Shaded}

\begin{verbatim}

Call:
lm(formula = lpsa ~ lcavol_c * svi, data = prost)

Residuals:
    Min      1Q  Median      3Q     Max 
-1.6305 -0.5007  0.1266  0.4886  1.6847 

Coefficients:
             Estimate Std. Error t value Pr(>|t|)    
(Intercept)   2.33134    0.09128  25.540  < 2e-16 ***
lcavol_c      0.58640    0.08207   7.145 1.98e-10 ***
svi           0.60132    0.35833   1.678   0.0967 .  
lcavol_c:svi  0.06479    0.26614   0.243   0.8082    
---
Signif. codes:  0 '***' 0.001 '**' 0.01 '*' 0.05 '.' 0.1 ' ' 1

Residual standard error: 0.7595 on 93 degrees of freedom
Multiple R-squared:  0.5806,    Adjusted R-squared:  0.5671 
F-statistic: 42.92 on 3 and 93 DF,  p-value: < 2.2e-16
\end{verbatim}

If you've forgotten the details of the pieces of this summary, review
the Part C Notes from 431.

\subsection{\texorpdfstring{Adjusted
R\textsuperscript{2}}{Adjusted R2}}\label{adjusted-r2}

R\textsuperscript{2} is greedy.

\begin{itemize}
\tightlist
\item
  R\textsuperscript{2} will always suggest that we make our models as
  big as possible, often including variables of dubious predictive
  value.
\item
  As a result, there are various methods for penalizing
  R\textsuperscript{2} so that we wind up with smaller models.
\item
  The \textbf{adjusted R\textsuperscript{2}} is often a useful way to
  compare multiple models for the same response.

  \begin{itemize}
  \tightlist
  \item
    \(R^2_{adj} = 1 - \frac{(1-R^2)(n - 1)}{n - k}\), where \(n\) = the
    number of observations and \(k\) is the number of coefficients
    estimated by the regression (including the intercept and any
    slopes).
  \item
    So, in this case,
    \(R^2_{adj} = 1 - \frac{(1 - 0.5806)(97 - 1)}{97 - 4} = 0.5671\)
  \item
    The adjusted R\textsuperscript{2} value is not, technically, a
    proportion of anything, but it is comparable across models for the
    same outcome.
  \item
    The adjusted R\textsuperscript{2} will always be less than the
    (unadjusted) R\textsuperscript{2}.
  \end{itemize}
\end{itemize}

\subsection{Coefficient Confidence
Intervals}\label{coefficient-confidence-intervals}

Here are the 90\% confidence intervals for the coefficients in Model A.
Adjust the \texttt{level} to get different intervals.

\begin{Shaded}
\begin{Highlighting}[]
\KeywordTok{confint}\NormalTok{(c5_prost_A, }\DataTypeTok{level =} \FloatTok{0.90}\NormalTok{)}
\end{Highlighting}
\end{Shaded}

\begin{verbatim}
                     5 %      95 %
(Intercept)   2.17968697 2.4830012
lcavol_c      0.45004577 0.7227462
svi           0.00599401 1.1966454
lcavol_c:svi -0.37737623 0.5069622
\end{verbatim}

What can we conclude from this about the utility of the interaction
term?

\subsection{\texorpdfstring{ANOVA for Model
\texttt{c5\_prost\_A}}{ANOVA for Model c5\_prost\_A}}\label{anova-for-model-c5_prost_a}

The interaction term appears unnecessary. We might wind up fitting the
model without it. A complete ANOVA test is available, including a
\emph{p} value, if you want it.

\begin{Shaded}
\begin{Highlighting}[]
\KeywordTok{anova}\NormalTok{(c5_prost_A)}
\end{Highlighting}
\end{Shaded}

\begin{verbatim}
Analysis of Variance Table

Response: lpsa
             Df Sum Sq Mean Sq  F value    Pr(>F)    
lcavol_c      1 69.003  69.003 119.6289 < 2.2e-16 ***
svi           1  5.237   5.237   9.0801  0.003329 ** 
lcavol_c:svi  1  0.034   0.034   0.0593  0.808191    
Residuals    93 53.643   0.577                       
---
Signif. codes:  0 '***' 0.001 '**' 0.01 '*' 0.05 '.' 0.1 ' ' 1
\end{verbatim}

Note that the \texttt{anova} approach for a \texttt{lm} object is
sequential. The first row shows the impact of \texttt{lcavol\_c} as
compared to a model with no predictors (just an intercept). The second
row shows the impact of adding \texttt{svi} to a model that already
contains \texttt{lcavol\_c}. The third row shows the impact of adding
the interaction (product) term to the model with the two main effects.
So the order in which the variables are added to the regression model
matters for this ANOVA. The F tests here describe the incremental impact
of each covariate in turn.

\subsection{\texorpdfstring{Residuals, Fitted Values and Standard Errors
with
\texttt{augment}}{Residuals, Fitted Values and Standard Errors with augment}}\label{residuals-fitted-values-and-standard-errors-with-augment}

The \texttt{augment} function in the \texttt{broom} package builds a
data frame including the data used in the model, along with predictions
(fitted values), residuals and other useful information.

\begin{Shaded}
\begin{Highlighting}[]
\NormalTok{c5_prost_A_frame <-}\StringTok{ }\KeywordTok{augment}\NormalTok{(c5_prost_A) }\OperatorTok\StringTok{ }\NormalTok{tbl_df}
\KeywordTok{skim}\NormalTok{(c5_prost_A_frame)}
\end{Highlighting}
\end{Shaded}

\begin{verbatim}
Skim summary statistics
 n obs: 97 
 n variables: 10 

Variable type: integer 
 variable missing complete  n mean   sd p0 p25 median p75 p100
      svi       0       97 97 0.22 0.41  0   0      0   0    1

Variable type: numeric 
   variable missing complete  n     mean     sd       p0      p25 median
    .cooksd       0       97 97  0.011   0.02    6.9e-06  0.00078 0.0035
    .fitted       0       97 97  2.48    0.88    0.75     1.84    2.4   
       .hat       0       97 97  0.041   0.041   0.013    0.016   0.025 
     .resid       0       97 97 -6.9e-17 0.75   -1.63    -0.5     0.13  
    .se.fit       0       97 97  0.14    0.061   0.087    0.095   0.12  
     .sigma       0       97 97  0.76    0.0052  0.74     0.76    0.76  
 .std.resid       0       97 97  0.0012  1.01   -2.19    -0.69    0.17  
   lcavol_c       0       97 97  5.4e-17 1.18   -2.7     -0.84    0.097 
       lpsa       0       97 97  2.48    1.15   -0.43     1.73    2.59  
   p75 p100
 0.01  0.13
 3.07  4.54
 0.049 0.25
 0.49  1.68
 0.17  0.38
 0.76  0.76
 0.65  2.26
 0.78  2.47
 3.06  5.58
\end{verbatim}

Elements shown here include:

\begin{itemize}
\tightlist
\item
  \texttt{.fitted} Fitted values of model (or predicted values)
\item
  \texttt{.se.fit} Standard errors of fitted values
\item
  \texttt{.resid} Residuals (observed - fitted values)
\item
  \texttt{.hat} Diagonal of the hat matrix (these indicate
  \emph{leverage} - points with high leverage indicate unusual
  combinations of predictors - values more than 2-3 times the mean
  leverage are worth some study - leverage is always between 0 and 1,
  and measures the amount by which the predicted value would change if
  the observation's y value was increased by one unit - a point with
  leverage 1 would cause the line to follow that point perfectly)
\item
  \texttt{.sigma} Estimate of residual standard deviation when
  corresponding observation is dropped from model
\item
  \texttt{.cooksd} Cook's distance, which helps identify influential
  points (values of Cook's d \textgreater{} 0.5 may be influential,
  values \textgreater{} 1.0 almost certainly are - an influential point
  changes the fit substantially when it is removed from the data)
\item
  \texttt{.std.resid} Standardized residuals (values above 2 in absolute
  value are worth some study - treat these as normal deviates {[}Z
  scores{]}, essentially)
\end{itemize}

See \texttt{?augment.lm} in R for more details.

\subsection{\texorpdfstring{Making Predictions with
\texttt{c5\_prost\_A}}{Making Predictions with c5\_prost\_A}}\label{making-predictions-with-c5_prost_a}

Suppose we want to predict the \texttt{lpsa} for a patient with cancer
volume equal to this group's mean, for both a patient with and without
seminal vesicle invasion, and in each case, we want to use a 90\%
prediction interval?

\begin{Shaded}
\begin{Highlighting}[]
\NormalTok{newdata <-}\StringTok{ }\KeywordTok{data.frame}\NormalTok{(}\DataTypeTok{lcavol_c =} \KeywordTok{c}\NormalTok{(}\DecValTok{0}\NormalTok{,}\DecValTok{0}\NormalTok{), }\DataTypeTok{svi =} \KeywordTok{c}\NormalTok{(}\DecValTok{0}\NormalTok{,}\DecValTok{1}\NormalTok{))}
\KeywordTok{predict}\NormalTok{(c5_prost_A, newdata, }\DataTypeTok{interval =} \StringTok{"prediction"}\NormalTok{, }\DataTypeTok{level =} \FloatTok{0.90}\NormalTok{)}
\end{Highlighting}
\end{Shaded}

\begin{verbatim}
       fit      lwr      upr
1 2.331344 1.060462 3.602226
2 2.932664 1.545742 4.319586
\end{verbatim}

Since the predicted value in \texttt{fit} refers to the natural
logarithm of PSA, to make the predictions in terms of PSA, we would need
to exponentiate. The code below will accomplish that task.

\begin{Shaded}
\begin{Highlighting}[]
\NormalTok{pred <-}\StringTok{ }\KeywordTok{predict}\NormalTok{(c5_prost_A, newdata, }\DataTypeTok{interval =} \StringTok{"prediction"}\NormalTok{, }\DataTypeTok{level =} \FloatTok{0.90}\NormalTok{)}
\KeywordTok{exp}\NormalTok{(pred)}
\end{Highlighting}
\end{Shaded}

\begin{verbatim}
       fit      lwr      upr
1 10.29177 2.887706 36.67978
2 18.77758 4.691450 75.15750
\end{verbatim}

\section{\texorpdfstring{Plotting Model
\texttt{c5\_prost\_A}}{Plotting Model c5\_prost\_A}}\label{plotting-model-c5_prost_a}

\subsubsection{Plot logs conventionally}\label{plot-logs-conventionally}

Here, we'll use \texttt{ggplot2} to plot the logarithms of the variables
as they came to us, on a conventional coordinate scale. Note that the
lines are nearly parallel. What does this suggest about our Model A?

\begin{Shaded}
\begin{Highlighting}[]
\KeywordTok{ggplot}\NormalTok{(prost, }\KeywordTok{aes}\NormalTok{(}\DataTypeTok{x =}\NormalTok{ lcavol, }\DataTypeTok{y =}\NormalTok{ lpsa, }\DataTypeTok{group =}\NormalTok{ svi_f, }\DataTypeTok{color =}\NormalTok{ svi_f)) }\OperatorTok{+}
\StringTok{    }\KeywordTok{geom_point}\NormalTok{() }\OperatorTok{+}
\StringTok{    }\KeywordTok{geom_smooth}\NormalTok{(}\DataTypeTok{method =} \StringTok{"lm"}\NormalTok{, }\DataTypeTok{se =} \OtherTok{FALSE}\NormalTok{) }\OperatorTok{+}\StringTok{ }
\StringTok{    }\KeywordTok{scale_color_discrete}\NormalTok{(}\DataTypeTok{name =} \StringTok{"Seminal Vesicle Invasion?"}\NormalTok{) }\OperatorTok{+}
\StringTok{    }\KeywordTok{theme_bw}\NormalTok{() }\OperatorTok{+}
\StringTok{    }\KeywordTok{labs}\NormalTok{(}\DataTypeTok{x =} \StringTok{"Log (cancer volume, cc)"}\NormalTok{, }
         \DataTypeTok{y =} \StringTok{"Log (Prostate Specific Antigen, ng/ml)"}\NormalTok{, }
         \DataTypeTok{title =} \StringTok{"Two Predictor Model c5_prost_A, including Interaction"}\NormalTok{)}
\end{Highlighting}
\end{Shaded}

\includegraphics{bookdown-demo_files/figure-latex/unnamed-chunk-53-1.pdf}

\subsubsection{Plot on log-log scale}\label{plot-on-log-log-scale}

Another approach (which might be easier in some settings) would be to
plot the raw values of Cancer Volume and PSA, but use logarithmic axes,
again using the natural (base \emph{e}) logarithm, as follows. If we use
the default choice with `trans = ``log'', we'll find a need to select
some useful break points for the grid, as I've done in what follows.

\begin{Shaded}
\begin{Highlighting}[]
\KeywordTok{ggplot}\NormalTok{(prost, }\KeywordTok{aes}\NormalTok{(}\DataTypeTok{x =}\NormalTok{ cavol, }\DataTypeTok{y =}\NormalTok{ psa, }\DataTypeTok{group =}\NormalTok{ svi_f, }\DataTypeTok{color =}\NormalTok{ svi_f)) }\OperatorTok{+}
\StringTok{    }\KeywordTok{geom_point}\NormalTok{() }\OperatorTok{+}
\StringTok{    }\KeywordTok{geom_smooth}\NormalTok{(}\DataTypeTok{method =} \StringTok{"lm"}\NormalTok{, }\DataTypeTok{se =} \OtherTok{FALSE}\NormalTok{) }\OperatorTok{+}\StringTok{ }
\StringTok{    }\KeywordTok{scale_color_discrete}\NormalTok{(}\DataTypeTok{name =} \StringTok{"Seminal Vesicle Invasion?"}\NormalTok{) }\OperatorTok{+}
\StringTok{    }\KeywordTok{scale_x_continuous}\NormalTok{(}\DataTypeTok{trans =} \StringTok{"log"}\NormalTok{, }
                       \DataTypeTok{breaks =} \KeywordTok{c}\NormalTok{(}\FloatTok{0.5}\NormalTok{, }\DecValTok{1}\NormalTok{, }\DecValTok{2}\NormalTok{, }\DecValTok{5}\NormalTok{, }\DecValTok{10}\NormalTok{, }\DecValTok{25}\NormalTok{, }\DecValTok{50}\NormalTok{)) }\OperatorTok{+}
\StringTok{    }\KeywordTok{scale_y_continuous}\NormalTok{(}\DataTypeTok{trans =} \StringTok{"log"}\NormalTok{, }
                       \DataTypeTok{breaks =} \KeywordTok{c}\NormalTok{(}\DecValTok{1}\NormalTok{, }\DecValTok{2}\NormalTok{, }\DecValTok{4}\NormalTok{, }\DecValTok{10}\NormalTok{, }\DecValTok{25}\NormalTok{, }\DecValTok{50}\NormalTok{, }\DecValTok{100}\NormalTok{, }\DecValTok{200}\NormalTok{)) }\OperatorTok{+}
\StringTok{    }\KeywordTok{theme_bw}\NormalTok{() }\OperatorTok{+}
\StringTok{    }\KeywordTok{labs}\NormalTok{(}\DataTypeTok{x =} \StringTok{"Cancer volume, in cubic centimeters"}\NormalTok{, }
         \DataTypeTok{y =} \StringTok{"Prostate Specific Antigen, in ng/ml"}\NormalTok{, }
         \DataTypeTok{title =} \StringTok{"Two Predictor Model c5_prost_A, including Interaction"}\NormalTok{)}
\end{Highlighting}
\end{Shaded}

\includegraphics{bookdown-demo_files/figure-latex/unnamed-chunk-54-1.pdf}

I've used the break point of 4 on the Y axis because of the old rule
suggesting further testing for asymptomatic men with PSA of 4 or higher,
but the other break points are arbitrary - they seemed to work for me,
and used round numbers.

\subsection{\texorpdfstring{Residual Plots of
\texttt{c5\_prost\_A}}{Residual Plots of c5\_prost\_A}}\label{residual-plots-of-c5_prost_a}

\begin{Shaded}
\begin{Highlighting}[]
\KeywordTok{plot}\NormalTok{(c5_prost_A, }\DataTypeTok{which =} \DecValTok{1}\NormalTok{)}
\end{Highlighting}
\end{Shaded}

\includegraphics{bookdown-demo_files/figure-latex/unnamed-chunk-55-1.pdf}

\begin{Shaded}
\begin{Highlighting}[]
\KeywordTok{plot}\NormalTok{(c5_prost_A, }\DataTypeTok{which =} \DecValTok{5}\NormalTok{)}
\end{Highlighting}
\end{Shaded}

\includegraphics{bookdown-demo_files/figure-latex/unnamed-chunk-56-1.pdf}

\section{\texorpdfstring{Cross-Validation of Model
\texttt{c5\_prost\_A}}{Cross-Validation of Model c5\_prost\_A}}\label{cross-validation-of-model-c5_prost_a}

Suppose we want to evaluate whether our model \texttt{c5\_prost\_A}
predicts effectively in new data.

One approach (used, for instance, in 431) would be to split our sample
into a separate training (perhaps 70\% of the data) and test (perhaps
30\% of the data) samples, and then:

\begin{itemize}
\item
  \begin{enumerate}
  \def\labelenumi{\arabic{enumi}.}
  \tightlist
  \item
    fit the model in the training sample,
  \end{enumerate}
\item
  \begin{enumerate}
  \def\labelenumi{\arabic{enumi}.}
  \setcounter{enumi}{1}
  \tightlist
  \item
    use the resulting model to make predictions for \texttt{lpsa} in the
    test sample, and
  \end{enumerate}
\item
  \begin{enumerate}
  \def\labelenumi{\arabic{enumi}.}
  \setcounter{enumi}{2}
  \tightlist
  \item
    evaluate the quality of those predictions, perhaps by comparing the
    results to what we'd get using a different model.
  \end{enumerate}
\end{itemize}

One problem with this approach is that with a small data set like this,
we may be reluctant to cut our sample size for the training or the
testing down because we're afraid that our model building and testing
will be hampered by a small sample size. A potential solution is the
idea of \textbf{cross-validation}, which involves partitioning our data
into a series of training-test subsets, multiple times, and then
combining the results.

The rest of this section is built on some material by David Robinson at
\url{https://rpubs.com/dgrtwo/cv-modelr}.

Suppose that we want to perform what is called \emph{10-crossfold
separation}. In words, this approach splits the 97 observations in our
\texttt{prost} data frame into 10 exclusive partitions of about 90\% (so
about 87-88 observations) into a training sample, and the remaining 10\%
(9-10 observations) in a test sample\footnote{If we did 5-crossfold
  validation, we'd have 5 partitions into samples of 80\% training and
  20\% test samples.}. We then refit a model of interest using the
training data, and fit the resulting model on the test data using the
\texttt{broom} package's \texttt{augment} function. This process is then
repeated (a total of 10 times) so that each observation is used 9 times
in the training sample, and once in the test sample.

To code this in R, we'll make use of a few new ideas. Our goal will be
to cross-validate model \texttt{c5\_prost\_A}, which, you'll recall,
uses \texttt{lcavol\_c}, \texttt{svi} and their interaction, to predict
\texttt{lpsa} in the \texttt{prost} data.

\begin{enumerate}
\def\labelenumi{\arabic{enumi}.}
\tightlist
\item
  First, we set a seed for the validation algorithm, so we can replicate
  our results down the line.
\item
  Then we use the \texttt{crossv\_kfold} function from the
  \texttt{modelr} package to split the \texttt{prost} data into ten
  different partitions, and then use each partition for a split into
  training and test samples, which the machine indexes with
  \texttt{train} and \texttt{test}.
\item
  Then we use some magic and the \texttt{map} function from the
  \texttt{purrr} package (part of the core \texttt{tidyverse}) to fit a
  new \texttt{lm(lpsa\ \textasciitilde{}\ lcavol\_c\ *\ svi)} model to
  each of the training samples generated by \texttt{crossv\_kfold}.
\item
  Finally, some additional magic with the \texttt{unnest} and
  \texttt{map2} functions applies each of these new models to the
  appropriate test sample, and generate predictions (\texttt{.fitted})
  and standard errors for each prediction (\texttt{.se.fit}).
\end{enumerate}

\begin{Shaded}
\begin{Highlighting}[]
\KeywordTok{set.seed}\NormalTok{(}\DecValTok{4320308}\NormalTok{)}

\NormalTok{prost_models <-}\StringTok{ }\NormalTok{prost }\OperatorTok
\StringTok{    }\KeywordTok{crossv_kfold}\NormalTok{(}\DataTypeTok{k =} \DecValTok{10}\NormalTok{) }\OperatorTok
\StringTok{    }\KeywordTok{mutate}\NormalTok{(}\DataTypeTok{model =} \KeywordTok{map}\NormalTok{(train, }\OperatorTok{~}\StringTok{ }\KeywordTok{lm}\NormalTok{(lpsa }\OperatorTok{~}\StringTok{ }\NormalTok{lcavol_c }\OperatorTok{*}\StringTok{ }\NormalTok{svi, }\DataTypeTok{data =}\NormalTok{ .)))}

\NormalTok{prost_predictions <-}\StringTok{ }\NormalTok{prost_models }\OperatorTok
\StringTok{    }\KeywordTok{unnest}\NormalTok{(}\KeywordTok{map2}\NormalTok{(model, test, }\OperatorTok{~}\StringTok{ }\KeywordTok{augment}\NormalTok{(.x, }\DataTypeTok{newdata =}\NormalTok{ .y)))}

\KeywordTok{head}\NormalTok{(prost_predictions)}
\end{Highlighting}
\end{Shaded}

\begin{verbatim}
# A tibble: 6 x 19
  .id   subject   lpsa   lcavol lweight   age bph      svi    lcp gleason
  <chr>   <int>  <dbl>    <dbl>   <dbl> <int> <fct>  <int>  <dbl> <fct>  
1 01          3 -0.163 -0.511      2.69    74 Low        0 -1.39  7      
2 01         12  1.27  -1.35       3.60    63 Medium     0 -1.39  6      
3 01         16  1.45   1.54       3.06    66 Low        0 -1.39  6      
4 01         18  1.49   2.29       3.65    66 Low        0  0.372 6      
5 01         30  1.89   2.41       3.38    65 Low        0  1.62  6      
6 01         34  2.02   0.00995    3.27    54 Low        0 -1.39  6      
# ... with 9 more variables: pgg45 <int>, svi_f <fct>, gleason_f <fct>,
#   bph_f <fct>, lcavol_c <dbl>, cavol <dbl>, psa <dbl>, .fitted <dbl>,
#   .se.fit <dbl>
\end{verbatim}

The results are a set of predictions based on the splits into training
and test groups (remember there are 10 such splits, indexed by
\texttt{.id}) that describe the complete set of 97 subjects again.

\subsection{Cross-Validated Summaries of Prediction
Quality}\label{cross-validated-summaries-of-prediction-quality}

Now, we can calculate the root Mean Squared Prediction Error (RMSE) and
Mean Absolute Prediction Error (MAE) for this modeling approach (using
\texttt{lcavol\_c} and \texttt{svi} to predict \texttt{lpsa}) across
these observations.

\begin{Shaded}
\begin{Highlighting}[]
\NormalTok{prost_predictions }\OperatorTok
\StringTok{    }\KeywordTok{summarize}\NormalTok{(}\DataTypeTok{RMSE_ourmodel =} \KeywordTok{sqrt}\NormalTok{(}\KeywordTok{mean}\NormalTok{((lpsa }\OperatorTok{-}\StringTok{ }\NormalTok{.fitted) }\OperatorTok{^}\DecValTok{2}\NormalTok{)),}
              \DataTypeTok{MAE_ourmodel =} \KeywordTok{mean}\NormalTok{(}\KeywordTok{abs}\NormalTok{(lpsa }\OperatorTok{-}\StringTok{ }\NormalTok{.fitted)))}
\end{Highlighting}
\end{Shaded}

\begin{verbatim}
# A tibble: 1 x 2
  RMSE_ourmodel MAE_ourmodel
          <dbl>        <dbl>
1         0.783        0.638
\end{verbatim}

For now, we'll compare our model to the ``intercept only'' model that
simply predicts the mean \texttt{lpsa} across all patients.

\begin{Shaded}
\begin{Highlighting}[]
\NormalTok{prost_predictions }\OperatorTok
\StringTok{    }\KeywordTok{summarize}\NormalTok{(}\DataTypeTok{RMSE_intercept =} \KeywordTok{sqrt}\NormalTok{(}\KeywordTok{mean}\NormalTok{((lpsa }\OperatorTok{-}\StringTok{ }\KeywordTok{mean}\NormalTok{(lpsa)) }\OperatorTok{^}\DecValTok{2}\NormalTok{)),}
              \DataTypeTok{MAE_intercept =} \KeywordTok{mean}\NormalTok{(}\KeywordTok{abs}\NormalTok{(lpsa }\OperatorTok{-}\StringTok{ }\KeywordTok{mean}\NormalTok{(lpsa))))}
\end{Highlighting}
\end{Shaded}

\begin{verbatim}
# A tibble: 1 x 2
  RMSE_intercept MAE_intercept
           <dbl>         <dbl>
1           1.15         0.891
\end{verbatim}

So our model looks meaningfully better than the ``intercept only''
model, in that both the RMSE and MAE are much lower (better) with our
model.

Another thing we could do with this tibble of predictions we have
created is to graph the size of the prediction errors (observed
\texttt{lpsa} minus predicted values in \texttt{.fitted}) that our
modeling approach makes.

\begin{Shaded}
\begin{Highlighting}[]
\NormalTok{prost_predictions }\OperatorTok
\StringTok{    }\KeywordTok{mutate}\NormalTok{(}\DataTypeTok{errors =}\NormalTok{ lpsa }\OperatorTok{-}\StringTok{ }\NormalTok{.fitted) }\OperatorTok
\StringTok{    }\KeywordTok{ggplot}\NormalTok{(., }\KeywordTok{aes}\NormalTok{(}\DataTypeTok{x =}\NormalTok{ errors)) }\OperatorTok{+}
\StringTok{    }\KeywordTok{geom_histogram}\NormalTok{(}\DataTypeTok{bins =} \DecValTok{30}\NormalTok{, }\DataTypeTok{fill =} \StringTok{"darkviolet"}\NormalTok{, }\DataTypeTok{col =} \StringTok{"yellow"}\NormalTok{) }\OperatorTok{+}\StringTok{ }
\StringTok{    }\KeywordTok{labs}\NormalTok{(}\DataTypeTok{title =} \StringTok{"Cross-Validated Errors in Prediction of log(PSA)"}\NormalTok{,}
         \DataTypeTok{subtitle =} \StringTok{"Using a model (`c5_prostA`) including lcavol_c and svi and their interaction"}\NormalTok{,}
         \DataTypeTok{x =} \StringTok{"Error in predicting log(PSA)"}\NormalTok{)}
\end{Highlighting}
\end{Shaded}

\includegraphics{bookdown-demo_files/figure-latex/validation_c5_prost_A_10fold_errors_histogram-1.pdf}

This suggests that some of our results are off by quite a bit, on the
log(PSA) scale, which is summarized for the original data below.

\begin{Shaded}
\begin{Highlighting}[]
\NormalTok{prost }\OperatorTok\StringTok{ }\KeywordTok{skim}\NormalTok{(lpsa)}
\end{Highlighting}
\end{Shaded}

\begin{verbatim}
Skim summary statistics
 n obs: 97 
 n variables: 16 

Variable type: numeric 
 variable missing complete  n mean   sd    p0  p25 median  p75 p100
     lpsa       0       97 97 2.48 1.15 -0.43 1.73   2.59 3.06 5.58
\end{verbatim}

If we like, we could transform the predictions and observed values back
to the scale of PSA (unlogged) and then calculate and display errors, as
follows:

\begin{Shaded}
\begin{Highlighting}[]
\NormalTok{prost_predictions }\OperatorTok
\StringTok{    }\KeywordTok{mutate}\NormalTok{(}\DataTypeTok{err.psa =} \KeywordTok{exp}\NormalTok{(lpsa) }\OperatorTok{-}\StringTok{ }\KeywordTok{exp}\NormalTok{(.fitted)) }\OperatorTok
\StringTok{    }\KeywordTok{ggplot}\NormalTok{(., }\KeywordTok{aes}\NormalTok{(}\DataTypeTok{x =}\NormalTok{ err.psa)) }\OperatorTok{+}
\StringTok{    }\KeywordTok{geom_histogram}\NormalTok{(}\DataTypeTok{bins =} \DecValTok{30}\NormalTok{, }\DataTypeTok{fill =} \StringTok{"darkorange"}\NormalTok{, }\DataTypeTok{col =} \StringTok{"yellow"}\NormalTok{) }\OperatorTok{+}\StringTok{ }
\StringTok{    }\KeywordTok{labs}\NormalTok{(}\DataTypeTok{title =} \StringTok{"Cross-Validated Errors in Prediction of PSA"}\NormalTok{,}
         \DataTypeTok{subtitle =} \StringTok{"Using a model (`c5_prostA`) including lcavol_c and svi and their interaction"}\NormalTok{,}
         \DataTypeTok{x =} \StringTok{"Error in predicting PSA"}\NormalTok{)}
\end{Highlighting}
\end{Shaded}

\includegraphics{bookdown-demo_files/figure-latex/validation_c5_prost_A_10fold_errorsonPSA_histogram-1.pdf}

This suggests that some of our results are off by quite a bit, on the
original scale of PSA, which is summarized below.

\begin{Shaded}
\begin{Highlighting}[]
\NormalTok{prost }\OperatorTok\StringTok{ }\KeywordTok{mutate}\NormalTok{(}\DataTypeTok{psa =} \KeywordTok{exp}\NormalTok{(lpsa)) }\OperatorTok\StringTok{ }\KeywordTok{skim}\NormalTok{(psa)}
\end{Highlighting}
\end{Shaded}

\begin{verbatim}
Skim summary statistics
 n obs: 97 
 n variables: 16 

Variable type: numeric 
 variable missing complete  n  mean    sd   p0  p25 median   p75   p100
      psa       0       97 97 23.74 40.83 0.65 5.65  13.35 21.25 265.85
\end{verbatim}

We'll return to the notion of cross-validation again, but for now, let's
consider the problem of considering adding more predictors to our model,
and then making sensible selections as to which predictors actually
should be incorporated.

\chapter{Stepwise Variable Selection}\label{stepwise-variable-selection}

\section{Strategy for Model
Selection}\label{strategy-for-model-selection}

\citet{RamseySchafer2002} suggest a strategy for dealing with many
potential explanatory variables should include the following elements:

\begin{enumerate}
\def\labelenumi{\arabic{enumi}.}
\tightlist
\item
  Identify the key objectives.
\item
  Screen the available variables, deciding on a list that is sensitive
  to the objectives and excludes obvious redundancies.
\item
  Perform exploratory analysis, examining graphical displays and
  correlation coefficients.
\item
  Perform transformations, as necessary.
\item
  Examine a residual plot after fitting a rich model, performing further
  transformations and considering outliers.
\item
  Find a suitable subset of the predictors, exerting enough control over
  any semi-automated selection procedure to be sensitive to the
  questions of interest.
\item
  Proceed with the analysis, using the selected explanatory variables.
\end{enumerate}

The Two Key Aspects of Model Selection are:

\begin{enumerate}
\def\labelenumi{\arabic{enumi}.}
\tightlist
\item
  Evaluating each potential subset of predictor variables
\item
  Deciding on the collection of potential subsets
\end{enumerate}

\subsection{How Do We Choose Potential Subsets of
Predictors?}\label{how-do-we-choose-potential-subsets-of-predictors}

Choosing potential subsets of predictor variables usually involves
either:

\begin{enumerate}
\def\labelenumi{\arabic{enumi}.}
\tightlist
\item
  Stepwise approaches
\item
  All possible subset (or best possible subset) searches
\end{enumerate}

Note that the use of any variable selection procedure changes the
properties of \ldots{}

\begin{itemize}
\tightlist
\item
  the estimated coefficients, which are biased, and
\item
  the associated tests and confidence intervals, which are overly
  optimistic.
\end{itemize}

\citet{Leeb2005} summarize the key issues:

\begin{enumerate}
\def\labelenumi{\arabic{enumi}.}
\tightlist
\item
  Regardless of sample size, the model selection step typically has a
  dramatic effect on the sampling properties of the estimators that
  cannot be ignored. In particular, the sampling properties of
  post-model-selection estimators are typically significantly different
  from the nominal distributions that arise if a fixed model is
  supposed.
\item
  As a consequence, use of inference procedures that do not take into
  account the model selection step (e.g.~using standard t-intervals as
  if the selected model has been given prior to the statistical
  analysis) can be highly misleading.
\end{enumerate}

\section{\texorpdfstring{A ``Kitchen Sink'' Model (Model
\texttt{c5\_prost\_ks})}{A Kitchen Sink Model (Model c5\_prost\_ks)}}\label{a-kitchen-sink-model-model-c5_prost_ks}

Suppose that we now consider a model for the \texttt{prost} data we have
been working with, which includes main effects (and, in this case, only
the main effects) of all eight candidate predictors for \texttt{lpsa},
as follows.

\begin{Shaded}
\begin{Highlighting}[]
\NormalTok{c5_prost_ks <-}\StringTok{ }\KeywordTok{lm}\NormalTok{(lpsa }\OperatorTok{~}\StringTok{ }\NormalTok{lcavol }\OperatorTok{+}\StringTok{ }\NormalTok{lweight }\OperatorTok{+}\StringTok{ }\NormalTok{age }\OperatorTok{+}\StringTok{ }\NormalTok{bph_f }\OperatorTok{+}\StringTok{ }\NormalTok{svi_f }\OperatorTok{+}\StringTok{ }
\StringTok{                }\NormalTok{lcp }\OperatorTok{+}\StringTok{ }\NormalTok{gleason_f }\OperatorTok{+}\StringTok{ }\NormalTok{pgg45, }\DataTypeTok{data =}\NormalTok{ prost)}

\KeywordTok{tidy}\NormalTok{(c5_prost_ks)}
\end{Highlighting}
\end{Shaded}

\begin{verbatim}
          term     estimate   std.error  statistic      p.value
1  (Intercept)  0.169937821 0.931332512  0.1824674 8.556454e-01
2       lcavol  0.544313829 0.087979210  6.1868461 2.010505e-08
3      lweight  0.702237531 0.203013089  3.4590751 8.455164e-04
4          age -0.023857982 0.011081414 -2.1529727 3.412099e-02
5  bph_fMedium  0.364036274 0.182575941  1.9938896 4.933267e-02
6    bph_fHigh  0.248789989 0.195975792  1.2694935 2.076898e-01
7     svi_fYes  0.710949408 0.241990241  2.9379259 4.240326e-03
8          lcp -0.119311781 0.089458946 -1.3337043 1.858223e-01
9   gleason_f7  0.220746268 0.343065609  0.6434520 5.216430e-01
10  gleason_f6 -0.053096704 0.430098039 -0.1234526 9.020368e-01
11       pgg45  0.003984574 0.004146495  0.9609499 3.392714e-01
\end{verbatim}

\begin{Shaded}
\begin{Highlighting}[]
\KeywordTok{glance}\NormalTok{(c5_prost_ks)}
\end{Highlighting}
\end{Shaded}

\begin{verbatim}
  r.squared adj.r.squared     sigma statistic      p.value df    logLik
1 0.6790343     0.6417127 0.6909479  18.19414 2.373796e-17 11 -95.93939
       AIC      BIC deviance df.residual
1 215.8788 246.7753 41.05718          86
\end{verbatim}

We'll often refer to this (all predictors on board) approach as a
``kitchen sink'' model{[}This refers to the English idiom ``\ldots{}
everything but the kitchen sink'' which describes, essentially,
everything imaginable. A ``kitchen sink regression'' is often used as a
pejorative term, since no special skill or insight is required to
identify it, given a list of potential predictors. For more, yes, there
is a
\href{https://en.wikipedia.org/wiki/Kitchen_sink_regression}{Wikipedia
page}.{]}.

\section{Sequential Variable Selection: Stepwise
Approaches}\label{sequential-variable-selection-stepwise-approaches}

\begin{itemize}
\tightlist
\item
  Forward Selection

  \begin{itemize}
  \tightlist
  \item
    We begin with a constant mean and then add potential predictors one
    at a time according to some criterion (R defaults to minimizing the
    Akaike Information Criterion) until no further addition
    significantly improves the fit.
  \item
    Each categorical factor variable is represented in the regression
    model as a set of indicator variables. In the absence of a good
    reason to do something else, the set is added to the model as a
    single unit, and R does this automatically.
  \end{itemize}
\item
  Backwards Elimination

  \begin{itemize}
  \tightlist
  \item
    Start with the ``kitchen sink'' model and then delete potential
    predictors one at a time.
  \item
    Backwards Elimination is less likely than Forward Selection, to omit
    negatively confounded sets of variables, though all stepwise
    procedures have problems.
  \end{itemize}
\item
  Stepwise Regression can also be done by combining these methods.
\end{itemize}

\subsection{The Big Problems with Stepwise
Regression}\label{the-big-problems-with-stepwise-regression}

There is no reason to assume that a single best model can be found.

\begin{itemize}
\tightlist
\item
  The use of forward selection, or backwards elimination, or stepwise
  regression including both procedures, will NOT always find the same
  model.
\item
  It also appears to be essentially useless to try different stepwise
  methods to look for agreement.
\end{itemize}

Users of stepwise regression frequently place all of their attention on
the particular explanatory variables included in the resulting model,
when there's \textbf{no reason} (in most cases) to assume that model is
in any way optimal.

Despite all of its problems, let's use stepwise regression to help
predict \texttt{lpsa} given a subset of the eight predictors in
\texttt{c5\_prost\_ks}.

\section{\texorpdfstring{Forward Selection with the \texttt{step}
function}{Forward Selection with the step function}}\label{forward-selection-with-the-step-function}

\begin{enumerate}
\def\labelenumi{\arabic{enumi}.}
\tightlist
\item
  Specify the null model (intercept only)
\item
  Specify the variables R should consider as predictors (in the scope
  element of the step function)
\item
  Specify forward selection only
\item
  R defaults to using AIC as its stepwise criterion
\end{enumerate}

\begin{Shaded}
\begin{Highlighting}[]
\KeywordTok{with}\NormalTok{(prost, }
     \KeywordTok{step}\NormalTok{(}\KeywordTok{lm}\NormalTok{(lpsa }\OperatorTok{~}\StringTok{ }\DecValTok{1}\NormalTok{), }
     \DataTypeTok{scope=}\NormalTok{(}\OperatorTok{~}\StringTok{ }\NormalTok{lcavol }\OperatorTok{+}\StringTok{ }\NormalTok{lweight }\OperatorTok{+}\StringTok{ }\NormalTok{age }\OperatorTok{+}\StringTok{ }\NormalTok{bph_f }\OperatorTok{+}\StringTok{ }\NormalTok{svi_f }\OperatorTok{+}\StringTok{ }
\StringTok{                }\NormalTok{lcp }\OperatorTok{+}\StringTok{ }\NormalTok{gleason_f }\OperatorTok{+}\StringTok{ }\NormalTok{pgg45), }
     \DataTypeTok{direction=}\StringTok{"forward"}\NormalTok{))}
\end{Highlighting}
\end{Shaded}

\begin{verbatim}
Start:  AIC=28.84
lpsa ~ 1

            Df Sum of Sq     RSS     AIC
+ lcavol     1    69.003  58.915 -44.366
+ svi_f      1    41.011  86.907  -6.658
+ lcp        1    38.528  89.389  -3.926
+ gleason_f  2    30.121  97.796   6.793
+ lweight    1    24.019 103.899  10.665
+ pgg45      1    22.814 105.103  11.783
+ age        1     3.679 124.239  28.007
<none>                   127.918  28.838
+ bph_f      2     4.681 123.237  29.221

Step:  AIC=-44.37
lpsa ~ lcavol

            Df Sum of Sq    RSS     AIC
+ lweight    1    7.1726 51.742 -54.958
+ svi_f      1    5.2375 53.677 -51.397
+ bph_f      2    3.2994 55.615 -45.956
+ pgg45      1    1.6980 57.217 -45.203
+ gleason_f  2    2.7834 56.131 -45.061
<none>                   58.915 -44.366
+ lcp        1    0.6562 58.259 -43.452
+ age        1    0.0025 58.912 -42.370

Step:  AIC=-54.96
lpsa ~ lcavol + lweight

            Df Sum of Sq    RSS     AIC
+ svi_f      1    5.1737 46.568 -63.177
+ pgg45      1    1.8158 49.926 -56.424
+ gleason_f  2    2.6770 49.065 -56.111
<none>                   51.742 -54.958
+ lcp        1    0.8187 50.923 -54.506
+ age        1    0.6456 51.097 -54.176
+ bph_f      2    1.4583 50.284 -53.731

Step:  AIC=-63.18
lpsa ~ lcavol + lweight + svi_f

            Df Sum of Sq    RSS     AIC
<none>                   46.568 -63.177
+ gleason_f  2   1.60467 44.964 -62.579
+ age        1   0.62301 45.945 -62.484
+ bph_f      2   1.50046 45.068 -62.354
+ pgg45      1   0.50069 46.068 -62.226
+ lcp        1   0.06937 46.499 -61.322
\end{verbatim}

\begin{verbatim}

Call:
lm(formula = lpsa ~ lcavol + lweight + svi_f)

Coefficients:
(Intercept)       lcavol      lweight     svi_fYes  
    -0.7772       0.5259       0.6618       0.6657  
\end{verbatim}

The resulting model, arrived at after three forward selection steps,
includes \texttt{lcavol}, \texttt{lweight} and \texttt{svi\_f}.

\begin{Shaded}
\begin{Highlighting}[]
\NormalTok{model.fs <-}\StringTok{ }\KeywordTok{lm}\NormalTok{(lpsa }\OperatorTok{~}\StringTok{ }\NormalTok{lcavol }\OperatorTok{+}\StringTok{ }\NormalTok{lweight }\OperatorTok{+}\StringTok{ }\NormalTok{svi_f, }
               \DataTypeTok{data=}\NormalTok{prost)}
\KeywordTok{summary}\NormalTok{(model.fs)}\OperatorTok{$}\NormalTok{adj.r.squared}
\end{Highlighting}
\end{Shaded}

\begin{verbatim}
[1] 0.6242063
\end{verbatim}

\begin{Shaded}
\begin{Highlighting}[]
\KeywordTok{extractAIC}\NormalTok{(model.fs)}
\end{Highlighting}
\end{Shaded}

\begin{verbatim}
[1]   4.00000 -63.17744
\end{verbatim}

The adjusted R\textsuperscript{2} value for this model is 0.624, and the
AIC value used by the stepwise procedure is -63.18, on 4 effective
degrees of freedom.

\section{\texorpdfstring{Backward Elimination using the \texttt{step}
function}{Backward Elimination using the step function}}\label{backward-elimination-using-the-step-function}

In this case, the backward elimination approach, using reduction in AIC
for a criterion, comes to the same conclusion about the ``best'' model.

\begin{Shaded}
\begin{Highlighting}[]
\KeywordTok{with}\NormalTok{(prost, }
     \KeywordTok{step}\NormalTok{(}\KeywordTok{lm}\NormalTok{(lpsa }\OperatorTok{~}\StringTok{ }\NormalTok{lcavol }\OperatorTok{+}\StringTok{ }\NormalTok{lweight }\OperatorTok{+}\StringTok{ }\NormalTok{age }\OperatorTok{+}\StringTok{ }\NormalTok{bph_f }\OperatorTok{+}\StringTok{ }
\StringTok{                 }\NormalTok{svi_f }\OperatorTok{+}\StringTok{ }\NormalTok{lcp }\OperatorTok{+}\StringTok{ }\NormalTok{gleason_f }\OperatorTok{+}\StringTok{ }\NormalTok{pgg45), }
          \DataTypeTok{direction=}\StringTok{"backward"}\NormalTok{))}
\end{Highlighting}
\end{Shaded}

\begin{verbatim}
Start:  AIC=-61.4
lpsa ~ lcavol + lweight + age + bph_f + svi_f + lcp + gleason_f + 
    pgg45

            Df Sum of Sq    RSS     AIC
- gleason_f  2    1.1832 42.240 -62.639
- pgg45      1    0.4409 41.498 -62.359
- lcp        1    0.8492 41.906 -61.409
<none>                   41.057 -61.395
- bph_f      2    2.0299 43.087 -60.714
- age        1    2.2129 43.270 -58.303
- svi_f      1    4.1207 45.178 -54.118
- lweight    1    5.7123 46.769 -50.760
- lcavol     1   18.2738 59.331 -27.683

Step:  AIC=-62.64
lpsa ~ lcavol + lweight + age + bph_f + svi_f + lcp + pgg45

          Df Sum of Sq    RSS     AIC
- lcp      1    0.8470 43.087 -62.713
<none>                 42.240 -62.639
- pgg45    1    1.2029 43.443 -61.916
- bph_f    2    2.2515 44.492 -61.602
- age      1    2.0730 44.313 -59.992
- svi_f    1    4.6431 46.884 -54.523
- lweight  1    5.5988 47.839 -52.566
- lcavol   1   21.4956 63.736 -24.736

Step:  AIC=-62.71
lpsa ~ lcavol + lweight + age + bph_f + svi_f + pgg45

          Df Sum of Sq    RSS     AIC
- pgg45    1    0.5860 43.673 -63.403
<none>                 43.087 -62.713
- bph_f    2    2.0214 45.109 -62.266
- age      1    1.7101 44.798 -60.938
- svi_f    1    3.7964 46.884 -56.523
- lweight  1    5.6462 48.734 -52.769
- lcavol   1   22.5152 65.603 -23.936

Step:  AIC=-63.4
lpsa ~ lcavol + lweight + age + bph_f + svi_f

          Df Sum of Sq    RSS     AIC
<none>                 43.673 -63.403
- bph_f    2    2.2720 45.945 -62.484
- age      1    1.3945 45.068 -62.354
- svi_f    1    5.2747 48.948 -54.343
- lweight  1    5.3319 49.005 -54.230
- lcavol   1   25.5538 69.227 -20.720
\end{verbatim}

\begin{verbatim}

Call:
lm(formula = lpsa ~ lcavol + lweight + age + bph_f + svi_f)

Coefficients:
(Intercept)       lcavol      lweight          age  bph_fMedium  
    0.14329      0.54022      0.67283     -0.01819      0.37607  
  bph_fHigh     svi_fYes  
    0.27216      0.68174  
\end{verbatim}

The backwards elimination approach in this case lands on a model with
five inputs (one of which includes two \texttt{bph} indicators,)
eliminating only \texttt{gleason\_f}, \texttt{pgg45} and \texttt{lcp}.

\section{Allen-Cady Modified Backward
Elimination}\label{allen-cady-modified-backward-elimination}

Ranking candidate predictors by importance in advance of backwards
elimination can help avoid false-positives, while reducing model size.
See \citet{Vittinghoff2012}, Section 10.3 for more details.

\begin{enumerate}
\def\labelenumi{\arabic{enumi}.}
\tightlist
\item
  First, force into the model any predictors of primary interest, and
  any confounders necessary for face validity of the final model.

  \begin{itemize}
  \tightlist
  \item
    ``Some variables in the hypothesized causal model may be such
    well-established causal antecedents of the outcome that it makes
    sense to include them, essentially to establish the face validity of
    the model and without regard to the strength or statistical
    significance of their associations with the primary predictor and
    outcome \ldots{}''
  \end{itemize}
\item
  Rank the remaining candidate predictors in order of importance.
\item
  Starting from an initial model with all candidate predictors included,
  delete predictors in order of ascending importance until the first
  variable meeting a criterion to stay in the model hits. Then stop.
\end{enumerate}

Only the remaining variable hypothesized to be least important is
eligible for removal at each step. When we are willing to do this
sorting before collecting (or analyzing) the data, then we can do
Allen-Cady backwards elimination using the \texttt{drop1} command in R.

\subsection{Demonstration of the Allen-Cady
approach}\label{demonstration-of-the-allen-cady-approach}

Suppose, for the moment that we decided to fit a model for the log of
\texttt{psa} and we decided (before we saw the data) that we would:

lcavol + lweight + svi\_f + age + bph\_f + gleason\_f + lcp + pgg45

\begin{itemize}
\tightlist
\item
  force the \texttt{gleason\_f} variable to be in the model, due to
  prior information about its importance,
\item
  and then rated the importance of the other variables as
  \texttt{lcavol} (most important), then \texttt{svi\_f} then
  \texttt{age}, and then \texttt{bph\_f}, then \texttt{lweight} and
  \texttt{lcp} followed by \texttt{pgg45} (least important)
\end{itemize}

When we are willing to do this sorting before collecting (or analyzing)
the data, then we can do Allen-Cady backwards elimination using the
\texttt{drop1} command in R.

\textbf{Step 1.} Fit the full model, then see if removing \texttt{pgg45}
improves AIC\ldots{}

\begin{Shaded}
\begin{Highlighting}[]
\KeywordTok{with}\NormalTok{(prost, }\KeywordTok{drop1}\NormalTok{(}\KeywordTok{lm}\NormalTok{(lpsa }\OperatorTok{~}\StringTok{ }\NormalTok{gleason_f }\OperatorTok{+}\StringTok{ }\NormalTok{lcavol }\OperatorTok{+}\StringTok{ }\NormalTok{svi_f }\OperatorTok{+}\StringTok{ }
\StringTok{              }\NormalTok{age }\OperatorTok{+}\StringTok{ }\NormalTok{bph_f }\OperatorTok{+}\StringTok{ }\NormalTok{lweight }\OperatorTok{+}\StringTok{ }\NormalTok{lcp }\OperatorTok{+}\StringTok{ }\NormalTok{pgg45),}
              \DataTypeTok{scope =}\NormalTok{ (}\OperatorTok{~}\StringTok{ }\NormalTok{pgg45)))}
\end{Highlighting}
\end{Shaded}

\begin{verbatim}
Single term deletions

Model:
lpsa ~ gleason_f + lcavol + svi_f + age + bph_f + lweight + lcp + 
    pgg45
       Df Sum of Sq    RSS     AIC
<none>              41.057 -61.395
pgg45   1   0.44085 41.498 -62.359
\end{verbatim}

Since -62.3 is smaller (i.e.~more negative) than -61.4, we delete
\texttt{pgg45} and move on to assess whether we can remove the variable
we deemed next least important (\texttt{lcp})

\textbf{Step 2.} Let's see if removing \texttt{lcp} from this model
improves AIC\ldots{}

\begin{Shaded}
\begin{Highlighting}[]
\KeywordTok{with}\NormalTok{(prost, }\KeywordTok{drop1}\NormalTok{(}\KeywordTok{lm}\NormalTok{(lpsa }\OperatorTok{~}\StringTok{ }\NormalTok{gleason_f }\OperatorTok{+}\StringTok{ }\NormalTok{lcavol }\OperatorTok{+}\StringTok{ }\NormalTok{svi_f }\OperatorTok{+}\StringTok{ }
\StringTok{              }\NormalTok{age }\OperatorTok{+}\StringTok{ }\NormalTok{bph_f }\OperatorTok{+}\StringTok{ }\NormalTok{lweight  }\OperatorTok{+}\StringTok{ }\NormalTok{lcp),}
              \DataTypeTok{scope =}\NormalTok{ (}\OperatorTok{~}\StringTok{ }\NormalTok{lcp)))}
\end{Highlighting}
\end{Shaded}

\begin{verbatim}
Single term deletions

Model:
lpsa ~ gleason_f + lcavol + svi_f + age + bph_f + lweight + lcp
       Df Sum of Sq    RSS     AIC
<none>              41.498 -62.359
lcp     1   0.56767 42.066 -63.041
\end{verbatim}

Again, since -63.0 is smaller than -62.4, we delete \texttt{lcp} and
next assess whether we should delete \texttt{lweight}.

\textbf{Step 3.} Does removing \texttt{lweight} from this model improves
AIC\ldots{}

\begin{Shaded}
\begin{Highlighting}[]
\KeywordTok{with}\NormalTok{(prost, }\KeywordTok{drop1}\NormalTok{(}\KeywordTok{lm}\NormalTok{(lpsa }\OperatorTok{~}\StringTok{ }\NormalTok{gleason_f }\OperatorTok{+}\StringTok{ }\NormalTok{lcavol }\OperatorTok{+}\StringTok{ }\NormalTok{svi_f }\OperatorTok{+}\StringTok{ }
\StringTok{              }\NormalTok{age }\OperatorTok{+}\StringTok{ }\NormalTok{bph_f }\OperatorTok{+}\StringTok{ }\NormalTok{lweight),}
              \DataTypeTok{scope =}\NormalTok{ (}\OperatorTok{~}\StringTok{ }\NormalTok{lweight)))}
\end{Highlighting}
\end{Shaded}

\begin{verbatim}
Single term deletions

Model:
lpsa ~ gleason_f + lcavol + svi_f + age + bph_f + lweight
        Df Sum of Sq    RSS     AIC
<none>               42.066 -63.041
lweight  1     5.678 47.744 -52.760
\end{verbatim}

Since the AIC for the model after the removal of \texttt{lweight} is
larger (i.e.~less negative), we stop, and declare our final model by the
Allen-Cady approach to include \texttt{gleason\_f}, \texttt{lcavol},
\texttt{svi\_f}, \texttt{age}, \texttt{bph\_f} and \texttt{lweight}.

\section{Summarizing the Results}\label{summarizing-the-results}

\begin{longtable}[]{@{}rl@{}}
\toprule
Method & Suggested Predictors\tabularnewline
\midrule
\endhead
Forward selection & \texttt{lcavol}, \texttt{lweight},
\texttt{svi\_f}\tabularnewline
Backwards elimination & \texttt{lcavol}, \texttt{lweight},
\texttt{svi\_f}, \texttt{age}, \texttt{bph\_f}\tabularnewline
Allen-Cady approach & \texttt{lcavol}, \texttt{lweight},
\texttt{svi\_f}, \texttt{age}, \texttt{bph\_f},
\texttt{gleason\_f}\tabularnewline
\bottomrule
\end{longtable}

\subsection{In-Sample Testing and
Summaries}\label{in-sample-testing-and-summaries}

Since these models are nested in each other, let's look at the summary
statistics (like R\textsuperscript{2}, and AIC) and also run an
ANOVA-based comparison of these nested models to each other and to the
model with the intercept alone, and the kitchen sink model with all
available predictors.

\begin{Shaded}
\begin{Highlighting}[]
\NormalTok{prost_m_int <-}\StringTok{ }\KeywordTok{lm}\NormalTok{(lpsa }\OperatorTok{~}\StringTok{ }\DecValTok{1}\NormalTok{, }\DataTypeTok{data =}\NormalTok{ prost)}
\NormalTok{prost_m_fw <-}\StringTok{ }\KeywordTok{lm}\NormalTok{(lpsa }\OperatorTok{~}\StringTok{ }\NormalTok{lcavol }\OperatorTok{+}\StringTok{ }\NormalTok{lweight }\OperatorTok{+}\StringTok{ }\NormalTok{svi_f, }\DataTypeTok{data =}\NormalTok{ prost)}
\NormalTok{prost_m_bw <-}\StringTok{ }\KeywordTok{lm}\NormalTok{(lpsa }\OperatorTok{~}\StringTok{ }\NormalTok{lcavol }\OperatorTok{+}\StringTok{ }\NormalTok{lweight }\OperatorTok{+}\StringTok{ }\NormalTok{svi_f }\OperatorTok{+}\StringTok{ }
\StringTok{              }\NormalTok{age }\OperatorTok{+}\StringTok{ }\NormalTok{bph_f }\OperatorTok{+}\StringTok{ }\NormalTok{gleason_f, }\DataTypeTok{data =}\NormalTok{ prost)}
\NormalTok{prost_m_ac <-}\StringTok{ }\KeywordTok{lm}\NormalTok{(lpsa }\OperatorTok{~}\StringTok{ }\NormalTok{lcavol }\OperatorTok{+}\StringTok{ }\NormalTok{lweight }\OperatorTok{+}\StringTok{ }\NormalTok{svi_f }\OperatorTok{+}\StringTok{ }
\StringTok{              }\NormalTok{age }\OperatorTok{+}\StringTok{ }\NormalTok{bph_f }\OperatorTok{+}\StringTok{ }\NormalTok{gleason_f }\OperatorTok{+}\StringTok{ }\NormalTok{lcp, }\DataTypeTok{data =}\NormalTok{ prost)}
\NormalTok{prost_m_ks <-}\StringTok{ }\KeywordTok{lm}\NormalTok{(lpsa }\OperatorTok{~}\StringTok{ }\NormalTok{lcavol }\OperatorTok{+}\StringTok{ }\NormalTok{lweight }\OperatorTok{+}\StringTok{ }\NormalTok{svi_f }\OperatorTok{+}\StringTok{ }
\StringTok{              }\NormalTok{age }\OperatorTok{+}\StringTok{ }\NormalTok{bph_f }\OperatorTok{+}\StringTok{ }\NormalTok{gleason_f }\OperatorTok{+}\StringTok{ }\NormalTok{lcp }\OperatorTok{+}\StringTok{ }\NormalTok{pgg45, }\DataTypeTok{data =}\NormalTok{ prost)}
\end{Highlighting}
\end{Shaded}

\subsubsection{\texorpdfstring{Model Fit Summaries (in-sample) from
\texttt{glance}}{Model Fit Summaries (in-sample) from glance}}\label{model-fit-summaries-in-sample-from-glance}

Here are the models, at a \texttt{glance} from the \texttt{broom}
package.

\begin{Shaded}
\begin{Highlighting}[]
\NormalTok{prost_sum <-}\StringTok{ }\KeywordTok{bind_rows}\NormalTok{(}\KeywordTok{glance}\NormalTok{(prost_m_int), }\KeywordTok{glance}\NormalTok{(prost_m_fw),}
                       \KeywordTok{glance}\NormalTok{(prost_m_bw), }\KeywordTok{glance}\NormalTok{(prost_m_ac), }
                       \KeywordTok{glance}\NormalTok{(prost_m_ks)) }\OperatorTok\StringTok{ }\KeywordTok{round}\NormalTok{(., }\DecValTok{3}\NormalTok{)}
\NormalTok{prost_sum}\OperatorTok{$}\NormalTok{names <-}\StringTok{ }\KeywordTok{c}\NormalTok{(}\StringTok{"intercept"}\NormalTok{, }\StringTok{"lcavol + lweight + svi"}\NormalTok{, }
                      \StringTok{"... + age + bhp + gleason"}\NormalTok{, }\StringTok{"... + lcp"}\NormalTok{, }\StringTok{"... + pgg45"}\NormalTok{)}
\NormalTok{prost_sum <-}\StringTok{ }\NormalTok{prost_sum }\OperatorTok
\StringTok{    }\KeywordTok{select}\NormalTok{(names, r.squared, adj.r.squared, AIC, BIC, sigma, df, df.residual)}

\NormalTok{prost_sum}
\end{Highlighting}
\end{Shaded}

\begin{verbatim}
                      names r.squared adj.r.squared     AIC     BIC sigma
1                 intercept     0.000         0.000 306.112 311.261 1.154
2    lcavol + lweight + svi     0.636         0.624 214.097 226.970 0.708
3 ... + age + bhp + gleason     0.671         0.641 214.233 239.980 0.691
4                 ... + lcp     0.676         0.642 214.915 243.237 0.691
5               ... + pgg45     0.679         0.642 215.879 246.775 0.691
  df df.residual
1  1          96
2  4          93
3  9          88
4 10          87
5 11          86
\end{verbatim}

From these summaries, it looks like:

\begin{itemize}
\tightlist
\item
  the adjusted R\textsuperscript{2} is essentially indistinguishable
  between the three largest models, but a bit less strong with the
  three-predictor (4 df) model, and
\item
  the AIC and BIC are (slightly) better (lower) with the three-predictor
  model (4 df) than any other.
\end{itemize}

So we might be motivated by these summaries to select any of the three
models we're studying closely here.

\subsubsection{Model Testing via ANOVA
(in-sample)}\label{model-testing-via-anova-in-sample}

To obtain ANOVA-based test results, we'll run\ldots{}

\begin{Shaded}
\begin{Highlighting}[]
\KeywordTok{anova}\NormalTok{(prost_m_int, prost_m_fw, prost_m_bw, prost_m_ac, prost_m_ks)}
\end{Highlighting}
\end{Shaded}

\begin{verbatim}
Analysis of Variance Table

Model 1: lpsa ~ 1
Model 2: lpsa ~ lcavol + lweight + svi_f
Model 3: lpsa ~ lcavol + lweight + svi_f + age + bph_f + gleason_f
Model 4: lpsa ~ lcavol + lweight + svi_f + age + bph_f + gleason_f + lcp
Model 5: lpsa ~ lcavol + lweight + svi_f + age + bph_f + gleason_f + lcp + 
    pgg45
  Res.Df     RSS Df Sum of Sq       F Pr(>F)    
1     96 127.918                                
2     93  46.568  3    81.349 56.7991 <2e-16 ***
3     88  42.066  5     4.503  1.8863 0.1050    
4     87  41.498  1     0.568  1.1891 0.2786    
5     86  41.057  1     0.441  0.9234 0.3393    
---
Signif. codes:  0 '***' 0.001 '**' 0.01 '*' 0.05 '.' 0.1 ' ' 1
\end{verbatim}

What conclusions can we draw on the basis of these ANOVA tests?

\begin{itemize}
\tightlist
\item
  There is a statistically significant improvement in predictive value
  for Model 2 (the forward selection approach) as compared to Model 1
  (the intercept only.)
\item
  The ANOVA test comparing Model 5 (kitchen sink) to Model 4 (Allen-Cady
  result) shows no statistically significant improvement in predictive
  value.
\item
  Neither does the ANOVA test comparing Model 3 to Model 2 or Model 4 to
  Model 3.
\end{itemize}

This suggests that, \textbf{if we are willing to let the ANOVA test
decide our best model} than that would be the model produced by forward
selection, with predictors \texttt{lcavol}, \texttt{lweight} and
\texttt{svi\_f}. But we haven't validated the models.

\begin{enumerate}
\def\labelenumi{\arabic{enumi}.}
\tightlist
\item
  If the purpose of the model is to predict new data, some sort of
  out-of-sample or cross-validation approach will be necessary, and
\item
  Even if our goal isn't prediction but merely description of the
  current data, we would still want to build diagnostic plots to
  regression assumptions in each model, and
\item
  There is no reason to assume in advance that any of these models is in
  fact correct, or that any one of these stepwise approaches is
  necessarily better than any other, and
\item
  The mere act of running a stepwise regression model, as we'll see, can
  increase the bias in our findings if we accept the results at face
  value.
\end{enumerate}

So we'll need some ways to validate the results once we complete the
selection process.

\subsection{Validating the Results of the Various
Models}\label{validating-the-results-of-the-various-models}

We can use a 5-fold cross-validation approach to assess the predictions
made by our potential models and then compare them. Let's compare our
three models:

\begin{itemize}
\tightlist
\item
  the three predictor model obtained by forward selection, including
  \texttt{lcavol}, \texttt{lweight}, and \texttt{svi\_f}
\item
  the five predictor model obtained by backwards elimination, including
  \texttt{lcavol}, \texttt{lweight}, \texttt{svi\_f}, and also
  \texttt{age}, and \texttt{bph\_f}
\item
  the six predictor model obtained by the Allen-Cady approach, adding
  \texttt{gleason\_f} to the previous model.
\end{itemize}

Here's the 5-fold validation work (and resulting RMSE and MAE estimates)
for the three-predictor model.

\begin{Shaded}
\begin{Highlighting}[]
\KeywordTok{set.seed}\NormalTok{(}\DecValTok{43201012}\NormalTok{)}

\NormalTok{prost3_models <-}\StringTok{ }\NormalTok{prost }\OperatorTok
\StringTok{    }\KeywordTok{crossv_kfold}\NormalTok{(}\DataTypeTok{k =} \DecValTok{5}\NormalTok{) }\OperatorTok
\StringTok{    }\KeywordTok{mutate}\NormalTok{(}\DataTypeTok{model =} \KeywordTok{map}\NormalTok{(train, }\OperatorTok{~}\StringTok{ }\KeywordTok{lm}\NormalTok{(lpsa }\OperatorTok{~}\StringTok{ }\NormalTok{lcavol }\OperatorTok{+}\StringTok{ }\NormalTok{lweight }\OperatorTok{+}\StringTok{ }
\StringTok{                                       }\NormalTok{svi_f, }\DataTypeTok{data =}\NormalTok{ .)))}

\NormalTok{prost3_preds <-}\StringTok{ }\NormalTok{prost3_models }\OperatorTok
\StringTok{    }\KeywordTok{unnest}\NormalTok{(}\KeywordTok{map2}\NormalTok{(model, test, }\OperatorTok{~}\StringTok{ }\KeywordTok{augment}\NormalTok{(.x, }\DataTypeTok{newdata =}\NormalTok{ .y)))}

\NormalTok{prost3_preds }\OperatorTok
\StringTok{    }\KeywordTok{summarize}\NormalTok{(}\DataTypeTok{RMSE_prost3 =} \KeywordTok{sqrt}\NormalTok{(}\KeywordTok{mean}\NormalTok{((lpsa }\OperatorTok{-}\StringTok{ }\NormalTok{.fitted) }\OperatorTok{^}\DecValTok{2}\NormalTok{)),}
              \DataTypeTok{MAE_prost3 =} \KeywordTok{mean}\NormalTok{(}\KeywordTok{abs}\NormalTok{(lpsa }\OperatorTok{-}\StringTok{ }\NormalTok{.fitted)))}
\end{Highlighting}
\end{Shaded}

\begin{verbatim}
# A tibble: 1 x 2
  RMSE_prost3 MAE_prost3
        <dbl>      <dbl>
1       0.745      0.587
\end{verbatim}

Now, we'll generate the RMSE and MAE estimates for the five-predictor
model.

\begin{Shaded}
\begin{Highlighting}[]
\KeywordTok{set.seed}\NormalTok{(}\DecValTok{43206879}\NormalTok{)}

\NormalTok{prost5_models <-}\StringTok{ }\NormalTok{prost }\OperatorTok
\StringTok{    }\KeywordTok{crossv_kfold}\NormalTok{(}\DataTypeTok{k =} \DecValTok{5}\NormalTok{) }\OperatorTok
\StringTok{    }\KeywordTok{mutate}\NormalTok{(}\DataTypeTok{model =} \KeywordTok{map}\NormalTok{(train, }\OperatorTok{~}\StringTok{ }\KeywordTok{lm}\NormalTok{(lpsa }\OperatorTok{~}\StringTok{ }\NormalTok{lcavol }\OperatorTok{+}\StringTok{ }\NormalTok{lweight }\OperatorTok{+}\StringTok{ }
\StringTok{                                       }\NormalTok{svi_f }\OperatorTok{+}\StringTok{ }\NormalTok{age }\OperatorTok{+}\StringTok{ }\NormalTok{bph_f, }\DataTypeTok{data =}\NormalTok{ .)))}

\NormalTok{prost5_preds <-}\StringTok{ }\NormalTok{prost5_models }\OperatorTok
\StringTok{    }\KeywordTok{unnest}\NormalTok{(}\KeywordTok{map2}\NormalTok{(model, test, }\OperatorTok{~}\StringTok{ }\KeywordTok{augment}\NormalTok{(.x, }\DataTypeTok{newdata =}\NormalTok{ .y)))}

\NormalTok{prost5_preds }\OperatorTok
\StringTok{    }\KeywordTok{summarize}\NormalTok{(}\DataTypeTok{RMSE_prost5 =} \KeywordTok{sqrt}\NormalTok{(}\KeywordTok{mean}\NormalTok{((lpsa }\OperatorTok{-}\StringTok{ }\NormalTok{.fitted) }\OperatorTok{^}\DecValTok{2}\NormalTok{)),}
              \DataTypeTok{MAE_prost5 =} \KeywordTok{mean}\NormalTok{(}\KeywordTok{abs}\NormalTok{(lpsa }\OperatorTok{-}\StringTok{ }\NormalTok{.fitted)))}
\end{Highlighting}
\end{Shaded}

\begin{verbatim}
# A tibble: 1 x 2
  RMSE_prost5 MAE_prost5
        <dbl>      <dbl>
1       0.750      0.581
\end{verbatim}

And at last, we'll generate the RMSE and MAE estimates for the
six-predictor model.

\begin{Shaded}
\begin{Highlighting}[]
\KeywordTok{set.seed}\NormalTok{(}\DecValTok{43236198}\NormalTok{)}

\NormalTok{prost6_models <-}\StringTok{ }\NormalTok{prost }\OperatorTok
\StringTok{    }\KeywordTok{crossv_kfold}\NormalTok{(}\DataTypeTok{k =} \DecValTok{5}\NormalTok{) }\OperatorTok
\StringTok{    }\KeywordTok{mutate}\NormalTok{(}\DataTypeTok{model =} \KeywordTok{map}\NormalTok{(train, }\OperatorTok{~}\StringTok{ }\KeywordTok{lm}\NormalTok{(lpsa }\OperatorTok{~}\StringTok{ }\NormalTok{lcavol }\OperatorTok{+}\StringTok{ }\NormalTok{lweight }\OperatorTok{+}\StringTok{ }
\StringTok{                                       }\NormalTok{svi_f }\OperatorTok{+}\StringTok{ }\NormalTok{age }\OperatorTok{+}\StringTok{ }\NormalTok{bph_f }\OperatorTok{+}\StringTok{ }\NormalTok{gleason_f, }\DataTypeTok{data =}\NormalTok{ .)))}

\NormalTok{prost6_preds <-}\StringTok{ }\NormalTok{prost6_models }\OperatorTok
\StringTok{    }\KeywordTok{unnest}\NormalTok{(}\KeywordTok{map2}\NormalTok{(model, test, }\OperatorTok{~}\StringTok{ }\KeywordTok{augment}\NormalTok{(.x, }\DataTypeTok{newdata =}\NormalTok{ .y)))}

\NormalTok{prost6_preds }\OperatorTok
\StringTok{    }\KeywordTok{summarize}\NormalTok{(}\DataTypeTok{RMSE_prost6 =} \KeywordTok{sqrt}\NormalTok{(}\KeywordTok{mean}\NormalTok{((lpsa }\OperatorTok{-}\StringTok{ }\NormalTok{.fitted) }\OperatorTok{^}\DecValTok{2}\NormalTok{)),}
              \DataTypeTok{MAE_prost6 =} \KeywordTok{mean}\NormalTok{(}\KeywordTok{abs}\NormalTok{(lpsa }\OperatorTok{-}\StringTok{ }\NormalTok{.fitted)))}
\end{Highlighting}
\end{Shaded}

\begin{verbatim}
# A tibble: 1 x 2
  RMSE_prost6 MAE_prost6
        <dbl>      <dbl>
1       0.720      0.551
\end{verbatim}

It appears that the six-predictor model does better than either of the
other two approaches, with smaller RMSE and MAE. The three-predictor
model does slightly better in terms of root mean square prediction error
and slightly worse in terms of mean absolute prediction error than the
five-predictor model.

OK. A mixed bag, with different conclusions depending on which summary
we want to look at. But of course, stepwise regression isn't the only
way to do variable selection. Let's consider a broader range of
potential predictor sets.

\chapter{\texorpdfstring{``Best Subsets'' Variable Selection in our
Prostate Cancer
Study}{Best Subsets Variable Selection in our Prostate Cancer Study}}\label{best-subsets-variable-selection-in-our-prostate-cancer-study}

A second approach to model selection involved fitting all possible
subset models and identifying the ones that look best according to some
meaningful criterion and ideally one that includes enough variables to
model the response appropriately without including lots of redundant or
unnecessary terms.

\section{Four Key Summaries We'll Use to Evaluate Potential
Models}\label{four-key-summaries-well-use-to-evaluate-potential-models}

\begin{enumerate}
\def\labelenumi{\arabic{enumi}.}
\tightlist
\item
  Adjusted R\textsuperscript{2}, which we try to maximize.
\item
  Akaike's Information Criterion (AIC), which we try to minimize, and a
  Bias-Corrected version of AIC due to \citet{HurvichTsai1989}, which we
  use when the sample size is small, specifically when the sample size
  \(n\) and the number of predictors being studied \(k\) are such that
  \(n/k \leq 40\). We also try to minimize this bias-corrected AIC.
\item
  Bayesian Information Criterion (BIC), which we also try to minimize.
\item
  Mallows' C\textsubscript{p} statistic, which we (essentially) try to
  minimize.
\end{enumerate}

Choosing between AIC and BIC can be challenging.

\begin{quote}
For model selection purposes, there is no clear choice between AIC and
BIC. Given a family of models, including the true model, the probability
that BIC will select the correct model approaches one as the sample size
n approaches infinity - thus BIC is asymptotically consistent, which AIC
is not. {[}But, for practical purposes,{]} BIC often chooses models that
are too simple {[}relative to AIC{]} because of its heavy penalty on
complexity.
\end{quote}

\begin{itemize}
\tightlist
\item
  Source: \citet{Hastie2001}, page 208.
\end{itemize}

Several useful tools for running ``all subsets'' or ``best subsets''
regression comparisons are developed in R's \texttt{leaps} package.

\section{\texorpdfstring{Using \texttt{regsubsets} in the \texttt{leaps}
package}{Using regsubsets in the leaps package}}\label{using-regsubsets-in-the-leaps-package}

We can use the \texttt{leaps} package to obtain results in the
\texttt{prost} study from looking at all possible subsets of the
candidate predictors. The \texttt{leaps} package isn't particularly
friendly to the tidyverse. In particular, we \textbf{cannot have any
character variables} in our predictor set. We specify our ``kitchen
sink'' model, and apply the \texttt{regsubsets} function from
\texttt{leaps}, which identifies the set of models.

To start, we'll ask R to find the one best subset (with 1 predictor
variable {[}in addition to the intercept{]}, then with 2 predictors, and
then with each of 3, 4, \ldots{} 8 predictor variables) according to an
exhaustive search without forcing any of the variables to be in or out.

\begin{itemize}
\tightlist
\item
  Use the \texttt{nvmax} command within the \texttt{regsubsets} function
  to limit the number of regression inputs to a maximum.
\item
  Use the \texttt{nbest} command to identify how many subsets you want
  to identify for each predictor count.
\item
  If all of your predictors are \textbf{quantitative} or \textbf{binary}
  then you can skip the \texttt{preds} step, and simply place your
  kitchen sink model into \texttt{regsubsets}.
\item
  But if you have multi-categorical variables (like \texttt{gleason\_f}
  or \texttt{svi\_f} in our case) then you must create a \texttt{preds}
  group, as follows.
\end{itemize}

\begin{Shaded}
\begin{Highlighting}[]
\NormalTok{preds <-}\StringTok{ }\KeywordTok{with}\NormalTok{(prost, }\KeywordTok{cbind}\NormalTok{(lcavol, lweight, age, bph_f, }
\NormalTok{                           svi_f, lcp, gleason_f, pgg45))}

\NormalTok{rs.ks <-}\StringTok{ }\KeywordTok{regsubsets}\NormalTok{(preds, }\DataTypeTok{y =}\NormalTok{ prost}\OperatorTok{$}\NormalTok{lpsa, }
                    \DataTypeTok{nvmax =} \DecValTok{8}\NormalTok{, }\DataTypeTok{nbest =} \DecValTok{1}\NormalTok{)}
\NormalTok{rs.summ <-}\StringTok{ }\KeywordTok{summary}\NormalTok{(rs.ks)}
\NormalTok{rs.summ}
\end{Highlighting}
\end{Shaded}

\begin{verbatim}
Subset selection object
8 Variables  (and intercept)
          Forced in Forced out
lcavol        FALSE      FALSE
lweight       FALSE      FALSE
age           FALSE      FALSE
bph_f         FALSE      FALSE
svi_f         FALSE      FALSE
lcp           FALSE      FALSE
gleason_f     FALSE      FALSE
pgg45         FALSE      FALSE
1 subsets of each size up to 8
Selection Algorithm: exhaustive
         lcavol lweight age bph_f svi_f lcp gleason_f pgg45
1  ( 1 ) "*"    " "     " " " "   " "   " " " "       " "  
2  ( 1 ) "*"    "*"     " " " "   " "   " " " "       " "  
3  ( 1 ) "*"    "*"     " " " "   "*"   " " " "       " "  
4  ( 1 ) "*"    "*"     " " "*"   "*"   " " " "       " "  
5  ( 1 ) "*"    "*"     "*" "*"   "*"   " " " "       " "  
6  ( 1 ) "*"    "*"     "*" "*"   "*"   " " "*"       " "  
7  ( 1 ) "*"    "*"     "*" "*"   "*"   "*" "*"       " "  
8  ( 1 ) "*"    "*"     "*" "*"   "*"   "*" "*"       "*"  
\end{verbatim}

So\ldots{}

\begin{itemize}
\tightlist
\item
  the best one-predictor model used \texttt{lcavol}
\item
  the best two-predictor model used \texttt{lcavol} and \texttt{lweight}
\item
  the best three-predictor model used \texttt{lcavol}, \texttt{lweight}
  and \texttt{svi\_f}
\item
  the best four-predictor model added \texttt{bph\_f}, and
\item
  the best five-predictor model added \texttt{age}
\item
  the best six-input model added \texttt{gleason\_f},
\item
  the best seven-input model added \texttt{lcp},
\item
  and the eight-input model adds \texttt{pgg45}.
\end{itemize}

All of these ``best subsets'' are hierarchical, in that each model is a
subset of the one below it. This isn't inevitably true.

\begin{itemize}
\tightlist
\item
  To determine which model is best, we can plot key summaries of model
  fit (adjusted R\textsuperscript{2}, Mallows' \(C_p\), bias-corrected
  AIC, and BIC) using either base R plotting techniques (what I've done
  in the past) or \texttt{ggplot2} (what I use now.) I'll show both
  types of plotting approaches in the next two sections.
\end{itemize}

\subsection{\texorpdfstring{Identifying the models with \texttt{which}
and
\texttt{outmat}}{Identifying the models with which and outmat}}\label{identifying-the-models-with-which-and-outmat}

To see the models selected by the system, we use:

\begin{Shaded}
\begin{Highlighting}[]
\NormalTok{rs.summ}\OperatorTok{$}\NormalTok{which}
\end{Highlighting}
\end{Shaded}

\begin{verbatim}
  (Intercept) lcavol lweight   age bph_f svi_f   lcp gleason_f pgg45
1        TRUE   TRUE   FALSE FALSE FALSE FALSE FALSE     FALSE FALSE
2        TRUE   TRUE    TRUE FALSE FALSE FALSE FALSE     FALSE FALSE
3        TRUE   TRUE    TRUE FALSE FALSE  TRUE FALSE     FALSE FALSE
4        TRUE   TRUE    TRUE FALSE  TRUE  TRUE FALSE     FALSE FALSE
5        TRUE   TRUE    TRUE  TRUE  TRUE  TRUE FALSE     FALSE FALSE
6        TRUE   TRUE    TRUE  TRUE  TRUE  TRUE FALSE      TRUE FALSE
7        TRUE   TRUE    TRUE  TRUE  TRUE  TRUE  TRUE      TRUE FALSE
8        TRUE   TRUE    TRUE  TRUE  TRUE  TRUE  TRUE      TRUE  TRUE
\end{verbatim}

Another version of this formatted for printing is:

\begin{Shaded}
\begin{Highlighting}[]
\NormalTok{rs.summ}\OperatorTok{$}\NormalTok{outmat}
\end{Highlighting}
\end{Shaded}

\begin{verbatim}
         lcavol lweight age bph_f svi_f lcp gleason_f pgg45
1  ( 1 ) "*"    " "     " " " "   " "   " " " "       " "  
2  ( 1 ) "*"    "*"     " " " "   " "   " " " "       " "  
3  ( 1 ) "*"    "*"     " " " "   "*"   " " " "       " "  
4  ( 1 ) "*"    "*"     " " "*"   "*"   " " " "       " "  
5  ( 1 ) "*"    "*"     "*" "*"   "*"   " " " "       " "  
6  ( 1 ) "*"    "*"     "*" "*"   "*"   " " "*"       " "  
7  ( 1 ) "*"    "*"     "*" "*"   "*"   "*" "*"       " "  
8  ( 1 ) "*"    "*"     "*" "*"   "*"   "*" "*"       "*"  
\end{verbatim}

We built one subset of each size up to eight predictors, and if we add
the intercept term, this means we have models of size k = 2, 3, 4, 5, 6,
7, 8 and 9.

The models are:

\begin{longtable}[]{@{}rl@{}}
\toprule
Size k & Predictors included (besides intercept)\tabularnewline
\midrule
\endhead
2 & \texttt{lcavol}\tabularnewline
3 & \texttt{lcavol} and \texttt{lweight}\tabularnewline
4 & add \texttt{svi\_f}\tabularnewline
5 & add \texttt{bph\_f}\tabularnewline
6 & add \texttt{age}\tabularnewline
7 & add \texttt{gleason\_f}\tabularnewline
8 & add \texttt{lcp}\tabularnewline
9 & add \texttt{pgg45}\tabularnewline
\bottomrule
\end{longtable}

\section{Calculating bias-corrected
AIC}\label{calculating-bias-corrected-aic}

The bias-corrected AIC formula developed in \citet{HurvichTsai1989}
requires three inputs:

\begin{itemize}
\tightlist
\item
  the residual sum of squares for a model
\item
  the sample size (n) or number of observations used to fit the model
\item
  the number of regression inputs, k, including the intercept, used in
  the model
\end{itemize}

So, for a particular model fit to \emph{n} observations, on \emph{k}
predictors (including the intercept) and a residual sum of squares equal
to RSS, we have:

\[
AIC_c = n log(\frac{RSS}{n}) + 2k + \frac{2k (k+1)}{n-k-1}
\]

Note that the corrected \(AIC_c\) can be related to the original AIC
via:

\[
AIC_c = AIC + \frac{2k (k+1)}{n - k - 1}
\]

\subsection{Calculation of aic.c in our
setting}\label{calculation-of-aic.c-in-our-setting}

In our case, we have \(n\) = 97 observations, and built a series of
models with \(k\) = \texttt{2:9} predictors (including the intercept in
each case), so we will insert those values into the general formula for
bias-corrected AIC which is:

\begin{verbatim}
aic.c <- n * log( rs.summ$rss / n) + 2 * k + 
                      (2 * k * (k + 1) / (n - k - 1))
\end{verbatim}

We can obtain the residual sum of squares explained by each model by
pulling \texttt{rss} from the \texttt{regsubsets} summary contained here
in \texttt{rs.summ}.

\begin{Shaded}
\begin{Highlighting}[]
\KeywordTok{data_frame}\NormalTok{(}\DataTypeTok{k =} \DecValTok{2}\OperatorTok{:}\DecValTok{9}\NormalTok{, }\DataTypeTok{RSS =}\NormalTok{ rs.summ}\OperatorTok{$}\NormalTok{rss)}
\end{Highlighting}
\end{Shaded}

\begin{verbatim}
# A tibble: 8 x 2
      k   RSS
  <int> <dbl>
1     2  58.9
2     3  51.7
3     4  46.6
4     5  45.7
5     6  44.6
6     7  43.7
7     8  43.0
8     9  42.8
\end{verbatim}

In this case, we have:

\begin{Shaded}
\begin{Highlighting}[]
\NormalTok{rs.summ}\OperatorTok{$}\NormalTok{aic.c <-}\StringTok{ }\DecValTok{97}\OperatorTok{*}\KeywordTok{log}\NormalTok{(rs.summ}\OperatorTok{$}\NormalTok{rss }\OperatorTok{/}\StringTok{ }\DecValTok{97}\NormalTok{) }\OperatorTok{+}\StringTok{ }\DecValTok{2}\OperatorTok{*}\NormalTok{(}\DecValTok{2}\OperatorTok{:}\DecValTok{9}\NormalTok{) }\OperatorTok{+}
\StringTok{               }\NormalTok{(}\DecValTok{2} \OperatorTok{*}\StringTok{ }\NormalTok{(}\DecValTok{2}\OperatorTok{:}\DecValTok{9}\NormalTok{) }\OperatorTok{*}\StringTok{ }\NormalTok{((}\DecValTok{2}\OperatorTok{:}\DecValTok{9}\NormalTok{)}\OperatorTok{+}\DecValTok{1}\NormalTok{) }\OperatorTok{/}\StringTok{ }\NormalTok{(}\DecValTok{97} \OperatorTok{-}\StringTok{ }\NormalTok{(}\DecValTok{2}\OperatorTok{:}\DecValTok{9}\NormalTok{) }\OperatorTok{-}\StringTok{ }\DecValTok{1}\NormalTok{))}

\KeywordTok{round}\NormalTok{(rs.summ}\OperatorTok{$}\NormalTok{aic.c,}\DecValTok{2}\NormalTok{) }\CommentTok{# bias-corrected}
\end{Highlighting}
\end{Shaded}

\begin{verbatim}
[1] -44.24 -54.70 -62.74 -62.29 -62.34 -62.11 -61.17 -59.36
\end{verbatim}

The impact of this bias correction can be modest but important. Here's a
little table looking closely at the results in this problem. The
uncorrected AIC are obtained using \texttt{extractAIC}, as described in
the next section.

\begin{longtable}[]{@{}rrrrrrrrr@{}}
\toprule
Size & 2 & 3 & 4 & 5 & 6 & 7 & 8 & 9\tabularnewline
\midrule
\endhead
Bias-corrected AIC & -44.2 & -54.7 & -62.7 & -62.3 & -62.3 & -62.1 &
-61.2 & -59.4\tabularnewline
Uncorrected AIC & -44.4 & -55.0 & -63.2 & -62.4 & -63.4 & -63.0 & -62.4
& -61.4\tabularnewline
\bottomrule
\end{longtable}

\subsection{The Uncorrected AIC provides no more useful information
here}\label{the-uncorrected-aic-provides-no-more-useful-information-here}

We could, if necessary, also calculate the \emph{uncorrected}
\texttt{aic} value for each model, but we won't make any direct use of
that, because that will not provide any new information not already
gathered by the \(C_p\) statistic for a linear regression model. If you
wanted to find the uncorrected AIC for a given model, you can use the
\texttt{extractAIC} function.

\begin{Shaded}
\begin{Highlighting}[]
\KeywordTok{extractAIC}\NormalTok{(}\KeywordTok{lm}\NormalTok{(lpsa }\OperatorTok{~}\StringTok{ }\NormalTok{lcavol, }\DataTypeTok{data =}\NormalTok{ prost))}
\end{Highlighting}
\end{Shaded}

\begin{verbatim}
[1]   2.00000 -44.36603
\end{verbatim}

\begin{Shaded}
\begin{Highlighting}[]
\KeywordTok{extractAIC}\NormalTok{(}\KeywordTok{lm}\NormalTok{(lpsa }\OperatorTok{~}\StringTok{ }\NormalTok{lcavol }\OperatorTok{+}\StringTok{ }\NormalTok{lweight, }\DataTypeTok{data =}\NormalTok{ prost))}
\end{Highlighting}
\end{Shaded}

\begin{verbatim}
[1]   3.00000 -54.95846
\end{verbatim}

Note that:

\begin{itemize}
\tightlist
\item
  these results are fairly comparable to the bias-corrected AIC we built
  above, and
\item
  the \texttt{extractAIC} and \texttt{AIC} functions look like they give
  very different results, but they really don't.
\end{itemize}

\begin{Shaded}
\begin{Highlighting}[]
\KeywordTok{AIC}\NormalTok{(}\KeywordTok{lm}\NormalTok{(lpsa }\OperatorTok{~}\StringTok{ }\NormalTok{lcavol, }\DataTypeTok{data =}\NormalTok{ prost))}
\end{Highlighting}
\end{Shaded}

\begin{verbatim}
[1] 232.908
\end{verbatim}

\begin{Shaded}
\begin{Highlighting}[]
\KeywordTok{AIC}\NormalTok{(}\KeywordTok{lm}\NormalTok{(lpsa }\OperatorTok{~}\StringTok{ }\NormalTok{lcavol }\OperatorTok{+}\StringTok{ }\NormalTok{lweight, }\DataTypeTok{data =}\NormalTok{ prost))}
\end{Highlighting}
\end{Shaded}

\begin{verbatim}
[1] 222.3156
\end{verbatim}

But notice that the differences in AIC are the same, either way,
comparing these two models:

\begin{Shaded}
\begin{Highlighting}[]
\KeywordTok{extractAIC}\NormalTok{(}\KeywordTok{lm}\NormalTok{(lpsa }\OperatorTok{~}\StringTok{ }\NormalTok{lcavol, }\DataTypeTok{data =}\NormalTok{ prost)) }\OperatorTok{-}\StringTok{ }\KeywordTok{extractAIC}\NormalTok{(}\KeywordTok{lm}\NormalTok{(lpsa }\OperatorTok{~}\StringTok{ }\NormalTok{lcavol }\OperatorTok{+}\StringTok{ }\NormalTok{lweight, }\DataTypeTok{data =}\NormalTok{ prost))}
\end{Highlighting}
\end{Shaded}

\begin{verbatim}
[1] -1.00000 10.59243
\end{verbatim}

\begin{Shaded}
\begin{Highlighting}[]
\KeywordTok{AIC}\NormalTok{(}\KeywordTok{lm}\NormalTok{(lpsa }\OperatorTok{~}\StringTok{ }\NormalTok{lcavol, }\DataTypeTok{data =}\NormalTok{ prost)) }\OperatorTok{-}\StringTok{ }\KeywordTok{AIC}\NormalTok{(}\KeywordTok{lm}\NormalTok{(lpsa }\OperatorTok{~}\StringTok{ }\NormalTok{lcavol }\OperatorTok{+}\StringTok{ }\NormalTok{lweight, }\DataTypeTok{data =}\NormalTok{ prost))}
\end{Highlighting}
\end{Shaded}

\begin{verbatim}
[1] 10.59243
\end{verbatim}

\begin{itemize}
\tightlist
\item
  AIC is only defined up to an additive constant.
\item
  Since the difference between two models using either \texttt{AIC} or
  \texttt{extractAIC} is the same, this doesn't actually matter which
  one we use, so long as we use the same one consistently.
\end{itemize}

\subsection{Building a Tibble containing the necessary
information}\label{building-a-tibble-containing-the-necessary-information}

Again, note the use of 2:9 for the values of \(k\), because we're
fitting one model for each size from 2 through 9.

\begin{Shaded}
\begin{Highlighting}[]
\NormalTok{best_mods_}\DecValTok{1}\NormalTok{ <-}\StringTok{ }\KeywordTok{data_frame}\NormalTok{(}
    \DataTypeTok{k =} \DecValTok{2}\OperatorTok{:}\DecValTok{9}\NormalTok{,}
    \DataTypeTok{r2 =}\NormalTok{ rs.summ}\OperatorTok{$}\NormalTok{rsq,}
    \DataTypeTok{adjr2 =}\NormalTok{ rs.summ}\OperatorTok{$}\NormalTok{adjr2,}
    \DataTypeTok{cp =}\NormalTok{ rs.summ}\OperatorTok{$}\NormalTok{cp,}
    \DataTypeTok{aic.c =}\NormalTok{ rs.summ}\OperatorTok{$}\NormalTok{aic.c,}
    \DataTypeTok{bic =}\NormalTok{ rs.summ}\OperatorTok{$}\NormalTok{bic}
\NormalTok{)}

\NormalTok{best_mods <-}\StringTok{ }\KeywordTok{cbind}\NormalTok{(best_mods_}\DecValTok{1}\NormalTok{, rs.summ}\OperatorTok{$}\NormalTok{which)}

\NormalTok{best_mods}
\end{Highlighting}
\end{Shaded}

\begin{verbatim}
  k        r2     adjr2        cp     aic.c       bic (Intercept) lcavol
1 2 0.5394320 0.5345839 28.213914 -44.23838 -66.05416        TRUE   TRUE
2 3 0.5955040 0.5868977 15.456669 -54.70040 -74.07188        TRUE   TRUE
3 4 0.6359499 0.6242063  6.811986 -62.74265 -79.71614        TRUE   TRUE
4 5 0.6425479 0.6270065  7.075509 -62.29223 -76.91557        TRUE   TRUE
5 6 0.6509970 0.6318211  6.851826 -62.33858 -74.66120        TRUE   TRUE
6 7 0.6584484 0.6356783  6.890739 -62.10692 -72.17992        TRUE   TRUE
7 8 0.6634967 0.6370302  7.562119 -61.17338 -69.04961        TRUE   TRUE
8 9 0.6656326 0.6352355  9.000000 -59.35841 -65.09253        TRUE   TRUE
  lweight   age bph_f svi_f   lcp gleason_f pgg45
1   FALSE FALSE FALSE FALSE FALSE     FALSE FALSE
2    TRUE FALSE FALSE FALSE FALSE     FALSE FALSE
3    TRUE FALSE FALSE  TRUE FALSE     FALSE FALSE
4    TRUE FALSE  TRUE  TRUE FALSE     FALSE FALSE
5    TRUE  TRUE  TRUE  TRUE FALSE     FALSE FALSE
6    TRUE  TRUE  TRUE  TRUE FALSE      TRUE FALSE
7    TRUE  TRUE  TRUE  TRUE  TRUE      TRUE FALSE
8    TRUE  TRUE  TRUE  TRUE  TRUE      TRUE  TRUE
\end{verbatim}

\section{\texorpdfstring{Plotting the Best Subsets Results using
\texttt{ggplot2}}{Plotting the Best Subsets Results using ggplot2}}\label{plotting-the-best-subsets-results-using-ggplot2}

\subsection{\texorpdfstring{The Adjusted R\textsuperscript{2}
Plot}{The Adjusted R2 Plot}}\label{the-adjusted-r2-plot}

\begin{Shaded}
\begin{Highlighting}[]
\NormalTok{p1 <-}\StringTok{ }\KeywordTok{ggplot}\NormalTok{(best_mods, }\KeywordTok{aes}\NormalTok{(}\DataTypeTok{x =}\NormalTok{ k, }\DataTypeTok{y =}\NormalTok{ adjr2,}
                            \DataTypeTok{label =} \KeywordTok{round}\NormalTok{(adjr2,}\DecValTok{2}\NormalTok{))) }\OperatorTok{+}
\StringTok{    }\KeywordTok{geom_line}\NormalTok{() }\OperatorTok{+}
\StringTok{    }\KeywordTok{geom_label}\NormalTok{() }\OperatorTok{+}
\StringTok{    }\KeywordTok{geom_label}\NormalTok{(}\DataTypeTok{data =} \KeywordTok{subset}\NormalTok{(best_mods,}
\NormalTok{                             adjr2 }\OperatorTok{==}\StringTok{ }\KeywordTok{max}\NormalTok{(adjr2)),}
               \KeywordTok{aes}\NormalTok{(}\DataTypeTok{x =}\NormalTok{ k, }\DataTypeTok{y =}\NormalTok{ adjr2, }\DataTypeTok{label =} \KeywordTok{round}\NormalTok{(adjr2,}\DecValTok{2}\NormalTok{)),}
               \DataTypeTok{fill =} \StringTok{"yellow"}\NormalTok{, }\DataTypeTok{col =} \StringTok{"blue"}\NormalTok{) }\OperatorTok{+}
\StringTok{    }\KeywordTok{theme_bw}\NormalTok{() }\OperatorTok{+}
\StringTok{    }\KeywordTok{scale_x_continuous}\NormalTok{(}\DataTypeTok{breaks =} \DecValTok{2}\OperatorTok{:}\DecValTok{9}\NormalTok{) }\OperatorTok{+}
\StringTok{    }\KeywordTok{labs}\NormalTok{(}\DataTypeTok{x =} \StringTok{"# of predictors (including intercept)"}\NormalTok{,}
         \DataTypeTok{y =} \StringTok{"Adjusted R-squared"}\NormalTok{)}

\NormalTok{p1}
\end{Highlighting}
\end{Shaded}

\includegraphics{bookdown-demo_files/figure-latex/unnamed-chunk-77-1.pdf}

Models 4-9 all look like reasonable choices here. The maximum adjusted
R\textsuperscript{2} is seen in the model of size 8.

\subsection{\texorpdfstring{Mallows'
\(C_p\)}{Mallows' C\_p}}\label{mallows-c_p}

The \(C_p\) statistic focuses directly on the tradeoff between
\textbf{bias} (due to excluding important predictors from the model) and
extra \textbf{variance} (due to including too many unimportant
predictors in the model.)

If N is the sample size, and we select \(p\) regression predictors from
a set of \(K\) (where \(p < K\)), then the \(C_p\) statistic is

\(C_p = \frac{SSE_p}{MSE_K} - N + 2p\)

where:

\begin{itemize}
\tightlist
\item
  \(SSE_p\) is the sum of squares for error (residual) in the model with
  \(p\) predictors
\item
  \(MSE_K\) is the residual mean square after regression in the model
  with all \(K\) predictors
\end{itemize}

As it turns out, this is just measuring the particular model's lack of
fit, and then adding a penalty for the number of terms in the model
(specifically \(2p - N\) is the penalty since the lack of fit is
measured as \((N-p) \frac{SSE_p}{MSE_K}\).

\begin{itemize}
\tightlist
\item
  If a model has no meaningful lack of fit (i.e.~no substantial bias)
  then the expected value of \(C_p\) is roughly \(p\).
\item
  Otherwise, the expectation is \(p\) plus a positive bias term.
\item
  In general, we want to see \emph{smaller} values of \(C_p\).
\item
  We usually select a ``winning model'' by choosing a subset of
  predictors that have \(C_p\) near the value of \(p\).
\end{itemize}

\subsection{\texorpdfstring{The \(C_p\)
Plot}{The C\_p Plot}}\label{the-c_p-plot}

The \(C_p\) plot is just a scatterplot of \(C_p\) on the Y-axis, and the
size of the model (coefficients plus intercept) \(p = k\) on the X-axis.

Each of the various predictor subsets we will study is represented in a
single point. A model without bias should have \(C_p\) roughly equal to
\(p\), so we'll frequently draw a line at \(C_p = p\) to make that
clear. We then select our model from among all models with small \(C_p\)
statistics.

\begin{itemize}
\tightlist
\item
  My typical approach is to identify the models where
  \(C_p - p \geq 0\), then select from among those models the model
  where \(C_p - p\) is minimized, and if there is a tie, select the
  model where \(p\) is minimized.
\item
  Another good candidate might be a slightly overfit model (where
  \(C_p - p < 0\) but just barely.)
\end{itemize}

\begin{Shaded}
\begin{Highlighting}[]
\NormalTok{p2 <-}\StringTok{ }\KeywordTok{ggplot}\NormalTok{(best_mods, }\KeywordTok{aes}\NormalTok{(}\DataTypeTok{x =}\NormalTok{ k, }\DataTypeTok{y =}\NormalTok{ cp,}
                            \DataTypeTok{label =} \KeywordTok{round}\NormalTok{(cp,}\DecValTok{1}\NormalTok{))) }\OperatorTok{+}
\StringTok{    }\KeywordTok{geom_line}\NormalTok{() }\OperatorTok{+}
\StringTok{    }\KeywordTok{geom_label}\NormalTok{() }\OperatorTok{+}
\StringTok{    }\KeywordTok{geom_abline}\NormalTok{(}\DataTypeTok{intercept =} \DecValTok{0}\NormalTok{, }\DataTypeTok{slope =} \DecValTok{1}\NormalTok{,}
                \DataTypeTok{col =} \StringTok{"red"}\NormalTok{) }\OperatorTok{+}
\StringTok{    }\KeywordTok{theme_bw}\NormalTok{() }\OperatorTok{+}
\StringTok{    }\KeywordTok{scale_x_continuous}\NormalTok{(}\DataTypeTok{breaks =} \DecValTok{2}\OperatorTok{:}\DecValTok{9}\NormalTok{) }\OperatorTok{+}
\StringTok{    }\KeywordTok{labs}\NormalTok{(}\DataTypeTok{x =} \StringTok{"# of predictors (including intercept)"}\NormalTok{,}
         \DataTypeTok{y =} \StringTok{"Mallows' Cp"}\NormalTok{)}

\NormalTok{p2}
\end{Highlighting}
\end{Shaded}

\includegraphics{bookdown-demo_files/figure-latex/unnamed-chunk-78-1.pdf}

\begin{itemize}
\tightlist
\item
  Model 6 is a possibility here, with the difference \(C_p - p\)
  minimized among all models with \(C_p >= p\).
\item
  Model 7 also looks pretty good, with C\textsubscript{p} just barely
  smaller than the size (p = 7) of the model.
\end{itemize}

\subsection{\texorpdfstring{``All Subsets'' Regression and Information
Criteria}{All Subsets Regression and Information Criteria}}\label{all-subsets-regression-and-information-criteria}

We might consider any of three main information criteria:

\begin{itemize}
\tightlist
\item
  the Bayesian Information Criterion, called BIC
\item
  the Akaike Information Criterion (used by R's default stepwise
  approaches,) called AIC
\item
  a corrected version of AIC due to \citet{HurvichTsai1989}, called
  AIC\textsubscript{c} or \texttt{aic.c}
\end{itemize}

Each of these indicates better models by getting smaller. Since the
\(C_p\) and AIC results will lead to the same model, I'll focus on
plotting the bias-corrected AIC and on BIC.

\subsection{The bias-corrected AIC
plot}\label{the-bias-corrected-aic-plot}

\begin{Shaded}
\begin{Highlighting}[]
\NormalTok{p3 <-}\StringTok{ }\KeywordTok{ggplot}\NormalTok{(best_mods, }\KeywordTok{aes}\NormalTok{(}\DataTypeTok{x =}\NormalTok{ k, }\DataTypeTok{y =}\NormalTok{ aic.c,}
                             \DataTypeTok{label =} \KeywordTok{round}\NormalTok{(aic.c,}\DecValTok{1}\NormalTok{))) }\OperatorTok{+}
\StringTok{    }\KeywordTok{geom_line}\NormalTok{() }\OperatorTok{+}
\StringTok{    }\KeywordTok{geom_label}\NormalTok{() }\OperatorTok{+}
\StringTok{    }\KeywordTok{geom_label}\NormalTok{(}\DataTypeTok{data =} \KeywordTok{subset}\NormalTok{(best_mods, aic.c }\OperatorTok{==}\StringTok{ }\KeywordTok{min}\NormalTok{(aic.c)),}
               \KeywordTok{aes}\NormalTok{(}\DataTypeTok{x =}\NormalTok{ k, }\DataTypeTok{y =}\NormalTok{ aic.c), }\DataTypeTok{fill =} \StringTok{"pink"}\NormalTok{, }
               \DataTypeTok{col =} \StringTok{"red"}\NormalTok{) }\OperatorTok{+}
\StringTok{    }\KeywordTok{theme_bw}\NormalTok{() }\OperatorTok{+}
\StringTok{    }\KeywordTok{scale_x_continuous}\NormalTok{(}\DataTypeTok{breaks =} \DecValTok{2}\OperatorTok{:}\DecValTok{9}\NormalTok{) }\OperatorTok{+}
\StringTok{    }\KeywordTok{labs}\NormalTok{(}\DataTypeTok{x =} \StringTok{"# of predictors (including intercept)"}\NormalTok{,}
         \DataTypeTok{y =} \StringTok{"Bias-Corrected AIC"}\NormalTok{)}

\NormalTok{p3}
\end{Highlighting}
\end{Shaded}

\includegraphics{bookdown-demo_files/figure-latex/unnamed-chunk-79-1.pdf}

The smallest AIC\textsubscript{c} values occur in models 4 and later,
especially model 4 itself.

\subsection{The BIC plot}\label{the-bic-plot}

\begin{Shaded}
\begin{Highlighting}[]
\NormalTok{p4 <-}\StringTok{ }\KeywordTok{ggplot}\NormalTok{(best_mods, }\KeywordTok{aes}\NormalTok{(}\DataTypeTok{x =}\NormalTok{ k, }\DataTypeTok{y =}\NormalTok{ bic,}
                            \DataTypeTok{label =} \KeywordTok{round}\NormalTok{(bic,}\DecValTok{1}\NormalTok{))) }\OperatorTok{+}
\StringTok{    }\KeywordTok{geom_line}\NormalTok{() }\OperatorTok{+}
\StringTok{    }\KeywordTok{geom_label}\NormalTok{() }\OperatorTok{+}
\StringTok{    }\KeywordTok{geom_label}\NormalTok{(}\DataTypeTok{data =} \KeywordTok{subset}\NormalTok{(best_mods, bic }\OperatorTok{==}\StringTok{ }\KeywordTok{min}\NormalTok{(bic)),}
               \KeywordTok{aes}\NormalTok{(}\DataTypeTok{x =}\NormalTok{ k, }\DataTypeTok{y =}\NormalTok{ bic),}
               \DataTypeTok{fill =} \StringTok{"lightgreen"}\NormalTok{, }\DataTypeTok{col =} \StringTok{"blue"}\NormalTok{) }\OperatorTok{+}
\StringTok{    }\KeywordTok{theme_bw}\NormalTok{() }\OperatorTok{+}
\StringTok{    }\KeywordTok{scale_x_continuous}\NormalTok{(}\DataTypeTok{breaks =} \DecValTok{2}\OperatorTok{:}\DecValTok{9}\NormalTok{) }\OperatorTok{+}
\StringTok{    }\KeywordTok{labs}\NormalTok{(}\DataTypeTok{x =} \StringTok{"# of predictors (including intercept)"}\NormalTok{,}
         \DataTypeTok{y =} \StringTok{"BIC"}\NormalTok{)}

\NormalTok{p4}
\end{Highlighting}
\end{Shaded}

\includegraphics{bookdown-demo_files/figure-latex/unnamed-chunk-80-1.pdf}

\subsection{All Four Plots in One Figure (via
ggplot2)}\label{all-four-plots-in-one-figure-via-ggplot2}

\begin{Shaded}
\begin{Highlighting}[]
\NormalTok{gridExtra}\OperatorTok{::}\KeywordTok{grid.arrange}\NormalTok{(p1, p2, p3, p4, }\DataTypeTok{nrow =} \DecValTok{2}\NormalTok{)}
\end{Highlighting}
\end{Shaded}

\includegraphics{bookdown-demo_files/figure-latex/unnamed-chunk-81-1.pdf}

\section{Table of Key Results}\label{table-of-key-results}

We can build a big table, like this:

\begin{Shaded}
\begin{Highlighting}[]
\NormalTok{best_mods}
\end{Highlighting}
\end{Shaded}

\begin{verbatim}
  k        r2     adjr2        cp     aic.c       bic (Intercept) lcavol
1 2 0.5394320 0.5345839 28.213914 -44.23838 -66.05416        TRUE   TRUE
2 3 0.5955040 0.5868977 15.456669 -54.70040 -74.07188        TRUE   TRUE
3 4 0.6359499 0.6242063  6.811986 -62.74265 -79.71614        TRUE   TRUE
4 5 0.6425479 0.6270065  7.075509 -62.29223 -76.91557        TRUE   TRUE
5 6 0.6509970 0.6318211  6.851826 -62.33858 -74.66120        TRUE   TRUE
6 7 0.6584484 0.6356783  6.890739 -62.10692 -72.17992        TRUE   TRUE
7 8 0.6634967 0.6370302  7.562119 -61.17338 -69.04961        TRUE   TRUE
8 9 0.6656326 0.6352355  9.000000 -59.35841 -65.09253        TRUE   TRUE
  lweight   age bph_f svi_f   lcp gleason_f pgg45
1   FALSE FALSE FALSE FALSE FALSE     FALSE FALSE
2    TRUE FALSE FALSE FALSE FALSE     FALSE FALSE
3    TRUE FALSE FALSE  TRUE FALSE     FALSE FALSE
4    TRUE FALSE  TRUE  TRUE FALSE     FALSE FALSE
5    TRUE  TRUE  TRUE  TRUE FALSE     FALSE FALSE
6    TRUE  TRUE  TRUE  TRUE FALSE      TRUE FALSE
7    TRUE  TRUE  TRUE  TRUE  TRUE      TRUE FALSE
8    TRUE  TRUE  TRUE  TRUE  TRUE      TRUE  TRUE
\end{verbatim}

\section{Models Worth Considering?}\label{models-worth-considering}

\begin{longtable}[]{@{}rrll@{}}
\toprule
\(k\) & Predictors & Reason\tabularnewline
\midrule
\endhead
4 & \texttt{lcavol\ lweight\ svi\_f} & minimizes BIC,
AIC\textsubscript{c}\tabularnewline
7 & \texttt{+\ age\ bph\_f\ gleason\_f} & \(C_p\) near
\emph{p}\tabularnewline
8 & \texttt{+\ lcp} & max \(R^2_{adj}\)\tabularnewline
\bottomrule
\end{longtable}

\section{Compare these candidate models
in-sample?}\label{compare-these-candidate-models-in-sample}

\subsection{\texorpdfstring{Using \texttt{anova} to compare nested
models}{Using anova to compare nested models}}\label{using-anova-to-compare-nested-models}

Let's run an ANOVA-based comparison of these nested models to each other
and to the model with the intercept alone.

\begin{itemize}
\tightlist
\item
  The models are \textbf{nested} because \texttt{m04} is a subset of the
  predictors in \texttt{m07}, which includes a subset of the predictors
  in \texttt{m08}.
\end{itemize}

\begin{Shaded}
\begin{Highlighting}[]
\NormalTok{m.int <-}\StringTok{ }\KeywordTok{lm}\NormalTok{(lpsa }\OperatorTok{~}\StringTok{ }\DecValTok{1}\NormalTok{, }\DataTypeTok{data =}\NormalTok{ prost)}
\NormalTok{m04 <-}\StringTok{ }\KeywordTok{lm}\NormalTok{(lpsa }\OperatorTok{~}\StringTok{ }\NormalTok{lcavol }\OperatorTok{+}\StringTok{ }\NormalTok{lweight }\OperatorTok{+}\StringTok{ }\NormalTok{svi_f, }\DataTypeTok{data =}\NormalTok{ prost)}
\NormalTok{m07 <-}\StringTok{ }\KeywordTok{lm}\NormalTok{(lpsa }\OperatorTok{~}\StringTok{ }\NormalTok{lcavol }\OperatorTok{+}\StringTok{ }\NormalTok{lweight }\OperatorTok{+}\StringTok{ }\NormalTok{svi_f }\OperatorTok{+}\StringTok{ }
\StringTok{              }\NormalTok{age }\OperatorTok{+}\StringTok{ }\NormalTok{bph_f }\OperatorTok{+}\StringTok{ }\NormalTok{gleason_f, }\DataTypeTok{data =}\NormalTok{ prost)}
\NormalTok{m08 <-}\StringTok{ }\KeywordTok{lm}\NormalTok{(lpsa }\OperatorTok{~}\StringTok{ }\NormalTok{lcavol }\OperatorTok{+}\StringTok{ }\NormalTok{lweight }\OperatorTok{+}\StringTok{ }\NormalTok{svi_f }\OperatorTok{+}\StringTok{ }
\StringTok{              }\NormalTok{age }\OperatorTok{+}\StringTok{ }\NormalTok{bph_f }\OperatorTok{+}\StringTok{ }\NormalTok{gleason_f }\OperatorTok{+}\StringTok{ }\NormalTok{lcp, }\DataTypeTok{data =}\NormalTok{ prost)}
\NormalTok{m.full <-}\StringTok{ }\KeywordTok{lm}\NormalTok{(lpsa }\OperatorTok{~}\StringTok{ }\NormalTok{lcavol }\OperatorTok{+}\StringTok{ }\NormalTok{lweight }\OperatorTok{+}\StringTok{ }\NormalTok{svi_f }\OperatorTok{+}\StringTok{ }
\StringTok{              }\NormalTok{age }\OperatorTok{+}\StringTok{ }\NormalTok{bph_f }\OperatorTok{+}\StringTok{ }\NormalTok{gleason_f }\OperatorTok{+}\StringTok{ }\NormalTok{lcp }\OperatorTok{+}\StringTok{ }\NormalTok{pgg45, }\DataTypeTok{data =}\NormalTok{ prost)}
\end{Highlighting}
\end{Shaded}

Next, we'll run\ldots{}

\begin{Shaded}
\begin{Highlighting}[]
\KeywordTok{anova}\NormalTok{(m.full, m08, m07, m04, m.int)}
\end{Highlighting}
\end{Shaded}

\begin{verbatim}
Analysis of Variance Table

Model 1: lpsa ~ lcavol + lweight + svi_f + age + bph_f + gleason_f + lcp + 
    pgg45
Model 2: lpsa ~ lcavol + lweight + svi_f + age + bph_f + gleason_f + lcp
Model 3: lpsa ~ lcavol + lweight + svi_f + age + bph_f + gleason_f
Model 4: lpsa ~ lcavol + lweight + svi_f
Model 5: lpsa ~ 1
  Res.Df     RSS Df Sum of Sq       F Pr(>F)    
1     86  41.057                                
2     87  41.498 -1    -0.441  0.9234 0.3393    
3     88  42.066 -1    -0.568  1.1891 0.2786    
4     93  46.568 -5    -4.503  1.8863 0.1050    
5     96 127.918 -3   -81.349 56.7991 <2e-16 ***
---
Signif. codes:  0 '***' 0.001 '**' 0.01 '*' 0.05 '.' 0.1 ' ' 1
\end{verbatim}

What conclusions can we draw here, on the basis of these ANOVA tests?

\begin{itemize}
\tightlist
\item
  The first \emph{p} value, of 0.3393, compares what the \texttt{anova}
  called Model 1, and what we call \texttt{m.full} to what the
  \texttt{anova} called Model 2, and what we call \texttt{m08}. So
  there's no significant decline in predictive value observed when we
  drop from the \texttt{m.full} model to the \texttt{m08} model. This
  suggests that the \texttt{m08} model may be a better choice.
\item
  The second \emph{p} value, of 0.2786, compares \texttt{m08} to
  \texttt{m07}, and suggests that we lose no significant predictive
  value by dropping down to \texttt{m07}.
\item
  The third \emph{p} value, of 0.1050, compares \texttt{m07} to
  \texttt{m04}, and suggests that we lose no significant predictive
  value by dropping down to \texttt{m04}.
\item
  But the fourth \emph{p} value, of 2e-16 (or, functionally, zero),
  compares \texttt{m04} to \texttt{m.int} and suggests that we do gain
  significant predictive value by including the predictors in
  \texttt{m04} as compared to a model with an intercept alone.
\item
  So, by the significance tests, the model we'd select would be
  \texttt{m04}, but, of course, in-sample statistical significance alone
  isn't a good enough reason to select a model if we want to do
  prediction well.
\end{itemize}

\section{AIC and BIC comparisons, within the training
sample}\label{aic-and-bic-comparisons-within-the-training-sample}

Next, we'll compare the three candidate models (ignoring the
intercept-only and kitchen sink models) in terms of their AIC values and
BIC values, again using the same sample we used to fit the models in the
first place.

\begin{Shaded}
\begin{Highlighting}[]
\KeywordTok{AIC}\NormalTok{(m04, m07, m08)}
\end{Highlighting}
\end{Shaded}

\begin{verbatim}
    df      AIC
m04  5 214.0966
m07 10 214.2327
m08 11 214.9148
\end{verbatim}

\begin{Shaded}
\begin{Highlighting}[]
\KeywordTok{BIC}\NormalTok{(m04, m07, m08)}
\end{Highlighting}
\end{Shaded}

\begin{verbatim}
    df      BIC
m04  5 226.9702
m07 10 239.9798
m08 11 243.2366
\end{verbatim}

\begin{itemize}
\tightlist
\item
  The model with the smallest AIC value shows the best performance
  within the sample on that measure.
\item
  Similarly, smaller BIC values are associated with predictor sets that
  perform better in sample on that criterion.
\item
  BIC often suggests smaller models (with fewer regression inputs) than
  does AIC. Does that happen in this case?
\item
  Note that \texttt{AIC} and \texttt{BIC} can be calculated in a few
  different ways, so we may see some variation if we don't compare
  apples to apples with regard to the R functions involved.
\end{itemize}

\section{Cross-Validation of Candidate Models out of
Sample}\label{cross-validation-of-candidate-models-out-of-sample}

\subsection{\texorpdfstring{20-fold Cross-Validation of model
\texttt{m04}}{20-fold Cross-Validation of model m04}}\label{fold-cross-validation-of-model-m04}

Model \texttt{m04} uses \texttt{lcavol}, \texttt{lweight} and
\texttt{svi\_f} to predict the \texttt{lpsa} outcome. Let's do 20-fold
cross-validation of this modeling approach, and calculate the root mean
squared prediction error and the mean absolute prediction error for that
modeling scheme.

\begin{Shaded}
\begin{Highlighting}[]
\KeywordTok{set.seed}\NormalTok{(}\DecValTok{43201}\NormalTok{)}

\NormalTok{cv_m04 <-}\StringTok{ }\NormalTok{prost }\OperatorTok
\StringTok{    }\KeywordTok{crossv_kfold}\NormalTok{(}\DataTypeTok{k =} \DecValTok{20}\NormalTok{) }\OperatorTok
\StringTok{    }\KeywordTok{mutate}\NormalTok{(}\DataTypeTok{model =} \KeywordTok{map}\NormalTok{(train, }
                       \OperatorTok{~}\StringTok{ }\KeywordTok{lm}\NormalTok{(lpsa }\OperatorTok{~}\StringTok{ }\NormalTok{lcavol }\OperatorTok{+}\StringTok{ }\NormalTok{lweight }\OperatorTok{+}\StringTok{ }\NormalTok{svi_f,}
                                   \DataTypeTok{data =}\NormalTok{ .)))}

\NormalTok{cv_m04_pred <-}\StringTok{ }\NormalTok{cv_m04 }\OperatorTok
\StringTok{    }\KeywordTok{unnest}\NormalTok{(}\KeywordTok{map2}\NormalTok{(model, test, }\OperatorTok{~}\StringTok{ }\KeywordTok{augment}\NormalTok{(.x, }\DataTypeTok{newdata =}\NormalTok{ .y)))}

\NormalTok{cv_m04_results <-}\StringTok{ }\NormalTok{cv_m04_pred }\OperatorTok
\StringTok{    }\KeywordTok{summarize}\NormalTok{(}\DataTypeTok{Model =} \StringTok{"m04"}\NormalTok{, }
              \DataTypeTok{RMSE =} \KeywordTok{sqrt}\NormalTok{(}\KeywordTok{mean}\NormalTok{((lpsa }\OperatorTok{-}\StringTok{ }\NormalTok{.fitted) }\OperatorTok{^}\DecValTok{2}\NormalTok{)),}
              \DataTypeTok{MAE =} \KeywordTok{mean}\NormalTok{(}\KeywordTok{abs}\NormalTok{(lpsa }\OperatorTok{-}\StringTok{ }\NormalTok{.fitted)))}

\NormalTok{cv_m04_results}
\end{Highlighting}
\end{Shaded}

\begin{verbatim}
# A tibble: 1 x 3
  Model  RMSE   MAE
  <chr> <dbl> <dbl>
1 m04   0.725 0.574
\end{verbatim}

\subsection{\texorpdfstring{20-fold Cross-Validation of model
\texttt{m07}}{20-fold Cross-Validation of model m07}}\label{fold-cross-validation-of-model-m07}

Model \texttt{m07} uses \texttt{lcavol}, \texttt{lweight},
\texttt{svi\_f}, \texttt{age}, \texttt{bph\_f}, and \texttt{gleason\_f}
to predict the \texttt{lpsa} outcome. Let's now do 20-fold
cross-validation of this modeling approach, and calculate the root mean
squared prediction error and the mean absolute prediction error for that
modeling scheme. Note the small changes required, as compared to our
cross-validation of model \texttt{m04} a moment ago.

\begin{Shaded}
\begin{Highlighting}[]
\KeywordTok{set.seed}\NormalTok{(}\DecValTok{43202}\NormalTok{)}

\NormalTok{cv_m07 <-}\StringTok{ }\NormalTok{prost }\OperatorTok
\StringTok{    }\KeywordTok{crossv_kfold}\NormalTok{(}\DataTypeTok{k =} \DecValTok{20}\NormalTok{) }\OperatorTok
\StringTok{    }\KeywordTok{mutate}\NormalTok{(}\DataTypeTok{model =} \KeywordTok{map}\NormalTok{(train, }
                       \OperatorTok{~}\StringTok{ }\KeywordTok{lm}\NormalTok{(lpsa }\OperatorTok{~}\StringTok{ }\NormalTok{lcavol }\OperatorTok{+}\StringTok{ }\NormalTok{lweight }\OperatorTok{+}\StringTok{ }
\StringTok{                                }\NormalTok{svi_f }\OperatorTok{+}\StringTok{ }\NormalTok{age }\OperatorTok{+}\StringTok{ }\NormalTok{bph_f }\OperatorTok{+}\StringTok{ }
\StringTok{                                }\NormalTok{gleason_f,}
                                   \DataTypeTok{data =}\NormalTok{ .)))}

\NormalTok{cv_m07_pred <-}\StringTok{ }\NormalTok{cv_m07 }\OperatorTok
\StringTok{    }\KeywordTok{unnest}\NormalTok{(}\KeywordTok{map2}\NormalTok{(model, test, }\OperatorTok{~}\StringTok{ }\KeywordTok{augment}\NormalTok{(.x, }\DataTypeTok{newdata =}\NormalTok{ .y)))}

\NormalTok{cv_m07_results <-}\StringTok{ }\NormalTok{cv_m07_pred }\OperatorTok
\StringTok{    }\KeywordTok{summarize}\NormalTok{(}\DataTypeTok{Model =} \StringTok{"m07"}\NormalTok{, }
              \DataTypeTok{RMSE =} \KeywordTok{sqrt}\NormalTok{(}\KeywordTok{mean}\NormalTok{((lpsa }\OperatorTok{-}\StringTok{ }\NormalTok{.fitted) }\OperatorTok{^}\DecValTok{2}\NormalTok{)),}
              \DataTypeTok{MAE =} \KeywordTok{mean}\NormalTok{(}\KeywordTok{abs}\NormalTok{(lpsa }\OperatorTok{-}\StringTok{ }\NormalTok{.fitted)))}

\NormalTok{cv_m07_results}
\end{Highlighting}
\end{Shaded}

\begin{verbatim}
# A tibble: 1 x 3
  Model  RMSE   MAE
  <chr> <dbl> <dbl>
1 m07   0.730 0.556
\end{verbatim}

\subsection{\texorpdfstring{20-fold Cross-Validation of model
\texttt{m08}}{20-fold Cross-Validation of model m08}}\label{fold-cross-validation-of-model-m08}

Model \texttt{m08} uses \texttt{lcavol}, \texttt{lweight},
\texttt{svi\_f}, \texttt{age}, \texttt{bph\_f}, \texttt{gleason\_f} and
\texttt{lcp} to predict the \texttt{lpsa} outcome. Let's now do 20-fold
cross-validation of this modeling approach.

\begin{Shaded}
\begin{Highlighting}[]
\KeywordTok{set.seed}\NormalTok{(}\DecValTok{43202}\NormalTok{)}

\NormalTok{cv_m08 <-}\StringTok{ }\NormalTok{prost }\OperatorTok
\StringTok{    }\KeywordTok{crossv_kfold}\NormalTok{(}\DataTypeTok{k =} \DecValTok{20}\NormalTok{) }\OperatorTok
\StringTok{    }\KeywordTok{mutate}\NormalTok{(}\DataTypeTok{model =} \KeywordTok{map}\NormalTok{(train, }
                       \OperatorTok{~}\StringTok{ }\KeywordTok{lm}\NormalTok{(lpsa }\OperatorTok{~}\StringTok{ }\NormalTok{lcavol }\OperatorTok{+}\StringTok{ }\NormalTok{lweight }\OperatorTok{+}\StringTok{ }
\StringTok{                                }\NormalTok{svi_f }\OperatorTok{+}\StringTok{ }\NormalTok{age }\OperatorTok{+}\StringTok{ }\NormalTok{bph_f }\OperatorTok{+}\StringTok{ }
\StringTok{                                }\NormalTok{gleason_f }\OperatorTok{+}\StringTok{ }\NormalTok{lcp,}
                                   \DataTypeTok{data =}\NormalTok{ .)))}

\NormalTok{cv_m08_pred <-}\StringTok{ }\NormalTok{cv_m08 }\OperatorTok
\StringTok{    }\KeywordTok{unnest}\NormalTok{(}\KeywordTok{map2}\NormalTok{(model, test, }\OperatorTok{~}\StringTok{ }\KeywordTok{augment}\NormalTok{(.x, }\DataTypeTok{newdata =}\NormalTok{ .y)))}

\NormalTok{cv_m08_results <-}\StringTok{ }\NormalTok{cv_m08_pred }\OperatorTok
\StringTok{    }\KeywordTok{summarize}\NormalTok{(}\DataTypeTok{Model =} \StringTok{"m08"}\NormalTok{, }
              \DataTypeTok{RMSE =} \KeywordTok{sqrt}\NormalTok{(}\KeywordTok{mean}\NormalTok{((lpsa }\OperatorTok{-}\StringTok{ }\NormalTok{.fitted) }\OperatorTok{^}\DecValTok{2}\NormalTok{)),}
              \DataTypeTok{MAE =} \KeywordTok{mean}\NormalTok{(}\KeywordTok{abs}\NormalTok{(lpsa }\OperatorTok{-}\StringTok{ }\NormalTok{.fitted)))}

\NormalTok{cv_m08_results}
\end{Highlighting}
\end{Shaded}

\begin{verbatim}
# A tibble: 1 x 3
  Model  RMSE   MAE
  <chr> <dbl> <dbl>
1 m08   0.729 0.557
\end{verbatim}

\subsection{Comparing the Results of the
Cross-Validations}\label{comparing-the-results-of-the-cross-validations}

\begin{Shaded}
\begin{Highlighting}[]
\KeywordTok{bind_rows}\NormalTok{(cv_m04_results, cv_m07_results, cv_m08_results)}
\end{Highlighting}
\end{Shaded}

\begin{verbatim}
# A tibble: 3 x 3
  Model  RMSE   MAE
  <chr> <dbl> <dbl>
1 m04   0.725 0.574
2 m07   0.730 0.556
3 m08   0.729 0.557
\end{verbatim}

It appears that model \texttt{m04} has the smallest RMSE and MAE in this
case. So, that's the model with the strongest cross-validated predictive
accuracy, by these two standards.

\section{What about Interaction
Terms?}\label{what-about-interaction-terms}

Suppose we consider for a moment a much smaller and less realistic
problem. We want to use best subsets to identify a model out of a set of
three predictors for \texttt{lpsa}: specifically \texttt{lcavol},
\texttt{age} and \texttt{svi\_f}, but now we also want to consider the
interaction of \texttt{svi\_f} with \texttt{lcavol} as a potential
addition. Remember that \texttt{svi} is the 1/0 numeric version of
\texttt{svi\_f}. We could simply add a numerical product term to our
model, as follows.

\begin{Shaded}
\begin{Highlighting}[]
\NormalTok{pred2 <-}\StringTok{ }\KeywordTok{with}\NormalTok{(prost, }\KeywordTok{cbind}\NormalTok{(lcavol, age, svi_f, }\DataTypeTok{svixlcavol =}\NormalTok{ svi}\OperatorTok{*}\NormalTok{lcavol))}

\NormalTok{rs.ks2 <-}\StringTok{ }\KeywordTok{regsubsets}\NormalTok{(pred2, }\DataTypeTok{y =}\NormalTok{ prost}\OperatorTok{$}\NormalTok{lpsa, }
                    \DataTypeTok{nvmax =} \OtherTok{NULL}\NormalTok{, }\DataTypeTok{nbest =} \DecValTok{1}\NormalTok{)}
\NormalTok{rs.summ2 <-}\StringTok{ }\KeywordTok{summary}\NormalTok{(rs.ks2)}
\NormalTok{rs.summ2}
\end{Highlighting}
\end{Shaded}

\begin{verbatim}
Subset selection object
4 Variables  (and intercept)
           Forced in Forced out
lcavol         FALSE      FALSE
age            FALSE      FALSE
svi_f          FALSE      FALSE
svixlcavol     FALSE      FALSE
1 subsets of each size up to 4
Selection Algorithm: exhaustive
         lcavol age svi_f svixlcavol
1  ( 1 ) "*"    " " " "   " "       
2  ( 1 ) "*"    " " "*"   " "       
3  ( 1 ) "*"    " " "*"   "*"       
4  ( 1 ) "*"    "*" "*"   "*"       
\end{verbatim}

In this case, best subsets doesn't identify the interaction term as an
attractive predictor until it has already included the main effects that
go into it. So that's fine. But if that isn't the case, we would have a
problem.

To resolve this, we could:

\begin{enumerate}
\def\labelenumi{\arabic{enumi}.}
\tightlist
\item
  Consider interactions beforehand, and force them in if desired.
\item
  Consider interaction terms outside of best subsets, and only after the
  selection of main effects.
\item
  Use another approach to deal with variable selection for interaction
  terms.
\end{enumerate}

\chapter{Adding Non-linear Terms to a Linear Regression
Model}\label{adding-non-linear-terms-to-a-linear-regression-model}

\section{\texorpdfstring{The \texttt{pollution}
data}{The pollution data}}\label{the-pollution-data}

Consider the \texttt{pollution} data set, which contain 15 independent
variables and a measure of mortality, describing 60 US metropolitan
areas in 1959-1961. The data come from \citet{McDonald1973}, and are
available at
\url{http://www4.stat.ncsu.edu/~boos/var.select/pollution.html} and our
web site.

\begin{Shaded}
\begin{Highlighting}[]
\NormalTok{pollution}
\end{Highlighting}
\end{Shaded}

\begin{verbatim}
# A tibble: 60 x 16
      x1    x2    x3    x4    x5    x6    x7    x8     x9   x10   x11
   <int> <int> <int> <dbl> <dbl> <dbl> <dbl> <int>  <dbl> <dbl> <dbl>
 1    36    27    71  8.10  3.34 11.4   81.5  3243  8.80   42.6  11.7
 2    35    23    72 11.1   3.14 11.0   78.8  4281  3.50   50.7  14.4
 3    44    29    74 10.4   3.21  9.80  81.6  4260  0.800  39.4  12.4
 4    47    45    79  6.50  3.41 11.1   77.5  3125 27.1    50.2  20.6
 5    43    35    77  7.60  3.44  9.60  84.6  6441 24.4    43.7  14.3
 6    53    45    80  7.70  3.45 10.2   66.8  3325 38.5    43.1  25.5
 7    43    30    74 10.9   3.23 12.1   83.9  4679  3.50   49.2  11.3
 8    45    30    73  9.30  3.29 10.6   86.0  2140  5.30   40.4  10.5
 9    36    24    70  9.00  3.31 10.5   83.2  6582  8.10   42.5  12.6
10    36    27    72  9.50  3.36 10.7   79.3  4213  6.70   41.0  13.2
# ... with 50 more rows, and 5 more variables: x12 <int>, x13 <int>,
#   x14 <int>, x15 <int>, y <dbl>
\end{verbatim}

Here's a codebook:

\begin{longtable}[]{@{}rl@{}}
\toprule
Variable & Description\tabularnewline
\midrule
\endhead
\texttt{y} & Total Age Adjusted Mortality Rate\tabularnewline
\texttt{x1} & Mean annual precipitation in inches\tabularnewline
\texttt{x2} & Mean January temperature in degrees
Fahrenheit\tabularnewline
\texttt{x3} & Mean July temperature in degrees Fahrenheit\tabularnewline
\texttt{x4} & Percent of 1960 SMSA population that is 65 years of age or
over\tabularnewline
\texttt{x5} & Population per household, 1960 SMSA\tabularnewline
\texttt{x6} & Median school years completed for those over 25 in 1960
SMSA\tabularnewline
\texttt{x7} & Percent of housing units that are found with
facilities\tabularnewline
\texttt{x8} & Population per square mile in urbanized area in
1960\tabularnewline
\texttt{x9} & Percent of 1960 urbanized area population that is
non-white\tabularnewline
\texttt{x10} & Percent employment in white-collar occupations in 1960
urbanized area\tabularnewline
\texttt{x11} & Percent of families with income under 3; 000 in 1960
urbanized area\tabularnewline
\texttt{x12} & Relative population potential of hydrocarbons,
HC\tabularnewline
\texttt{x13} & Relative pollution potential of oxides of nitrogen,
NOx\tabularnewline
\texttt{x14} & Relative pollution potential of sulfur dioxide,
SO2\tabularnewline
\texttt{x15} & Percent relative humidity, annual average at 1
p.m.\tabularnewline
\bottomrule
\end{longtable}

\section{\texorpdfstring{Fitting a straight line model to predict
\texttt{y} from
\texttt{x2}}{Fitting a straight line model to predict y from x2}}\label{fitting-a-straight-line-model-to-predict-y-from-x2}

Consider the relationship between \texttt{y}, the age-adjusted mortality
rate, and \texttt{x2}, the mean January temperature, across these 60
areas. I'll include both a linear model (in blue) and a loess smooth (in
red.) Does the relationship appear to be linear?

\begin{Shaded}
\begin{Highlighting}[]
\KeywordTok{ggplot}\NormalTok{(pollution, }\KeywordTok{aes}\NormalTok{(}\DataTypeTok{x =}\NormalTok{ x2, }\DataTypeTok{y =}\NormalTok{ y)) }\OperatorTok{+}
\StringTok{    }\KeywordTok{geom_point}\NormalTok{() }\OperatorTok{+}
\StringTok{    }\KeywordTok{geom_smooth}\NormalTok{(}\DataTypeTok{method =} \StringTok{"lm"}\NormalTok{, }\DataTypeTok{col =} \StringTok{"blue"}\NormalTok{, }\DataTypeTok{se =}\NormalTok{ F) }\OperatorTok{+}
\StringTok{    }\KeywordTok{geom_smooth}\NormalTok{(}\DataTypeTok{method =} \StringTok{"loess"}\NormalTok{, }\DataTypeTok{col =} \StringTok{"red"}\NormalTok{, }\DataTypeTok{se =}\NormalTok{ F)}
\end{Highlighting}
\end{Shaded}

\includegraphics{bookdown-demo_files/figure-latex/c9_lm_and_loess_y_on_x2-1.pdf}

Suppose we plot the residuals that emerge from the linear model shown in
blue, above. Do we see a curve in a plot of residuals against fitted
values?

\begin{Shaded}
\begin{Highlighting}[]
\KeywordTok{plot}\NormalTok{(}\KeywordTok{lm}\NormalTok{(y }\OperatorTok{~}\StringTok{ }\NormalTok{x2, }\DataTypeTok{data =}\NormalTok{ pollution), }\DataTypeTok{which =} \DecValTok{1}\NormalTok{)}
\end{Highlighting}
\end{Shaded}

\includegraphics{bookdown-demo_files/figure-latex/fit_c9_m1_and_check_residuals_for_curve-1.pdf}

\section{\texorpdfstring{Quadratic polynomial model to predict
\texttt{y} using
\texttt{x2}}{Quadratic polynomial model to predict y using x2}}\label{quadratic-polynomial-model-to-predict-y-using-x2}

A polynomial in the variable \texttt{x} of degree D is a linear
combination of the powers of \texttt{x} up to D.

For example:

\begin{itemize}
\tightlist
\item
  Linear: \(y = \beta_0 + \beta_1 x\)
\item
  Quadratic: \(y = \beta_0 + \beta_1 x + \beta_2 x^2\)
\item
  Cubic: \(y = \beta_0 + \beta_1 x + \beta_2 x^2 + \beta_3 x^3\)
\item
  Quartic:
  \(y = \beta_0 + \beta_1 x + \beta_2 x^2 + \beta_3 x^3 + \beta_4 x^4\)
\item
  Quintic:
  \(y = \beta_0 + \beta_1 x + \beta_2 x^2 + \beta_3 x^3 + \beta_4 x^4 + \beta_5 x^5\)
\end{itemize}

Fitting such a model creates a **polynomial regression*.

\subsection{The raw quadratic model}\label{the-raw-quadratic-model}

Let's look at a \textbf{quadratic model} which predicts \texttt{y} using
\texttt{x2} and the square of \texttt{x2}, so that our model is of the
form:

\[
y = \beta_0 + \beta_1 x_2 + \beta_2 x_2^2 + error
\]

There are several ways to fit this exact model.

\begin{itemize}
\tightlist
\item
  One approach is to calculate the square of \texttt{x2} within our
  \texttt{pollution} data set, and then feed both \texttt{x2} and
  \texttt{x2squared} to \texttt{lm}.
\item
  Another approach uses the I function within our \texttt{lm} to specify
  the use of both \texttt{x2} and its square.
\item
  Yet another approach uses the \texttt{poly} function within our
  \texttt{lm}, which can be used to specify raw models including
  \texttt{x2} and \texttt{x2squared}.
\end{itemize}

\begin{Shaded}
\begin{Highlighting}[]
\NormalTok{pollution <-}\StringTok{ }\NormalTok{pollution }\OperatorTok
\StringTok{    }\KeywordTok{mutate}\NormalTok{(}\DataTypeTok{x2squared =}\NormalTok{ x2}\OperatorTok{^}\DecValTok{2}\NormalTok{)}

\NormalTok{mod2a <-}\StringTok{ }\KeywordTok{lm}\NormalTok{(y }\OperatorTok{~}\StringTok{ }\NormalTok{x2 }\OperatorTok{+}\StringTok{ }\NormalTok{x2squared, }\DataTypeTok{data =}\NormalTok{ pollution)}
\NormalTok{mod2b <-}\StringTok{ }\KeywordTok{lm}\NormalTok{(y }\OperatorTok{~}\StringTok{ }\NormalTok{x2 }\OperatorTok{+}\StringTok{ }\KeywordTok{I}\NormalTok{(x2}\OperatorTok{^}\DecValTok{2}\NormalTok{), }\DataTypeTok{data =}\NormalTok{ pollution)}
\NormalTok{mod2c <-}\StringTok{ }\KeywordTok{lm}\NormalTok{(y }\OperatorTok{~}\StringTok{ }\KeywordTok{poly}\NormalTok{(x2, }\DataTypeTok{degree =} \DecValTok{2}\NormalTok{, }\DataTypeTok{raw =} \OtherTok{TRUE}\NormalTok{), }\DataTypeTok{data =}\NormalTok{ pollution)}
\end{Highlighting}
\end{Shaded}

Each of these approaches produces the same model, as they are just
different ways of expressing the same idea.

\begin{Shaded}
\begin{Highlighting}[]
\KeywordTok{summary}\NormalTok{(mod2a)}
\end{Highlighting}
\end{Shaded}

\begin{verbatim}

Call:
lm(formula = y ~ x2 + x2squared, data = pollution)

Residuals:
     Min       1Q   Median       3Q      Max 
-148.977  -38.651    6.889   35.312  189.346 

Coefficients:
             Estimate Std. Error t value Pr(>|t|)    
(Intercept) 785.77449   79.54086   9.879 5.87e-14 ***
x2            8.87640    4.27394   2.077   0.0423 *  
x2squared    -0.11704    0.05429  -2.156   0.0353 *  
---
Signif. codes:  0 '***' 0.001 '**' 0.01 '*' 0.05 '.' 0.1 ' ' 1

Residual standard error: 60.83 on 57 degrees of freedom
Multiple R-squared:  0.07623,   Adjusted R-squared:  0.04382 
F-statistic: 2.352 on 2 and 57 DF,  p-value: 0.1044
\end{verbatim}

And if we plot the fitted values for this \texttt{mod2} using whatever
approach you like, we get exactly the same result.

\begin{Shaded}
\begin{Highlighting}[]
\NormalTok{mod2a.aug <-}\StringTok{ }\KeywordTok{augment}\NormalTok{(mod2a)}
\NormalTok{mod2a.aug}\OperatorTok{$}\NormalTok{x2 <-}\StringTok{ }\NormalTok{pollution}\OperatorTok{$}\NormalTok{x2}

\KeywordTok{ggplot}\NormalTok{(pollution, }\KeywordTok{aes}\NormalTok{(}\DataTypeTok{x =}\NormalTok{ x2, }\DataTypeTok{y =}\NormalTok{ y)) }\OperatorTok{+}
\StringTok{    }\KeywordTok{geom_point}\NormalTok{() }\OperatorTok{+}
\StringTok{    }\KeywordTok{geom_line}\NormalTok{(}\DataTypeTok{data =}\NormalTok{ mod2a.aug, }\KeywordTok{aes}\NormalTok{(}\DataTypeTok{x =}\NormalTok{ x2, }\DataTypeTok{y =}\NormalTok{ .fitted), }
              \DataTypeTok{col =} \StringTok{"red"}\NormalTok{) }\OperatorTok{+}
\StringTok{    }\KeywordTok{labs}\NormalTok{(}\DataTypeTok{title =} \StringTok{"Model 2a: Quadratic fit using x2 and x2^2"}\NormalTok{)}
\end{Highlighting}
\end{Shaded}

\includegraphics{bookdown-demo_files/figure-latex/unnamed-chunk-94-1.pdf}

\begin{Shaded}
\begin{Highlighting}[]
\NormalTok{mod2b.aug <-}\StringTok{ }\KeywordTok{augment}\NormalTok{(mod2b)}
\NormalTok{mod2b.aug}\OperatorTok{$}\NormalTok{x2 <-}\StringTok{ }\NormalTok{pollution}\OperatorTok{$}\NormalTok{x2}

\NormalTok{mod2c.aug <-}\StringTok{ }\KeywordTok{augment}\NormalTok{(mod2c)}
\NormalTok{mod2c.aug}\OperatorTok{$}\NormalTok{x2 <-}\StringTok{ }\NormalTok{pollution}\OperatorTok{$}\NormalTok{x2}

\NormalTok{p1 <-}\StringTok{ }\KeywordTok{ggplot}\NormalTok{(pollution, }\KeywordTok{aes}\NormalTok{(}\DataTypeTok{x =}\NormalTok{ x2, }\DataTypeTok{y =}\NormalTok{ y)) }\OperatorTok{+}
\StringTok{    }\KeywordTok{geom_point}\NormalTok{() }\OperatorTok{+}
\StringTok{    }\KeywordTok{geom_line}\NormalTok{(}\DataTypeTok{data =}\NormalTok{ mod2b.aug, }\KeywordTok{aes}\NormalTok{(}\DataTypeTok{x =}\NormalTok{ x2, }\DataTypeTok{y =}\NormalTok{ .fitted), }
              \DataTypeTok{col =} \StringTok{"red"}\NormalTok{) }\OperatorTok{+}
\StringTok{    }\KeywordTok{labs}\NormalTok{(}\DataTypeTok{title =} \StringTok{"Model 2b: Quadratic fit"}\NormalTok{)}

\NormalTok{p2 <-}\StringTok{ }\KeywordTok{ggplot}\NormalTok{(pollution, }\KeywordTok{aes}\NormalTok{(}\DataTypeTok{x =}\NormalTok{ x2, }\DataTypeTok{y =}\NormalTok{ y)) }\OperatorTok{+}
\StringTok{    }\KeywordTok{geom_point}\NormalTok{() }\OperatorTok{+}
\StringTok{    }\KeywordTok{geom_line}\NormalTok{(}\DataTypeTok{data =}\NormalTok{ mod2c.aug, }\KeywordTok{aes}\NormalTok{(}\DataTypeTok{x =}\NormalTok{ x2, }\DataTypeTok{y =}\NormalTok{ .fitted), }
              \DataTypeTok{col =} \StringTok{"blue"}\NormalTok{) }\OperatorTok{+}
\StringTok{    }\KeywordTok{labs}\NormalTok{(}\DataTypeTok{title =} \StringTok{"Model 2c: Quadratic fit"}\NormalTok{)}

\NormalTok{gridExtra}\OperatorTok{::}\KeywordTok{grid.arrange}\NormalTok{(p1, p2, }\DataTypeTok{nrow =} \DecValTok{1}\NormalTok{)}
\end{Highlighting}
\end{Shaded}

\includegraphics{bookdown-demo_files/figure-latex/unnamed-chunk-95-1.pdf}

\subsection{\texorpdfstring{Raw quadratic fit after centering
\texttt{x2}}{Raw quadratic fit after centering x2}}\label{raw-quadratic-fit-after-centering-x2}

Sometimes, we'll center (and perhaps rescale, too) the x2 variable
before including it in a quadratic fit like this.

\begin{Shaded}
\begin{Highlighting}[]
\NormalTok{pollution <-}\StringTok{ }\NormalTok{pollution }\OperatorTok
\StringTok{    }\KeywordTok{mutate}\NormalTok{(}\DataTypeTok{x2_c =}\NormalTok{ x2 }\OperatorTok{-}\StringTok{ }\KeywordTok{mean}\NormalTok{(x2))}

\NormalTok{mod2d <-}\StringTok{ }\KeywordTok{lm}\NormalTok{(y }\OperatorTok{~}\StringTok{ }\NormalTok{x2_c }\OperatorTok{+}\StringTok{ }\KeywordTok{I}\NormalTok{(x2_c}\OperatorTok{^}\DecValTok{2}\NormalTok{), }\DataTypeTok{data =}\NormalTok{ pollution)}

\KeywordTok{summary}\NormalTok{(mod2d)}
\end{Highlighting}
\end{Shaded}

\begin{verbatim}

Call:
lm(formula = y ~ x2_c + I(x2_c^2), data = pollution)

Residuals:
     Min       1Q   Median       3Q      Max 
-148.977  -38.651    6.889   35.312  189.346 

Coefficients:
             Estimate Std. Error t value Pr(>|t|)    
(Intercept) 952.25941    9.59896  99.204   <2e-16 ***
x2_c          0.92163    0.93237   0.988   0.3271    
I(x2_c^2)    -0.11704    0.05429  -2.156   0.0353 *  
---
Signif. codes:  0 '***' 0.001 '**' 0.01 '*' 0.05 '.' 0.1 ' ' 1

Residual standard error: 60.83 on 57 degrees of freedom
Multiple R-squared:  0.07623,   Adjusted R-squared:  0.04382 
F-statistic: 2.352 on 2 and 57 DF,  p-value: 0.1044
\end{verbatim}

Note that this model looks very different, with the exception of the
second order quadratic term. But, it produces the same fitted values as
the models we fit previously.

\begin{Shaded}
\begin{Highlighting}[]
\NormalTok{mod2d.aug <-}\StringTok{ }\KeywordTok{augment}\NormalTok{(mod2d)}
\NormalTok{mod2d.aug}\OperatorTok{$}\NormalTok{x2 <-}\StringTok{ }\NormalTok{pollution}\OperatorTok{$}\NormalTok{x2}

\KeywordTok{ggplot}\NormalTok{(pollution, }\KeywordTok{aes}\NormalTok{(}\DataTypeTok{x =}\NormalTok{ x2, }\DataTypeTok{y =}\NormalTok{ y)) }\OperatorTok{+}
\StringTok{    }\KeywordTok{geom_point}\NormalTok{() }\OperatorTok{+}
\StringTok{    }\KeywordTok{geom_line}\NormalTok{(}\DataTypeTok{data =}\NormalTok{ mod2d.aug, }\KeywordTok{aes}\NormalTok{(}\DataTypeTok{x =}\NormalTok{ x2, }\DataTypeTok{y =}\NormalTok{ .fitted), }
              \DataTypeTok{col =} \StringTok{"red"}\NormalTok{) }\OperatorTok{+}
\StringTok{    }\KeywordTok{labs}\NormalTok{(}\DataTypeTok{title =} \StringTok{"Model 2d: Quadratic fit using centered x2 and x2^2"}\NormalTok{)}
\end{Highlighting}
\end{Shaded}

\includegraphics{bookdown-demo_files/figure-latex/unnamed-chunk-97-1.pdf}

Or, if you don't believe me yet, look at the four sets of fitted values
another way.

\begin{Shaded}
\begin{Highlighting}[]
\NormalTok{mod2a.aug }\OperatorTok\StringTok{ }\KeywordTok{skim}\NormalTok{(.fitted)}
\end{Highlighting}
\end{Shaded}

\begin{verbatim}
Skim summary statistics
 n obs: 60 
 n variables: 10 

Variable type: numeric 
 variable missing complete  n   mean    sd    p0    p25 median    p75
  .fitted       0       60 60 940.36 17.18 855.1 936.72  945.6 950.29
   p100
 954.07
\end{verbatim}

\begin{Shaded}
\begin{Highlighting}[]
\NormalTok{mod2b.aug }\OperatorTok\StringTok{ }\KeywordTok{skim}\NormalTok{(.fitted)}
\end{Highlighting}
\end{Shaded}

\begin{verbatim}
Skim summary statistics
 n obs: 60 
 n variables: 10 

Variable type: numeric 
 variable missing complete  n   mean    sd    p0    p25 median    p75
  .fitted       0       60 60 940.36 17.18 855.1 936.72  945.6 950.29
   p100
 954.07
\end{verbatim}

\begin{Shaded}
\begin{Highlighting}[]
\NormalTok{mod2c.aug }\OperatorTok\StringTok{ }\KeywordTok{skim}\NormalTok{(.fitted)}
\end{Highlighting}
\end{Shaded}

\begin{verbatim}
Skim summary statistics
 n obs: 60 
 n variables: 10 

Variable type: numeric 
 variable missing complete  n   mean    sd    p0    p25 median    p75
  .fitted       0       60 60 940.36 17.18 855.1 936.72  945.6 950.29
   p100
 954.07
\end{verbatim}

\begin{Shaded}
\begin{Highlighting}[]
\NormalTok{mod2d.aug }\OperatorTok\StringTok{ }\KeywordTok{skim}\NormalTok{(.fitted)}
\end{Highlighting}
\end{Shaded}

\begin{verbatim}
Skim summary statistics
 n obs: 60 
 n variables: 11 

Variable type: numeric 
 variable missing complete  n   mean    sd    p0    p25 median    p75
  .fitted       0       60 60 940.36 17.18 855.1 936.72  945.6 950.29
   p100
 954.07
\end{verbatim}

\section{Orthogonal Polynomials}\label{orthogonal-polynomials}

Now, let's fit an orthogonal polynomial of degree 2 to predict
\texttt{y} using \texttt{x2}.

\begin{Shaded}
\begin{Highlighting}[]
\NormalTok{mod2_orth <-}\StringTok{ }\KeywordTok{lm}\NormalTok{(y }\OperatorTok{~}\StringTok{ }\KeywordTok{poly}\NormalTok{(x2, }\DecValTok{2}\NormalTok{), }\DataTypeTok{data =}\NormalTok{ pollution)}

\KeywordTok{summary}\NormalTok{(mod2_orth)}
\end{Highlighting}
\end{Shaded}

\begin{verbatim}

Call:
lm(formula = y ~ poly(x2, 2), data = pollution)

Residuals:
     Min       1Q   Median       3Q      Max 
-148.977  -38.651    6.889   35.312  189.346 

Coefficients:
             Estimate Std. Error t value Pr(>|t|)    
(Intercept)   940.358      7.853 119.746   <2e-16 ***
poly(x2, 2)1  -14.345     60.829  -0.236   0.8144    
poly(x2, 2)2 -131.142     60.829  -2.156   0.0353 *  
---
Signif. codes:  0 '***' 0.001 '**' 0.01 '*' 0.05 '.' 0.1 ' ' 1

Residual standard error: 60.83 on 57 degrees of freedom
Multiple R-squared:  0.07623,   Adjusted R-squared:  0.04382 
F-statistic: 2.352 on 2 and 57 DF,  p-value: 0.1044
\end{verbatim}

Now this looks very different in the equation, but, again, we can see
that this produces exactly the same fitted values as our previous
models, and the same model fit summaries. Is it, in fact, the same
model? Here, we'll plot the fitted Model 2a in a red line, and this new
Model 2 with Orthogonal Polynomials as blue points.

\begin{Shaded}
\begin{Highlighting}[]
\NormalTok{mod2orth.aug <-}\StringTok{ }\KeywordTok{augment}\NormalTok{(mod2_orth)}
\NormalTok{mod2orth.aug}\OperatorTok{$}\NormalTok{x2 <-}\StringTok{ }\NormalTok{pollution}\OperatorTok{$}\NormalTok{x2}

\KeywordTok{ggplot}\NormalTok{(pollution, }\KeywordTok{aes}\NormalTok{(}\DataTypeTok{x =}\NormalTok{ x2, }\DataTypeTok{y =}\NormalTok{ y)) }\OperatorTok{+}
\StringTok{    }\KeywordTok{geom_point}\NormalTok{() }\OperatorTok{+}
\StringTok{    }\KeywordTok{geom_line}\NormalTok{(}\DataTypeTok{data =}\NormalTok{ mod2a.aug, }\KeywordTok{aes}\NormalTok{(}\DataTypeTok{x =}\NormalTok{ x2, }\DataTypeTok{y =}\NormalTok{ .fitted),}
              \DataTypeTok{col =} \StringTok{"red"}\NormalTok{) }\OperatorTok{+}
\StringTok{    }\KeywordTok{geom_point}\NormalTok{(}\DataTypeTok{data =}\NormalTok{ mod2orth.aug, }\KeywordTok{aes}\NormalTok{(}\DataTypeTok{x =}\NormalTok{ x2, }\DataTypeTok{y =}\NormalTok{ .fitted), }
              \DataTypeTok{col =} \StringTok{"blue"}\NormalTok{) }\OperatorTok{+}
\StringTok{    }\KeywordTok{labs}\NormalTok{(}\DataTypeTok{title =} \StringTok{"Model 2 with Orthogonal Polynomial, degree 2"}\NormalTok{)}
\end{Highlighting}
\end{Shaded}

\includegraphics{bookdown-demo_files/figure-latex/unnamed-chunk-100-1.pdf}

Yes, it is again the same model in terms of the predictions it makes for
\texttt{y}.

By default, with \texttt{raw\ =\ FALSE}, the \texttt{poly()} function
within a linear model computes what is called an \textbf{orthogonal
polynomial}. An orthogonal polynomial sets up a model design matrix
using the coding we've seen previously: \texttt{x2} and \texttt{x2}\^{}2
in our case, and then scales those columns so that each column is
\textbf{orthogonal} to the previous ones. This eliminates the
collinearity (correlation between predictors) and lets our t tests tell
us whether the addition of any particular polynomial term improves the
fit of the model over the lower orders.

Would the addition of a cubic term help us much in predicting \texttt{y}
from \texttt{x2}?

\begin{Shaded}
\begin{Highlighting}[]
\NormalTok{mod3 <-}\StringTok{ }\KeywordTok{lm}\NormalTok{(y }\OperatorTok{~}\StringTok{ }\KeywordTok{poly}\NormalTok{(x2, }\DecValTok{3}\NormalTok{), }\DataTypeTok{data =}\NormalTok{ pollution)}
\KeywordTok{summary}\NormalTok{(mod3)}
\end{Highlighting}
\end{Shaded}

\begin{verbatim}

Call:
lm(formula = y ~ poly(x2, 3), data = pollution)

Residuals:
     Min       1Q   Median       3Q      Max 
-146.262  -39.679    5.569   35.984  191.536 

Coefficients:
             Estimate Std. Error t value Pr(>|t|)    
(Intercept)   940.358      7.917 118.772   <2e-16 ***
poly(x2, 3)1  -14.345     61.328  -0.234   0.8159    
poly(x2, 3)2 -131.142     61.328  -2.138   0.0369 *  
poly(x2, 3)3   16.918     61.328   0.276   0.7837    
---
Signif. codes:  0 '***' 0.001 '**' 0.01 '*' 0.05 '.' 0.1 ' ' 1

Residual standard error: 61.33 on 56 degrees of freedom
Multiple R-squared:  0.07748,   Adjusted R-squared:  0.02806 
F-statistic: 1.568 on 3 and 56 DF,  p-value: 0.2073
\end{verbatim}

It doesn't appear that the cubic term adds much here, if anything. The
\emph{p} value is not significant for the third degree polynomial, the
summaries of fit quality aren't much improved, and as we can see from
the plot below, the predictions don't actually change all that much.

\begin{Shaded}
\begin{Highlighting}[]
\NormalTok{mod3.aug <-}\StringTok{ }\KeywordTok{augment}\NormalTok{(mod3)}
\NormalTok{mod3.aug}\OperatorTok{$}\NormalTok{x2 <-}\StringTok{ }\NormalTok{pollution}\OperatorTok{$}\NormalTok{x2}

\KeywordTok{ggplot}\NormalTok{(pollution, }\KeywordTok{aes}\NormalTok{(}\DataTypeTok{x =}\NormalTok{ x2, }\DataTypeTok{y =}\NormalTok{ y)) }\OperatorTok{+}
\StringTok{    }\KeywordTok{geom_point}\NormalTok{() }\OperatorTok{+}
\StringTok{    }\KeywordTok{geom_line}\NormalTok{(}\DataTypeTok{data =}\NormalTok{ mod2orth.aug, }\KeywordTok{aes}\NormalTok{(}\DataTypeTok{x =}\NormalTok{ x2, }\DataTypeTok{y =}\NormalTok{ .fitted),}
              \DataTypeTok{col =} \StringTok{"red"}\NormalTok{) }\OperatorTok{+}
\StringTok{    }\KeywordTok{geom_line}\NormalTok{(}\DataTypeTok{data =}\NormalTok{ mod3.aug, }\KeywordTok{aes}\NormalTok{(}\DataTypeTok{x =}\NormalTok{ x2, }\DataTypeTok{y =}\NormalTok{ .fitted), }
              \DataTypeTok{col =} \StringTok{"blue"}\NormalTok{) }\OperatorTok{+}
\StringTok{    }\KeywordTok{labs}\NormalTok{(}\DataTypeTok{title =} \StringTok{"Quadratic (red) vs. Cubic (blue) Polynomial Fits"}\NormalTok{)}
\end{Highlighting}
\end{Shaded}

\includegraphics{bookdown-demo_files/figure-latex/unnamed-chunk-102-1.pdf}

\section{\texorpdfstring{Fit a cubic polynomial to predict \texttt{y}
from
\texttt{x3}}{Fit a cubic polynomial to predict y from x3}}\label{fit-a-cubic-polynomial-to-predict-y-from-x3}

What if we consider another predictor instead? Let's look at
\texttt{x3}, the Mean July temperature in degrees Fahrenheit. Here is
the \texttt{loess} smooth.

\begin{Shaded}
\begin{Highlighting}[]
\KeywordTok{ggplot}\NormalTok{(pollution, }\KeywordTok{aes}\NormalTok{(}\DataTypeTok{x =}\NormalTok{ x3, }\DataTypeTok{y =}\NormalTok{ y)) }\OperatorTok{+}
\StringTok{    }\KeywordTok{geom_point}\NormalTok{() }\OperatorTok{+}
\StringTok{    }\KeywordTok{geom_smooth}\NormalTok{(}\DataTypeTok{method =} \StringTok{"loess"}\NormalTok{)}
\end{Highlighting}
\end{Shaded}

\includegraphics{bookdown-demo_files/figure-latex/unnamed-chunk-103-1.pdf}

That looks pretty curvy - perhaps we need a more complex polynomial.
We'll consider a linear model (\texttt{mod4\_L}), a quadratic fit
(\texttt{mod4\_Q}) and a polynomial of degree 3: a \textbf{cubic} fit
(\texttt{mod\_4C})

\begin{Shaded}
\begin{Highlighting}[]
\NormalTok{mod4_L <-}\StringTok{ }\KeywordTok{lm}\NormalTok{(y }\OperatorTok{~}\StringTok{ }\NormalTok{x3, }\DataTypeTok{data =}\NormalTok{ pollution)}
\KeywordTok{summary}\NormalTok{(mod4_L)}
\end{Highlighting}
\end{Shaded}

\begin{verbatim}

Call:
lm(formula = y ~ x3, data = pollution)

Residuals:
     Min       1Q   Median       3Q      Max 
-139.813  -34.341    4.271   38.197  149.587 

Coefficients:
            Estimate Std. Error t value Pr(>|t|)    
(Intercept)  670.529    123.140   5.445  1.1e-06 ***
x3             3.618      1.648   2.196   0.0321 *  
---
Signif. codes:  0 '***' 0.001 '**' 0.01 '*' 0.05 '.' 0.1 ' ' 1

Residual standard error: 60.29 on 58 degrees of freedom
Multiple R-squared:  0.07674,   Adjusted R-squared:  0.06082 
F-statistic: 4.821 on 1 and 58 DF,  p-value: 0.03213
\end{verbatim}

\begin{Shaded}
\begin{Highlighting}[]
\NormalTok{mod4_Q <-}\StringTok{ }\KeywordTok{lm}\NormalTok{(y }\OperatorTok{~}\StringTok{ }\KeywordTok{poly}\NormalTok{(x3, }\DecValTok{2}\NormalTok{), }\DataTypeTok{data =}\NormalTok{ pollution)}
\KeywordTok{summary}\NormalTok{(mod4_Q)}
\end{Highlighting}
\end{Shaded}

\begin{verbatim}

Call:
lm(formula = y ~ poly(x3, 2), data = pollution)

Residuals:
     Min       1Q   Median       3Q      Max 
-132.004  -42.184    4.069   47.126  157.396 

Coefficients:
             Estimate Std. Error t value Pr(>|t|)    
(Intercept)   940.358      7.553 124.503   <2e-16 ***
poly(x3, 2)1  132.364     58.504   2.262   0.0275 *  
poly(x3, 2)2 -125.270     58.504  -2.141   0.0365 *  
---
Signif. codes:  0 '***' 0.001 '**' 0.01 '*' 0.05 '.' 0.1 ' ' 1

Residual standard error: 58.5 on 57 degrees of freedom
Multiple R-squared:  0.1455,    Adjusted R-squared:  0.1155 
F-statistic: 4.852 on 2 and 57 DF,  p-value: 0.01133
\end{verbatim}

\begin{Shaded}
\begin{Highlighting}[]
\NormalTok{mod4_C <-}\StringTok{ }\KeywordTok{lm}\NormalTok{(y }\OperatorTok{~}\StringTok{ }\KeywordTok{poly}\NormalTok{(x3, }\DecValTok{3}\NormalTok{), }\DataTypeTok{data =}\NormalTok{ pollution)}
\KeywordTok{summary}\NormalTok{(mod4_C)}
\end{Highlighting}
\end{Shaded}

\begin{verbatim}

Call:
lm(formula = y ~ poly(x3, 3), data = pollution)

Residuals:
     Min       1Q   Median       3Q      Max 
-148.004  -29.998    1.441   34.579  141.396 

Coefficients:
             Estimate Std. Error t value Pr(>|t|)    
(Intercept)   940.358      7.065 133.095  < 2e-16 ***
poly(x3, 3)1  132.364     54.728   2.419  0.01886 *  
poly(x3, 3)2 -125.270     54.728  -2.289  0.02588 *  
poly(x3, 3)3 -165.439     54.728  -3.023  0.00377 ** 
---
Signif. codes:  0 '***' 0.001 '**' 0.01 '*' 0.05 '.' 0.1 ' ' 1

Residual standard error: 54.73 on 56 degrees of freedom
Multiple R-squared:  0.2654,    Adjusted R-squared:  0.226 
F-statistic: 6.742 on 3 and 56 DF,  p-value: 0.0005799
\end{verbatim}

It looks like the cubic polynomial term is of some real importance here.
Do the linear, quadratic and cubic model fitted values look different?

\begin{Shaded}
\begin{Highlighting}[]
\NormalTok{mod4_L.aug <-}\StringTok{ }\KeywordTok{augment}\NormalTok{(mod4_L)}
\NormalTok{mod4_L.aug}\OperatorTok{$}\NormalTok{x3 <-}\StringTok{ }\NormalTok{pollution}\OperatorTok{$}\NormalTok{x3}

\NormalTok{mod4_Q.aug <-}\StringTok{ }\KeywordTok{augment}\NormalTok{(mod4_Q)}
\NormalTok{mod4_Q.aug}\OperatorTok{$}\NormalTok{x3 <-}\StringTok{ }\NormalTok{pollution}\OperatorTok{$}\NormalTok{x3}

\NormalTok{mod4_C.aug <-}\StringTok{ }\KeywordTok{augment}\NormalTok{(mod4_C)}
\NormalTok{mod4_C.aug}\OperatorTok{$}\NormalTok{x3 <-}\StringTok{ }\NormalTok{pollution}\OperatorTok{$}\NormalTok{x3}

\KeywordTok{ggplot}\NormalTok{(pollution, }\KeywordTok{aes}\NormalTok{(}\DataTypeTok{x =}\NormalTok{ x3, }\DataTypeTok{y =}\NormalTok{ y)) }\OperatorTok{+}
\StringTok{    }\KeywordTok{geom_point}\NormalTok{() }\OperatorTok{+}
\StringTok{    }\KeywordTok{geom_line}\NormalTok{(}\DataTypeTok{data =}\NormalTok{ mod4_L.aug, }\KeywordTok{aes}\NormalTok{(}\DataTypeTok{x =}\NormalTok{ x3, }\DataTypeTok{y =}\NormalTok{ .fitted), }
              \DataTypeTok{col =} \StringTok{"blue"}\NormalTok{, }\DataTypeTok{size =} \FloatTok{1.25}\NormalTok{) }\OperatorTok{+}
\StringTok{    }\KeywordTok{geom_line}\NormalTok{(}\DataTypeTok{data =}\NormalTok{ mod4_Q.aug, }\KeywordTok{aes}\NormalTok{(}\DataTypeTok{x =}\NormalTok{ x3, }\DataTypeTok{y =}\NormalTok{ .fitted),}
              \DataTypeTok{col =} \StringTok{"black"}\NormalTok{, }\DataTypeTok{size =} \FloatTok{1.25}\NormalTok{) }\OperatorTok{+}
\StringTok{    }\KeywordTok{geom_line}\NormalTok{(}\DataTypeTok{data =}\NormalTok{ mod4_C.aug, }\KeywordTok{aes}\NormalTok{(}\DataTypeTok{x =}\NormalTok{ x3, }\DataTypeTok{y =}\NormalTok{ .fitted),}
              \DataTypeTok{col =} \StringTok{"red"}\NormalTok{, }\DataTypeTok{size =} \FloatTok{1.25}\NormalTok{) }\OperatorTok{+}
\StringTok{    }\KeywordTok{geom_text}\NormalTok{(}\DataTypeTok{x =} \DecValTok{66}\NormalTok{, }\DataTypeTok{y =} \DecValTok{930}\NormalTok{, }\DataTypeTok{label =} \StringTok{"Linear Fit"}\NormalTok{, }\DataTypeTok{col =} \StringTok{"blue"}\NormalTok{) }\OperatorTok{+}
\StringTok{    }\KeywordTok{geom_text}\NormalTok{(}\DataTypeTok{x =} \DecValTok{64}\NormalTok{, }\DataTypeTok{y =} \DecValTok{820}\NormalTok{, }\DataTypeTok{label =} \StringTok{"Quadratic Fit"}\NormalTok{, }\DataTypeTok{col =} \StringTok{"black"}\NormalTok{) }\OperatorTok{+}
\StringTok{    }\KeywordTok{geom_text}\NormalTok{(}\DataTypeTok{x =} \DecValTok{83}\NormalTok{, }\DataTypeTok{y =} \DecValTok{900}\NormalTok{, }\DataTypeTok{label =} \StringTok{"Cubic Fit"}\NormalTok{, }\DataTypeTok{col =} \StringTok{"red"}\NormalTok{) }\OperatorTok{+}
\StringTok{    }\KeywordTok{labs}\NormalTok{(}\DataTypeTok{title =} \StringTok{"Linear, Quadratic and Cubic Fits predicting y with x3"}\NormalTok{) }\OperatorTok{+}
\StringTok{    }\KeywordTok{theme_bw}\NormalTok{()}
\end{Highlighting}
\end{Shaded}

\includegraphics{bookdown-demo_files/figure-latex/unnamed-chunk-105-1.pdf}

\section{Fitting a restricted cubic spline in a linear
regression}\label{fitting-a-restricted-cubic-spline-in-a-linear-regression}

\begin{itemize}
\tightlist
\item
  A \textbf{linear spline} is a continuous function formed by connecting
  points (called \textbf{knots} of the spline) by line segments.
\item
  A \textbf{restricted cubic spline} is a way to build highly
  complicated curves into a regression equation in a fairly easily
  structured way.
\item
  A restricted cubic spline is a series of polynomial functions joined
  together at the knots.

  \begin{itemize}
  \tightlist
  \item
    Such a spline gives us a way to flexibly account for non-linearity
    without over-fitting the model.
  \item
    Restricted cubic splines can fit many different types of
    non-linearities.
  \item
    Specifying the number of knots is all you need to do in R to get a
    reasonable result from a restricted cubic spline.
  \end{itemize}
\end{itemize}

The most common choices are 3, 4, or 5 knots. Each additional knot adds
to the non-linearity, and spends an additional degree of freedom:

\begin{itemize}
\tightlist
\item
  3 Knots, 2 degrees of freedom, allows the curve to ``bend'' once.
\item
  4 Knots, 3 degrees of freedom, lets the curve ``bend'' twice.
\item
  5 Knots, 4 degrees of freedom, lets the curve ``bend'' three times.
\end{itemize}

For most applications, three to five knots strike a nice balance between
complicating the model needlessly and fitting data pleasingly. Let's
consider a restricted cubic spline model for our \texttt{y} based on
\texttt{x3} again, but now with:

\begin{itemize}
\tightlist
\item
  in \texttt{mod5a}, 3 knots,
\item
  in \texttt{mod5b}, 4 knots, and
\item
  in \texttt{mod5c}, 5 knots
\end{itemize}

\begin{Shaded}
\begin{Highlighting}[]
\NormalTok{mod5a_rcs <-}\StringTok{ }\KeywordTok{lm}\NormalTok{(y }\OperatorTok{~}\StringTok{ }\KeywordTok{rcs}\NormalTok{(x3, }\DecValTok{3}\NormalTok{), }\DataTypeTok{data =}\NormalTok{ pollution)}
\NormalTok{mod5b_rcs <-}\StringTok{ }\KeywordTok{lm}\NormalTok{(y }\OperatorTok{~}\StringTok{ }\KeywordTok{rcs}\NormalTok{(x3, }\DecValTok{4}\NormalTok{), }\DataTypeTok{data =}\NormalTok{ pollution)}
\NormalTok{mod5c_rcs <-}\StringTok{ }\KeywordTok{lm}\NormalTok{(y }\OperatorTok{~}\StringTok{ }\KeywordTok{rcs}\NormalTok{(x3, }\DecValTok{5}\NormalTok{), }\DataTypeTok{data =}\NormalTok{ pollution)}
\end{Highlighting}
\end{Shaded}

Here, for instance, is the summary of the 5-knot model:

\begin{Shaded}
\begin{Highlighting}[]
\KeywordTok{summary}\NormalTok{(mod5c_rcs)}
\end{Highlighting}
\end{Shaded}

\begin{verbatim}

Call:
lm(formula = y ~ rcs(x3, 5), data = pollution)

Residuals:
     Min       1Q   Median       3Q      Max 
-141.522  -32.009    1.674   31.971  147.878 

Coefficients:
                Estimate Std. Error t value Pr(>|t|)
(Intercept)      468.113    396.319   1.181    0.243
rcs(x3, 5)x3       6.447      5.749   1.121    0.267
rcs(x3, 5)x3'    -25.633     46.810  -0.548    0.586
rcs(x3, 5)x3''   323.137    293.065   1.103    0.275
rcs(x3, 5)x3''' -612.578    396.270  -1.546    0.128

Residual standard error: 54.35 on 55 degrees of freedom
Multiple R-squared:  0.2883,    Adjusted R-squared:  0.2366 
F-statistic: 5.571 on 4 and 55 DF,  p-value: 0.0007734
\end{verbatim}

We'll begin by storing the fitted values from these three models and
other summaries, for plotting.

\begin{Shaded}
\begin{Highlighting}[]
\NormalTok{mod5a.aug <-}\StringTok{ }\KeywordTok{augment}\NormalTok{(mod5a_rcs)}
\NormalTok{mod5a.aug}\OperatorTok{$}\NormalTok{x3 <-}\StringTok{ }\NormalTok{pollution}\OperatorTok{$}\NormalTok{x3}

\NormalTok{mod5b.aug <-}\StringTok{ }\KeywordTok{augment}\NormalTok{(mod5b_rcs)}
\NormalTok{mod5b.aug}\OperatorTok{$}\NormalTok{x3 <-}\StringTok{ }\NormalTok{pollution}\OperatorTok{$}\NormalTok{x3}

\NormalTok{mod5c.aug <-}\StringTok{ }\KeywordTok{augment}\NormalTok{(mod5c_rcs)}
\NormalTok{mod5c.aug}\OperatorTok{$}\NormalTok{x3 <-}\StringTok{ }\NormalTok{pollution}\OperatorTok{$}\NormalTok{x3}
\end{Highlighting}
\end{Shaded}

\begin{Shaded}
\begin{Highlighting}[]
\NormalTok{p2 <-}\StringTok{ }\KeywordTok{ggplot}\NormalTok{(pollution, }\KeywordTok{aes}\NormalTok{(}\DataTypeTok{x =}\NormalTok{ x3, }\DataTypeTok{y =}\NormalTok{ y)) }\OperatorTok{+}
\StringTok{    }\KeywordTok{geom_point}\NormalTok{() }\OperatorTok{+}
\StringTok{    }\KeywordTok{geom_smooth}\NormalTok{(}\DataTypeTok{method =} \StringTok{"loess"}\NormalTok{, }\DataTypeTok{col =} \StringTok{"purple"}\NormalTok{, }\DataTypeTok{se =}\NormalTok{ F) }\OperatorTok{+}
\StringTok{    }\KeywordTok{labs}\NormalTok{(}\DataTypeTok{title =} \StringTok{"Loess Smooth"}\NormalTok{) }\OperatorTok{+}
\StringTok{    }\KeywordTok{theme_bw}\NormalTok{()}

\NormalTok{p3 <-}\StringTok{ }\KeywordTok{ggplot}\NormalTok{(pollution, }\KeywordTok{aes}\NormalTok{(}\DataTypeTok{x =}\NormalTok{ x3, }\DataTypeTok{y =}\NormalTok{ y)) }\OperatorTok{+}
\StringTok{    }\KeywordTok{geom_point}\NormalTok{() }\OperatorTok{+}
\StringTok{    }\KeywordTok{geom_line}\NormalTok{(}\DataTypeTok{data =}\NormalTok{ mod5a.aug, }\KeywordTok{aes}\NormalTok{(}\DataTypeTok{x =}\NormalTok{ x3, }\DataTypeTok{y =}\NormalTok{ .fitted), }
              \DataTypeTok{col =} \StringTok{"blue"}\NormalTok{, }\DataTypeTok{size =} \FloatTok{1.25}\NormalTok{) }\OperatorTok{+}
\StringTok{    }\KeywordTok{labs}\NormalTok{(}\DataTypeTok{title =} \StringTok{"RCS, 3 knots"}\NormalTok{) }\OperatorTok{+}
\StringTok{    }\KeywordTok{theme_bw}\NormalTok{()}

\NormalTok{p4 <-}\StringTok{ }\KeywordTok{ggplot}\NormalTok{(pollution, }\KeywordTok{aes}\NormalTok{(}\DataTypeTok{x =}\NormalTok{ x3, }\DataTypeTok{y =}\NormalTok{ y)) }\OperatorTok{+}
\StringTok{    }\KeywordTok{geom_point}\NormalTok{() }\OperatorTok{+}
\StringTok{    }\KeywordTok{geom_line}\NormalTok{(}\DataTypeTok{data =}\NormalTok{ mod5b.aug, }\KeywordTok{aes}\NormalTok{(}\DataTypeTok{x =}\NormalTok{ x3, }\DataTypeTok{y =}\NormalTok{ .fitted), }
              \DataTypeTok{col =} \StringTok{"black"}\NormalTok{, }\DataTypeTok{size =} \FloatTok{1.25}\NormalTok{) }\OperatorTok{+}
\StringTok{    }\KeywordTok{labs}\NormalTok{(}\DataTypeTok{title =} \StringTok{"RCS, 4 knots"}\NormalTok{) }\OperatorTok{+}
\StringTok{    }\KeywordTok{theme_bw}\NormalTok{()}

\NormalTok{p5 <-}\StringTok{ }\KeywordTok{ggplot}\NormalTok{(pollution, }\KeywordTok{aes}\NormalTok{(}\DataTypeTok{x =}\NormalTok{ x3, }\DataTypeTok{y =}\NormalTok{ y)) }\OperatorTok{+}
\StringTok{    }\KeywordTok{geom_point}\NormalTok{() }\OperatorTok{+}
\StringTok{    }\KeywordTok{geom_line}\NormalTok{(}\DataTypeTok{data =}\NormalTok{ mod5c.aug, }\KeywordTok{aes}\NormalTok{(}\DataTypeTok{x =}\NormalTok{ x3, }\DataTypeTok{y =}\NormalTok{ .fitted), }
              \DataTypeTok{col =} \StringTok{"red"}\NormalTok{, }\DataTypeTok{size =} \FloatTok{1.25}\NormalTok{) }\OperatorTok{+}
\StringTok{    }\KeywordTok{labs}\NormalTok{(}\DataTypeTok{title =} \StringTok{"RCS, 5 knots"}\NormalTok{) }\OperatorTok{+}
\StringTok{    }\KeywordTok{theme_bw}\NormalTok{()}

\NormalTok{gridExtra}\OperatorTok{::}\KeywordTok{grid.arrange}\NormalTok{(p2, p3, p4, p5, }\DataTypeTok{nrow =} \DecValTok{2}\NormalTok{)}
\end{Highlighting}
\end{Shaded}

\includegraphics{bookdown-demo_files/figure-latex/unnamed-chunk-109-1.pdf}

Does it seem like the fit improves markedly (perhaps approaching the
loess smooth result) as we increase the number of knots?

\begin{Shaded}
\begin{Highlighting}[]
\KeywordTok{anova}\NormalTok{(mod5a_rcs, mod5b_rcs, mod5c_rcs)}
\end{Highlighting}
\end{Shaded}

\begin{verbatim}
Analysis of Variance Table

Model 1: y ~ rcs(x3, 3)
Model 2: y ~ rcs(x3, 4)
Model 3: y ~ rcs(x3, 5)
  Res.Df    RSS Df Sum of Sq      F   Pr(>F)   
1     57 194935                                
2     56 171448  1   23486.9 7.9503 0.006672 **
3     55 162481  1    8967.2 3.0354 0.087057 . 
---
Signif. codes:  0 '***' 0.001 '**' 0.01 '*' 0.05 '.' 0.1 ' ' 1
\end{verbatim}

Based on an ANOVA comparison, the fourth knot adds significant
predictive value (p = 0.0067), but the fifth knot is borderline (p =
0.0871). From the \texttt{glance} function in the \texttt{broom}
package, we can also look at some key summaries.

\begin{Shaded}
\begin{Highlighting}[]
\KeywordTok{glance}\NormalTok{(mod5a_rcs)}
\end{Highlighting}
\end{Shaded}

\begin{verbatim}
  r.squared adj.r.squared    sigma statistic    p.value df    logLik
1  0.146184     0.1162256 58.48006  4.879558 0.01106323  3 -327.7187
       AIC      BIC deviance df.residual
1 663.4373 671.8147 194935.3          57
\end{verbatim}

\begin{Shaded}
\begin{Highlighting}[]
\KeywordTok{glance}\NormalTok{(mod5b_rcs)}
\end{Highlighting}
\end{Shaded}

\begin{verbatim}
  r.squared adj.r.squared    sigma statistic   p.value df    logLik
1 0.2490566     0.2088274 55.33153  6.190953 0.0010423  4 -323.8671
       AIC      BIC deviance df.residual
1 657.7342 668.2059 171448.4          56
\end{verbatim}

\begin{Shaded}
\begin{Highlighting}[]
\KeywordTok{glance}\NormalTok{(mod5c_rcs)}
\end{Highlighting}
\end{Shaded}

\begin{verbatim}
  r.squared adj.r.squared    sigma statistic      p.value df    logLik
1 0.2883327     0.2365751 54.35259  5.570826 0.0007734418  5 -322.2555
      AIC      BIC deviance df.residual
1 656.511 669.0771 162481.2          55
\end{verbatim}

\begin{longtable}[]{@{}rrrrrr@{}}
\toprule
Model & Knots & R\textsuperscript{2} & Adj. R\textsuperscript{2} & AIC &
BIC\tabularnewline
\midrule
\endhead
5a & 3 & 0.146 & 0.116 & 663.4 & 671.8\tabularnewline
5b & 4 & 0.249 & 0.209 & 657.7 & \textbf{668.2}\tabularnewline
5c & 5 & 0.288 & \textbf{0.237} & \textbf{656.5} & 669.1\tabularnewline
\bottomrule
\end{longtable}

Within our sample, the five-knot RCS outperforms the 3- and 4-knot
versions on adjusted R\textsuperscript{2} and AIC (barely) and does a
little worse than the 4-knot RCS on BIC.

Of course, we could also use the cross-validation methods we've
developed for other linear regressions to assess predictive capacity of
these models. I'll skip that for now.

To see the values of \texttt{x3} where the splines place their knots, we
can use the \texttt{attributes} function.

\begin{Shaded}
\begin{Highlighting}[]
\KeywordTok{attributes}\NormalTok{(}\KeywordTok{rcs}\NormalTok{(pollution}\OperatorTok{$}\NormalTok{x3, }\DecValTok{5}\NormalTok{))}
\end{Highlighting}
\end{Shaded}

\begin{verbatim}
$dim
[1] 60  4

$dimnames
$dimnames[[1]]
NULL

$dimnames[[2]]
[1] "pollution"    "pollution'"   "pollution''"  "pollution'''"


$class
[1] "rms"

$name
[1] "pollution"

$label
[1] "pollution"

$assume
[1] "rcspline"

$assume.code
[1] 4

$parms
[1] 68 72 74 77 82

$nonlinear
[1] FALSE  TRUE  TRUE  TRUE

$colnames
[1] "pollution"    "pollution'"   "pollution''"  "pollution'''"
\end{verbatim}

The knots in this particular 5-knot spline are placed by the computer at
68, 72, 74, 77 and 82, it seems.

There are two kinds of Multivariate Regression Models

\begin{enumerate}
\def\labelenumi{\arabic{enumi}.}
\tightlist
\item
  {[}Prediction{]} Those that are built so that we can make accurate
  predictions.
\item
  {[}Explanatory{]} Those that are built to help understand underlying
  phenomena.
\end{enumerate}

While those two notions overlap considerably, they do imply different
things about how we strategize about model-building and model
assessment. Harrell's primary concern is effective use of the available
data for \textbf{prediction} - this implies some things that will be
different from what we've seen in the past.

Harrell refers to multivariable regression modeling strategy as the
process of \textbf{spending degrees of freedom}. The main job in
strategizing about multivariate modeling is to

\begin{enumerate}
\def\labelenumi{\arabic{enumi}.}
\tightlist
\item
  Decide the number of degrees of freedom that can be spent
\item
  Decide where to spend them
\item
  Spend them, wisely.
\end{enumerate}

What this means is essentially linked to making decisions about
predictor complexity, both in terms of how many predictors will be
included in the regression model, and about how we'll include those
predictors.

\section{\texorpdfstring{``Spending'' Degrees of
Freedom}{Spending Degrees of Freedom}}\label{spending-degrees-of-freedom}

\begin{itemize}
\tightlist
\item
  ``Spending'' df includes

  \begin{itemize}
  \tightlist
  \item
    fitting parameter estimates in models, or
  \item
    examining figures built using the outcome variable Y that tell you
    how to model the predictors.
  \end{itemize}
\end{itemize}

If you use a scatterplot of Y vs.~X or the residuals of the Y-X
regression model vs.~X to decide whether a linear model is appropriate,
then how many degrees of freedom have you actually spent?

Grambsch and O'Brien conclude that if you wish to preserve the key
statistical properties of the various estimation and fitting procedures
used in building a model, you can't retrieve these degrees of freedom
once they have been spent.

\subsection{Overfitting and Limits on the \# of
Predictors}\label{overfitting-and-limits-on-the-of-predictors}

Suppose you have a total sample size of \(n\) observations, then you
really shouldn't be thinking about estimating more than \(n / 15\)
regression coefficients, at the most.

\begin{itemize}
\tightlist
\item
  If \(k\) is the number of parameters in a full model containing all
  candidate predictors for a stepwise analysis, then \(k\) should be no
  greater than \(n / 15\).
\item
  \(k\) should include all variables screened for association with the
  response, including interaction terms.
\end{itemize}

So if you have 97 observations in your data, then you can probably just
barely justify the use of a stepwise analysis using the main effects
alone of 5 candidate variables (with one additional DF for the intercept
term.)

\citet{Harrell2001} also mentions that if you have a \textbf{narrowly
distributed} predictor, without a lot of variation to work with, then an
even larger sample size \(n\) should be required. See
\citet{Vittinghoff2012}, Section 10.3 for more details.

\subsection{The Importance of
Collinearity}\label{the-importance-of-collinearity}

\begin{quote}
Collinearity denotes correlation between predictors high enough to
degrade the precision of the regression coefficient estimates
substantially for some or all of the correlated predictors
\end{quote}

\begin{itemize}
\item
  \citet{Vittinghoff2012}, section 10.4.1
\item
  Can one predictor in a model be predicted well using the other
  predictors in the model?

  \begin{itemize}
  \tightlist
  \item
    Strong correlations (for instance, \(r \geq 0.8\)) are especially
    troublesome.
  \end{itemize}
\item
  Effects of collinearity

  \begin{itemize}
  \tightlist
  \item
    decreases precision, in the sense of increasing the standard errors
    of the parameter estimates
  \item
    decreases power
  \item
    increases the difficulty of interpreting individual predictor
    effects
  \item
    overall F test is significant, but individual t tests may not be
  \end{itemize}
\end{itemize}

Suppose we want to assess whether variable \(X_j\) is collinear with the
other predictors in a model. We run a regression predicting \(X_j\)
using the other predictors, and obtain the R\textsuperscript{2}. The VIF
is defined as 1 / (1 - this R\textsuperscript{2}), and we usually
interpret VIFs above 5 as indicating a serious multicollinearity problem
(i.e.~R\textsuperscript{2} values for this predictor of 0.8 and above
would thus concern us.)

\begin{Shaded}
\begin{Highlighting}[]
\KeywordTok{vif}\NormalTok{(}\KeywordTok{lm}\NormalTok{(y }\OperatorTok{~}\StringTok{ }\NormalTok{x1 }\OperatorTok{+}\StringTok{ }\NormalTok{x2 }\OperatorTok{+}\StringTok{ }\NormalTok{x3 }\OperatorTok{+}\StringTok{ }\NormalTok{x4 }\OperatorTok{+}\StringTok{ }\NormalTok{x5 }\OperatorTok{+}\StringTok{ }\NormalTok{x6, }\DataTypeTok{data =}\NormalTok{ pollution))}
\end{Highlighting}
\end{Shaded}

\begin{verbatim}
      x1       x2       x3       x4       x5       x6 
2.238862 2.058731 2.153044 4.174448 3.447399 1.792996 
\end{verbatim}

Occasionally, you'll see the inverse of VIF reported, and this is called
\emph{tolerance}.

\begin{itemize}
\tightlist
\item
  tolerance = 1 / VIF
\end{itemize}

\subsection{Collinearity in an Explanatory
Model}\label{collinearity-in-an-explanatory-model}

\begin{itemize}
\tightlist
\item
  When we are attempting to \textbf{identify multiple independent
  predictors} (the explanatory model approach), then we will need to
  choose between collinear variables

  \begin{itemize}
  \tightlist
  \item
    options suggested by \citet{Vittinghoff2012}, p.~422, include
    choosing on the basis of plausibility as a causal factor,
  \item
    choosing the variable that has higher data quality (is measured more
    accurately or has fewer missing values.)
  \item
    Often, we choose to include a variable that is statistically
    significant as a predictor, and drop others, should we be so lucky.
  \end{itemize}
\item
  Larger effects, especially if they are associated with predictors that
  have minimal correlation with the other predictors under study, cause
  less trouble in terms of potential violation of the \(n/15\) rule for
  what constitutes a reasonable number of predictors.
\end{itemize}

\subsection{Collinearity in a Prediction
Model}\label{collinearity-in-a-prediction-model}

\begin{itemize}
\tightlist
\item
  If we are primarily building a \textbf{prediction model} for which
  inference on the individual predictors is not of interest, then it is
  totally reasonable to use both predictors in the model, if doing so
  reduces prediction error.

  \begin{itemize}
  \tightlist
  \item
    Collinearity doesn't affect predictions in our model development
    sample.
  \item
    Collinearity doesn't affect predictions on new data so long as the
    new data have similar relationships between predictors.
  \item
    If our key predictor is correlated strongly with a confounder, then
    if the predictor remains significant after adjustment for the
    confounder, then this suggests a meaningful independent effect.

    \begin{itemize}
    \tightlist
    \item
      If the effects of the predictor are clearly confounded by the
      adjustment variable, we again have a clear result.
    \item
      If neither is statistically significant after adjustment, the data
      may be inadequate.
    \end{itemize}
  \item
    If the collinearity is between adjustment variables, but doesn't
    involve the key predictor, then inclusion of the collinear variables
    is unlikely to cause substantial problems.
  \end{itemize}
\end{itemize}

\section{\texorpdfstring{Spending DF on Non-Linearity: The Spearman
\(\rho^2\)
Plot}{Spending DF on Non-Linearity: The Spearman \textbackslash{}rho\^{}2 Plot}}\label{spending-df-on-non-linearity-the-spearman-rho2-plot}

We need a flexible approach to assessing non-linearity and fitting
models with non-linear predictors. This will lead us to a measure of
what \citet{Harrell2001} calls \textbf{potential predictive punch} which
hides the true form of the regression from the analyst so as to preserve
statistical properties, but that lets us make sensible decisions about
whether a predictor should be included in a model, and the number of
parameters (degrees of freedom, essentially) we are willing to devote to
it.

What if we want to consider where best to spend our degrees of freedom
on non-linear predictor terms, like interactions, polynomial functions
or curved splines to represent our input data? The approach we'll find
useful in the largest variety of settings is a combination of

\begin{enumerate}
\def\labelenumi{\arabic{enumi}.}
\tightlist
\item
  a rank correlation assessment of potential predictive punch (using a
  Spearman \(\rho^2\) plot, available in the \texttt{Hmisc} package),
  followed by
\item
  the application of restricted cubic splines to fit and assess models.
\end{enumerate}

Suppose, for instance, that we want to create a model for \texttt{y}
using some combination of linear and non-linear terms drawn from the
complete set of 15 predictors available in the \texttt{pollution} data.
I'd begin by running a Spearman \(\rho^2\) plot:

\begin{Shaded}
\begin{Highlighting}[]
\KeywordTok{plot}\NormalTok{(Hmisc}\OperatorTok{::}\KeywordTok{spearman2}\NormalTok{(y }\OperatorTok{~}\StringTok{ }\NormalTok{x1 }\OperatorTok{+}\StringTok{ }\NormalTok{x2 }\OperatorTok{+}\StringTok{ }\NormalTok{x3 }\OperatorTok{+}\StringTok{ }\NormalTok{x4 }\OperatorTok{+}\StringTok{ }\NormalTok{x5 }\OperatorTok{+}\StringTok{ }\NormalTok{x6 }\OperatorTok{+}\StringTok{ }\NormalTok{x7 }\OperatorTok{+}
\StringTok{                          }\NormalTok{x8 }\OperatorTok{+}\StringTok{ }\NormalTok{x9 }\OperatorTok{+}\StringTok{ }\NormalTok{x10 }\OperatorTok{+}\StringTok{ }\NormalTok{x11 }\OperatorTok{+}\StringTok{ }\NormalTok{x12 }\OperatorTok{+}\StringTok{ }\NormalTok{x13 }\OperatorTok{+}
\StringTok{                          }\NormalTok{x14 }\OperatorTok{+}\StringTok{ }\NormalTok{x15, }\DataTypeTok{data =}\NormalTok{ pollution))}
\end{Highlighting}
\end{Shaded}

\includegraphics{bookdown-demo_files/figure-latex/unnamed-chunk-114-1.pdf}

The variable with the largest adjusted squared Spearman \(\rho\)
statistic in this setting is \texttt{x9}, followed by \texttt{x6} and
\texttt{x14}. With only 60 observations, we might well want to restrict
ourselves to a very small model. What the Spearman plot suggests is that
we focus any non-linear terms on \texttt{x9} first, and then perhaps
\texttt{x6} and \texttt{x14} as they have some potential predictive
power. It may or may not work out that the non-linear terms are
productive.

\subsection{\texorpdfstring{Fitting a Big Model to the
\texttt{pollution}
data}{Fitting a Big Model to the pollution data}}\label{fitting-a-big-model-to-the-pollution-data}

So, one possible model built in reaction this plot might be to fit:

\begin{itemize}
\tightlist
\item
  a restricted cubic spline with 5 knots on x9
\item
  a restricted cubic spline with 3 knots on x6
\item
  and a quadratic polynomial on x14
\item
  plus a linear fit to x1 and x13
\end{itemize}

That's way more degrees of freedom (4 for \texttt{x9}, 2 for
\texttt{x6}, 2 for \texttt{x14} and 1 each for x1 and x13 makes a total
of 10 without the intercept term) than we can really justify with a
sample of 60 observations. But let's see what happens.

\begin{Shaded}
\begin{Highlighting}[]
\NormalTok{mod_big <-}\StringTok{ }\KeywordTok{lm}\NormalTok{(y }\OperatorTok{~}\StringTok{ }\KeywordTok{rcs}\NormalTok{(x9, }\DecValTok{5}\NormalTok{) }\OperatorTok{+}\StringTok{ }\KeywordTok{rcs}\NormalTok{(x6, }\DecValTok{3}\NormalTok{) }\OperatorTok{+}\StringTok{ }\KeywordTok{poly}\NormalTok{(x14, }\DecValTok{2}\NormalTok{) }\OperatorTok{+}\StringTok{ }\NormalTok{x1 }\OperatorTok{+}\StringTok{ }\NormalTok{x13, }\DataTypeTok{data =}\NormalTok{ pollution)}

\KeywordTok{anova}\NormalTok{(mod_big)}
\end{Highlighting}
\end{Shaded}

\begin{verbatim}
Analysis of Variance Table

Response: y
             Df Sum Sq Mean Sq F value    Pr(>F)    
rcs(x9, 5)    4 100164 25040.9 17.8482 4.229e-09 ***
rcs(x6, 3)    2  38306 19152.8 13.6513 1.939e-05 ***
poly(x14, 2)  2  15595  7797.7  5.5579  0.006677 ** 
x1            1   4787  4787.3  3.4122  0.070759 .  
x13           1    712   711.9  0.5074  0.479635    
Residuals    49  68747  1403.0                      
---
Signif. codes:  0 '***' 0.001 '**' 0.01 '*' 0.05 '.' 0.1 ' ' 1
\end{verbatim}

This \texttt{anova} suggests that we have at least some predictive value
in each spline (\texttt{x9} and \texttt{x6}) and some additional value
in \texttt{x14}, although it's not as clear that the linear terms
(\texttt{x1} and \texttt{x13}) did much good.

\subsection{\texorpdfstring{Limitations of \texttt{lm} for fitting
complex linear regression
models}{Limitations of lm for fitting complex linear regression models}}\label{limitations-of-lm-for-fitting-complex-linear-regression-models}

We can certainly assess this big, complex model using \texttt{lm} in
comparison to other models:

\begin{itemize}
\tightlist
\item
  with in-sample summary statistics like adjusted R\textsuperscript{2},
  AIC and BIC,
\item
  we can assess its assumptions with residual plots, and
\item
  we can also compare out-of-sample predictive quality through
  cross-validation,
\end{itemize}

But to really delve into the details of how well this complex model
works, and to help plot what is actually being fit, we'll probably want
to fit the model using an alternative method for fitting linear models,
called \texttt{ols}, from the \texttt{rms} package developed by Frank
Harrell and colleagues. That will be the focus of our next chapter.

\chapter{\texorpdfstring{Using \texttt{ols} from the \texttt{rms}
package to fit linear
models}{Using ols from the rms package to fit linear models}}\label{using-ols-from-the-rms-package-to-fit-linear-models}

At the end of the previous chapter, we had fit a model to the
\texttt{pollution} data that predicted our outcome \texttt{y} =
Age-Adjusted Mortality Rate, using:

\begin{itemize}
\tightlist
\item
  a restricted cubic spline with 5 knots on \texttt{x9}
\item
  a restricted cubic spline with 3 knots on \texttt{x6}
\item
  a polynomial in 2 degrees on \texttt{x14}
\item
  linear terms for \texttt{x1} and \texttt{x13}
\end{itemize}

but this model was hard to evaluate in some ways. Now, instead of using
\texttt{lm} to fit this model, we'll use a new function called
\texttt{ols} from the \texttt{rms} package developed by Frank Harrell
and colleagues, in part to support ideas developed in
\citet{Harrell2001} for clinical prediction models.

\section{\texorpdfstring{Fitting a model with
\texttt{ols}}{Fitting a model with ols}}\label{fitting-a-model-with-ols}

We will use the \texttt{datadist} approach when fitting a linear model
with \texttt{ols} from the \texttt{rms} package, so as to store
additional important elements of the model fit.

\begin{Shaded}
\begin{Highlighting}[]
\KeywordTok{library}\NormalTok{(rms)}

\NormalTok{d <-}\StringTok{ }\KeywordTok{datadist}\NormalTok{(pollution)}
\KeywordTok{options}\NormalTok{(}\DataTypeTok{datadist =} \StringTok{"d"}\NormalTok{)}
\end{Highlighting}
\end{Shaded}

Next, we'll fit the model using \texttt{ols} and place its results in
\texttt{newmod}.

\begin{Shaded}
\begin{Highlighting}[]
\NormalTok{newmod <-}\StringTok{ }\KeywordTok{ols}\NormalTok{(y }\OperatorTok{~}\StringTok{ }\KeywordTok{rcs}\NormalTok{(x9, }\DecValTok{5}\NormalTok{) }\OperatorTok{+}\StringTok{ }\KeywordTok{rcs}\NormalTok{(x6, }\DecValTok{3}\NormalTok{) }\OperatorTok{+}\StringTok{ }\KeywordTok{pol}\NormalTok{(x14, }\DecValTok{2}\NormalTok{) }\OperatorTok{+}\StringTok{ }
\StringTok{                  }\NormalTok{x1 }\OperatorTok{+}\StringTok{ }\NormalTok{x13, }
              \DataTypeTok{data =}\NormalTok{ pollution, }\DataTypeTok{x =} \OtherTok{TRUE}\NormalTok{, }\DataTypeTok{y =} \OtherTok{TRUE}\NormalTok{)}
\NormalTok{newmod}
\end{Highlighting}
\end{Shaded}

\begin{verbatim}
Linear Regression Model
 
 ols(formula = y ~ rcs(x9, 5) + rcs(x6, 3) + pol(x14, 2) + x1 + 
     x13, data = pollution, x = TRUE, y = TRUE)
 
                 Model Likelihood     Discrimination    
                    Ratio Test           Indexes        
 Obs       60    LR chi2     72.02    R2       0.699    
 sigma37.4566    d.f.           10    R2 adj   0.637    
 d.f.      49    Pr(> chi2) 0.0000    g       58.961    
 
 Residuals
 
     Min      1Q  Median      3Q     Max 
 -86.189 -18.554  -1.799  18.645 104.307 
 
 
           Coef      S.E.     t     Pr(>|t|)
 Intercept  796.2658 162.3269  4.91 <0.0001 
 x9          -2.6328   6.3504 -0.41 0.6803  
 x9'        121.4651 124.4827  0.98 0.3340  
 x9''      -219.8025 227.6775 -0.97 0.3391  
 x9'''      151.5700 171.3867  0.88 0.3808  
 x6           7.6817  15.5230  0.49 0.6229  
 x6'        -29.4388  18.0531 -1.63 0.1094  
 x14          0.5652   0.2547  2.22 0.0311  
 x14^2       -0.0010   0.0010 -0.96 0.3407  
 x1           1.0717   0.7317  1.46 0.1494  
 x13         -0.1028   0.1443 -0.71 0.4796  
 
\end{verbatim}

Some of the advantages and disadvantages of fitting linear regression
models with \texttt{ols} or \texttt{lm} will reveal themselves over
time. For now, one advantage for \texttt{ols} is that the entire
variance-covariance matrix is saved. Most of the time, there will be
some value to considering both \texttt{ols} and \texttt{lm} approaches.

Most of this output should be familiar, but a few pieces are different.

\subsection{The Model Likelihood Ratio
Test}\label{the-model-likelihood-ratio-test}

The \textbf{Model Likelihood Ratio Test} compares \texttt{newmod} to the
null model with only an intercept term. It is a goodness-of-fit test
that we'll use in several types of model settings this semester.

\begin{itemize}
\tightlist
\item
  In many settings, the logarithm of the likelihood ratio, multiplied by
  -2, yields a value which can be compared to a \(\chi^2\) distribution.
  So here, the value 72.02 is -2(log likelihood), and is compared to a
  \(\chi^2\) distribution with 10 degrees of freedom. We reject the null
  hypothesis that \texttt{newmod} is no better than the null model, and
  conclude instead that at least of these predictors adds statistically
  significant value.

  \begin{itemize}
  \tightlist
  \item
    For \texttt{ols}, interpret the model likelihood ratio test like the
    global (ANOVA) F test in \texttt{lm}.
  \item
    The likelihood function is the probability of observing our data
    under the specified model.
  \item
    We can compare two nested models by evaluating the difference in
    their likelihood ratios and degrees of freedom, then comparing the
    result to a \(\chi^2\) distribution.
  \end{itemize}
\end{itemize}

\subsection{The g statistic}\label{the-g-statistic}

The \textbf{g statistic} is new and is referred to as the g-index. it's
based on Gini's mean difference and is purported to be a robust and
highly efficient measure of variation.

\begin{itemize}
\tightlist
\item
  Here, g = 58.9, which implies that if you randomly select two of the
  60 areas included in the model, the average difference in predicted
  \texttt{y} (Age-Adjusted Mortality Rate) using this model will be
  58.9.

  \begin{itemize}
  \tightlist
  \item
    Technically, g is Gini's mean difference of the predicted values.
  \end{itemize}
\end{itemize}

\section{\texorpdfstring{ANOVA for an \texttt{ols}
model}{ANOVA for an ols model}}\label{anova-for-an-ols-model}

One advantage of the \texttt{ols} approach is that when you apply an
\texttt{anova} to it, it separates out the linear and non-linear
components of restricted cubic splines and polynomial terms (as well as
product terms, if your model includes them.)

\begin{Shaded}
\begin{Highlighting}[]
\KeywordTok{anova}\NormalTok{(newmod)}
\end{Highlighting}
\end{Shaded}

\begin{verbatim}
                Analysis of Variance          Response: y 

 Factor          d.f. Partial SS  MS         F     P     
 x9               4    35219.7647  8804.9412  6.28 0.0004
  Nonlinear       3     1339.3081   446.4360  0.32 0.8121
 x6               2     9367.6008  4683.8004  3.34 0.0437
  Nonlinear       1     3730.7388  3730.7388  2.66 0.1094
 x14              2    18679.6957  9339.8478  6.66 0.0028
  Nonlinear       1     1298.7625  1298.7625  0.93 0.3407
 x1               1     3009.1829  3009.1829  2.14 0.1494
 x13              1      711.9108   711.9108  0.51 0.4796
 TOTAL NONLINEAR  5     6656.1824  1331.2365  0.95 0.4582
 REGRESSION      10   159563.8285 15956.3829 11.37 <.0001
 ERROR           49    68746.8004  1402.9959             
\end{verbatim}

Unlike the \texttt{anova} approach in \texttt{lm}, in \texttt{ols}
ANOVA, \emph{partial} F tests are presented - each predictor is assessed
as ``last predictor in'' much like the usual \emph{t} tests in
\texttt{lm}. In essence, the partial sums of squares and F tests here
describe the marginal impact of removing each covariate from
\texttt{newmod}.

We conclude that the non-linear parts of \texttt{x9} and \texttt{x6} and
\texttt{x14} combined don't seem to add much value, but that overall,
\texttt{x9}, \texttt{x6} and \texttt{x14} seem to be valuable. So it
must be the linear parts of those variables within our model that are
doing the lion's share of the work.

\section{Effect Estimates}\label{effect-estimates}

A particularly useful thing to get out of the \texttt{ols} approach that
is not as easily available in \texttt{lm} (without recoding or
standardizing our predictors) is a summary of the effects of each
predictor in an interesting scale.

\begin{Shaded}
\begin{Highlighting}[]
\KeywordTok{summary}\NormalTok{(newmod)}
\end{Highlighting}
\end{Shaded}

\begin{verbatim}
             Effects              Response : y 

 Factor Low   High  Diff. Effect   S.E.    Lower 0.95 Upper 0.95
 x9      4.95 15.65 10.70  40.4060 14.0790  12.1120   68.6990   
 x6     10.40 11.50  1.10 -18.2930  8.1499 -34.6710   -1.9153   
 x14    11.00 69.00 58.00  28.3480 10.6480   6.9503   49.7460   
 x1     32.75 43.25 10.50  11.2520  7.6833  -4.1878   26.6930   
 x13     4.00 23.75 19.75  -2.0303  2.8502  -7.7579    3.6973   
\end{verbatim}

This ``effects summary'' shows the effect on \texttt{y} of moving from
the 25th to the 75th percentile of each variable (along with a standard
error and 95\% confidence interval) while holding the other variable at
the level specified at the bottom of the output.

The most useful way to look at this sort of analysis is often a plot.

\begin{Shaded}
\begin{Highlighting}[]
\KeywordTok{plot}\NormalTok{(}\KeywordTok{summary}\NormalTok{(newmod))}
\end{Highlighting}
\end{Shaded}

\includegraphics{bookdown-demo_files/figure-latex/unnamed-chunk-120-1.pdf}

For \texttt{x9} note from the \texttt{summary} above that the 25th
percentile is 4.95 and the 75th is 15.65. Our conclusion is that the
estimated effect of moving \texttt{x9} from 4.95 to 15.65 is an increase
of 40.4 on \texttt{y}, with a 95\% CI of (12.1, 68.7).

For a categorical variable, the low level is shown first and then the
high level.

The plot shows the point estimate (arrow head) and then the 90\%
(narrowest bar), 95\% (middle bar) and 99\% (widest bar in lightest
color) confidence intervals for each predictor's effect.

\begin{itemize}
\tightlist
\item
  It's easier to distinguish this in the \texttt{x9} plot than the one
  for \texttt{x13}.
\item
  Remember that what is being compared is the first value to the second
  value's impact on the outcome, with other predictors held constant.
\end{itemize}

\subsection{Simultaneous Confidence
Intervals}\label{simultaneous-confidence-intervals}

These confidence intervals make no effort to deal with the multiple
comparisons problem, but just fit individual 95\% (or whatever level you
choose) confidence intervals for each predictor. The natural alternative
is to make an adjustment for multiple comparisons in fitting the
confidence intervals, so that the set of (in this case, 2) confidence
intervals for effect sizes has a family-wise 95\% confidence level.
You'll note that the effect estimates and standard errors are unchanged,
but the confidence limits are a bit wider.

\begin{Shaded}
\begin{Highlighting}[]
\KeywordTok{summary}\NormalTok{(newmod, }\DataTypeTok{conf.type=}\KeywordTok{c}\NormalTok{(}\StringTok{'simultaneous'}\NormalTok{))}
\end{Highlighting}
\end{Shaded}

\begin{verbatim}
             Effects              Response : y 

 Factor Low   High  Diff. Effect   S.E.    Lower 0.95 Upper 0.95
 x9      4.95 15.65 10.70  40.4060 14.0790   3.12280  77.6890   
 x6     10.40 11.50  1.10 -18.2930  8.1499 -39.87400   3.2882   
 x14    11.00 69.00 58.00  28.3480 10.6480   0.15192  56.5440   
 x1     32.75 43.25 10.50  11.2520  7.6833  -9.09340  31.5980   
 x13     4.00 23.75 19.75  -2.0303  2.8502  -9.57760   5.5171   
\end{verbatim}

Remember that if you're looking for the usual \texttt{lm} summary for an
\texttt{ols} object, use \texttt{summary.lm}, and that the
\texttt{display} function from \texttt{arm} does not recognize
\texttt{ols} objects.

\section{\texorpdfstring{The \texttt{Predict} function for an
\texttt{ols}
model}{The Predict function for an ols model}}\label{the-predict-function-for-an-ols-model}

The \texttt{Predict} function is very flexible, and can be used to
produce individual or simultaneous confidence limits.

\begin{Shaded}
\begin{Highlighting}[]
\KeywordTok{Predict}\NormalTok{(newmod, }\DataTypeTok{x9 =} \DecValTok{12}\NormalTok{, }\DataTypeTok{x6 =} \DecValTok{12}\NormalTok{, }\DataTypeTok{x14 =} \DecValTok{40}\NormalTok{, }\DataTypeTok{x1 =} \DecValTok{40}\NormalTok{, }\DataTypeTok{x13 =} \DecValTok{20}\NormalTok{) }\CommentTok{# individual limits}
\end{Highlighting}
\end{Shaded}

\begin{verbatim}
  x9 x6 x14 x1 x13     yhat    lower   upper
1 12 12  40 40  20 923.0982 893.0984 953.098

Response variable (y): y 

Limits are 0.95 confidence limits
\end{verbatim}

\begin{Shaded}
\begin{Highlighting}[]
\KeywordTok{Predict}\NormalTok{(newmod, }\DataTypeTok{x9 =} \DecValTok{5}\OperatorTok{:}\DecValTok{15}\NormalTok{) }\CommentTok{# individual limits}
\end{Highlighting}
\end{Shaded}

\begin{verbatim}
   x9    x6 x14 x1 x13     yhat    lower    upper
1   5 11.05  30 38   9 913.7392 889.4802 937.9983
2   6 11.05  30 38   9 916.3490 892.0082 940.6897
3   7 11.05  30 38   9 921.3093 898.9657 943.6529
4   8 11.05  30 38   9 927.6464 907.0355 948.2574
5   9 11.05  30 38   9 934.3853 913.3761 955.3946
6  10 11.05  30 38   9 940.5510 917.8371 963.2648
7  11 11.05  30 38   9 945.2225 921.9971 968.4479
8  12 11.05  30 38   9 948.2885 926.4576 970.1194
9  13 11.05  30 38   9 950.2608 930.3003 970.2213
10 14 11.05  30 38   9 951.6671 932.2370 971.0971
11 15 11.05  30 38   9 953.0342 932.1662 973.9021

Response variable (y): y 

Adjust to: x6=11.05 x14=30 x1=38 x13=9  

Limits are 0.95 confidence limits
\end{verbatim}

\begin{Shaded}
\begin{Highlighting}[]
\KeywordTok{Predict}\NormalTok{(newmod, }\DataTypeTok{x9 =} \DecValTok{5}\OperatorTok{:}\DecValTok{15}\NormalTok{, }\DataTypeTok{conf.type =} \StringTok{'simult'}\NormalTok{)}
\end{Highlighting}
\end{Shaded}

\begin{verbatim}
   x9    x6 x14 x1 x13     yhat    lower    upper
1   5 11.05  30 38   9 913.7392 882.4311 945.0473
2   6 11.05  30 38   9 916.3490 884.9354 947.7625
3   7 11.05  30 38   9 921.3093 892.4733 950.1454
4   8 11.05  30 38   9 927.6464 901.0465 954.2464
5   9 11.05  30 38   9 934.3853 907.2713 961.4993
6  10 11.05  30 38   9 940.5510 911.2371 969.8649
7  11 11.05  30 38   9 945.2225 915.2484 975.1966
8  12 11.05  30 38   9 948.2885 920.1141 976.4629
9  13 11.05  30 38   9 950.2608 924.5003 976.0212
10 14 11.05  30 38   9 951.6671 926.5912 976.7430
11 15 11.05  30 38   9 953.0342 926.1025 979.9658

Response variable (y): y 

Adjust to: x6=11.05 x14=30 x1=38 x13=9  

Limits are 0.95 confidence limits
\end{verbatim}

The plot below shows the individual effects in \texttt{newmod} in five
subpanels, using the default approach of displaying the same range of
values as are seen in the data. Note that each panel shows point and
interval estimates of the effects, and spot the straight lines in
\texttt{x1} and \texttt{x13}, the single bends in \texttt{x14} and
\texttt{x6} and the wiggles in \texttt{x9}, corresponding to the amount
of non-linearity specified in the model.

\begin{Shaded}
\begin{Highlighting}[]
\KeywordTok{ggplot}\NormalTok{(}\KeywordTok{Predict}\NormalTok{(newmod))}
\end{Highlighting}
\end{Shaded}

\includegraphics{bookdown-demo_files/figure-latex/unnamed-chunk-123-1.pdf}

\section{\texorpdfstring{Checking Influence via
\texttt{dfbeta}}{Checking Influence via dfbeta}}\label{checking-influence-via-dfbeta}

For an \texttt{ols} object, we have several tools for looking at
residuals. The most interesting to me is \texttt{which.influence} which
is reliant on the notion of \texttt{dfbeta}.

\begin{itemize}
\tightlist
\item
  DFBETA is estimated for each observation in the data, and each
  coefficient in the model.
\item
  The DFBETA is the difference in the estimated coefficient caused by
  deleting the observation, scaled by the coefficient's standard error
  estimated with the observation deleted.
\item
  The \texttt{which.influence} command applied to an \texttt{ols} model
  produces a list of all of the predictors estimated by the model,
  including the intercept.

  \begin{itemize}
  \tightlist
  \item
    For each predictor, the command lists all observations (by row
    number) that, if removed from the model, would cause the estimated
    coefficient (the ``beta'') for that predictor to change by at least
    some particular cutoff.
  \item
    The default is that the DFBETA for that predictor is 0.2 or more.
  \end{itemize}
\end{itemize}

\begin{Shaded}
\begin{Highlighting}[]
\KeywordTok{which.influence}\NormalTok{(newmod)}
\end{Highlighting}
\end{Shaded}

\begin{verbatim}
$Intercept
[1]  2 11 28 32 37 49 59

$x9
[1]  2  3  6  9 31 35 49 57 58

$x6
[1]  2 11 15 28 32 37 50 56 59

$x14
[1]  2  6  7 12 13 16 32 37

$x1
[1]  7 18 32 37 49 57

$x13
[1] 29 32 37
\end{verbatim}

The implication here, for instance, is that if we drop row 3 from our
data frame, and refit the model, this will have a meaningful impact on
the estimate of \texttt{x9} but not on the other coefficients. But if we
drop, say, row 37, we will affect the estimates of the intercept,
\texttt{x6}, \texttt{x14}, \texttt{x1}, and \texttt{x13}.

\subsection{\texorpdfstring{Using the \texttt{residuals} command for
\texttt{dfbetas}}{Using the residuals command for dfbetas}}\label{using-the-residuals-command-for-dfbetas}

To see the \texttt{dfbeta} values, standardized according to the
approach I used above, you can use the following code (I'll use
\texttt{head} to just show the first few rows of results) to get a
matrix of the results.

\begin{Shaded}
\begin{Highlighting}[]
\KeywordTok{head}\NormalTok{(}\KeywordTok{residuals}\NormalTok{(newmod, }\DataTypeTok{type =} \StringTok{"dfbetas"}\NormalTok{))}
\end{Highlighting}
\end{Shaded}

\begin{verbatim}
            [,1]         [,2]         [,3]        [,4]        [,5]
[1,]  0.03071160 -0.023775487 -0.004055111  0.01205425 -0.03260003
[2,] -0.38276573 -0.048404993 -0.142293606  0.17009666 -0.22350621
[3,]  0.17226780 -0.426153536  0.350913139 -0.32949129  0.25777913
[4,]  0.06175110 -0.006460916  0.024828272 -0.03009337  0.04154812
[5,]  0.16875200  0.039839994 -0.058178534  0.06449504 -0.07772208
[6,]  0.03322073  0.112699877 -0.203543632  0.23987378 -0.35201736
            [,6]        [,7]        [,8]         [,9]        [,10]
[1,] -0.02392315  0.01175375 -0.06494414  0.060929683 -0.011042644
[2,]  0.44737372 -0.48562818  0.19372285 -0.212186731 -0.107830147
[3,] -0.10263448  0.05005284 -0.02049877  0.014059330  0.010793169
[4,] -0.06254145  0.05498432  0.01135031 -0.001877983 -0.005490454
[5,] -0.18058630  0.16151742  0.02723710  0.065483158  0.003326357
[6,] -0.04075617  0.02900006 -0.21508009  0.171627718  0.019241676
           [,11]
[1,]  0.03425156
[2,] -0.01503250
[3,]  0.04924166
[4,] -0.01254111
[5,] -0.05570035
[6,]  0.05775536
\end{verbatim}

\subsection{\texorpdfstring{Using the \texttt{residuals} command for
other
summaries}{Using the residuals command for other summaries}}\label{using-the-residuals-command-for-other-summaries}

The \texttt{residuals} command will also let you get ordinary residuals,
leverage values and \texttt{dffits} values, which are the normalized
differences in predicted values when observations are omitted. See
\texttt{?residuals.ols} for more details.

\begin{Shaded}
\begin{Highlighting}[]
\NormalTok{temp <-}\StringTok{ }\KeywordTok{data.frame}\NormalTok{(}\DataTypeTok{area =} \DecValTok{1}\OperatorTok{:}\DecValTok{60}\NormalTok{)}
\NormalTok{temp}\OperatorTok{$}\NormalTok{residual <-}\StringTok{ }\KeywordTok{residuals}\NormalTok{(newmod, }\DataTypeTok{type =} \StringTok{"ordinary"}\NormalTok{)}
\NormalTok{temp}\OperatorTok{$}\NormalTok{leverage <-}\StringTok{ }\KeywordTok{residuals}\NormalTok{(newmod, }\DataTypeTok{type =} \StringTok{"hat"}\NormalTok{)}
\NormalTok{temp}\OperatorTok{$}\NormalTok{dffits <-}\StringTok{ }\KeywordTok{residuals}\NormalTok{(newmod, }\DataTypeTok{type =} \StringTok{"dffits"}\NormalTok{)}
\KeywordTok{tbl_df}\NormalTok{(temp)}
\end{Highlighting}
\end{Shaded}

\begin{verbatim}
# A tibble: 60 x 4
    area residual leverage  dffits
   <int>    <dbl>    <dbl>   <dbl>
 1     1   -13.3    0.0929 -0.119 
 2     2    81.0    0.0941  0.766 
 3     3    28.8    0.266   0.539 
 4     4   -12.5    0.117  -0.128 
 5     5    27.8    0.204   0.419 
 6     6   -40.4    0.416  -1.20  
 7     7    37.0    0.207   0.568 
 8     8   -14.3    0.145  -0.169 
 9     9    66.6    0.0863  0.587 
10    10   - 4.96   0.0997 -0.0460
# ... with 50 more rows
\end{verbatim}

\begin{Shaded}
\begin{Highlighting}[]
\KeywordTok{ggplot}\NormalTok{(temp, }\KeywordTok{aes}\NormalTok{(}\DataTypeTok{x =}\NormalTok{ area, }\DataTypeTok{y =}\NormalTok{ dffits)) }\OperatorTok{+}
\StringTok{    }\KeywordTok{geom_point}\NormalTok{() }\OperatorTok{+}
\StringTok{    }\KeywordTok{geom_line}\NormalTok{()}
\end{Highlighting}
\end{Shaded}

\includegraphics{bookdown-demo_files/figure-latex/unnamed-chunk-126-1.pdf}

It appears that point 37 has the largest (positive) \texttt{dffits}
value. Recall that point 37 seemed influential on several predictors and
the intercept term. Point 32 has the smallest (or largest negative)
\texttt{dffits}, and also appears to have been influential on several
predictors and the intercept.

\begin{Shaded}
\begin{Highlighting}[]
\KeywordTok{which.max}\NormalTok{(temp}\OperatorTok{$}\NormalTok{dffits)}
\end{Highlighting}
\end{Shaded}

\begin{verbatim}
[1] 37
\end{verbatim}

\begin{Shaded}
\begin{Highlighting}[]
\KeywordTok{which.min}\NormalTok{(temp}\OperatorTok{$}\NormalTok{dffits)}
\end{Highlighting}
\end{Shaded}

\begin{verbatim}
[1] 32
\end{verbatim}

\section{Model Validation and Correcting for
Optimism}\label{model-validation-and-correcting-for-optimism}

In 431, we learned about splitting our regression models into
\textbf{training} samples and \textbf{test} samples, performing variable
selection work on the training sample to identify two or three candidate
models (perhaps via a stepwise approach), and then comparing the
predictions made by those models in a test sample.

At the final project presentations, I mentioned (to many folks) that
there was a way to automate this process a bit in 432, that would
provide some ways to get the machine to split the data for you multiple
times, and then average over the results, using a bootstrap approach.
This is it.

The \texttt{validate} function allows us to perform cross-validation of
our models for some summary statistics (and then correct those
statistics for optimism in describing likely predictive accuracy) in an
easy way.

\texttt{validate} develops:

\begin{itemize}
\tightlist
\item
  Resampling validation with or without backward elimination of
  variables
\item
  Estimates of the \emph{optimism} in measures of predictive accuracy
\item
  Estimates of the intercept and slope of a calibration model
\end{itemize}

\begin{center}
(observed y) = Intercept + Slope (predicted y)
\end{center}

with the following code\ldots{}

\begin{Shaded}
\begin{Highlighting}[]
\KeywordTok{set.seed}\NormalTok{(}\DecValTok{432002}\NormalTok{); }\KeywordTok{validate}\NormalTok{(newmod, }\DataTypeTok{method =} \StringTok{"boot"}\NormalTok{, }\DataTypeTok{B =} \DecValTok{40}\NormalTok{)}
\end{Highlighting}
\end{Shaded}

\begin{verbatim}
          index.orig training      test  optimism index.corrected  n
R-square      0.6989   0.7500    0.5964    0.1536          0.5452 40
MSE        1145.7800 888.2564 1535.8277 -647.5714       1793.3514 40
g            58.9614  58.9323   55.2085    3.7238         55.2376 40
Intercept     0.0000   0.0000   86.5968  -86.5968         86.5968 40
Slope         1.0000   1.0000    0.9088    0.0912          0.9088 40
\end{verbatim}

So, for \texttt{R-square} we see that our original estimate was 0.6989

\begin{itemize}
\tightlist
\item
  Our estimated \texttt{R-square} across \texttt{n} = 40 training
  samples was 0.7500, but in the resulting tests, the average
  \texttt{R-square} was only 0.5964
\item
  This suggests an optimism of 0.7500 - 0.5964 = 0.1536 (after
  rounding).
\item
  We then apply that optimism to obtain a new estimate of
  R\textsuperscript{2} corrected for overfitting, at 0.5452, which is
  probably a better estimate of what our results might look like in new
  data that were similar to (but not the same as) the data we used in
  building \texttt{newmod} than our initial estimate of 0.6989
\end{itemize}

We also obtain optimism-corrected estimates of the mean squared error
(square of the residual standard deviation), the g index, and the
intercept and slope of the calibration model. The ``corrected'' slope is
a shrinkage factor that takes overfitting into account.

\section{Building a Nomogram for Our
Model}\label{building-a-nomogram-for-our-model}

Another nice feature of an \texttt{ols} model object is that we can
picture the model with a \textbf{nomogram} easily. Here is model
\texttt{newmod}.

\begin{Shaded}
\begin{Highlighting}[]
\KeywordTok{plot}\NormalTok{(}\KeywordTok{nomogram}\NormalTok{(newmod))}
\end{Highlighting}
\end{Shaded}

\includegraphics{bookdown-demo_files/figure-latex/unnamed-chunk-129-1.pdf}

For this model, we can use this plot to predict \texttt{y} as follows:

\begin{enumerate}
\def\labelenumi{\arabic{enumi}.}
\tightlist
\item
  find our values of \texttt{x9} on the appropriate line
\item
  draw a vertical line up to the points line to count the points
  associated with our subject
\item
  repeat the process to obtain the points associated with \texttt{x6},
  \texttt{x14}, \texttt{x1}, and \texttt{x13}. Sum the points.
\item
  draw a vertical line down from that number in the Total Points line to
  estimate \texttt{y} (the Linear Predictor) = Age-Adjusted Mortality
  Rate.
\end{enumerate}

The impact of the non-linearity is seen in the \texttt{x6} results, for
example, which turn around from 9-10 to 11-12. We also see
non-linearity's effects in the scales of the non-linear terms in terms
of points awarded.

An area with a combination of predictor values leading to a total of 100
points, for instance, would lead to a prediction of a Mortality Rate
near 905. An area with a total of 140 points would have a predicted
Mortality Rate of 955, roughly.

\chapter{Other Variable Selection
Strategies}\label{other-variable-selection-strategies}

\section{Why not use stepwise
procedures?}\label{why-not-use-stepwise-procedures}

\begin{enumerate}
\def\labelenumi{\arabic{enumi}.}
\tightlist
\item
  The R\textsuperscript{2} for a model selected in a stepwise manner is
  biased, high.
\item
  The coefficient estimates and standard errors are biased.
\item
  The \(p\) values for the individual-variable t tests are too small.
\item
  In stepwise analyses of prediction models, the final model represented
  noise 20-74\% of the time.
\item
  In stepwise analyses, the final model usually contained less than half
  of the actual number of real predictors.
\item
  It is not logical that a population regression coefficient would be
  exactly zero just because its estimate was not statistically
  significant.
\end{enumerate}

This last comment applies to things like our ``best subsets'' approach
as well as standard stepwise procedures.

Sander Greenland's comments on parsimony and stepwise approaches to
model selection are worth addressing\ldots{}

\begin{itemize}
\tightlist
\item
  Stepwise variable selection on confounders leaves important
  confounders uncontrolled.
\item
  Shrinkage approaches (like ridge regression and the lasso) are far
  superior to variable selection.
\item
  Variable selection does more damage to confidence interval widths than
  to point estimates.
\end{itemize}

If we are seriously concerned about \textbf{overfitting} - winding up
with a model that doesn't perform well on new data - then stepwise
approaches generally don't help.

\citet{Vittinghoff2012} suggest four strategies for minimizing the
chance of overfitting

\begin{enumerate}
\def\labelenumi{\arabic{enumi}.}
\tightlist
\item
  Pre-specify well-motivated predictors and how to model them.
\item
  Eliminate predictors without using the outcome.
\item
  Use the outcome, but cross-validate the target measure of prediction
  error.
\item
  Use the outcome, and \textbf{shrink} the coefficient estimates.
\end{enumerate}

The best subsets methods we have studied either include a variable or
drop it from the model. Often, this choice is based on only a tiny
difference in the quality of a fit to data.

\begin{itemize}
\tightlist
\item
  \citet{Harrell2001}: not reasonable to assume that a population
  regression coefficient would be exactly zero just because it failed to
  meet a criterion for significance.
\item
  Brad Efron has suggested that a stepwise approach is ``overly greedy,
  impulsively eliminating covariates which are correlated with other
  covariates.''
\end{itemize}

So, what's the alternative?

\section{Ridge Regression}\label{ridge-regression}

\textbf{Ridge regression} involves a more smooth transition between
useful and not useful predictors which can be obtained by constraining
the overall size of the regression coefficients.

Ridge regression assumes that the regression coefficients (after
normalization) should not be very large. This is reasonable to assume
when you have lots of predictors and you believe \emph{many} of them
have some effect on the outcome.

Pros:

\begin{enumerate}
\def\labelenumi{\arabic{enumi}.}
\tightlist
\item
  Some nice statistical properties
\item
  Can be calculated using only standard least squares approaches, so
  it's been around for a while.
\item
  Available in the \texttt{MASS} package.
\end{enumerate}

Ridge regression takes the sum of the squared estimated standardized
regression coefficients and constrains that sum to only be as large as
some value \(k\).

\[
\sum \hat{\beta_j}^2 \leq k.
\]

The value \(k\) is one of several available measures of the amount of
shrinkage, but the main one used in the \texttt{MASS} package is a value
\(\lambda\). As \(\lambda\) increases, the amount of shrinkage goes up,
and \(k\) goes down.

\subsection{Assessing a Ridge Regression
Approach}\label{assessing-a-ridge-regression-approach}

We'll look at a plot produced by the \texttt{lm.ridge} function for a
ridge regression for the prostate cancer study we worked on when
studying Stepwise Regression and Best Subsets methods earlier.

\begin{itemize}
\tightlist
\item
  Several (here 101) different values for \(\lambda\), our shrinkage
  parameter, will be tested.
\item
  Results are plotted so that we see the coefficients across the various
  (standardized) predictors.

  \begin{itemize}
  \tightlist
  \item
    Each selection of a \(\lambda\) value implies a different vector of
    covariate values across the predictors we are studying.
  \item
    The idea is to pick a value of \(\lambda\) for which the
    coefficients seem relatively stable.
  \end{itemize}
\end{itemize}

\begin{Shaded}
\begin{Highlighting}[]
\NormalTok{preds <-}\StringTok{ }\KeywordTok{with}\NormalTok{(prost, }\KeywordTok{cbind}\NormalTok{(lcavol, lweight, age, bph_f,}
\NormalTok{                           svi_f, lcp, gleason_f, pgg45))}

\NormalTok{x <-}\StringTok{ }\KeywordTok{lm.ridge}\NormalTok{(prost}\OperatorTok{$}\NormalTok{lpsa }\OperatorTok{~}\StringTok{ }\NormalTok{preds, }\DataTypeTok{lambda =} \DecValTok{0}\OperatorTok{:}\DecValTok{100}\NormalTok{)}
    
\KeywordTok{plot}\NormalTok{(x)}
\KeywordTok{title}\NormalTok{(}\StringTok{"Ridge Regression for prost data"}\NormalTok{)}
\KeywordTok{abline}\NormalTok{(}\DataTypeTok{h =} \DecValTok{0}\NormalTok{)}
\end{Highlighting}
\end{Shaded}

\includegraphics{bookdown-demo_files/figure-latex/ridge_prost_code-1.pdf}

Usually, you need to use trial and error to decide the range of
\(\lambda\) to be tested. Here, \texttt{0:100} means going from 0 (no
shrinkage) to 100 in steps of 1.

\subsection{\texorpdfstring{The \texttt{lm.ridge} plot - where do
coefficients
stabilize?}{The lm.ridge plot - where do coefficients stabilize?}}\label{the-lm.ridge-plot---where-do-coefficients-stabilize}

Does \(\lambda = 20\) seem like a stable spot here?

\begin{Shaded}
\begin{Highlighting}[]
\NormalTok{x <-}\StringTok{ }\KeywordTok{lm.ridge}\NormalTok{(prost}\OperatorTok{$}\NormalTok{lpsa }\OperatorTok{~}\StringTok{ }\NormalTok{preds, }\DataTypeTok{lambda =} \DecValTok{0}\OperatorTok{:}\DecValTok{100}\NormalTok{)}
\KeywordTok{plot}\NormalTok{(x)}
\KeywordTok{title}\NormalTok{(}\StringTok{"Ridge Regression for prost data"}\NormalTok{)}
\KeywordTok{abline}\NormalTok{(}\DataTypeTok{h =} \DecValTok{0}\NormalTok{)}
\KeywordTok{abline}\NormalTok{(}\DataTypeTok{v=}\DecValTok{20}\NormalTok{, }\DataTypeTok{lty=}\DecValTok{2}\NormalTok{, }\DataTypeTok{col=}\StringTok{"black"}\NormalTok{)}
\end{Highlighting}
\end{Shaded}

\includegraphics{bookdown-demo_files/figure-latex/ridge_prost_20-1.pdf}

The coefficients at \(\lambda\) = 20 can be determined from the
\texttt{lm.ridge} output. These are fully standardized coefficients. The
original predictors are centered by their means and then scaled by their
standard deviations and the outcome has also been centered, in these
models.

\begin{Shaded}
\begin{Highlighting}[]
\KeywordTok{round}\NormalTok{(x}\OperatorTok{$}\NormalTok{coef[,}\DecValTok{20}\NormalTok{],}\DecValTok{3}\NormalTok{)}
\end{Highlighting}
\end{Shaded}

\begin{verbatim}
   predslcavol   predslweight       predsage     predsbph_f     predssvi_f 
         0.482          0.248         -0.091          0.097          0.252 
      predslcp predsgleason_f     predspgg45 
         0.009         -0.099          0.061 
\end{verbatim}

Was an intercept used?

\begin{Shaded}
\begin{Highlighting}[]
\NormalTok{x}\OperatorTok{$}\NormalTok{Inter}
\end{Highlighting}
\end{Shaded}

\begin{verbatim}
[1] 1
\end{verbatim}

Yes, it was. There is an automated way to pick \(\lambda\). Use the
\texttt{select} function in the \texttt{MASS} package:

\begin{Shaded}
\begin{Highlighting}[]
\NormalTok{MASS}\OperatorTok{::}\KeywordTok{select}\NormalTok{(x)}
\end{Highlighting}
\end{Shaded}

\begin{verbatim}
modified HKB estimator is 4.210238 
modified L-W estimator is 3.32223 
smallest value of GCV  at 6 
\end{verbatim}

I'll use the GCV = generalized cross-validation to select \(\lambda\) =
6 instead.

\begin{Shaded}
\begin{Highlighting}[]
\NormalTok{x <-}\StringTok{ }\KeywordTok{lm.ridge}\NormalTok{(prost}\OperatorTok{$}\NormalTok{lpsa }\OperatorTok{~}\StringTok{ }\NormalTok{preds, }\DataTypeTok{lambda =} \DecValTok{0}\OperatorTok{:}\DecValTok{100}\NormalTok{)}
\KeywordTok{plot}\NormalTok{(x)}
\KeywordTok{title}\NormalTok{(}\StringTok{"Ridge Regression for prost data"}\NormalTok{)}
\KeywordTok{abline}\NormalTok{(}\DataTypeTok{h =} \DecValTok{0}\NormalTok{)}
\KeywordTok{abline}\NormalTok{(}\DataTypeTok{v=}\DecValTok{6}\NormalTok{, }\DataTypeTok{lty=}\DecValTok{2}\NormalTok{, }\DataTypeTok{col=}\StringTok{"black"}\NormalTok{)}
\end{Highlighting}
\end{Shaded}

\includegraphics{bookdown-demo_files/figure-latex/ridge for ptsdmale with 40 line-1.pdf}

\begin{Shaded}
\begin{Highlighting}[]
\NormalTok{x}\OperatorTok{$}\NormalTok{coef[,}\DecValTok{6}\NormalTok{]}
\end{Highlighting}
\end{Shaded}

\begin{verbatim}
   predslcavol   predslweight       predsage     predsbph_f     predssvi_f 
    0.58911149     0.26773757    -0.13715070     0.11862949     0.29491008 
      predslcp predsgleason_f     predspgg45 
   -0.09389545    -0.10477578     0.07250609 
\end{verbatim}

\subsection{Ridge Regression: The Bottom
Line}\label{ridge-regression-the-bottom-line}

The main problem with ridge regression is that all it does is shrink the
coefficient estimates, but it's not so useful in practical settings
because it still includes all variables.

\begin{enumerate}
\def\labelenumi{\arabic{enumi}.}
\tightlist
\item
  It's been easy to do ridge regression for many years, so you see it
  occasionally in the literature.
\item
  It leads to the \textbf{lasso}, which incorporates the positive
  features of shrinking regression coefficients with the ability to
  wisely select some variables to be eliminated from the predictor pool.
\end{enumerate}

\section{The Lasso}\label{the-lasso}

The lasso works by takes the sum of the absolute values of the estimated
standardized regression coefficients and constrains it to only be as
large as some value k.

\[
\sum \hat{|\beta_j|} \leq k.
\]

This looks like a minor change, but it's not.

\subsection{Consequences of the Lasso
Approach}\label{consequences-of-the-lasso-approach}

\begin{enumerate}
\def\labelenumi{\arabic{enumi}.}
\tightlist
\item
  In ridge regression, while the individual coefficients shrink and
  sometimes approach zero, they seldom reach zero and are thus excluded
  from the model. With the lasso, some coefficients do reach zero and
  thus, those predictors do drop out of the model.

  \begin{itemize}
  \tightlist
  \item
    So the lasso leads to more parsimonious models than does ridge
    regression.
  \item
    Ridge regression is a method of shrinkage but not model selection.
    The lasso accomplishes both tasks.
  \end{itemize}
\item
  If k is chosen to be too small, then the model may not capture
  important characteristics of the data. If k is too large, the model
  may over-fit the data in the sample and thus not represent the
  population of interest accurately.
\item
  The lasso is far more difficult computationally than ridge regression
  (the problem requires an algorithm called least angle regression
  published in 2004), although R has a library (\texttt{lars}) which can
  do the calculations pretty efficiently.
\end{enumerate}

The lasso is not an acronym, but rather refers to cowboys using a rope
to pull cattle from the herd, much as we will pull predictors from a
model.

\subsection{How The Lasso Works}\label{how-the-lasso-works}

The \texttt{lars} package lets us compute the lasso coefficient
estimates \textbf{and} do cross-validation to determine the appropriate
amount of shrinkage. The main tool is a pair of graphs.

\begin{enumerate}
\def\labelenumi{\arabic{enumi}.}
\tightlist
\item
  The first plot shows what coefficients get selected as we move from
  constraining all of the coefficients to zero (complete shrinkage)
  towards fewer constraints all the way up to ordinary least squares,
  showing which variables are included in the model at each point.
\item
  The second plot suggests where on the first plot we should look for a
  good model choice, according to a cross-validation approach.
\end{enumerate}

\begin{Shaded}
\begin{Highlighting}[]
\NormalTok{## requires lars package}
\NormalTok{lasso1 <-}\StringTok{ }\KeywordTok{lars}\NormalTok{(preds, prost}\OperatorTok{$}\NormalTok{lpsa, }\DataTypeTok{type=}\StringTok{"lasso"}\NormalTok{)}
\KeywordTok{plot}\NormalTok{(lasso1)}
\end{Highlighting}
\end{Shaded}

\includegraphics{bookdown-demo_files/figure-latex/lasso_graph1_forprost-1.pdf}

\begin{itemize}
\tightlist
\item
  The y axis shows standardized regression coefficients.

  \begin{itemize}
  \tightlist
  \item
    The \texttt{lars} package standardizes all variables so the
    shrinkage doesn't penalize some coefficients because of their scale.
  \end{itemize}
\item
  The x-axis is labeled
  \texttt{\textbar{}beta\textbar{}/max\textbar{}beta\textbar{}}.

  \begin{itemize}
  \tightlist
  \item
    This ranges from 0 to 1.
  \item
    0 means that the sum of the \(|\hat{\beta_j}|\) is zero (completely
    shrunk)
  \item
    1 means the ordinary least squares unbiased estimates.
  \end{itemize}
\end{itemize}

The lasso graph starts at constraining all of the coefficients to zero,
and then moves toward ordinary least squares.

Identifiers for the predictors (numbers) are shown to the right of the
graph.

The vertical lines in the lasso plot show when a variable has been
eliminated from the model, and in fact these are the only points that
are actually shown in the default lasso graph. The labels on the top of
the graph tell you how many predictors are in the model at that stage.

\begin{Shaded}
\begin{Highlighting}[]
\KeywordTok{summary}\NormalTok{(lasso1)}
\end{Highlighting}
\end{Shaded}

\begin{verbatim}
LARS/LASSO
Call: lars(x = preds, y = prost$lpsa, type = "lasso")
  Df     Rss       Cp
0  1 127.918 168.1835
1  2  76.392  64.1722
2  3  70.247  53.5293
3  4  50.598  15.1017
4  5  49.065  13.9485
5  6  48.550  14.8898
6  7  46.284  12.2276
7  8  44.002   9.5308
8  9  42.772   9.0000
\end{verbatim}

Based on the C\textsubscript{p} statistics, it looks like the
improvements continue throughout, and don't really finish happening
until we get pretty close to the full model with 9 df.

\subsection{Cross-Validation with the
Lasso}\label{cross-validation-with-the-lasso}

Normally, cross-validation methods are used to determine how much
shrinkage should be used. We'll use the \texttt{cv.lars} function.

\begin{itemize}
\tightlist
\item
  10-fold (K = 10) cross-validation

  \begin{itemize}
  \tightlist
  \item
    the data are randomly divided into 10 groups.
  \item
    Nine groups are used to predict the remaining group for each group
    in turn.
  \item
    Overall prediction performance is computed, and the machine
    calculates a cross-validation criterion (mean squared error) and
    standard error for that criterion.
  \end{itemize}
\end{itemize}

The cross-validation plot is the second lasso plot.

\begin{Shaded}
\begin{Highlighting}[]
\KeywordTok{set.seed}\NormalTok{(}\DecValTok{432}\NormalTok{)}
\NormalTok{lassocv <-}\StringTok{ }\KeywordTok{cv.lars}\NormalTok{(preds, prost}\OperatorTok{$}\NormalTok{lpsa, }\DataTypeTok{K=}\DecValTok{10}\NormalTok{)}
\end{Highlighting}
\end{Shaded}

\includegraphics{bookdown-demo_files/figure-latex/lasso_graph2-1.pdf}

\begin{Shaded}
\begin{Highlighting}[]
\NormalTok{## default cv.lars K is 10}
\end{Highlighting}
\end{Shaded}

We're looking to minimize cross-validated mean squared error in this
plot, which doesn't seem to happen until the fraction gets very close to
1.

\subsection{What value of the key fraction minimizes cross-validated
MSE?}\label{what-value-of-the-key-fraction-minimizes-cross-validated-mse}

\begin{Shaded}
\begin{Highlighting}[]
\NormalTok{frac <-}\StringTok{ }\NormalTok{lassocv}\OperatorTok{$}\NormalTok{index[}\KeywordTok{which.min}\NormalTok{(lassocv}\OperatorTok{$}\NormalTok{cv)]}
\NormalTok{frac}
\end{Highlighting}
\end{Shaded}

\begin{verbatim}
[1] 0.989899
\end{verbatim}

The cross-validation plot suggests we use a fraction of about 0.3,
that's suggesting a model with 4-5 predictors, based on the top LASSO
plot.

\begin{Shaded}
\begin{Highlighting}[]
\KeywordTok{par}\NormalTok{(}\DataTypeTok{mfrow=}\KeywordTok{c}\NormalTok{(}\DecValTok{2}\NormalTok{,}\DecValTok{1}\NormalTok{))}
\NormalTok{lasso1 <-}\StringTok{ }\KeywordTok{lars}\NormalTok{(preds, prost}\OperatorTok{$}\NormalTok{lpsa, }\DataTypeTok{type=}\StringTok{"lasso"}\NormalTok{)}
\KeywordTok{plot}\NormalTok{(lasso1)}
\KeywordTok{set.seed}\NormalTok{(}\DecValTok{432}\NormalTok{)}
\NormalTok{lassocv <-}\StringTok{ }\KeywordTok{cv.lars}\NormalTok{(preds, prost}\OperatorTok{$}\NormalTok{lpsa, }\DataTypeTok{K=}\DecValTok{10}\NormalTok{)}
\end{Highlighting}
\end{Shaded}

\includegraphics{bookdown-demo_files/figure-latex/lasso_bothplots-1.pdf}

\begin{Shaded}
\begin{Highlighting}[]
\KeywordTok{par}\NormalTok{(}\DataTypeTok{mfrow=}\KeywordTok{c}\NormalTok{(}\DecValTok{1}\NormalTok{,}\DecValTok{1}\NormalTok{))}
\end{Highlighting}
\end{Shaded}

\subsection{Coefficients for the Model Identified by the
Cross-Validation}\label{coefficients-for-the-model-identified-by-the-cross-validation}

\begin{Shaded}
\begin{Highlighting}[]
\NormalTok{coef.cv <-}\StringTok{ }\KeywordTok{coef}\NormalTok{(lasso1, }\DataTypeTok{s=}\NormalTok{frac, }\DataTypeTok{mode=}\StringTok{"fraction"}\NormalTok{)}
\KeywordTok{round}\NormalTok{(coef.cv,}\DecValTok{4}\NormalTok{)}
\end{Highlighting}
\end{Shaded}

\begin{verbatim}
   lcavol   lweight       age     bph_f     svi_f       lcp gleason_f 
   0.5529    0.6402   -0.0217    0.1535    0.7750   -0.1155   -0.1826 
    pgg45 
   0.0030 
\end{verbatim}

So the model suggested by the lasso still includes all sight of these
predictors.

\subsection{Obtaining Fitted Values from
Lasso}\label{obtaining-fitted-values-from-lasso}

\begin{Shaded}
\begin{Highlighting}[]
\NormalTok{fits.cv <-}\StringTok{ }\KeywordTok{predict.lars}\NormalTok{(lasso1, preds, }\DataTypeTok{s=}\NormalTok{frac, }
                        \DataTypeTok{type=}\StringTok{"fit"}\NormalTok{, }\DataTypeTok{mode=}\StringTok{"fraction"}\NormalTok{)}
\NormalTok{fits.cv}
\end{Highlighting}
\end{Shaded}

\begin{verbatim}
$s
[1] 0.989899

$fraction
[1] 0.989899

$mode
[1] "fraction"

$fit
 [1] 0.7995838 0.7493971 0.5111634 0.6098520 1.7001847 0.8338020 1.8288518
 [8] 2.1302316 1.2487955 1.2661752 1.4704969 0.7782005 2.0755860 1.9129272
[15] 2.1533975 1.8124981 1.2713610 2.3993624 1.3232566 1.7709029 1.9757841
[22] 2.7451649 1.1658326 2.4825521 1.8036338 1.9112578 2.0144298 1.7829219
[29] 1.9706111 2.1688199 2.0377131 1.8657882 1.6955904 1.3580186 1.0516394
[36] 2.9097450 2.1898622 1.0454123 3.8896481 1.7971270 2.0932871 2.3253395
[43] 2.0809295 2.5303655 2.4451523 2.5827203 4.0692397 2.6845105 2.7034959
[50] 1.9590266 2.4522082 2.9801227 2.1902084 3.0559124 3.3447025 2.9765233
[57] 1.7620182 2.3424646 2.2856404 2.6188548 2.3056410 3.5568662 2.9756755
[64] 3.6764122 2.5097586 2.6579014 2.9482717 3.0892917 1.5113015 3.0282296
[71] 3.2887119 2.1083273 2.8889223 3.4903026 3.6959516 3.6070031 3.2749993
[78] 3.4518575 3.4049180 3.1814731 2.0496216 2.8986175 3.6743113 3.3292860
[85] 2.6965297 3.8339856 2.9892543 3.0555536 4.2903885 3.0986508 3.3784385
[92] 4.0205201 3.8309974 4.7531590 3.6290575 4.1347645 4.0982744
\end{verbatim}

\subsection{Complete Set of Fitted Values from the
Lasso}\label{complete-set-of-fitted-values-from-the-lasso}

\begin{Shaded}
\begin{Highlighting}[]
\KeywordTok{round}\NormalTok{(fits.cv}\OperatorTok{$}\NormalTok{fit,}\DecValTok{3}\NormalTok{)}
\end{Highlighting}
\end{Shaded}

\begin{verbatim}
 [1] 0.800 0.749 0.511 0.610 1.700 0.834 1.829 2.130 1.249 1.266 1.470
[12] 0.778 2.076 1.913 2.153 1.812 1.271 2.399 1.323 1.771 1.976 2.745
[23] 1.166 2.483 1.804 1.911 2.014 1.783 1.971 2.169 2.038 1.866 1.696
[34] 1.358 1.052 2.910 2.190 1.045 3.890 1.797 2.093 2.325 2.081 2.530
[45] 2.445 2.583 4.069 2.685 2.703 1.959 2.452 2.980 2.190 3.056 3.345
[56] 2.977 1.762 2.342 2.286 2.619 2.306 3.557 2.976 3.676 2.510 2.658
[67] 2.948 3.089 1.511 3.028 3.289 2.108 2.889 3.490 3.696 3.607 3.275
[78] 3.452 3.405 3.181 2.050 2.899 3.674 3.329 2.697 3.834 2.989 3.056
[89] 4.290 3.099 3.378 4.021 3.831 4.753 3.629 4.135 4.098
\end{verbatim}

To assess the quality of these predictions, we might plot them against
the observed values of our outcome (\texttt{lpsa}), or we might look at
residuals vs.~these fitted values.

\begin{Shaded}
\begin{Highlighting}[]
\NormalTok{prost_lasso_res <-}\StringTok{ }\KeywordTok{data_frame}\NormalTok{(}\DataTypeTok{fitted =}\NormalTok{ fits.cv}\OperatorTok{$}\NormalTok{fit, }
                             \DataTypeTok{actual =}\NormalTok{ prost}\OperatorTok{$}\NormalTok{lpsa, }
                             \DataTypeTok{resid =}\NormalTok{ actual }\OperatorTok{-}\StringTok{ }\NormalTok{fitted)}

\KeywordTok{ggplot}\NormalTok{(prost_lasso_res, }\KeywordTok{aes}\NormalTok{(}\DataTypeTok{x =}\NormalTok{ actual, }\DataTypeTok{y =}\NormalTok{ fitted)) }\OperatorTok{+}\StringTok{ }
\StringTok{    }\KeywordTok{geom_point}\NormalTok{() }\OperatorTok{+}\StringTok{ }
\StringTok{    }\KeywordTok{geom_abline}\NormalTok{(}\DataTypeTok{slope =} \DecValTok{1}\NormalTok{, }\DataTypeTok{intercept =} \DecValTok{0}\NormalTok{) }\OperatorTok{+}
\StringTok{    }\KeywordTok{labs}\NormalTok{(}\DataTypeTok{y =} \StringTok{"Fitted log(PSA) from Cross-Validated LASSO"}\NormalTok{,}
         \DataTypeTok{x =} \StringTok{"Observed values of log(PSA)"}\NormalTok{,}
         \DataTypeTok{title =} \StringTok{"Fitted vs. Actual Values of log(PSA)"}\NormalTok{)}
\end{Highlighting}
\end{Shaded}

\includegraphics{bookdown-demo_files/figure-latex/unnamed-chunk-133-1.pdf}

\begin{Shaded}
\begin{Highlighting}[]
\KeywordTok{ggplot}\NormalTok{(prost_lasso_res, }\KeywordTok{aes}\NormalTok{(}\DataTypeTok{x =}\NormalTok{ fitted, }\DataTypeTok{y =}\NormalTok{ resid)) }\OperatorTok{+}\StringTok{ }
\StringTok{    }\KeywordTok{geom_point}\NormalTok{() }\OperatorTok{+}\StringTok{ }
\StringTok{    }\KeywordTok{geom_hline}\NormalTok{(}\DataTypeTok{yintercept =} \DecValTok{0}\NormalTok{, }\DataTypeTok{col =} \StringTok{"red"}\NormalTok{) }\OperatorTok{+}
\StringTok{    }\KeywordTok{geom_smooth}\NormalTok{(}\DataTypeTok{method =} \StringTok{"loess"}\NormalTok{, }\DataTypeTok{col =} \StringTok{"blue"}\NormalTok{, }\DataTypeTok{se =}\NormalTok{ F) }\OperatorTok{+}
\StringTok{    }\KeywordTok{labs}\NormalTok{(}\DataTypeTok{x =} \StringTok{"LASSO-fitted log(PSA)"}\NormalTok{,}
         \DataTypeTok{y =} \StringTok{"Residuals from Cross-Validated LASSO model"}\NormalTok{,}
         \DataTypeTok{title =} \StringTok{"Residuals vs. Fitted Values of log(PSA) from LASSO"}\NormalTok{,}
         \DataTypeTok{subtitle =} \StringTok{"with loess smooth"}\NormalTok{)}
\end{Highlighting}
\end{Shaded}

\includegraphics{bookdown-demo_files/figure-latex/unnamed-chunk-134-1.pdf}

\subsection{When is the Lasso Most
Useful?}\label{when-is-the-lasso-most-useful}

As \citet{Faraway2015} suggests, the lasso is particularly useful when
we believe the effects are sparse, in the sense that we believe that few
of the many predictors we are evaluating have a meaningful effect.

Consider, for instance, the analysis of gene expression data, where we
have good reason to believe that only a small number of genes have an
influence on our response of interest.

Or, in medical claims data, where we can have thousands of available
codes to search through that may apply to some of the people included in
a large analysis relating health care costs to outcomes.

\section{\texorpdfstring{Applying the Lasso to the \texttt{pollution}
data}{Applying the Lasso to the pollution data}}\label{applying-the-lasso-to-the-pollution-data}

Let's consider the lasso approach in application to the
\texttt{pollution} data we've seen previously. Recall that we have 60
observations on an outcome, \texttt{y}, and 15 predictors, labeled x1
through x15.

\begin{Shaded}
\begin{Highlighting}[]
\NormalTok{preds <-}\StringTok{ }\KeywordTok{with}\NormalTok{(pollution, }\KeywordTok{cbind}\NormalTok{(x1, x2, x3, x4, x5, x6, x7,}
\NormalTok{                               x8, x9, x10, x11, x12, x13,}
\NormalTok{                               x14, x15))}

\NormalTok{lasso_p1 <-}\StringTok{ }\KeywordTok{lars}\NormalTok{(preds, pollution}\OperatorTok{$}\NormalTok{y, }\DataTypeTok{type=}\StringTok{"lasso"}\NormalTok{)}
\KeywordTok{plot}\NormalTok{(lasso_p1)}
\end{Highlighting}
\end{Shaded}

\includegraphics{bookdown-demo_files/figure-latex/lasso_graph1_forpollution-1.pdf}

\begin{Shaded}
\begin{Highlighting}[]
\KeywordTok{summary}\NormalTok{(lasso_p1)}
\end{Highlighting}
\end{Shaded}

\begin{verbatim}
LARS/LASSO
Call: lars(x = preds, y = pollution$y, type = "lasso")
   Df    Rss       Cp
0   1 228311 129.1367
1   2 185419  95.9802
2   3 149370  68.4323
3   4 143812  65.8764
4   5  92077  25.4713
5   6  83531  20.4668
6   7  69532  10.9922
7   8  67682  11.4760
8   9  60689   7.7445
9  10  60167   9.3163
10 11  59609  10.8588
11 12  58287  11.7757
12 13  57266  12.9383
13 14  56744  14.5107
14 13  56159  12.0311
15 14  55238  13.2765
16 15  53847  14.1361
17 16  53681  16.0000
\end{verbatim}

Based on the C\textsubscript{p} statistics, it looks like the big
improvements occur somewhere around the move from 6 to 7 df. Let's look
at the cross-validation

\begin{Shaded}
\begin{Highlighting}[]
\KeywordTok{set.seed}\NormalTok{(}\DecValTok{432012}\NormalTok{)}
\NormalTok{pollution_lassocv <-}\StringTok{ }\KeywordTok{cv.lars}\NormalTok{(preds, pollution}\OperatorTok{$}\NormalTok{y, }\DataTypeTok{K=}\DecValTok{10}\NormalTok{)}
\end{Highlighting}
\end{Shaded}

\includegraphics{bookdown-demo_files/figure-latex/unnamed-chunk-135-1.pdf}

Here it looks like cross-validated MSE happens somewhere between a
fraction of 0.2 and 0.4.

\begin{Shaded}
\begin{Highlighting}[]
\NormalTok{frac <-}\StringTok{ }\NormalTok{pollution_lassocv}\OperatorTok{$}\NormalTok{index[}\KeywordTok{which.min}\NormalTok{(pollution_lassocv}\OperatorTok{$}\NormalTok{cv)]}
\NormalTok{frac}
\end{Highlighting}
\end{Shaded}

\begin{verbatim}
[1] 0.3535354
\end{verbatim}

\begin{Shaded}
\begin{Highlighting}[]
\KeywordTok{par}\NormalTok{(}\DataTypeTok{mfrow=}\KeywordTok{c}\NormalTok{(}\DecValTok{2}\NormalTok{,}\DecValTok{1}\NormalTok{))}
\NormalTok{lasso_p1 <-}\StringTok{ }\KeywordTok{lars}\NormalTok{(preds, pollution}\OperatorTok{$}\NormalTok{y, }\DataTypeTok{type=}\StringTok{"lasso"}\NormalTok{)}
\KeywordTok{plot}\NormalTok{(lasso_p1)}
\KeywordTok{set.seed}\NormalTok{(}\DecValTok{432012}\NormalTok{)}
\NormalTok{pollution_lassocv <-}\StringTok{ }\KeywordTok{cv.lars}\NormalTok{(preds, pollution}\OperatorTok{$}\NormalTok{y, }\DataTypeTok{K=}\DecValTok{10}\NormalTok{)}
\end{Highlighting}
\end{Shaded}

\includegraphics{bookdown-demo_files/figure-latex/unnamed-chunk-137-1.pdf}

\begin{Shaded}
\begin{Highlighting}[]
\KeywordTok{par}\NormalTok{(}\DataTypeTok{mfrow=}\KeywordTok{c}\NormalTok{(}\DecValTok{1}\NormalTok{,}\DecValTok{1}\NormalTok{))}
\end{Highlighting}
\end{Shaded}

It looks like a model with 6-8 predictors will be the most useful. The
cross-validated coefficients are as follows:

\begin{Shaded}
\begin{Highlighting}[]
\NormalTok{poll.cv <-}\StringTok{ }\KeywordTok{coef}\NormalTok{(lasso_p1, }\DataTypeTok{s=}\NormalTok{frac, }\DataTypeTok{mode=}\StringTok{"fraction"}\NormalTok{)}
\KeywordTok{round}\NormalTok{(poll.cv,}\DecValTok{3}\NormalTok{)}
\end{Highlighting}
\end{Shaded}

\begin{verbatim}
     x1      x2      x3      x4      x5      x6      x7      x8      x9 
  1.471  -1.164  -1.102   0.000   0.000 -10.610  -0.457   0.003   3.918 
    x10     x11     x12     x13     x14     x15 
  0.000   0.000   0.000   0.000   0.228   0.000 
\end{verbatim}

Note that by this cross-validated lasso selection, not only are the
coefficients for the 8 variables remaining in the model shrunken, but
variables \texttt{x4}, \texttt{x5}, \texttt{x10}, \texttt{x11},
\texttt{x12}, \texttt{x13} and \texttt{x15} are all dropped from the
model, and model \texttt{x8} almost is, as well.

\begin{Shaded}
\begin{Highlighting}[]
\NormalTok{poll_fits <-}\StringTok{ }\KeywordTok{predict.lars}\NormalTok{(lasso_p1, preds, }\DataTypeTok{s=}\NormalTok{frac, }
                        \DataTypeTok{type=}\StringTok{"fit"}\NormalTok{, }\DataTypeTok{mode=}\StringTok{"fraction"}\NormalTok{)}
\KeywordTok{round}\NormalTok{(poll_fits}\OperatorTok{$}\NormalTok{fit,}\DecValTok{3}\NormalTok{)}
\end{Highlighting}
\end{Shaded}

\begin{verbatim}
 [1]  932.627  918.415  921.904  987.396 1050.184 1065.837  912.424
 [8]  916.605  949.647  926.168  996.625 1017.362  977.730  954.550
[15]  931.455  894.263  931.551  868.599  973.471  940.937  881.867
[22]  906.666  973.609  919.640  933.821  956.352  913.018  925.650
[29]  874.528  983.829 1042.870  915.002  937.760  885.464  989.947
[36]  931.709 1013.795  969.729 1003.962  983.813  896.042  918.446
[43]  934.609 1004.565  910.273  976.747  831.132  907.996  826.485
[50]  895.082  909.398  917.969  926.777  917.381  991.266  879.972
[57]  942.867  913.737  960.952  949.030
\end{verbatim}

Here's a plot of the actual \texttt{pollution} y values, against these
fitted values.

\begin{Shaded}
\begin{Highlighting}[]
\NormalTok{poll_lasso_res <-}\StringTok{ }\KeywordTok{data_frame}\NormalTok{(}\DataTypeTok{fitted =}\NormalTok{ poll_fits}\OperatorTok{$}\NormalTok{fit, }
                             \DataTypeTok{actual =}\NormalTok{ pollution}\OperatorTok{$}\NormalTok{y, }
                             \DataTypeTok{resid =}\NormalTok{ actual }\OperatorTok{-}\StringTok{ }\NormalTok{fitted)}

\KeywordTok{ggplot}\NormalTok{(poll_lasso_res, }\KeywordTok{aes}\NormalTok{(}\DataTypeTok{x =}\NormalTok{ actual, }\DataTypeTok{y =}\NormalTok{ fitted)) }\OperatorTok{+}\StringTok{ }
\StringTok{    }\KeywordTok{geom_point}\NormalTok{() }\OperatorTok{+}\StringTok{ }
\StringTok{    }\KeywordTok{geom_abline}\NormalTok{(}\DataTypeTok{slope =} \DecValTok{1}\NormalTok{, }\DataTypeTok{intercept =} \DecValTok{0}\NormalTok{) }\OperatorTok{+}
\StringTok{    }\KeywordTok{labs}\NormalTok{(}\DataTypeTok{y =} \StringTok{"Fitted y values from Cross-Validated LASSO"}\NormalTok{,}
         \DataTypeTok{x =} \StringTok{"Observed values of y = Age-Adjusted Mortality Rate"}\NormalTok{,}
         \DataTypeTok{title =} \StringTok{"Fitted vs. Actual Values of Age-Adjusted Mortality"}\NormalTok{)}
\end{Highlighting}
\end{Shaded}

\includegraphics{bookdown-demo_files/figure-latex/unnamed-chunk-139-1.pdf}

And now, here's a plot or residuals vs.~fitted values.

\begin{Shaded}
\begin{Highlighting}[]
\KeywordTok{ggplot}\NormalTok{(poll_lasso_res, }\KeywordTok{aes}\NormalTok{(}\DataTypeTok{x =}\NormalTok{ fitted, }\DataTypeTok{y =}\NormalTok{ resid)) }\OperatorTok{+}\StringTok{ }
\StringTok{    }\KeywordTok{geom_point}\NormalTok{() }\OperatorTok{+}\StringTok{ }
\StringTok{    }\KeywordTok{geom_hline}\NormalTok{(}\DataTypeTok{yintercept =} \DecValTok{0}\NormalTok{, }\DataTypeTok{col =} \StringTok{"red"}\NormalTok{) }\OperatorTok{+}
\StringTok{    }\KeywordTok{geom_smooth}\NormalTok{(}\DataTypeTok{method =} \StringTok{"loess"}\NormalTok{, }\DataTypeTok{col =} \StringTok{"blue"}\NormalTok{, }\DataTypeTok{se =}\NormalTok{ F) }\OperatorTok{+}
\StringTok{    }\KeywordTok{labs}\NormalTok{(}\DataTypeTok{x =} \StringTok{"LASSO-fitted Age-Adjusted Mortality"}\NormalTok{,}
         \DataTypeTok{y =} \StringTok{"Residuals from Cross-Validated LASSO model"}\NormalTok{,}
         \DataTypeTok{title =} \StringTok{"Residuals vs. Fitted Values of Age-Adjusted Mortality from LASSO"}\NormalTok{,}
         \DataTypeTok{subtitle =} \StringTok{"with loess smooth"}\NormalTok{)}
\end{Highlighting}
\end{Shaded}

\includegraphics{bookdown-demo_files/figure-latex/unnamed-chunk-140-1.pdf}

\chapter{\texorpdfstring{Logistic Regression and the \texttt{resect}
data}{Logistic Regression and the resect data}}\label{logistic-regression-and-the-resect-data}

\section{\texorpdfstring{The \texttt{resect}
data}{The resect data}}\label{the-resect-data}

My source for these data was \citet{Riffenburgh2006}. The data describe
134 patients who had undergone resection of the tracheal carina (most
often this is done to address tumors in the trachea), and the
\texttt{resect.csv} data file contains the following variables:

\begin{itemize}
\tightlist
\item
  \texttt{id} = a patient ID \#,
\item
  \texttt{age}= the patient's age at surgery,
\item
  \texttt{prior} = prior tracheal surgery (1 = yes, 0 = no),
\item
  \texttt{resection} = extent of the resection (in cm),
\item
  \texttt{intubated} = whether intubation was required at the end of
  surgery (1 = yes, 0 = no), and
\item
  \texttt{died} = the patient's death status (1 = dead, 0 = alive).
\end{itemize}

\begin{Shaded}
\begin{Highlighting}[]
\NormalTok{resect }\OperatorTok\StringTok{ }\KeywordTok{group_by}\NormalTok{(died) }\OperatorTok\StringTok{ }\KeywordTok{skim}\NormalTok{(}\OperatorTok{-}\NormalTok{id)}
\end{Highlighting}
\end{Shaded}

\begin{verbatim}
Skim summary statistics
 n obs: 134 
 n variables: 6 
 group variables: died 

Variable type: integer 
 died  variable missing complete   n   mean    sd p0 p25 median p75 p100
    0       age       0      117 117 48.05  16.01  8  36     51  62   80
    0 intubated       0      117 117  0.068  0.25  0   0      0   0    1
    0     prior       0      117 117  0.24   0.43  0   0      0   0    1
    1       age       0       17  17 46.41  14.46 26  33     46  60   66
    1 intubated       0       17  17  0.65   0.49  0   0      1   1    1
    1     prior       0       17  17  0.35   0.49  0   0      0   1    1

Variable type: numeric 
 died  variable missing complete   n mean   sd p0 p25 median p75 p100
    0 resection       0      117 117 2.82 1.21  1 2      2.5 3.5    6
    1 resection       0       17  17 3.97 1     2 3.5    4   4.5    6
\end{verbatim}

We have no missing data, and 17 of the 134 patients died. Our goal will
be to understand the characteristics of the patients, and how they
relate to the binary outcome of interest, death.

\section{Running A Simple Logistic Regression
Model}\label{running-a-simple-logistic-regression-model}

In the most common scenario, a logistic regression model is used to
predict a binary outcome (which can take on the values 0 or 1.) We will
eventually fit a logistic regression model in two ways.

\begin{enumerate}
\def\labelenumi{\arabic{enumi}.}
\tightlist
\item
  Through the \texttt{glm} function in the base package of R (similar to
  \texttt{lm} for linear regression)
\item
  Through the \texttt{lrm} function available in the \texttt{rms}
  package (similar to \texttt{ols} for linear regression)
\end{enumerate}

We'll focus on the \texttt{glm} approach first, and save the
\texttt{lrm} ideas for later in this Chapter.

\subsection{Logistic Regression Can Be Harder than Linear
Regression}\label{logistic-regression-can-be-harder-than-linear-regression}

\begin{itemize}
\tightlist
\item
  Logistic regression models are fitted using the method of maximum
  likelihood in \texttt{glm}, which requires multiple iterations until
  convergence is reached.
\item
  Logistic regression models are harder to interpret (for most people)
  than linear regressions.
\item
  Logistic regression models don't have the same set of assumptions as
  linear models.
\item
  Logistic regression outcomes (yes/no) carry much less information than
  quantitative outcomes. As a result, fitting a reasonable logistic
  regression requires more data than a linear model of similar size.

  \begin{itemize}
  \tightlist
  \item
    The rule I learned in graduate school was that a logistic regression
    requires 100 observations to fit an intercept plus another 15
    observations for each candidate predictor. That's not terrible, but
    it's a very large sample size.
  \item
    Frank Harrell recommends that 96 observations + a function of the
    number of candidate predictors (which depends on the amount of
    variation in the predictors, but 15 x the number of such predictors
    isn't too bad if the signal to noise ratio is pretty good) are
    required just to get reasonable confidence intervals around your
    predictions.

    \begin{itemize}
    \tightlist
    \item
      In a
      \href{https://twitter.com/f2harrell/status/936230071219707913}{twitter
      note}, Frank suggests that 96 + 8 times the number of candidate
      parameters might be reasonable so long as the smallest cell of
      interest (combination of an outcome and a split of the covariates)
      is 96 or more observations.
    \end{itemize}
  \item
    \citet{Peduzzi1996} suggest that if we let \(\pi\) be the smaller of
    the proportions of ``yes'' or ``no'' cases in the population of
    interest, and \emph{k} be the number of inputs under consideration,
    then \(N = 10k/\pi\) is the minimum number of cases to include,
    except that if N \textless{} 100 by this standard, you should
    increase it to 100, according to \citet{Long1997}.

    \begin{itemize}
    \tightlist
    \item
      That suggests that if you have an outcome that happens 10\% of the
      time, and you are running a model with 3 predictors, then you
      could get away with \((10 \times 3)/(0.10) = 300\) observations,
      but if your outcome happened 40\% of the time, you could get away
      with only \((10 \times 3)/(0.40) = 75\) observations, which you'd
      round up to 100.
    \end{itemize}
  \end{itemize}
\end{itemize}

\section{\texorpdfstring{Logistic Regression using
\texttt{glm}}{Logistic Regression using glm}}\label{logistic-regression-using-glm}

We'll begin by attempting to predict death based on the extent of the
resection.

\begin{Shaded}
\begin{Highlighting}[]
\NormalTok{res_modA <-}\StringTok{ }\KeywordTok{glm}\NormalTok{(died }\OperatorTok{~}\StringTok{ }\NormalTok{resection, }\DataTypeTok{data=}\NormalTok{resect, }
               \DataTypeTok{family=}\StringTok{"binomial"}\NormalTok{(}\DataTypeTok{link=}\StringTok{"logit"}\NormalTok{))}

\NormalTok{res_modA}
\end{Highlighting}
\end{Shaded}

\begin{verbatim}

Call:  glm(formula = died ~ resection, family = binomial(link = "logit"), 
    data = resect)

Coefficients:
(Intercept)    resection  
    -4.4337       0.7417  

Degrees of Freedom: 133 Total (i.e. Null);  132 Residual
Null Deviance:      101.9 
Residual Deviance: 89.49    AIC: 93.49
\end{verbatim}

Note that the \texttt{logit} link is the default approach with the
\texttt{binomial} family, so we could also have used:

\begin{Shaded}
\begin{Highlighting}[]
\NormalTok{res_modA <-}\StringTok{ }\KeywordTok{glm}\NormalTok{(died }\OperatorTok{~}\StringTok{ }\NormalTok{resection, }\DataTypeTok{data =}\NormalTok{ resect, }
                \DataTypeTok{family =} \StringTok{"binomial"}\NormalTok{)}
\end{Highlighting}
\end{Shaded}

which yields the same model.

\subsection{Interpreting the Coefficients of a Logistic Regression
Model}\label{interpreting-the-coefficients-of-a-logistic-regression-model}

Our model is:

\[
logit(died = 1) = log\left(\frac{Pr(died = 1)}{1 - Pr(died = 1)}\right) = \beta_0 + \beta_1 x = -4.4337 + 0.7417 \times resection
\]

The predicted log odds of death for a subject with a resection of 4 cm
is:

\[
log\left(\frac{Pr(died = 1)}{1 - Pr(died = 1)}\right) = -4.4337 + 0.7417 \times 4 = -1.467
\]

The predicted odds of death for a subject with a resection of 4 cm is
thus:

\[
\frac{Pr(died = 1)}{1 - Pr(died = 1)} = e^{-4.4337 + 0.7417 \times 4} = e^{-1.467} = 0.2306
\]

Since the odds are less than 1, we should find that the probability of
death is less than 1/2. With a little algebra, we see that the predicted
probability of death for a subject with a resection of 4 cm is:

\[
Pr(died = 1) = \frac{e^{-4.4337 + 0.7417 \times 4}}{1 + e^{-4.4337 + 0.7417 \times 4}} = \frac{e^{-1.467}}{1 + e^{-1.467}} = \frac{0.2306}{1.2306} = 0.187
\]

In general, we can frame the model in terms of a statement about
probabilities, like this:

\[
Pr(died = 1) = \frac{e^{\beta_0 + \beta_1 x}}{1 + {e^{\beta_0 + \beta_1 x}}} = \frac{e^{-4.4337 + 0.7417 \times resection}}{1 + e^{-4.4337 + 0.7417 \times resection}}
\]

and so by substituting in values for \texttt{resection}, we can estimate
the model's fitted probabilities of death.

\subsection{\texorpdfstring{Using \texttt{predict} to describe the
model's
fits}{Using predict to describe the model's fits}}\label{using-predict-to-describe-the-models-fits}

To obtain these fitted odds and probabilities in R, we can use the
\texttt{predict} function.

\begin{itemize}
\tightlist
\item
  The default predictions are on the scale of the log odds. These
  predictions are also available through the \texttt{type\ =\ "link"}
  command within the \texttt{predict} function for a generalized linear
  model like logistic regression.
\item
  Here are the predicted log odds of death for a subject (Sally) with a
  4 cm resection and a subject (Harry) who had a 5 cm resection.
\end{itemize}

\begin{Shaded}
\begin{Highlighting}[]
\KeywordTok{predict}\NormalTok{(res_modA, }\DataTypeTok{newdata =} \KeywordTok{data_frame}\NormalTok{(}\DataTypeTok{resection =} \KeywordTok{c}\NormalTok{(}\DecValTok{4}\NormalTok{,}\DecValTok{5}\NormalTok{)))}
\end{Highlighting}
\end{Shaded}

\begin{verbatim}
         1          2 
-1.4669912 -0.7253027 
\end{verbatim}

\begin{itemize}
\tightlist
\item
  We can also obtain predictions for each subject on the original
  response (here, probability) scale, backing out of the logit link.
\end{itemize}

\begin{Shaded}
\begin{Highlighting}[]
\KeywordTok{predict}\NormalTok{(res_modA, }\DataTypeTok{newdata =} \KeywordTok{data_frame}\NormalTok{(}\DataTypeTok{resection =} \KeywordTok{c}\NormalTok{(}\DecValTok{4}\NormalTok{, }\DecValTok{5}\NormalTok{)), }
        \DataTypeTok{type =} \StringTok{"response"}\NormalTok{)}
\end{Highlighting}
\end{Shaded}

\begin{verbatim}
        1         2 
0.1874004 0.3262264 
\end{verbatim}

So the predicted probability of death is 0.187 for Sally, the subject
with a 4 cm resection, and 0.326 for Harry, the subject with a 5 cm
resection.

\subsection{Odds Ratio interpretation of
Coefficients}\label{odds-ratio-interpretation-of-coefficients}

Often, we will exponentiate the estimated slope coefficients of a
logistic regression model to help us understand the impact of changing a
predictor on the odds of our outcome.

\begin{Shaded}
\begin{Highlighting}[]
\KeywordTok{exp}\NormalTok{(}\KeywordTok{coef}\NormalTok{(res_modA))}
\end{Highlighting}
\end{Shaded}

\begin{verbatim}
(Intercept)   resection 
 0.01186995  2.09947754 
\end{verbatim}

To interpret this finding, suppose we have two subjects, Harry and
Sally. Harry had a resection that was 1 cm larger than Sally. This
estimated coefficient suggests that the estimated odds for death
associated with Harry is 2.099 times larger than the odds for death
associated with Sally. In general, the odds ratio comparing two subjects
who differ by 1 cm on the resection length is 2.099.

To illustrate, again let's assume that Harry's resection was 5 cm, and
Sally's was 4 cm. Then we have:

\[
log\left(\frac{Pr(Harry died)}{1 - Pr(Harry died)}\right) = -4.4337 + 0.7417 \times 5 = -0.7253, \\
log\left(\frac{Pr(Sally died)}{1 - Pr(Sally died)}\right) = -4.4337 + 0.7417 \times 4 = -1.4667.
\]

which implies that our estimated odds of death for Harry and Sally are:

\[
Odds(Harry died) = \frac{Pr(Harry died)}{1 - Pr(Harry died)} = e^{-4.4337 + 0.7417 \times 5} = e^{-0.7253} = 0.4842 \\
Odds(Sally died) = \frac{Pr(Sally died)}{1 - Pr(Sally died)} = e^{-4.4337 + 0.7417 \times 4} = e^{-1.4667} = 0.2307
\]

and so the odds ratio is:

\[
OR = \frac{Odds(Harry died)}{Odds(Sally died)} = \frac{0.4842}{0.2307} = 2.099
\]

\begin{itemize}
\tightlist
\item
  If the odds ratio was 1, that would mean that Harry and Sally had the
  same estimated odds of death, and thus the same estimated probability
  of death, despite having different sizes of resections.
\item
  Since the odds ratio is greater than 1, it means that Harry has a
  higher estimated odds of death than Sally, and thus that Harry has a
  higher estimated probability of death than Sally.
\item
  If the odds ratio was less than 1, it would mean that Harry had a
  lower estimated odds of death than Sally, and thus that Harry had a
  lower estimated probability of death than Sally.
\end{itemize}

Remember that the odds ratio is a fraction describing two positive
numbers (odds can only be non-negative) so that the smallest possible
odds ratio is 0.

\subsection{\texorpdfstring{Interpreting the rest of the model output
from
\texttt{glm}}{Interpreting the rest of the model output from glm}}\label{interpreting-the-rest-of-the-model-output-from-glm}

\begin{Shaded}
\begin{Highlighting}[]
\NormalTok{res_modA}
\end{Highlighting}
\end{Shaded}

\begin{verbatim}

Call:  glm(formula = died ~ resection, family = "binomial", data = resect)

Coefficients:
(Intercept)    resection  
    -4.4337       0.7417  

Degrees of Freedom: 133 Total (i.e. Null);  132 Residual
Null Deviance:      101.9 
Residual Deviance: 89.49    AIC: 93.49
\end{verbatim}

In addition to specifying the logistic regression coefficients, we are
also presented with information on degrees of freedom, deviance (null
and residual) and AIC.

\begin{itemize}
\tightlist
\item
  The degrees of freedom indicate the sample size.

  \begin{itemize}
  \tightlist
  \item
    Recall that we had \emph{n} = 134 subjects in the data. The ``Null''
    model includes only an intercept term (which uses 1 df) and we thus
    have \emph{n} - 1 (here 133) degrees of freedom available for
    estimation.
  \item
    In our \texttt{res\_modA} model, a logistic regression is fit
    including a single slope (resection) and an intercept term. Each
    uses up one degree of freedom to build an estimate, so we have
    \emph{n} - 2 = 134 - 2 = 132 residual df remaining.
  \end{itemize}
\item
  The AIC or Akaike Information Criterion (lower values are better) is
  also provided. This is helpful if we're comparing multiple models for
  the same outcome.
\end{itemize}

\subsection{Deviance and Comparing Our Model to the Null
Model}\label{deviance-and-comparing-our-model-to-the-null-model}

\begin{itemize}
\tightlist
\item
  The deviance (a measure of the model's \emph{lack of fit}) is
  available for both the null model (the model with only an intercept)
  and for our model (\texttt{res\_modA}) predicting our outcome,
  mortality.
\item
  The deviance test, though available in R (see below) isn't really a
  test of whether the model works well. Instead, it assumes the model is
  true, and then tests to see if the coefficients are detectably
  different from zero. So it isn't of much practical use.

  \begin{itemize}
  \tightlist
  \item
    To compare the \texttt{deviance} statistics, we can subtract the
    residual deviance from the null deviance to describe the impact of
    our model on fit.
  \item
    Null Deviance - Residual Deviance can be compared to a \(\chi^2\)
    distribution with Null DF - Residual DF degrees of freedom to obtain
    a global test of the in-sample predictive power of our model.
  \item
    We can see this comparison more directly by running \texttt{anova}
    on our model:
  \end{itemize}
\end{itemize}

\begin{Shaded}
\begin{Highlighting}[]
\KeywordTok{anova}\NormalTok{(res_modA)}
\end{Highlighting}
\end{Shaded}

\begin{verbatim}
Analysis of Deviance Table

Model: binomial, link: logit

Response: died

Terms added sequentially (first to last)

          Df Deviance Resid. Df Resid. Dev
NULL                        133    101.943
resection  1    12.45       132     89.493
\end{verbatim}

To complete a deviance test and obtain a \emph{p} value, we can run the
following code to estimate the probability that a chi-square
distribution with a single degree of freedom would exhibit an
improvement in deviance as large as 12.45.

\begin{Shaded}
\begin{Highlighting}[]
\KeywordTok{pchisq}\NormalTok{(}\FloatTok{12.45}\NormalTok{, }\DecValTok{1}\NormalTok{, }\DataTypeTok{lower.tail =} \OtherTok{FALSE}\NormalTok{)}
\end{Highlighting}
\end{Shaded}

\begin{verbatim}
[1] 0.0004179918
\end{verbatim}

The \emph{p} value for the deviance test here is about 0.0004. But,
again, this isn't a test of whether the model is any good - it assumes
the model is true, and then tests some consequences.

\begin{itemize}
\tightlist
\item
  Specifically, it tests whether (if the model is TRUE) some of the
  model's coefficients are non-zero.
\item
  That's not so practically useful, so I discourage you from performing
  global tests of a logistic regression model with a deviance test.
\end{itemize}

\subsection{\texorpdfstring{Using \texttt{glance} with a logistic
regression
model}{Using glance with a logistic regression model}}\label{using-glance-with-a-logistic-regression-model}

We can use the \texttt{glance} function from the \texttt{broom} package
to obtain the null and residual deviance and degrees of freedom. Note
that the deviance for our model is related to the log likelihood by
-2*\texttt{logLik}.

\begin{Shaded}
\begin{Highlighting}[]
\KeywordTok{glance}\NormalTok{(res_modA)}
\end{Highlighting}
\end{Shaded}

\begin{verbatim}
  null.deviance df.null    logLik      AIC     BIC deviance df.residual
1      101.9431     133 -44.74646 93.49292 99.2886 89.49292         132
\end{verbatim}

The \texttt{glance} result also provides the AIC, and the BIC (Bayes
Information Criterion), each of which is helpful in understanding
comparisons between multiple models for the same outcome (with smaller
values of either criterion indicating better models.) The AIC is based
on the deviance, but penalizes you for making the model more
complicated. The BIC does the same sort of thing but with a different
penalty.

Again we see that we have a null deviance of 101.94 on 133 degrees of
freedom. Including the \texttt{resection} information in the model
decreased the deviance to 89.49 points on 132 degrees of freedom, so
that's a decrease of 12.45 points while using one degree of freedom, a
statistically significant reduction in deviance.

\section{Interpreting the Model
Summary}\label{interpreting-the-model-summary}

Let's get a more detailed summary of our \texttt{res\_modA} model,
including 95\% confidence intervals for the coefficients:

\begin{Shaded}
\begin{Highlighting}[]
\KeywordTok{summary}\NormalTok{(res_modA)}
\end{Highlighting}
\end{Shaded}

\begin{verbatim}

Call:
glm(formula = died ~ resection, family = "binomial", data = resect)

Deviance Residuals: 
    Min       1Q   Median       3Q      Max  
-1.1844  -0.5435  -0.3823  -0.2663   2.4501  

Coefficients:
            Estimate Std. Error z value Pr(>|z|)    
(Intercept)  -4.4337     0.8799  -5.039 4.67e-07 ***
resection     0.7417     0.2230   3.327 0.000879 ***
---
Signif. codes:  0 '***' 0.001 '**' 0.01 '*' 0.05 '.' 0.1 ' ' 1

(Dispersion parameter for binomial family taken to be 1)

    Null deviance: 101.943  on 133  degrees of freedom
Residual deviance:  89.493  on 132  degrees of freedom
AIC: 93.493

Number of Fisher Scoring iterations: 5
\end{verbatim}

\begin{Shaded}
\begin{Highlighting}[]
\KeywordTok{confint}\NormalTok{(res_modA, }\DataTypeTok{level =} \FloatTok{0.95}\NormalTok{)}
\end{Highlighting}
\end{Shaded}

\begin{verbatim}
Waiting for profiling to be done...
\end{verbatim}

\begin{verbatim}
                2.5 %    97.5 %
(Intercept) -6.344472 -2.855856
resection    0.322898  1.208311
\end{verbatim}

Some elements of this summary are very familiar from our work with
linear models.

\begin{itemize}
\tightlist
\item
  We still have a five-number summary of residuals, although these are
  called \emph{deviance} residuals.
\item
  We have a table of coefficients with standard errors, and hypothesis
  tests, although these are Wald z-tests, rather than the t tests we saw
  in linear modeling.
\item
  We have a summary of global fit in the comparison of null deviance and
  residual deviance, but without a formal p value. And we have the AIC,
  as discussed above.
\item
  We also have some new items related to a \emph{dispersion} parameter
  and to the number of Fisher Scoring Iterations.
\end{itemize}

Let's walk through each of these elements.

\subsection{Wald Z tests for Coefficients in a Logistic
Regression}\label{wald-z-tests-for-coefficients-in-a-logistic-regression}

The coefficients output provides the estimated coefficients, and their
standard errors, plus a Wald Z statistic, which is just the estimated
coefficient divided by its standard error. This is compared to a
standard Normal distribution to obtain the two-tailed p values
summarized in the \texttt{Pr(\textgreater{}\textbar{}z\textbar{})}
column.

\begin{itemize}
\tightlist
\item
  The interesting result is \texttt{resection}, which has a Wald Z =
  3.327, yielding a \emph{p} value of 0.00088.
\item
  The hypotheses being tested here are H\_0\_: \texttt{resection} does
  not have an effect on the log odds of \texttt{died} vs.~H\_A\_:
  \texttt{resection} does have such an effect.
\item
  Another way of stating this is that the \emph{p} value assesses
  whether the estimated coefficient of \texttt{resection}, 0.7417, is
  statistically detectably different from 0. If the coefficient (on the
  logit scale) for \texttt{resection} was truly 0, this would mean that:

  \begin{itemize}
  \tightlist
  \item
    the log odds of death did not change based on the \texttt{resection}
    size,
  \item
    the odds of death were unchanged based on the \texttt{resection}
    size (the odds ratio would be 1), and
  \item
    the probability of death was unchanged based on the
    \texttt{resection} size.
  \end{itemize}
\end{itemize}

In our case, we have a statistically detectable change in the log odds
of \texttt{died} associated with changes in \texttt{resection},
according to this \emph{p} value. We conclude that \texttt{resection}
size is associated with a positive impact on death rates (death rates
are generally higher for people with larger resections.)

\subsection{Confidence Intervals for the
Coefficients}\label{confidence-intervals-for-the-coefficients}

As in linear regression, we can obtain 95\% confidence intervals (to get
other levels, change the \texttt{level} parameter in \texttt{confint})
for the intercept and slope coefficients.

Here, we see, for example, that the coefficient of \texttt{resection}
has a point estimate of 0.7417, and a confidence interval of (0.3229,
1.208). Since this is on the logit scale, it's not that interpretable,
but we will often exponentiate the model and its confidence interval to
obtain a more interpretable result on the odds ratio scale.

\begin{Shaded}
\begin{Highlighting}[]
\KeywordTok{exp}\NormalTok{(}\KeywordTok{coef}\NormalTok{(res_modA))}
\end{Highlighting}
\end{Shaded}

\begin{verbatim}
(Intercept)   resection 
 0.01186995  2.09947754 
\end{verbatim}

\begin{Shaded}
\begin{Highlighting}[]
\KeywordTok{exp}\NormalTok{(}\KeywordTok{confint}\NormalTok{(res_modA))}
\end{Highlighting}
\end{Shaded}

\begin{verbatim}
Waiting for profiling to be done...
\end{verbatim}

\begin{verbatim}
                  2.5 %     97.5 %
(Intercept) 0.001756429 0.05750655
resection   1.381124459 3.34782604
\end{verbatim}

From this output, we can estimate the odds ratio for death associated
with a 1 cm increase in resection size is 2.099, with a 95\% CI of
(1.38, 3.35). - If the odds ratio was 1, it would indicate that the odds
of death did not change based on the change in resection size. - Here,
it's clear that the estimated odds of death will be larger (odds
\textgreater{} 1) for subjects with larger resection sizes. Larger odds
of death also indicate larger probabilities of death. This confidence
interval indicates that with 95\% confidence, we conclude that increases
in resection size are associated with statistically detectable increases
in the odds of death. - If the odds ratio was less than 1 (remember that
it cannot be less than 0) that would mean that subjects with larger
resection sizes were associated with smaller estimated odds of death.

\subsection{Deviance Residuals}\label{deviance-residuals}

In logistic regression, it's certainly a good idea to check to see how
well the model fits the data. However, there are a few different types
of residuals. The residuals presented here by default are called
deviance residuals. Other types of residuals are available for
generalized linear models, such as Pearson residuals, working residuals,
and response residuals. Logistic regression model diagnostics often make
use of multiple types of residuals.

The deviance residuals for each individual subject sum up to the
deviance statistic for the model, and describe the contribution of each
point to the model likelihood function.

The deviance residual, \(d_i\), for the i\textsuperscript{th}
observation in a model predicting \(y_i\) (a binary variable), with the
estimate being \(\hat{\pi}_i\) is:

\[
d_i = s_i \sqrt{-2 [y_i log \hat{\pi_i} + (1 - y_i) log(1 - \hat{\pi_i})]},
\]

where \(s_i\) is 1 if \(y_i = 1\) and \(s_i = -1\) if \(y_i = 0\).

Again, the sum of the deviance residuals is the deviance.

\subsection{Dispersion Parameter}\label{dispersion-parameter}

The dispersion parameter is taken to be 1 for \texttt{glm} fit using
either the \texttt{binomial} or \texttt{Poisson} families. For other
sorts of generalized linear models, the dispersion parameter will be of
some importance in estimating standard errors sensibly.

\subsection{Fisher Scoring iterations}\label{fisher-scoring-iterations}

The solution of a logistic regression model involves maximizing a
likelihood function. Fisher's scoring algorithm in our
\texttt{res\_modA} needed five iterations to perform the logistic
regression fit. All that this tells you is that the model converged, and
didn't require a lot of time to do so.

\section{Plotting a Simple Logistic Regression
Model}\label{plotting-a-simple-logistic-regression-model}

Let's plot the logistic regression model \texttt{res\_modA} for
\texttt{died} using the extent of the resection in terms of
probabilities. We can use either of two different approaches:

\begin{itemize}
\tightlist
\item
  we can plot the fitted values from our specific model against the
  original data, using the \texttt{augment} function from the
  \texttt{broom} package, or
\item
  we can create a smooth function called \texttt{binomial\_smooth} that
  plots a simple logistic model in an analogous way to
  \texttt{geom\_smooth(method\ =\ "lm")} for a simple linear regression.
\end{itemize}

\subsection{\texorpdfstring{Using \texttt{augment} to capture the fitted
probabilities}{Using augment to capture the fitted probabilities}}\label{using-augment-to-capture-the-fitted-probabilities}

\begin{Shaded}
\begin{Highlighting}[]
\NormalTok{res_A_aug <-}\StringTok{ }\KeywordTok{augment}\NormalTok{(res_modA, resect, }
                     \DataTypeTok{type.predict =} \StringTok{"response"}\NormalTok{)}
\KeywordTok{head}\NormalTok{(res_A_aug)}
\end{Highlighting}
\end{Shaded}

\begin{verbatim}
  id age prior resection intubated died    .fitted    .se.fit     .resid
1  1  34     1       2.5         0    0 0.07046791 0.02562381 -0.3822929
2  2  57     0       5.0         0    0 0.32622637 0.08605551 -0.8886631
3  3  60     1       4.0         1    1 0.18740037 0.04269795  1.8300317
4  4  62     1       4.2         0    0 0.21104240 0.04871389 -0.6885386
5  5  28     0       6.0         1    1 0.50409637 0.14302982  1.1704596
6  6  52     0       3.0         0    0 0.09897375 0.02867196 -0.4565542
         .hat    .sigma      .cooksd .std.resid
1 0.010024061 0.8258481 0.0003876961 -0.3842235
2 0.033691765 0.8227475 0.0087350915 -0.9040227
3 0.011972088 0.8107264 0.0265893468  1.8410857
4 0.014252277 0.8243062 0.0019617278 -0.6934983
5 0.081835623 0.8196110 0.0477480056  1.2215077
6 0.009218619 0.8255581 0.0005157780 -0.4586733
\end{verbatim}

This approach augments the \texttt{resect} data set with fitted,
residual and other summaries of each observation's impact on the fit,
using the ``response'' type of prediction, which yields the fitted
probabilities in the \texttt{.fitted} column.

\subsection{Plotting a Logistic Regression Model's Fitted
Values}\label{plotting-a-logistic-regression-models-fitted-values}

\begin{Shaded}
\begin{Highlighting}[]
\KeywordTok{ggplot}\NormalTok{(res_A_aug, }\KeywordTok{aes}\NormalTok{(}\DataTypeTok{x =}\NormalTok{ resection, }\DataTypeTok{y =}\NormalTok{ died)) }\OperatorTok{+}
\StringTok{    }\KeywordTok{geom_jitter}\NormalTok{(}\DataTypeTok{height =} \FloatTok{0.05}\NormalTok{) }\OperatorTok{+}
\StringTok{    }\KeywordTok{geom_line}\NormalTok{(}\KeywordTok{aes}\NormalTok{(}\DataTypeTok{x =}\NormalTok{ resection, }\DataTypeTok{y =}\NormalTok{ .fitted), }
              \DataTypeTok{col =} \StringTok{"blue"}\NormalTok{) }\OperatorTok{+}
\StringTok{    }\KeywordTok{labs}\NormalTok{(}\DataTypeTok{title =} \StringTok{"Logistic Regression from Model res_modA"}\NormalTok{)}
\end{Highlighting}
\end{Shaded}

\includegraphics{bookdown-demo_files/figure-latex/unnamed-chunk-153-1.pdf}

\subsection{\texorpdfstring{Plotting a Simple Logistic Model using
\texttt{binomial\_smooth}}{Plotting a Simple Logistic Model using binomial\_smooth}}\label{plotting-a-simple-logistic-model-using-binomial_smooth}

\begin{Shaded}
\begin{Highlighting}[]
\NormalTok{binomial_smooth <-}\StringTok{ }\ControlFlowTok{function}\NormalTok{(...) \{}
  \KeywordTok{geom_smooth}\NormalTok{(}\DataTypeTok{method =} \StringTok{"glm"}\NormalTok{, }
              \DataTypeTok{method.args =} \KeywordTok{list}\NormalTok{(}\DataTypeTok{family =} \StringTok{"binomial"}\NormalTok{), ...)}
\NormalTok{\}}

\KeywordTok{ggplot}\NormalTok{(resect, }\KeywordTok{aes}\NormalTok{(}\DataTypeTok{x =}\NormalTok{ resection, }\DataTypeTok{y =}\NormalTok{ died)) }\OperatorTok{+}
\StringTok{  }\KeywordTok{geom_jitter}\NormalTok{(}\DataTypeTok{height =} \FloatTok{0.05}\NormalTok{) }\OperatorTok{+}
\StringTok{  }\KeywordTok{binomial_smooth}\NormalTok{() }\OperatorTok{+}\StringTok{ }\NormalTok{## ...smooth(se=FALSE) to leave out interval}
\StringTok{  }\KeywordTok{labs}\NormalTok{(}\DataTypeTok{title =} \StringTok{"Logistic Regression from Model A"}\NormalTok{) }\OperatorTok{+}
\StringTok{  }\KeywordTok{theme_bw}\NormalTok{()}
\end{Highlighting}
\end{Shaded}

\includegraphics{bookdown-demo_files/figure-latex/unnamed-chunk-154-1.pdf}

As expected, we see an increase in the model probability of death as the
extent of the resection grows larger.

\section{Receiver Operating Characteristic Curve
Analysis}\label{receiver-operating-characteristic-curve-analysis}

One way to assess the predictive accuracy within the model development
sample in a logistic regression is to consider an analyses based on the
receiver operating characteristic (ROC) curve. ROC curves are commonly
used in assessing diagnoses in medical settings, and in signal detection
applications.

The accuracy of a ``test'' can be evaluated by considering two types of
errors: false positives and false negatives.

In our \texttt{res\_modA} model, we use \texttt{resection} size to
predict whether the patient \texttt{died}. Suppose we established a
value R, so that if the resection size was larger than R cm, we would
predict that the patient \texttt{died}, and otherwise we would predict
that the patient did not die.

A good outcome of our model's ``test'', then, would be when the
resection size is larger than R for a patient who actually died. Another
good outcome would be when the resection size is smaller than R for a
patient who survived.

But we can make errors, too.

\begin{itemize}
\tightlist
\item
  A false positive error in this setting would occur when the resection
  size is larger than R (so we predict the patient dies) but in fact the
  patient does not die.
\item
  A false negative error in this case would occur when the resection
  size is smaller than R (so we predict the patient survives) but in
  fact the patient dies.
\end{itemize}

Formally, the true positive fraction (TPF) for a specific resection
cutoff \(R\), is the probability of a positive test (a prediction that
the patient will die) among the people who have the outcome died = 1
(those who actually die).

\[
TPF(R) = Pr(resection > R | subject died)
\]

Similarly, the false positive fraction (FPF) for a specific cutoff \(R\)
is the probability of a positive test (prediction that the patient will
die) among the people with died = 0 (those who don't actually die)

\[
FPF(R) = Pr(resection > R | subject did not die)
\]

The True Positive Rate is referred to as the sensitivity of a diagnostic
test, and the True Negative rate (1 - the False Positive rate) is
referred to as the specificity of a diagnostic test.

Since the cutoff \(R\) is not fixed in advanced, we can plot the value
of TPF (on the y axis) against FPF (on the x axis) for all possible
values of \(R\), and this is what the ROC curve is. Others refer to the
Sensitivity on the Y axis, and 1-Specificity on the X axis, and this is
the same idea.

Before we get too far into the weeds, let me show you some simple
situations so you can understand what you might learn from the ROC
curve. The web page \url{http://blog.yhat.com/posts/roc-curves.html}
provides source materials.

\subsection{Interpreting the Area under the ROC
curve}\label{interpreting-the-area-under-the-roc-curve}

The AUC or Area under the ROC curve is the amount of space underneath
the ROC curve. Often referred to as the c statistic, the AUC represents
the quality of your TPR and FPR overall in a single number. The C
statistic ranges from 0 to 1, with C = 0.5 for a prediction that is no
better than random guessing, and C = 1 for a perfect prediction model.

Next, I'll build a simulation to demonstrate several possible ROC curves
in the sections that follow.

\begin{Shaded}
\begin{Highlighting}[]
\KeywordTok{set.seed}\NormalTok{(}\DecValTok{432999}\NormalTok{)}
\NormalTok{sim.temp <-}\StringTok{ }\KeywordTok{data_frame}\NormalTok{(}\DataTypeTok{x =} \KeywordTok{rnorm}\NormalTok{(}\DataTypeTok{n =} \DecValTok{200}\NormalTok{), }
                       \DataTypeTok{prob =} \KeywordTok{exp}\NormalTok{(x)}\OperatorTok{/}\NormalTok{(}\DecValTok{1} \OperatorTok{+}\StringTok{ }\KeywordTok{exp}\NormalTok{(x)), }
                       \DataTypeTok{y =} \KeywordTok{as.numeric}\NormalTok{(}\DecValTok{1} \OperatorTok{*}\StringTok{ }\KeywordTok{runif}\NormalTok{(}\DecValTok{200}\NormalTok{) }\OperatorTok{<}\StringTok{ }\NormalTok{prob))}

\NormalTok{sim.temp <-}\StringTok{ }\NormalTok{sim.temp }\OperatorTok
\StringTok{    }\KeywordTok{mutate}\NormalTok{(}\DataTypeTok{p_guess =} \DecValTok{1}\NormalTok{,}
           \DataTypeTok{p_perfect =}\NormalTok{ y, }
           \DataTypeTok{p_bad =} \KeywordTok{exp}\NormalTok{(}\OperatorTok{-}\DecValTok{2}\OperatorTok{*}\NormalTok{x) }\OperatorTok{/}\StringTok{ }\NormalTok{(}\DecValTok{1} \OperatorTok{+}\StringTok{ }\KeywordTok{exp}\NormalTok{(}\OperatorTok{-}\DecValTok{2}\OperatorTok{*}\NormalTok{x)),}
           \DataTypeTok{p_ok =}\NormalTok{ prob }\OperatorTok{+}\StringTok{ }\NormalTok{(}\DecValTok{1}\OperatorTok{-}\NormalTok{y)}\OperatorTok{*}\KeywordTok{runif}\NormalTok{(}\DecValTok{1}\NormalTok{, }\DecValTok{0}\NormalTok{, }\FloatTok{0.05}\NormalTok{),}
           \DataTypeTok{p_good =}\NormalTok{ prob }\OperatorTok{+}\StringTok{ }\NormalTok{y}\OperatorTok{*}\KeywordTok{runif}\NormalTok{(}\DecValTok{1}\NormalTok{, }\DecValTok{0}\NormalTok{, }\FloatTok{0.27}\NormalTok{))}
\end{Highlighting}
\end{Shaded}

\subsubsection{What if we are guessing?}\label{what-if-we-are-guessing}

If we're guessing completely at random, then the model should correctly
classify a subject (as died or not died) about 50\% of the time, so the
TPR and FPR will be equal. This yields a diagonal line in the ROC curve,
and an area under the curve (C statistic) of 0.5.

There are several ways to do this on the web, but I'll show this one,
which has some bizarre code, but that's a function of using a package
called \texttt{ROCR} to do the work. It comes from
\href{http://blog.yhat.com/posts/roc-curves.html}{this link}

\begin{Shaded}
\begin{Highlighting}[]
\NormalTok{pred_guess <-}\StringTok{ }\KeywordTok{prediction}\NormalTok{(sim.temp}\OperatorTok{$}\NormalTok{p_guess, sim.temp}\OperatorTok{$}\NormalTok{y)}
\NormalTok{perf_guess <-}\StringTok{ }\KeywordTok{performance}\NormalTok{(pred_guess, }\DataTypeTok{measure =} \StringTok{"tpr"}\NormalTok{, }\DataTypeTok{x.measure =} \StringTok{"fpr"}\NormalTok{)}
\NormalTok{auc_guess <-}\StringTok{ }\KeywordTok{performance}\NormalTok{(pred_guess, }\DataTypeTok{measure=}\StringTok{"auc"}\NormalTok{)}

\NormalTok{auc_guess <-}\StringTok{ }\KeywordTok{round}\NormalTok{(auc_guess}\OperatorTok{@}\NormalTok{y.values[[}\DecValTok{1}\NormalTok{]],}\DecValTok{3}\NormalTok{)}
\NormalTok{roc_guess <-}\StringTok{ }\KeywordTok{data.frame}\NormalTok{(}\DataTypeTok{fpr=}\KeywordTok{unlist}\NormalTok{(perf_guess}\OperatorTok{@}\NormalTok{x.values),}
                        \DataTypeTok{tpr=}\KeywordTok{unlist}\NormalTok{(perf_guess}\OperatorTok{@}\NormalTok{y.values),}
                        \DataTypeTok{model=}\StringTok{"GLM"}\NormalTok{)}

\KeywordTok{ggplot}\NormalTok{(roc_guess, }\KeywordTok{aes}\NormalTok{(}\DataTypeTok{x=}\NormalTok{fpr, }\DataTypeTok{ymin=}\DecValTok{0}\NormalTok{, }\DataTypeTok{ymax=}\NormalTok{tpr)) }\OperatorTok{+}
\StringTok{    }\KeywordTok{geom_ribbon}\NormalTok{(}\DataTypeTok{alpha=}\FloatTok{0.2}\NormalTok{, }\DataTypeTok{fill =} \StringTok{"blue"}\NormalTok{) }\OperatorTok{+}
\StringTok{    }\KeywordTok{geom_line}\NormalTok{(}\KeywordTok{aes}\NormalTok{(}\DataTypeTok{y=}\NormalTok{tpr), }\DataTypeTok{col =} \StringTok{"blue"}\NormalTok{) }\OperatorTok{+}
\StringTok{    }\KeywordTok{labs}\NormalTok{(}\DataTypeTok{title =} \KeywordTok{paste0}\NormalTok{(}\StringTok{"Guessing: ROC Curve w/ AUC="}\NormalTok{, auc_guess)) }\OperatorTok{+}
\StringTok{    }\KeywordTok{theme_bw}\NormalTok{()}
\end{Highlighting}
\end{Shaded}

\includegraphics{bookdown-demo_files/figure-latex/unnamed-chunk-156-1.pdf}

\subsubsection{What if we classify things
perfectly?}\label{what-if-we-classify-things-perfectly}

If we're classifying subjects perfectly, then we have a TPR of 1 and an
FPR of 0. That yields an ROC curve that looks like the upper and left
edges of a box. If our model correctly classifies a subject (as died or
not died) 100\% of the time, the area under the curve (c statistic) will
be 1.0. We'll add in the diagonal line here (in a dashed black line) to
show how this model compares to random guessing.

\begin{Shaded}
\begin{Highlighting}[]
\NormalTok{pred_perf <-}\StringTok{ }\KeywordTok{prediction}\NormalTok{(sim.temp}\OperatorTok{$}\NormalTok{p_perfect, sim.temp}\OperatorTok{$}\NormalTok{y)}
\NormalTok{perf_perf <-}\StringTok{ }\KeywordTok{performance}\NormalTok{(pred_perf, }\DataTypeTok{measure =} \StringTok{"tpr"}\NormalTok{, }\DataTypeTok{x.measure =} \StringTok{"fpr"}\NormalTok{)}
\NormalTok{auc_perf <-}\StringTok{ }\KeywordTok{performance}\NormalTok{(pred_perf, }\DataTypeTok{measure=}\StringTok{"auc"}\NormalTok{)}

\NormalTok{auc_perf <-}\StringTok{ }\KeywordTok{round}\NormalTok{(auc_perf}\OperatorTok{@}\NormalTok{y.values[[}\DecValTok{1}\NormalTok{]],}\DecValTok{3}\NormalTok{)}
\NormalTok{roc_perf <-}\StringTok{ }\KeywordTok{data.frame}\NormalTok{(}\DataTypeTok{fpr=}\KeywordTok{unlist}\NormalTok{(perf_perf}\OperatorTok{@}\NormalTok{x.values),}
                        \DataTypeTok{tpr=}\KeywordTok{unlist}\NormalTok{(perf_perf}\OperatorTok{@}\NormalTok{y.values),}
                        \DataTypeTok{model=}\StringTok{"GLM"}\NormalTok{)}

\KeywordTok{ggplot}\NormalTok{(roc_perf, }\KeywordTok{aes}\NormalTok{(}\DataTypeTok{x=}\NormalTok{fpr, }\DataTypeTok{ymin=}\DecValTok{0}\NormalTok{, }\DataTypeTok{ymax=}\NormalTok{tpr)) }\OperatorTok{+}
\StringTok{    }\KeywordTok{geom_ribbon}\NormalTok{(}\DataTypeTok{alpha=}\FloatTok{0.2}\NormalTok{, }\DataTypeTok{fill =} \StringTok{"blue"}\NormalTok{) }\OperatorTok{+}
\StringTok{    }\KeywordTok{geom_line}\NormalTok{(}\KeywordTok{aes}\NormalTok{(}\DataTypeTok{y=}\NormalTok{tpr), }\DataTypeTok{col =} \StringTok{"blue"}\NormalTok{) }\OperatorTok{+}
\StringTok{    }\KeywordTok{geom_abline}\NormalTok{(}\DataTypeTok{intercept =} \DecValTok{0}\NormalTok{, }\DataTypeTok{slope =} \DecValTok{1}\NormalTok{, }\DataTypeTok{lty =} \StringTok{"dashed"}\NormalTok{) }\OperatorTok{+}
\StringTok{    }\KeywordTok{labs}\NormalTok{(}\DataTypeTok{title =} \KeywordTok{paste0}\NormalTok{(}\StringTok{"Perfect Prediction: ROC Curve w/ AUC="}\NormalTok{, auc_perf)) }\OperatorTok{+}
\StringTok{    }\KeywordTok{theme_bw}\NormalTok{()}
\end{Highlighting}
\end{Shaded}

\includegraphics{bookdown-demo_files/figure-latex/unnamed-chunk-157-1.pdf}

\subsubsection{\texorpdfstring{What does ``worse than guessing'' look
like?}{What does worse than guessing look like?}}\label{what-does-worse-than-guessing-look-like}

A bad classifier will appear below and to the right of the diagonal line
we'd see if we were completely guessing. Such a model will have a c
statistic below 0.5, and will be valueless.

\begin{Shaded}
\begin{Highlighting}[]
\NormalTok{pred_bad <-}\StringTok{ }\KeywordTok{prediction}\NormalTok{(sim.temp}\OperatorTok{$}\NormalTok{p_bad, sim.temp}\OperatorTok{$}\NormalTok{y)}
\NormalTok{perf_bad <-}\StringTok{ }\KeywordTok{performance}\NormalTok{(pred_bad, }\DataTypeTok{measure =} \StringTok{"tpr"}\NormalTok{, }\DataTypeTok{x.measure =} \StringTok{"fpr"}\NormalTok{)}
\NormalTok{auc_bad <-}\StringTok{ }\KeywordTok{performance}\NormalTok{(pred_bad, }\DataTypeTok{measure=}\StringTok{"auc"}\NormalTok{)}

\NormalTok{auc_bad <-}\StringTok{ }\KeywordTok{round}\NormalTok{(auc_bad}\OperatorTok{@}\NormalTok{y.values[[}\DecValTok{1}\NormalTok{]],}\DecValTok{3}\NormalTok{)}
\NormalTok{roc_bad <-}\StringTok{ }\KeywordTok{data.frame}\NormalTok{(}\DataTypeTok{fpr=}\KeywordTok{unlist}\NormalTok{(perf_bad}\OperatorTok{@}\NormalTok{x.values),}
                        \DataTypeTok{tpr=}\KeywordTok{unlist}\NormalTok{(perf_bad}\OperatorTok{@}\NormalTok{y.values),}
                        \DataTypeTok{model=}\StringTok{"GLM"}\NormalTok{)}

\KeywordTok{ggplot}\NormalTok{(roc_bad, }\KeywordTok{aes}\NormalTok{(}\DataTypeTok{x=}\NormalTok{fpr, }\DataTypeTok{ymin=}\DecValTok{0}\NormalTok{, }\DataTypeTok{ymax=}\NormalTok{tpr)) }\OperatorTok{+}
\StringTok{    }\KeywordTok{geom_ribbon}\NormalTok{(}\DataTypeTok{alpha=}\FloatTok{0.2}\NormalTok{, }\DataTypeTok{fill =} \StringTok{"blue"}\NormalTok{) }\OperatorTok{+}
\StringTok{    }\KeywordTok{geom_line}\NormalTok{(}\KeywordTok{aes}\NormalTok{(}\DataTypeTok{y=}\NormalTok{tpr), }\DataTypeTok{col =} \StringTok{"blue"}\NormalTok{) }\OperatorTok{+}
\StringTok{    }\KeywordTok{geom_abline}\NormalTok{(}\DataTypeTok{intercept =} \DecValTok{0}\NormalTok{, }\DataTypeTok{slope =} \DecValTok{1}\NormalTok{, }\DataTypeTok{lty =} \StringTok{"dashed"}\NormalTok{) }\OperatorTok{+}
\StringTok{    }\KeywordTok{labs}\NormalTok{(}\DataTypeTok{title =} \KeywordTok{paste0}\NormalTok{(}\StringTok{"A Bad Model: ROC Curve w/ AUC="}\NormalTok{, auc_bad)) }\OperatorTok{+}
\StringTok{    }\KeywordTok{theme_bw}\NormalTok{()}
\end{Highlighting}
\end{Shaded}

\includegraphics{bookdown-demo_files/figure-latex/unnamed-chunk-158-1.pdf}

\subsubsection{\texorpdfstring{What does ``better than guessing'' look
like?}{What does better than guessing look like?}}\label{what-does-better-than-guessing-look-like}

An ``OK'' classifier will appear above and to the left of the diagonal
line we'd see if we were completely guessing. Such a model will have a c
statistic above 0.5, and might have some value. The plot below shows a
very fairly poor model, but at least it's better than guessing.

\begin{Shaded}
\begin{Highlighting}[]
\NormalTok{pred_ok <-}\StringTok{ }\KeywordTok{prediction}\NormalTok{(sim.temp}\OperatorTok{$}\NormalTok{p_ok, sim.temp}\OperatorTok{$}\NormalTok{y)}
\NormalTok{perf_ok <-}\StringTok{ }\KeywordTok{performance}\NormalTok{(pred_ok, }\DataTypeTok{measure =} \StringTok{"tpr"}\NormalTok{, }\DataTypeTok{x.measure =} \StringTok{"fpr"}\NormalTok{)}
\NormalTok{auc_ok <-}\StringTok{ }\KeywordTok{performance}\NormalTok{(pred_ok, }\DataTypeTok{measure=}\StringTok{"auc"}\NormalTok{)}

\NormalTok{auc_ok <-}\StringTok{ }\KeywordTok{round}\NormalTok{(auc_ok}\OperatorTok{@}\NormalTok{y.values[[}\DecValTok{1}\NormalTok{]],}\DecValTok{3}\NormalTok{)}
\NormalTok{roc_ok <-}\StringTok{ }\KeywordTok{data.frame}\NormalTok{(}\DataTypeTok{fpr=}\KeywordTok{unlist}\NormalTok{(perf_ok}\OperatorTok{@}\NormalTok{x.values),}
                        \DataTypeTok{tpr=}\KeywordTok{unlist}\NormalTok{(perf_ok}\OperatorTok{@}\NormalTok{y.values),}
                        \DataTypeTok{model=}\StringTok{"GLM"}\NormalTok{)}

\KeywordTok{ggplot}\NormalTok{(roc_ok, }\KeywordTok{aes}\NormalTok{(}\DataTypeTok{x=}\NormalTok{fpr, }\DataTypeTok{ymin=}\DecValTok{0}\NormalTok{, }\DataTypeTok{ymax=}\NormalTok{tpr)) }\OperatorTok{+}
\StringTok{    }\KeywordTok{geom_ribbon}\NormalTok{(}\DataTypeTok{alpha=}\FloatTok{0.2}\NormalTok{, }\DataTypeTok{fill =} \StringTok{"blue"}\NormalTok{) }\OperatorTok{+}
\StringTok{    }\KeywordTok{geom_line}\NormalTok{(}\KeywordTok{aes}\NormalTok{(}\DataTypeTok{y=}\NormalTok{tpr), }\DataTypeTok{col =} \StringTok{"blue"}\NormalTok{) }\OperatorTok{+}
\StringTok{    }\KeywordTok{geom_abline}\NormalTok{(}\DataTypeTok{intercept =} \DecValTok{0}\NormalTok{, }\DataTypeTok{slope =} \DecValTok{1}\NormalTok{, }\DataTypeTok{lty =} \StringTok{"dashed"}\NormalTok{) }\OperatorTok{+}
\StringTok{    }\KeywordTok{labs}\NormalTok{(}\DataTypeTok{title =} \KeywordTok{paste0}\NormalTok{(}\StringTok{"A Mediocre Model: ROC Curve w/ AUC="}\NormalTok{, auc_ok)) }\OperatorTok{+}
\StringTok{    }\KeywordTok{theme_bw}\NormalTok{()}
\end{Highlighting}
\end{Shaded}

\includegraphics{bookdown-demo_files/figure-latex/unnamed-chunk-159-1.pdf}

Sometimes people grasp for a rough guide as to the accuracy of a model's
predictions based on the area under the ROC curve. A common thought is
to assess the C statistic much like you would a class grade.

\begin{longtable}[]{@{}rl@{}}
\toprule
\begin{minipage}[b]{0.16\columnwidth}\raggedleft\strut
C statistic\strut
\end{minipage} & \begin{minipage}[b]{0.60\columnwidth}\raggedright\strut
Interpretation\strut
\end{minipage}\tabularnewline
\midrule
\endhead
\begin{minipage}[t]{0.16\columnwidth}\raggedleft\strut
0.90 to 1.00\strut
\end{minipage} & \begin{minipage}[t]{0.60\columnwidth}\raggedright\strut
model does an excellent job at discriminating ``yes'' from ``no''
(A)\strut
\end{minipage}\tabularnewline
\begin{minipage}[t]{0.16\columnwidth}\raggedleft\strut
0.80 to 0.90\strut
\end{minipage} & \begin{minipage}[t]{0.60\columnwidth}\raggedright\strut
model does a good job (B)\strut
\end{minipage}\tabularnewline
\begin{minipage}[t]{0.16\columnwidth}\raggedleft\strut
0.70 to 0.80\strut
\end{minipage} & \begin{minipage}[t]{0.60\columnwidth}\raggedright\strut
model does a fair job (C)\strut
\end{minipage}\tabularnewline
\begin{minipage}[t]{0.16\columnwidth}\raggedleft\strut
0.60 to 0.70\strut
\end{minipage} & \begin{minipage}[t]{0.60\columnwidth}\raggedright\strut
model does a poor job (D)\strut
\end{minipage}\tabularnewline
\begin{minipage}[t]{0.16\columnwidth}\raggedleft\strut
0.50 to 0.60\strut
\end{minipage} & \begin{minipage}[t]{0.60\columnwidth}\raggedright\strut
model fails (F)\strut
\end{minipage}\tabularnewline
\begin{minipage}[t]{0.16\columnwidth}\raggedleft\strut
below 0.50\strut
\end{minipage} & \begin{minipage}[t]{0.60\columnwidth}\raggedright\strut
model is worse than random guessing\strut
\end{minipage}\tabularnewline
\bottomrule
\end{longtable}

\subsubsection{\texorpdfstring{What does ``pretty good'' look
like?}{What does pretty good look like?}}\label{what-does-pretty-good-look-like}

A strong and good classifier will appear above and to the left of the
diagonal line we'd see if we were completely guessing, often with a nice
curve that is continually increasing and appears to be pulled up towards
the top left. Such a model will have a c statistic well above 0.5, but
not as large as 1. The plot below shows a stronger model, which appears
substantially better than guessing.

\begin{Shaded}
\begin{Highlighting}[]
\NormalTok{pred_good <-}\StringTok{ }\KeywordTok{prediction}\NormalTok{(sim.temp}\OperatorTok{$}\NormalTok{p_good, sim.temp}\OperatorTok{$}\NormalTok{y)}
\NormalTok{perf_good <-}\StringTok{ }\KeywordTok{performance}\NormalTok{(pred_good, }\DataTypeTok{measure =} \StringTok{"tpr"}\NormalTok{, }\DataTypeTok{x.measure =} \StringTok{"fpr"}\NormalTok{)}
\NormalTok{auc_good <-}\StringTok{ }\KeywordTok{performance}\NormalTok{(pred_good, }\DataTypeTok{measure=}\StringTok{"auc"}\NormalTok{)}

\NormalTok{auc_good <-}\StringTok{ }\KeywordTok{round}\NormalTok{(auc_good}\OperatorTok{@}\NormalTok{y.values[[}\DecValTok{1}\NormalTok{]],}\DecValTok{3}\NormalTok{)}
\NormalTok{roc_good <-}\StringTok{ }\KeywordTok{data.frame}\NormalTok{(}\DataTypeTok{fpr=}\KeywordTok{unlist}\NormalTok{(perf_good}\OperatorTok{@}\NormalTok{x.values),}
                        \DataTypeTok{tpr=}\KeywordTok{unlist}\NormalTok{(perf_good}\OperatorTok{@}\NormalTok{y.values),}
                        \DataTypeTok{model=}\StringTok{"GLM"}\NormalTok{)}

\KeywordTok{ggplot}\NormalTok{(roc_good, }\KeywordTok{aes}\NormalTok{(}\DataTypeTok{x=}\NormalTok{fpr, }\DataTypeTok{ymin=}\DecValTok{0}\NormalTok{, }\DataTypeTok{ymax=}\NormalTok{tpr)) }\OperatorTok{+}
\StringTok{    }\KeywordTok{geom_ribbon}\NormalTok{(}\DataTypeTok{alpha=}\FloatTok{0.2}\NormalTok{, }\DataTypeTok{fill =} \StringTok{"blue"}\NormalTok{) }\OperatorTok{+}
\StringTok{    }\KeywordTok{geom_line}\NormalTok{(}\KeywordTok{aes}\NormalTok{(}\DataTypeTok{y=}\NormalTok{tpr), }\DataTypeTok{col =} \StringTok{"blue"}\NormalTok{) }\OperatorTok{+}
\StringTok{    }\KeywordTok{geom_abline}\NormalTok{(}\DataTypeTok{intercept =} \DecValTok{0}\NormalTok{, }\DataTypeTok{slope =} \DecValTok{1}\NormalTok{, }\DataTypeTok{lty =} \StringTok{"dashed"}\NormalTok{) }\OperatorTok{+}
\StringTok{    }\KeywordTok{labs}\NormalTok{(}\DataTypeTok{title =} \KeywordTok{paste0}\NormalTok{(}\StringTok{"A Pretty Good Model: ROC Curve w/ AUC="}\NormalTok{, auc_good)) }\OperatorTok{+}
\StringTok{    }\KeywordTok{theme_bw}\NormalTok{()}
\end{Highlighting}
\end{Shaded}

\includegraphics{bookdown-demo_files/figure-latex/unnamed-chunk-160-1.pdf}

\section{\texorpdfstring{The ROC Plot for
\texttt{res\_modA}}{The ROC Plot for res\_modA}}\label{the-roc-plot-for-res_moda}

Let me show you the ROC curve for our \texttt{res\_modA} model.

\begin{Shaded}
\begin{Highlighting}[]
\NormalTok{## requires ROCR package}
\NormalTok{prob <-}\StringTok{ }\KeywordTok{predict}\NormalTok{(res_modA, resect, }\DataTypeTok{type=}\StringTok{"response"}\NormalTok{)}
\NormalTok{pred <-}\StringTok{ }\KeywordTok{prediction}\NormalTok{(prob, resect}\OperatorTok{$}\NormalTok{died)}
\NormalTok{perf <-}\StringTok{ }\KeywordTok{performance}\NormalTok{(pred, }\DataTypeTok{measure =} \StringTok{"tpr"}\NormalTok{, }\DataTypeTok{x.measure =} \StringTok{"fpr"}\NormalTok{)}
\NormalTok{auc <-}\StringTok{ }\KeywordTok{performance}\NormalTok{(pred, }\DataTypeTok{measure=}\StringTok{"auc"}\NormalTok{)}

\NormalTok{auc <-}\StringTok{ }\KeywordTok{round}\NormalTok{(auc}\OperatorTok{@}\NormalTok{y.values[[}\DecValTok{1}\NormalTok{]],}\DecValTok{3}\NormalTok{)}
\NormalTok{roc.data <-}\StringTok{ }\KeywordTok{data.frame}\NormalTok{(}\DataTypeTok{fpr=}\KeywordTok{unlist}\NormalTok{(perf}\OperatorTok{@}\NormalTok{x.values),}
                       \DataTypeTok{tpr=}\KeywordTok{unlist}\NormalTok{(perf}\OperatorTok{@}\NormalTok{y.values),}
                       \DataTypeTok{model=}\StringTok{"GLM"}\NormalTok{)}

\KeywordTok{ggplot}\NormalTok{(roc.data, }\KeywordTok{aes}\NormalTok{(}\DataTypeTok{x=}\NormalTok{fpr, }\DataTypeTok{ymin=}\DecValTok{0}\NormalTok{, }\DataTypeTok{ymax=}\NormalTok{tpr)) }\OperatorTok{+}
\StringTok{    }\KeywordTok{geom_ribbon}\NormalTok{(}\DataTypeTok{alpha=}\FloatTok{0.2}\NormalTok{, }\DataTypeTok{fill =} \StringTok{"blue"}\NormalTok{) }\OperatorTok{+}
\StringTok{    }\KeywordTok{geom_line}\NormalTok{(}\KeywordTok{aes}\NormalTok{(}\DataTypeTok{y=}\NormalTok{tpr), }\DataTypeTok{col =} \StringTok{"blue"}\NormalTok{) }\OperatorTok{+}
\StringTok{    }\KeywordTok{geom_abline}\NormalTok{(}\DataTypeTok{intercept =} \DecValTok{0}\NormalTok{, }\DataTypeTok{slope =} \DecValTok{1}\NormalTok{, }\DataTypeTok{lty =} \StringTok{"dashed"}\NormalTok{) }\OperatorTok{+}
\StringTok{    }\KeywordTok{labs}\NormalTok{(}\DataTypeTok{title =} \KeywordTok{paste0}\NormalTok{(}\StringTok{"ROC Curve w/ AUC="}\NormalTok{, auc)) }\OperatorTok{+}
\StringTok{    }\KeywordTok{theme_bw}\NormalTok{()}
\end{Highlighting}
\end{Shaded}

\includegraphics{bookdown-demo_files/figure-latex/unnamed-chunk-161-1.pdf}

Based on the C statistic (AUC = 0.771) this would rank somewhere near
the high end of a ``fair'' predictive model by this standard, not quite
to the level of a ``good'' model.

\subsection{Another way to plot the ROC
Curve}\label{another-way-to-plot-the-roc-curve}

If we've loaded the \texttt{pROC} package, we can also use the following
(admittedly simpler) approach to plot the ROC curve, without
\texttt{ggplot2}, and to obtain the C statistic, and a 95\% confidence
interval around that C statistic.

\begin{Shaded}
\begin{Highlighting}[]
\NormalTok{## requires pROC package}
\NormalTok{roc.modA <-}\StringTok{ }
\StringTok{    }\KeywordTok{roc}\NormalTok{(resect}\OperatorTok{$}\NormalTok{died }\OperatorTok{~}\StringTok{ }\KeywordTok{predict}\NormalTok{(res_modA, }\DataTypeTok{type=}\StringTok{"response"}\NormalTok{),}
        \DataTypeTok{ci =} \OtherTok{TRUE}\NormalTok{)}

\NormalTok{roc.modA}
\end{Highlighting}
\end{Shaded}

\begin{verbatim}

Call:
roc.formula(formula = resect$died ~ predict(res_modA, type = "response"),     ci = TRUE)

Data: predict(res_modA, type = "response") in 117 controls (resect$died 0) < 17 cases (resect$died 1).
Area under the curve: 0.7707
95% CI: 0.67-0.8715 (DeLong)
\end{verbatim}

\begin{Shaded}
\begin{Highlighting}[]
\KeywordTok{plot}\NormalTok{(roc.modA)}
\end{Highlighting}
\end{Shaded}

\includegraphics{bookdown-demo_files/figure-latex/unnamed-chunk-162-1.pdf}

\section{Assessing Residual Plots from Model
A}\label{assessing-residual-plots-from-model-a}

\begin{quote}
Residuals are certainly less informative for logistic regression than
they are for linear regression: not only do yes/no outcomes inherently
contain less information than continuous ones, but the fact that the
adjusted response depends on the fit hampers our ability to use
residuals as external checks on the model.
\end{quote}

\begin{quote}
This is mitigated to some extent, however, by the fact that we are also
making fewer distributional assumptions in logistic regression, so there
is no need to inspect residuals for, say, skewness or
heteroskedasticity.
\end{quote}

\begin{itemize}
\tightlist
\item
  Patrick Breheny, University of Kentucky,
  \href{https://web.as.uky.edu/statistics/users/pbreheny/760/S13/notes/3-26.pdf}{Slides
  on GLM Residuals and Diagnostics}
\end{itemize}

The usual residual plots are available in R for a logistic regression
model, but most of them are irrelevant in the logistic regression
setting. The residuals shouldn't follow a standard Normal distribution,
and they will not show constant variance over the range of the predictor
variables, so plots looking into those issues aren't helpful.

The only plot from the standard set that we'll look at in many settings
is plot 5, which helps us assess influence (via Cook's distance
contours), and a measure related to leverage (how unusual an observation
is in terms of the predictors) and standardized Pearson residuals.

\begin{Shaded}
\begin{Highlighting}[]
\KeywordTok{plot}\NormalTok{(res_modA, }\DataTypeTok{which =} \DecValTok{5}\NormalTok{)}
\end{Highlighting}
\end{Shaded}

\includegraphics{bookdown-demo_files/figure-latex/unnamed-chunk-163-1.pdf}

In this case, I don't see any highly influential points, as no points
fall outside of the Cook's distance (0.5 or 1) contours.

\section{\texorpdfstring{Model B: A ``Kitchen Sink'' Logistic Regression
Model}{Model B: A Kitchen Sink Logistic Regression Model}}\label{model-b-a-kitchen-sink-logistic-regression-model}

\begin{Shaded}
\begin{Highlighting}[]
\NormalTok{res_modB <-}\StringTok{ }\KeywordTok{glm}\NormalTok{(died }\OperatorTok{~}\StringTok{ }\NormalTok{resection }\OperatorTok{+}\StringTok{ }\NormalTok{age }\OperatorTok{+}\StringTok{ }\NormalTok{prior }\OperatorTok{+}\StringTok{ }\NormalTok{intubated,}
               \DataTypeTok{data =}\NormalTok{ resect, }\DataTypeTok{family =}\NormalTok{ binomial)}

\NormalTok{res_modB}
\end{Highlighting}
\end{Shaded}

\begin{verbatim}

Call:  glm(formula = died ~ resection + age + prior + intubated, family = binomial, 
    data = resect)

Coefficients:
(Intercept)    resection          age        prior    intubated  
  -5.152886     0.612211     0.001173     0.814691     2.810797  

Degrees of Freedom: 133 Total (i.e. Null);  129 Residual
Null Deviance:      101.9 
Residual Deviance: 67.36    AIC: 77.36
\end{verbatim}

\subsection{Comparing Model A to Model
B}\label{comparing-model-a-to-model-b}

\begin{Shaded}
\begin{Highlighting}[]
\KeywordTok{anova}\NormalTok{(res_modA, res_modB)}
\end{Highlighting}
\end{Shaded}

\begin{verbatim}
Analysis of Deviance Table

Model 1: died ~ resection
Model 2: died ~ resection + age + prior + intubated
  Resid. Df Resid. Dev Df Deviance
1       132     89.493            
2       129     67.359  3   22.134
\end{verbatim}

The addition of \texttt{age}, \texttt{prior} and \texttt{intubated}
reduces the lack of fit by 22.134 points, at a cost of 3 degrees of
freedom.

\begin{Shaded}
\begin{Highlighting}[]
\KeywordTok{glance}\NormalTok{(res_modA)}
\end{Highlighting}
\end{Shaded}

\begin{verbatim}
  null.deviance df.null    logLik      AIC     BIC deviance df.residual
1      101.9431     133 -44.74646 93.49292 99.2886 89.49292         132
\end{verbatim}

\begin{Shaded}
\begin{Highlighting}[]
\KeywordTok{glance}\NormalTok{(res_modB)}
\end{Highlighting}
\end{Shaded}

\begin{verbatim}
  null.deviance df.null   logLik     AIC     BIC deviance df.residual
1      101.9431     133 -33.6793 77.3586 91.8478  67.3586         129
\end{verbatim}

By either AIC or BIC, the larger model (\texttt{res\_modB}) looks more
effective.

\subsection{Interpreting Model B}\label{interpreting-model-b}

\begin{Shaded}
\begin{Highlighting}[]
\KeywordTok{summary}\NormalTok{(res_modB)}
\end{Highlighting}
\end{Shaded}

\begin{verbatim}

Call:
glm(formula = died ~ resection + age + prior + intubated, family = binomial, 
    data = resect)

Deviance Residuals: 
    Min       1Q   Median       3Q      Max  
-1.7831  -0.3741  -0.2386  -0.2014   2.5228  

Coefficients:
             Estimate Std. Error z value Pr(>|z|)    
(Intercept) -5.152886   1.469453  -3.507 0.000454 ***
resection    0.612211   0.282807   2.165 0.030406 *  
age          0.001173   0.020646   0.057 0.954700    
prior        0.814691   0.704785   1.156 0.247705    
intubated    2.810797   0.658395   4.269 1.96e-05 ***
---
Signif. codes:  0 '***' 0.001 '**' 0.01 '*' 0.05 '.' 0.1 ' ' 1

(Dispersion parameter for binomial family taken to be 1)

    Null deviance: 101.943  on 133  degrees of freedom
Residual deviance:  67.359  on 129  degrees of freedom
AIC: 77.359

Number of Fisher Scoring iterations: 6
\end{verbatim}

It appears that the \texttt{intubated} predictor adds significant value
to the model, by the Wald test.

Let's focus on the impact of these variables through odds ratios.

\begin{Shaded}
\begin{Highlighting}[]
\KeywordTok{exp}\NormalTok{(}\KeywordTok{coef}\NormalTok{(res_modB))}
\end{Highlighting}
\end{Shaded}

\begin{verbatim}
 (Intercept)    resection          age        prior    intubated 
 0.005782692  1.844504859  1.001173503  2.258476846 16.623153519 
\end{verbatim}

\begin{Shaded}
\begin{Highlighting}[]
\KeywordTok{exp}\NormalTok{(}\KeywordTok{confint}\NormalTok{(res_modB))}
\end{Highlighting}
\end{Shaded}

\begin{verbatim}
Waiting for profiling to be done...
\end{verbatim}

\begin{verbatim}
                   2.5 %     97.5 %
(Intercept) 0.0002408626  0.0837263
resection   1.0804548590  3.3495636
age         0.9618416869  1.0442885
prior       0.5485116610  9.1679931
intubated   4.7473282453 64.6456919
\end{verbatim}

At a 5\% significance level, we might conclude that:

\begin{itemize}
\tightlist
\item
  larger sized \texttt{resection}s are associated with a meaningful rise
  (est OR: 1.84, 95\% CI 1.08, 3.35) in the odds of death, holding all
  other predictors constant,
\item
  the need for \texttt{intubation} at the end of surgery is associated
  with a substantial rise (est OR: 16.6, 95\% CI 4.7, 64.7) in the odds
  of death, holding all other predictors constant, but that
\item
  older \texttt{age} as well as having a \texttt{prior} tracheal surgery
  appears to be associated with an increase in death risk, but not to an
  extent that we can declare statistically significant.
\end{itemize}

\section{Plotting Model B}\label{plotting-model-b}

Let's think about plotting the fitted values from our model, in terms of
probabilities.

\subsection{\texorpdfstring{Using \texttt{augment} to capture the fitted
probabilities}{Using augment to capture the fitted probabilities}}\label{using-augment-to-capture-the-fitted-probabilities-1}

\begin{Shaded}
\begin{Highlighting}[]
\NormalTok{res_B_aug <-}\StringTok{ }\KeywordTok{augment}\NormalTok{(res_modB, resect, }
                     \DataTypeTok{type.predict =} \StringTok{"response"}\NormalTok{)}
\KeywordTok{head}\NormalTok{(res_B_aug)}
\end{Highlighting}
\end{Shaded}

\begin{verbatim}
  id age prior resection intubated died    .fitted    .se.fit     .resid
1  1  34     1       2.5         0    0 0.05908963 0.03851118 -0.3490198
2  2  57     0       5.0         0    0 0.11660492 0.06253774 -0.4979613
3  3  60     1       4.0         1    1 0.72944600 0.15010423  0.7943172
4  4  62     1       4.2         0    0 0.15522494 0.09607978 -0.5808354
5  5  28     0       6.0         1    1 0.79641141 0.14588554  0.6747435
6  6  52     0       3.0         0    0 0.03713809 0.01933270 -0.2751191
        .hat    .sigma      .cooksd .std.resid
1 0.02667562 0.7247491 0.0003536652 -0.3537702
2 0.03796756 0.7240341 0.0010829917 -0.5076925
3 0.11416656 0.7215778 0.0107925872  0.8439524
4 0.07039819 0.7234665 0.0029937671 -0.6024273
5 0.13126049 0.7225958 0.0088920280  0.7239256
6 0.01045207 0.7250114 0.0000823406 -0.2765683
\end{verbatim}

\subsection{Plotting Model B Fits by Observed
Mortality}\label{plotting-model-b-fits-by-observed-mortality}

\begin{Shaded}
\begin{Highlighting}[]
\KeywordTok{ggplot}\NormalTok{(res_B_aug, }\KeywordTok{aes}\NormalTok{(}\DataTypeTok{x =} \KeywordTok{factor}\NormalTok{(died), }\DataTypeTok{y =}\NormalTok{ .fitted, }\DataTypeTok{col =} \KeywordTok{factor}\NormalTok{(died))) }\OperatorTok{+}
\StringTok{    }\KeywordTok{geom_boxplot}\NormalTok{() }\OperatorTok{+}
\StringTok{    }\KeywordTok{geom_jitter}\NormalTok{(}\DataTypeTok{width =} \FloatTok{0.1}\NormalTok{) }\OperatorTok{+}\StringTok{ }
\StringTok{    }\KeywordTok{guides}\NormalTok{(}\DataTypeTok{col =} \OtherTok{FALSE}\NormalTok{)}
\end{Highlighting}
\end{Shaded}

\includegraphics{bookdown-demo_files/figure-latex/unnamed-chunk-170-1.pdf}

Certainly it appears as though most of our predicted probabilities (of
death) for the subjects who actually survived are quite small, but not
all of them. We also have at least 6 big ``misses'' among the 17
subjects who actually died.

\subsection{The ROC curve for Model B}\label{the-roc-curve-for-model-b}

\begin{Shaded}
\begin{Highlighting}[]
\NormalTok{## requires ROCR package}
\NormalTok{prob <-}\StringTok{ }\KeywordTok{predict}\NormalTok{(res_modB, resect, }\DataTypeTok{type=}\StringTok{"response"}\NormalTok{)}
\NormalTok{pred <-}\StringTok{ }\KeywordTok{prediction}\NormalTok{(prob, resect}\OperatorTok{$}\NormalTok{died)}
\NormalTok{perf <-}\StringTok{ }\KeywordTok{performance}\NormalTok{(pred, }\DataTypeTok{measure =} \StringTok{"tpr"}\NormalTok{, }\DataTypeTok{x.measure =} \StringTok{"fpr"}\NormalTok{)}
\NormalTok{auc <-}\StringTok{ }\KeywordTok{performance}\NormalTok{(pred, }\DataTypeTok{measure=}\StringTok{"auc"}\NormalTok{)}

\NormalTok{auc <-}\StringTok{ }\KeywordTok{round}\NormalTok{(auc}\OperatorTok{@}\NormalTok{y.values[[}\DecValTok{1}\NormalTok{]],}\DecValTok{3}\NormalTok{)}
\NormalTok{roc.data <-}\StringTok{ }\KeywordTok{data.frame}\NormalTok{(}\DataTypeTok{fpr=}\KeywordTok{unlist}\NormalTok{(perf}\OperatorTok{@}\NormalTok{x.values),}
                       \DataTypeTok{tpr=}\KeywordTok{unlist}\NormalTok{(perf}\OperatorTok{@}\NormalTok{y.values),}
                       \DataTypeTok{model=}\StringTok{"GLM"}\NormalTok{)}

\KeywordTok{ggplot}\NormalTok{(roc.data, }\KeywordTok{aes}\NormalTok{(}\DataTypeTok{x=}\NormalTok{fpr, }\DataTypeTok{ymin=}\DecValTok{0}\NormalTok{, }\DataTypeTok{ymax=}\NormalTok{tpr)) }\OperatorTok{+}
\StringTok{    }\KeywordTok{geom_ribbon}\NormalTok{(}\DataTypeTok{alpha=}\FloatTok{0.2}\NormalTok{, }\DataTypeTok{fill =} \StringTok{"blue"}\NormalTok{) }\OperatorTok{+}
\StringTok{    }\KeywordTok{geom_line}\NormalTok{(}\KeywordTok{aes}\NormalTok{(}\DataTypeTok{y=}\NormalTok{tpr), }\DataTypeTok{col =} \StringTok{"blue"}\NormalTok{) }\OperatorTok{+}
\StringTok{    }\KeywordTok{geom_abline}\NormalTok{(}\DataTypeTok{intercept =} \DecValTok{0}\NormalTok{, }\DataTypeTok{slope =} \DecValTok{1}\NormalTok{, }\DataTypeTok{lty =} \StringTok{"dashed"}\NormalTok{) }\OperatorTok{+}
\StringTok{    }\KeywordTok{labs}\NormalTok{(}\DataTypeTok{title =} \KeywordTok{paste0}\NormalTok{(}\StringTok{"Model B: ROC Curve w/ AUC="}\NormalTok{, auc)) }\OperatorTok{+}
\StringTok{    }\KeywordTok{theme_bw}\NormalTok{()}
\end{Highlighting}
\end{Shaded}

\includegraphics{bookdown-demo_files/figure-latex/unnamed-chunk-171-1.pdf}

The area under the curve (C-statistic) is 0.86, which certainly looks
like a more discriminating fit than model A with resection alone.

\subsection{Residuals, Leverage and
Influence}\label{residuals-leverage-and-influence}

\begin{Shaded}
\begin{Highlighting}[]
\KeywordTok{plot}\NormalTok{(res_modB, }\DataTypeTok{which =} \DecValTok{5}\NormalTok{)}
\end{Highlighting}
\end{Shaded}

\includegraphics{bookdown-demo_files/figure-latex/unnamed-chunk-172-1.pdf}

Again, we see no signs of deeply influential points in this model.

\section{\texorpdfstring{Logistic Regression using
\texttt{lrm}}{Logistic Regression using lrm}}\label{logistic-regression-using-lrm}

To obtain the Nagelkerke \(R^2\) and the C statistic, as well as some
other summaries, I'll now demonstrate the use of \texttt{lrm} from the
\texttt{rms} package to fit a logistic regression model.

We'll return to the original model, predicting death using resection
size alone.

\begin{Shaded}
\begin{Highlighting}[]
\NormalTok{dd <-}\StringTok{ }\KeywordTok{datadist}\NormalTok{(resect)}
\KeywordTok{options}\NormalTok{(}\DataTypeTok{datadist=}\StringTok{"dd"}\NormalTok{)}

\NormalTok{res_modC <-}\StringTok{ }\KeywordTok{lrm}\NormalTok{(died }\OperatorTok{~}\StringTok{ }\NormalTok{resection, }\DataTypeTok{data=}\NormalTok{resect, }\DataTypeTok{x=}\OtherTok{TRUE}\NormalTok{, }\DataTypeTok{y=}\OtherTok{TRUE}\NormalTok{)}
\NormalTok{res_modC}
\end{Highlighting}
\end{Shaded}

\begin{verbatim}
Logistic Regression Model
 
 lrm(formula = died ~ resection, data = resect, x = TRUE, y = TRUE)
 
                      Model Likelihood     Discrimination    Rank Discrim.    
                         Ratio Test           Indexes           Indexes       
 Obs           134    LR chi2     12.45    R2       0.167    C       0.771    
  0            117    d.f.            1    g        1.037    Dxy     0.541    
  1             17    Pr(> chi2) 0.0004    gr       2.820    gamma   0.582    
 max |deriv| 2e-06                         gp       0.110    tau-a   0.121    
                                           Brier    0.103                     
 
           Coef    S.E.   Wald Z Pr(>|Z|)
 Intercept -4.4337 0.8799 -5.04  <0.0001 
 resection  0.7417 0.2230  3.33  0.0009  
 
\end{verbatim}

This output specifies the following:

\begin{itemize}
\tightlist
\item
  \texttt{Obs} = The number of observations used to fit the model, with
  \texttt{0} = the number of zeros and \texttt{1} = the number of ones
  in our outcome, \texttt{died}. Also specified is the maximum absolute
  value of the derivative at the point where the maximum likelihood
  function was estimated. I wouldn't worry about that practically, as
  all you will care about is whether the iterative function-fitting
  process converged, and R will warn you in other ways if it doesn't.
\item
  A likelihood ratio test (drop in deviance test) subtracting the
  residual deviance from the null deviance obtain the Likelihood Ratio
  \(\chi^2\) statistic, subtracting residual df from null df to obtain
  degrees of freedom, and comparing the resulting test statistic to a
  \(\chi^2\) distribution with the appropriate degrees of freedom to
  determine a \emph{p} value.
\item
  A series of discrimination indexes, including the Nagelkerke
  R\textsuperscript{2}, symbolized R2, and several others we'll discuss
  shortly.
\item
  A series of rank discrimination indexes, including the C statistic
  (area under the ROC curve) and Somers' D (Dxy), and several others.
\item
  A table of coefficients, standard errors, Wald Z statistics and
  \emph{p} values based on those Wald statistics.
\end{itemize}

The C statistic is estimated to be 0.771, with an associated
(Nagelkerke) \(R^2\) = 0.167, both indicating at best mediocre
performance for this model, as it turns out.

\subsection{\texorpdfstring{Interpreting Nagelkerke
R\textsuperscript{2}}{Interpreting Nagelkerke R2}}\label{interpreting-nagelkerke-r2}

There are many ways to calculate \(R^2\) for logistic regression.

\begin{itemize}
\tightlist
\item
  At the unfortunate
  \href{http://www.ats.ucla.edu/stat/mult_pkg/faq/general/Psuedo_RSquareds.htm}{URL
  linked here} (unfortunate because the term ``pseudo'' is misspelled)
  there is a nice summary of the key issue, which is that there are at
  least three different ways to think about \(R^2\) in linear regression
  that are equivalent in that context, but when you move to a
  categorical outcome, which interpretation you use leads you down a
  different path for extension to the new type of outcome.
\item
  Paul Allison, for instance, describes several at
  \href{http://statisticalhorizons.com/r2logistic}{this link} in a post
  entitled ``What's the Best R-Squared for Logistic Regression?''
\item
  Jonathan Bartlett looks at McFadden's pseudo \(R^2\) in some detail
  (including some R code) at
  \href{http://thestatsgeek.com/2014/02/08/r-squared-in-logistic-regression/}{this
  link}, in a post entitled ``R squared in logistic regression''
\end{itemize}

The Nagelkerke approach that is presented as \texttt{R2} in the
\texttt{lrm} output is as good as most of the available approaches, and
has the positive feature that it does reach 1 if the fitted model shows
as much improvement as possible over the null model (which predicts the
mean response for all subjects, and has R\textsuperscript{2} = 0). The
greater the improvement, the higher the Nagelkerke R\textsuperscript{2}.

For model A, our Nagelkerke R\textsuperscript{2} = 0.167, which is
pretty poor. It doesn't technically mean that 16.7\% of any sort of
variation has been explained, though.

\subsection{Interpreting the C statistic and Plotting the ROC
Curve}\label{interpreting-the-c-statistic-and-plotting-the-roc-curve}

The C statistic is a measure of the area under the receiver operating
characteristic curve.
\href{http://blog.yhat.com/posts/roc-curves.html}{This link} has some
nice material that provides some insight into the C statistic and ROC
curve.

\begin{itemize}
\tightlist
\item
  Recall that C ranges from 0 to 1. 0 = BAD, 1 = GOOD.

  \begin{itemize}
  \tightlist
  \item
    values of C less than 0.5 indicate that your prediction model is not
    even as good as simple random guessing of ``yes'' or ``no'' for your
    response.
  \item
    C = 0.5 for random guessing
  \item
    C = 1 indicates a perfect classification scheme - one that correctly
    guesses ``yes'' for all ``yes'' patients, and for none of the ``no''
    patients.
  \end{itemize}
\item
  The closer C is to 1, the happier we'll be, most of the time.

  \begin{itemize}
  \tightlist
  \item
    Often we'll call models with 0.5 \textless{} C \textless{} 0.8 poor
    or weak in terms of predictive ability by this measure
  \item
    0.8 \(\leq\) C \textless{} 0.9 are moderately strong in terms of
    predictive power (indicate good discrimination)
  \item
    C \(\geq\) 0.9 usually indicates a very strong model in this regard
    (indicate excellent discrimination)
  \end{itemize}
\end{itemize}

We've seen the ROC curve for this model before, when we looked at model
\texttt{res\_modA} fitted using \texttt{glm} in the previous chapter.
But, just for completeness, I'll include it.

\textbf{Note.} I change the initial \texttt{predict} call from
\texttt{type\ =\ "response"} for a \texttt{glm} fit to
\texttt{type\ =\ "fitted"} in a \texttt{lrm} fit. Otherwise, this is the
same approach.

\begin{Shaded}
\begin{Highlighting}[]
\NormalTok{## requires ROCR package}
\NormalTok{prob <-}\StringTok{ }\KeywordTok{predict}\NormalTok{(res_modC, resect, }\DataTypeTok{type=}\StringTok{"fitted"}\NormalTok{)}
\NormalTok{pred <-}\StringTok{ }\KeywordTok{prediction}\NormalTok{(prob, resect}\OperatorTok{$}\NormalTok{died)}
\NormalTok{perf <-}\StringTok{ }\KeywordTok{performance}\NormalTok{(pred, }\DataTypeTok{measure =} \StringTok{"tpr"}\NormalTok{, }\DataTypeTok{x.measure =} \StringTok{"fpr"}\NormalTok{)}
\NormalTok{auc <-}\StringTok{ }\KeywordTok{performance}\NormalTok{(pred, }\DataTypeTok{measure=}\StringTok{"auc"}\NormalTok{)}

\NormalTok{auc <-}\StringTok{ }\KeywordTok{round}\NormalTok{(auc}\OperatorTok{@}\NormalTok{y.values[[}\DecValTok{1}\NormalTok{]],}\DecValTok{3}\NormalTok{)}
\NormalTok{roc.data <-}\StringTok{ }\KeywordTok{data.frame}\NormalTok{(}\DataTypeTok{fpr=}\KeywordTok{unlist}\NormalTok{(perf}\OperatorTok{@}\NormalTok{x.values),}
                       \DataTypeTok{tpr=}\KeywordTok{unlist}\NormalTok{(perf}\OperatorTok{@}\NormalTok{y.values),}
                       \DataTypeTok{model=}\StringTok{"GLM"}\NormalTok{)}

\KeywordTok{ggplot}\NormalTok{(roc.data, }\KeywordTok{aes}\NormalTok{(}\DataTypeTok{x=}\NormalTok{fpr, }\DataTypeTok{ymin=}\DecValTok{0}\NormalTok{, }\DataTypeTok{ymax=}\NormalTok{tpr)) }\OperatorTok{+}
\StringTok{    }\KeywordTok{geom_ribbon}\NormalTok{(}\DataTypeTok{alpha=}\FloatTok{0.2}\NormalTok{, }\DataTypeTok{fill =} \StringTok{"blue"}\NormalTok{) }\OperatorTok{+}
\StringTok{    }\KeywordTok{geom_line}\NormalTok{(}\KeywordTok{aes}\NormalTok{(}\DataTypeTok{y=}\NormalTok{tpr), }\DataTypeTok{col =} \StringTok{"blue"}\NormalTok{) }\OperatorTok{+}
\StringTok{    }\KeywordTok{geom_abline}\NormalTok{(}\DataTypeTok{intercept =} \DecValTok{0}\NormalTok{, }\DataTypeTok{slope =} \DecValTok{1}\NormalTok{, }\DataTypeTok{lty =} \StringTok{"dashed"}\NormalTok{) }\OperatorTok{+}
\StringTok{    }\KeywordTok{labs}\NormalTok{(}\DataTypeTok{title =} \KeywordTok{paste0}\NormalTok{(}\StringTok{"Model C: ROC Curve w/ AUC="}\NormalTok{, auc)) }\OperatorTok{+}
\StringTok{    }\KeywordTok{theme_bw}\NormalTok{()}
\end{Highlighting}
\end{Shaded}

\includegraphics{bookdown-demo_files/figure-latex/unnamed-chunk-174-1.pdf}

\subsection{The C statistic and Somers'
D}\label{the-c-statistic-and-somers-d}

\begin{itemize}
\tightlist
\item
  The C statistic is directly related to \textbf{Somers' D statistic},
  abbreviated \(D_{xy}\), by the equation C = 0.5 + (D/2).

  \begin{itemize}
  \tightlist
  \item
    Somers' D and the ROC area only measure how well predicted values
    from the model can rank-order the responses. For example, predicted
    probabilities of 0.01 and 0.99 for a pair of subjects are no better
    than probabilities of 0.2 and 0.8 using rank measures, if the first
    subject had a lower response value than the second.
  \item
    Thus, the C statistic (or \(D_{xy}\)) may not be very sensitive ways
    to choose between models, even though they provide reasonable
    summaries of the models individually.
  \item
    This is especially true when the models are strong. The Nagelkerke
    R\textsuperscript{2} may be more sensitive.
  \end{itemize}
\item
  But as it turns out, we sometimes have to look at the ROC shapes, as
  the summary statistic alone isn't enough.
\end{itemize}

In our case, Somers D (Dxy) = .541, so the C statistic is 0.771.

\subsection{Validating the Logistic Regression Model Summary
Statistics}\label{validating-the-logistic-regression-model-summary-statistics}

Like other regression-fitting tools in \texttt{rms}, the \texttt{lrm}
function has a special \texttt{validate} tool to help perform resampling
validation of a model, with or without backwards step-wise variable
selection. Here, we'll validate our model's summary statistics using 100
bootstrap replications.

\begin{Shaded}
\begin{Highlighting}[]
\KeywordTok{set.seed}\NormalTok{(}\DecValTok{432001}\NormalTok{) }
\KeywordTok{validate}\NormalTok{(res_modC, }\DataTypeTok{B =} \DecValTok{100}\NormalTok{)}
\end{Highlighting}
\end{Shaded}

\begin{verbatim}
          index.orig training    test optimism index.corrected   n
Dxy           0.5415   0.5325  0.5415  -0.0090          0.5505 100
R2            0.1666   0.1664  0.1666  -0.0002          0.1668 100
Intercept     0.0000   0.0000  0.1425  -0.1425          0.1425 100
Slope         1.0000   1.0000  1.0742  -0.0742          1.0742 100
Emax          0.0000   0.0000  0.0416   0.0416          0.0416 100
D             0.0854   0.0872  0.0854   0.0017          0.0837 100
U            -0.0149  -0.0149 -0.0004  -0.0145         -0.0004 100
Q             0.1004   0.1021  0.0859   0.0162          0.0841 100
B             0.1025   0.1032  0.1046  -0.0014          0.1039 100
g             1.0369   1.0247  1.0369  -0.0122          1.0491 100
gp            0.1101   0.1082  0.1101  -0.0019          0.1119 100
\end{verbatim}

Recall that our area under the curve (C statistic) =
\texttt{0.5\ +\ (Dxy/2)}, so that we can also use the first row of
statistics to validate the C statistic. Accounting for optimism in this
manner, our corrected estimates are Dxy = 0.551, so C = 0.776, and
Nagelkerke R\textsuperscript{2} = 0.167.

\subsection{\texorpdfstring{Plotting the Summary of the \texttt{lrm}
approach}{Plotting the Summary of the lrm approach}}\label{plotting-the-summary-of-the-lrm-approach}

The \texttt{summary} function applied to an \texttt{lrm} fit shows the
effect size comparing the 25\textsuperscript{th} to the
75\textsuperscript{th} percentile of resection.

\begin{Shaded}
\begin{Highlighting}[]
\KeywordTok{plot}\NormalTok{(}\KeywordTok{summary}\NormalTok{(res_modC))}
\end{Highlighting}
\end{Shaded}

\includegraphics{bookdown-demo_files/figure-latex/unnamed-chunk-176-1.pdf}

\begin{Shaded}
\begin{Highlighting}[]
\KeywordTok{summary}\NormalTok{(res_modC)}
\end{Highlighting}
\end{Shaded}

\begin{verbatim}
             Effects              Response : died 

 Factor      Low High Diff. Effect S.E.    Lower 0.95 Upper 0.95
 resection   2   4    2     1.4834 0.44591 0.6094      2.3574   
  Odds Ratio 2   4    2     4.4078      NA 1.8393     10.5630   
\end{verbatim}

So, a move from a resection of 2 cm to a resection of 4 cm is associated
with an estimated effect on the log odds of death of 1.48 (with standard
error 0.45), or with an estimated effect on the odds ratio for death of
4.41, with 95\% CI (1.84, 10.56).

\subsection{\texorpdfstring{ANOVA from the \texttt{lrm}
approach}{ANOVA from the lrm approach}}\label{anova-from-the-lrm-approach}

\begin{Shaded}
\begin{Highlighting}[]
\KeywordTok{anova}\NormalTok{(res_modC)}
\end{Highlighting}
\end{Shaded}

\begin{verbatim}
                Wald Statistics          Response: died 

 Factor     Chi-Square d.f. P    
 resection  11.07      1    9e-04
 TOTAL      11.07      1    9e-04
\end{verbatim}

The ANOVA approach applied to a \texttt{lrm} fit provides a Wald test
for the model as a whole. Here, the use of \texttt{resection} is a
significant improvement over a null (intercept-only) model. The \emph{p}
value is 9 x 10\textsuperscript{-4}.

\subsection{Are any points particularly
influential?}\label{are-any-points-particularly-influential}

I'll use a cutoff for \texttt{dfbeta} here of 0.3, instead of the
default 0.2, because I want to focus on truly influential points. Note
that we have to use the data frame version of \texttt{resect} as
\texttt{show.influence} isn't tibble-friendly.

\begin{Shaded}
\begin{Highlighting}[]
\NormalTok{inf.C <-}\StringTok{ }\KeywordTok{which.influence}\NormalTok{(res_modC, }\DataTypeTok{cutoff=}\FloatTok{0.3}\NormalTok{)}
\NormalTok{inf.C}
\end{Highlighting}
\end{Shaded}

\begin{verbatim}
$Intercept
[1] "84"  "128"

$resection
[1] "84"
\end{verbatim}

\begin{Shaded}
\begin{Highlighting}[]
\KeywordTok{show.influence}\NormalTok{(}\DataTypeTok{object =}\NormalTok{ inf.C, }\DataTypeTok{dframe =} \KeywordTok{data.frame}\NormalTok{(resect))}
\end{Highlighting}
\end{Shaded}

\begin{verbatim}
    Count resection
84      2      *2.0
128     1       2.5
\end{verbatim}

It appears that observation 84 may have a meaningful effect on both the
intercept and the coefficient for \texttt{resection}.

\subsection{A plot of the model's calibration
curve}\label{a-plot-of-the-models-calibration-curve}

The \texttt{calibrate} function applied to a \texttt{lrm} fit provides
an assessment of the impact of overfitting on our model. The function
uses bootstrapping (or cross-validation) to get bias-corrected estimates
of predicted vs.~observed values based on nonparametric smoothers for
logistic regressions. In order to obtain this curve, you need to set
both x = TRUE and y = TRUE when fitting the model. The errors here refer
to the difference between the model predicted values and the
corresponding bias-corrected calibrated values.

\begin{Shaded}
\begin{Highlighting}[]
\KeywordTok{plot}\NormalTok{(}\KeywordTok{calibrate}\NormalTok{(res_modC))}
\end{Highlighting}
\end{Shaded}

\includegraphics{bookdown-demo_files/figure-latex/unnamed-chunk-179-1.pdf}

\begin{verbatim}

n=134   Mean absolute error=0.034   Mean squared error=0.00239
0.9 Quantile of absolute error=0.104
\end{verbatim}

Here, we see that this model deviates fairly substantially from the
``ideal'' model, especially as the predicted probabilities increase over
0.1.

\subsection{A Nomogram for Model C}\label{a-nomogram-for-model-c}

We use the \texttt{plogis} function within a nomogram call to get R to
produce fitted probabilities (of our outcome, \texttt{died}) in this
case.

\begin{Shaded}
\begin{Highlighting}[]
\KeywordTok{plot}\NormalTok{(}\KeywordTok{nomogram}\NormalTok{(res_modC, }\DataTypeTok{fun=}\NormalTok{plogis, }
              \DataTypeTok{fun.at=}\KeywordTok{c}\NormalTok{(}\FloatTok{0.05}\NormalTok{, }\KeywordTok{seq}\NormalTok{(}\FloatTok{0.1}\NormalTok{, }\FloatTok{0.9}\NormalTok{, }\DataTypeTok{by =} \FloatTok{0.1}\NormalTok{), }\FloatTok{0.95}\NormalTok{), }
              \DataTypeTok{funlabel=}\StringTok{"Pr(died)"}\NormalTok{))}
\end{Highlighting}
\end{Shaded}

\includegraphics{bookdown-demo_files/figure-latex/unnamed-chunk-180-1.pdf}

Since there's no non-linearity in the right hand side of our simple
logistic regression model, the nomogram is straightforward. We calculate
the points based on the resection by traveling up, and then travel down
in a straight vertical line from total points through the linear (log
odds) predictor straight to a fitted probability. Note that fitted
probabilities above 0.5 are not possible within the range of observed
\texttt{resection} values in this case.

\section{Model D: An Augmented Kitchen Sink
Model}\label{model-d-an-augmented-kitchen-sink-model}

Can we predict survival from the patient's age, whether the patient had
prior tracheal surgery or not, the extent of the resection, and whether
intubation was required at the end of surgery?

\subsection{\texorpdfstring{Spearman \(\rho^2\)
Plot}{Spearman \textbackslash{}rho\^{}2 Plot}}\label{spearman-rho2-plot}

Let's start by considering the limited use of non-linear terms for
predictors that look important in a Spearman \(\rho^2\) plot.

\begin{Shaded}
\begin{Highlighting}[]
\KeywordTok{plot}\NormalTok{(}\KeywordTok{spearman2}\NormalTok{(died }\OperatorTok{~}\StringTok{ }\NormalTok{age }\OperatorTok{+}\StringTok{ }\NormalTok{prior }\OperatorTok{+}\StringTok{ }\NormalTok{resection }\OperatorTok{+}\StringTok{ }\NormalTok{intubated, }\DataTypeTok{data=}\NormalTok{resect))}
\end{Highlighting}
\end{Shaded}

\includegraphics{bookdown-demo_files/figure-latex/unnamed-chunk-181-1.pdf}

The most important variable appears to be whether intubation was
required, so I'll include \texttt{intubated}'s interaction with the
linear effect of the next most (apparently) important variable,
\texttt{resection}, and also a cubic spline for \texttt{resection}, with
three knots. Since \texttt{prior} and \texttt{age} look less important,
I'll simply add them as linear terms.

\subsection{\texorpdfstring{Fitting Model D using
\texttt{lrm}}{Fitting Model D using lrm}}\label{fitting-model-d-using-lrm}

Note the use of \texttt{\%ia\%} here. This insures that only the linear
part of the \texttt{resection} term will be used in the interaction with
\texttt{intubated}.

\begin{Shaded}
\begin{Highlighting}[]
\NormalTok{dd <-}\StringTok{ }\KeywordTok{datadist}\NormalTok{(resect)}
\KeywordTok{options}\NormalTok{(}\DataTypeTok{datadist=}\StringTok{"dd"}\NormalTok{)}

\NormalTok{res_modD <-}\StringTok{ }\KeywordTok{lrm}\NormalTok{(died }\OperatorTok{~}\StringTok{ }\NormalTok{age }\OperatorTok{+}\StringTok{ }\NormalTok{prior }\OperatorTok{+}\StringTok{ }\KeywordTok{rcs}\NormalTok{(resection, }\DecValTok{3}\NormalTok{) }\OperatorTok{+}
\StringTok{                 }\NormalTok{intubated }\OperatorTok{+}\StringTok{ }\NormalTok{intubated }\OperatorTok\StringTok{ }\NormalTok{resection, }
               \DataTypeTok{data=}\NormalTok{resect, }\DataTypeTok{x=}\OtherTok{TRUE}\NormalTok{, }\DataTypeTok{y=}\OtherTok{TRUE}\NormalTok{)}
\end{Highlighting}
\end{Shaded}

\subsection{\texorpdfstring{Assessing Model D using \texttt{lrm}'s
tools}{Assessing Model D using lrm's tools}}\label{assessing-model-d-using-lrms-tools}

\begin{Shaded}
\begin{Highlighting}[]
\NormalTok{res_modD}
\end{Highlighting}
\end{Shaded}

\begin{verbatim}
Logistic Regression Model
 
 lrm(formula = died ~ age + prior + rcs(resection, 3) + intubated + 
     intubated %ia% resection, data = resect, x = TRUE, y = TRUE)
 
                       Model Likelihood     Discrimination    Rank Discrim.    
                          Ratio Test           Indexes           Indexes       
 Obs           134    LR chi2      38.08    R2       0.464    C       0.880    
  0            117    d.f.             6    g        2.382    Dxy     0.759    
  1             17    Pr(> chi2) <0.0001    gr      10.825    gamma   0.770    
 max |deriv| 9e-08                          gp       0.172    tau-a   0.169    
                                            Brier    0.067                     
 
                       Coef     S.E.   Wald Z Pr(>|Z|)
 Intercept             -11.3636 4.9099 -2.31  0.0206  
 age                     0.0000 0.0210  0.00  0.9988  
 prior                   0.6269 0.7367  0.85  0.3947  
 resection               3.3799 1.9700  1.72  0.0862  
 resection'             -4.2104 2.7035 -1.56  0.1194  
 intubated               0.4576 2.7848  0.16  0.8695  
 intubated * resection   0.6188 0.7306  0.85  0.3970  
 
\end{verbatim}

\begin{itemize}
\tightlist
\item
  The model likelihood ratio test suggests that at least some of these
  predictors are helpful.
\item
  The Nagelkerke R\textsuperscript{2} of 0.46, and the C statistic of
  0.88 indicate a meaningful improvement in discrimination over our
  model with \texttt{resection} alone.
\item
  The Wald Z tests see some potential need to prune the model, as none
  of the elements reaches statistical significance without the others.
  The product term between \texttt{intubated} and \texttt{resection}, in
  particular, doesn't appear to have helped much, once we already had
  the main effects.
\end{itemize}

\subsection{ANOVA and Wald Tests for Model
D}\label{anova-and-wald-tests-for-model-d}

\begin{Shaded}
\begin{Highlighting}[]
\KeywordTok{anova}\NormalTok{(res_modD)}
\end{Highlighting}
\end{Shaded}

\begin{verbatim}
                Wald Statistics          Response: died 

 Factor                                               Chi-Square d.f.
 age                                                   0.00      1   
 prior                                                 0.72      1   
 resection  (Factor+Higher Order Factors)              4.95      3   
  All Interactions                                     0.72      1   
  Nonlinear                                            2.43      1   
 intubated  (Factor+Higher Order Factors)             16.45      2   
  All Interactions                                     0.72      1   
 intubated * resection  (Factor+Higher Order Factors)  0.72      1   
 TOTAL NONLINEAR + INTERACTION                         2.56      2   
 TOTAL                                                23.90      6   
 P     
 0.9988
 0.3947
 0.1753
 0.3970
 0.1194
 0.0003
 0.3970
 0.3970
 0.2783
 0.0005
\end{verbatim}

Neither the interaction term nor the non-linearity from the cubic spline
appears to be statistically significant, based on the Wald tests via
ANOVA. However it is clear that \texttt{intubated} has a significant
impact as a main effect.

\subsection{Effect Sizes in Model D}\label{effect-sizes-in-model-d}

\begin{Shaded}
\begin{Highlighting}[]
\KeywordTok{plot}\NormalTok{(}\KeywordTok{summary}\NormalTok{(res_modD))}
\end{Highlighting}
\end{Shaded}

\includegraphics{bookdown-demo_files/figure-latex/unnamed-chunk-185-1.pdf}

\begin{Shaded}
\begin{Highlighting}[]
\KeywordTok{summary}\NormalTok{(res_modD)}
\end{Highlighting}
\end{Shaded}

\begin{verbatim}
             Effects              Response : died 

 Factor      Low High Diff. Effect      S.E.    Lower 0.95 Upper 0.95
 age         36  61   25    -0.00080933 0.52409 -1.02800     1.0264  
  Odds Ratio 36  61   25     0.99919000      NA  0.35772     2.7910  
 prior        0   1    1     0.62693000 0.73665 -0.81688     2.0707  
  Odds Ratio  0   1    1     1.87190000      NA  0.44181     7.9307  
 resection    2   4    2     2.42930000 1.43510 -0.38342     5.2419  
  Odds Ratio  2   4    2    11.35000000      NA  0.68153   189.0400  
 intubated    0   1    1     2.00470000 1.11220 -0.17513     4.1845  
  Odds Ratio  0   1    1     7.42380000      NA  0.83934    65.6610  

Adjusted to: resection=2.5 intubated=0  
\end{verbatim}

The effect sizes are perhaps best described in terms of odds ratios. The
odds ratio for death isn't significantly different from 1 for any
variable, but the impact of \texttt{resection} and \texttt{intubated},
though not strong enough to be significant, look more substantial (if
poorly estimated) than the effects of \texttt{age} and \texttt{prior}.

\subsection{Plotting the ROC curve for Model
D}\label{plotting-the-roc-curve-for-model-d}

Again, remember to use \texttt{type\ =\ "fitted"} with a \texttt{lrm}
fit.

\begin{Shaded}
\begin{Highlighting}[]
\NormalTok{## requires ROCR package}
\NormalTok{prob <-}\StringTok{ }\KeywordTok{predict}\NormalTok{(res_modD, resect, }\DataTypeTok{type=}\StringTok{"fitted"}\NormalTok{)}
\NormalTok{pred <-}\StringTok{ }\KeywordTok{prediction}\NormalTok{(prob, resect}\OperatorTok{$}\NormalTok{died)}
\NormalTok{perf <-}\StringTok{ }\KeywordTok{performance}\NormalTok{(pred, }\DataTypeTok{measure =} \StringTok{"tpr"}\NormalTok{, }\DataTypeTok{x.measure =} \StringTok{"fpr"}\NormalTok{)}
\NormalTok{auc <-}\StringTok{ }\KeywordTok{performance}\NormalTok{(pred, }\DataTypeTok{measure=}\StringTok{"auc"}\NormalTok{)}

\NormalTok{auc <-}\StringTok{ }\KeywordTok{round}\NormalTok{(auc}\OperatorTok{@}\NormalTok{y.values[[}\DecValTok{1}\NormalTok{]],}\DecValTok{3}\NormalTok{)}
\NormalTok{roc.data <-}\StringTok{ }\KeywordTok{data.frame}\NormalTok{(}\DataTypeTok{fpr=}\KeywordTok{unlist}\NormalTok{(perf}\OperatorTok{@}\NormalTok{x.values),}
                       \DataTypeTok{tpr=}\KeywordTok{unlist}\NormalTok{(perf}\OperatorTok{@}\NormalTok{y.values),}
                       \DataTypeTok{model=}\StringTok{"GLM"}\NormalTok{)}

\KeywordTok{ggplot}\NormalTok{(roc.data, }\KeywordTok{aes}\NormalTok{(}\DataTypeTok{x=}\NormalTok{fpr, }\DataTypeTok{ymin=}\DecValTok{0}\NormalTok{, }\DataTypeTok{ymax=}\NormalTok{tpr)) }\OperatorTok{+}
\StringTok{    }\KeywordTok{geom_ribbon}\NormalTok{(}\DataTypeTok{alpha=}\FloatTok{0.2}\NormalTok{, }\DataTypeTok{fill =} \StringTok{"blue"}\NormalTok{) }\OperatorTok{+}
\StringTok{    }\KeywordTok{geom_line}\NormalTok{(}\KeywordTok{aes}\NormalTok{(}\DataTypeTok{y=}\NormalTok{tpr), }\DataTypeTok{col =} \StringTok{"blue"}\NormalTok{) }\OperatorTok{+}
\StringTok{    }\KeywordTok{geom_abline}\NormalTok{(}\DataTypeTok{intercept =} \DecValTok{0}\NormalTok{, }\DataTypeTok{slope =} \DecValTok{1}\NormalTok{, }\DataTypeTok{lty =} \StringTok{"dashed"}\NormalTok{) }\OperatorTok{+}
\StringTok{    }\KeywordTok{labs}\NormalTok{(}\DataTypeTok{title =} \KeywordTok{paste0}\NormalTok{(}\StringTok{"ROC Curve w/ AUC="}\NormalTok{, auc)) }\OperatorTok{+}
\StringTok{    }\KeywordTok{theme_bw}\NormalTok{()}
\end{Highlighting}
\end{Shaded}

\includegraphics{bookdown-demo_files/figure-latex/unnamed-chunk-186-1.pdf}

The AUC fitted with \texttt{ROCR} (0.883) is slightly different than
what \texttt{lrm} has told us (0.880), and this also happens if we use
the \texttt{pROC} approach, demonstrated below.

\begin{Shaded}
\begin{Highlighting}[]
\NormalTok{## requires pROC package}
\NormalTok{roc.modD <-}\StringTok{ }
\StringTok{    }\KeywordTok{roc}\NormalTok{(resect}\OperatorTok{$}\NormalTok{died }\OperatorTok{~}\StringTok{ }\KeywordTok{predict}\NormalTok{(res_modD, }\DataTypeTok{type=}\StringTok{"fitted"}\NormalTok{),}
        \DataTypeTok{ci =} \OtherTok{TRUE}\NormalTok{)}

\NormalTok{roc.modD}
\end{Highlighting}
\end{Shaded}

\begin{verbatim}

Call:
roc.formula(formula = resect$died ~ predict(res_modD, type = "fitted"),     ci = TRUE)

Data: predict(res_modD, type = "fitted") in 117 controls (resect$died 0) < 17 cases (resect$died 1).
Area under the curve: 0.8826
95% CI: 0.7952-0.97 (DeLong)
\end{verbatim}

\begin{Shaded}
\begin{Highlighting}[]
\KeywordTok{plot}\NormalTok{(roc.modD)}
\end{Highlighting}
\end{Shaded}

\includegraphics{bookdown-demo_files/figure-latex/unnamed-chunk-187-1.pdf}

\subsection{Validation of Model D
summaries}\label{validation-of-model-d-summaries}

\begin{Shaded}
\begin{Highlighting}[]
\KeywordTok{set.seed}\NormalTok{(}\DecValTok{432002}\NormalTok{)}
\KeywordTok{validate}\NormalTok{(res_modD, }\DataTypeTok{B =} \DecValTok{100}\NormalTok{)}
\end{Highlighting}
\end{Shaded}

\begin{verbatim}

Divergence or singularity in 6 samples
\end{verbatim}

\begin{verbatim}
          index.orig training    test optimism index.corrected  n
Dxy           0.7652   0.8162  0.7142   0.1020          0.6632 94
R2            0.4643   0.5382  0.3991   0.1391          0.3252 94
Intercept     0.0000   0.0000 -0.3919   0.3919         -0.3919 94
Slope         1.0000   1.0000  0.7201   0.2799          0.7201 94
Emax          0.0000   0.0000  0.1530   0.1530          0.1530 94
D             0.2767   0.3258  0.2322   0.0936          0.1831 94
U            -0.0149  -0.0149  0.1998  -0.2147          0.1998 94
Q             0.2916   0.3407  0.0324   0.3083         -0.0167 94
B             0.0673   0.0595  0.0739  -0.0144          0.0817 94
g             2.3819   4.2214  2.2024   2.0190          0.3629 94
gp            0.1720   0.1777  0.1591   0.0186          0.1534 94
\end{verbatim}

The C statistic indicates fairly strong discrimination, at C = 0.88,
although after validation, this looks much weaker (based on Dxy =
0.6632, we would have C = 0.5 + 0.6632/2 = 0.83) and the Nagelkerke
\(R^2\) is also reasonably good, at 0.46, although this, too, is overly
optimistic, and we bias-correct through our validation study to 0.33.

\subsection{Calibration Plot for Model
D}\label{calibration-plot-for-model-d}

\begin{Shaded}
\begin{Highlighting}[]
\KeywordTok{plot}\NormalTok{(}\KeywordTok{calibrate}\NormalTok{(res_modD))}
\end{Highlighting}
\end{Shaded}

\begin{verbatim}

Divergence or singularity in 5 samples
\end{verbatim}

\includegraphics{bookdown-demo_files/figure-latex/unnamed-chunk-189-1.pdf}

\begin{verbatim}

n=134   Mean absolute error=0.023   Mean squared error=0.00151
0.9 Quantile of absolute error=0.062
\end{verbatim}

This larger model is perhaps a bit better calibrated than the simple
model on \texttt{resection} alone, but there's still a lot of variation
from the ideal (diagonal line.)

\section{Model E: Fitting a Reduced Model in light of Model
D}\label{model-e-fitting-a-reduced-model-in-light-of-model-d}

Can you suggest a reduced model (using a subset of the independent
variables in model D) that adequately predicts survival?

Based on the anova for model D and the Spearman rho-squared plot, it
appears that a two-predictor model using intubation and resection may be
sufficient. Neither of the other potential predictors shows a
statistically detectable effect in its Wald test.

\begin{Shaded}
\begin{Highlighting}[]
\NormalTok{res_modE <-}\StringTok{ }\KeywordTok{lrm}\NormalTok{(died }\OperatorTok{~}\StringTok{ }\NormalTok{intubated }\OperatorTok{+}\StringTok{ }\NormalTok{resection, }\DataTypeTok{data=}\NormalTok{resect, }
                \DataTypeTok{x=}\OtherTok{TRUE}\NormalTok{, }\DataTypeTok{y=}\OtherTok{TRUE}\NormalTok{)}
\NormalTok{res_modE}
\end{Highlighting}
\end{Shaded}

\begin{verbatim}
Logistic Regression Model
 
 lrm(formula = died ~ intubated + resection, data = resect, x = TRUE, 
     y = TRUE)
 
                       Model Likelihood     Discrimination    Rank Discrim.    
                          Ratio Test           Indexes           Indexes       
 Obs           134    LR chi2      33.27    R2       0.413    C       0.867    
  0            117    d.f.             2    g        1.397    Dxy     0.734    
  1             17    Pr(> chi2) <0.0001    gr       4.043    gamma   0.757    
 max |deriv| 5e-10                          gp       0.160    tau-a   0.164    
                                            Brier    0.073                     
 
           Coef    S.E.   Wald Z Pr(>|Z|)
 Intercept -4.6370 1.0430 -4.45  <0.0001 
 intubated  2.8640 0.6479  4.42  <0.0001 
 resection  0.5475 0.2689  2.04  0.0418  
 
\end{verbatim}

The model equation is that the log odds of death is -4.637 + 2.864
\texttt{intubated} + 0.548 \texttt{resection}.

This implies that:

\begin{itemize}
\tightlist
\item
  for intubated patients, the equation is -1.773 + 0.548
  \texttt{resection}, while
\item
  for non-intubated patients, the equation is -4.637 + 0.548
  \texttt{resection}.
\end{itemize}

We can use the \texttt{ilogit} function within the \texttt{faraway}
package to help plot this.

\subsection{A Plot comparing the two intubation
groups}\label{a-plot-comparing-the-two-intubation-groups}

\begin{Shaded}
\begin{Highlighting}[]
\KeywordTok{ggplot}\NormalTok{(resect, }\KeywordTok{aes}\NormalTok{(}\DataTypeTok{x =}\NormalTok{ resection, }\DataTypeTok{y =}\NormalTok{ died, }
                   \DataTypeTok{col =} \KeywordTok{factor}\NormalTok{(intubated))) }\OperatorTok{+}\StringTok{ }
\StringTok{    }\KeywordTok{scale_color_manual}\NormalTok{(}\DataTypeTok{values =} \KeywordTok{c}\NormalTok{(}\StringTok{"blue"}\NormalTok{, }\StringTok{"red"}\NormalTok{)) }\OperatorTok{+}
\StringTok{    }\KeywordTok{geom_jitter}\NormalTok{(}\DataTypeTok{size =} \DecValTok{2}\NormalTok{, }\DataTypeTok{height =} \FloatTok{0.1}\NormalTok{) }\OperatorTok{+}
\StringTok{    }\KeywordTok{geom_line}\NormalTok{(}\KeywordTok{aes}\NormalTok{(}\DataTypeTok{x =}\NormalTok{ resection, }
                  \DataTypeTok{y =}\NormalTok{ faraway}\OperatorTok{::}\KeywordTok{ilogit}\NormalTok{(}\OperatorTok{-}\FloatTok{4.637} \OperatorTok{+}\StringTok{ }\FloatTok{0.548}\OperatorTok{*}\NormalTok{resection)),}
              \DataTypeTok{col =} \StringTok{"blue"}\NormalTok{) }\OperatorTok{+}
\StringTok{    }\KeywordTok{geom_line}\NormalTok{(}\KeywordTok{aes}\NormalTok{(}\DataTypeTok{x =}\NormalTok{ resection,}
                  \DataTypeTok{y =}\NormalTok{ faraway}\OperatorTok{::}\KeywordTok{ilogit}\NormalTok{(}\OperatorTok{-}\FloatTok{1.773} \OperatorTok{+}\StringTok{ }\FloatTok{0.548}\OperatorTok{*}\NormalTok{resection)),}
              \DataTypeTok{col =} \StringTok{"red"}\NormalTok{) }\OperatorTok{+}
\StringTok{    }\KeywordTok{geom_text}\NormalTok{(}\DataTypeTok{x =} \DecValTok{4}\NormalTok{, }\DataTypeTok{y =} \FloatTok{0.2}\NormalTok{, }\DataTypeTok{label =} \StringTok{"Not Intubated"}\NormalTok{, }
              \DataTypeTok{col=}\StringTok{"blue"}\NormalTok{) }\OperatorTok{+}
\StringTok{    }\KeywordTok{geom_text}\NormalTok{(}\DataTypeTok{x =} \FloatTok{2.5}\NormalTok{, }\DataTypeTok{y =} \FloatTok{0.6}\NormalTok{, }\DataTypeTok{label =} \StringTok{"Intubated Patients"}\NormalTok{, }
              \DataTypeTok{col=}\StringTok{"red"}\NormalTok{) }\OperatorTok{+}
\StringTok{    }\KeywordTok{labs}\NormalTok{(}\DataTypeTok{x =} \StringTok{"Extent of Resection (in cm.)"}\NormalTok{,}
         \DataTypeTok{y =} \StringTok{"Death (1,0) and estimated probability of death"}\NormalTok{,}
         \DataTypeTok{title =} \StringTok{"resect data, Model E"}\NormalTok{)}
\end{Highlighting}
\end{Shaded}

\includegraphics{bookdown-demo_files/figure-latex/unnamed-chunk-191-1.pdf}

The effect of \texttt{intubation} appears to be very large, compared to
the resection size effect.

\subsection{Nomogram for Model E}\label{nomogram-for-model-e}

A nomogram of the model would help, too.

\begin{Shaded}
\begin{Highlighting}[]
\KeywordTok{plot}\NormalTok{(}\KeywordTok{nomogram}\NormalTok{(res_modE, }\DataTypeTok{fun=}\NormalTok{plogis, }
              \DataTypeTok{fun.at=}\KeywordTok{c}\NormalTok{(}\FloatTok{0.05}\NormalTok{, }\KeywordTok{seq}\NormalTok{(}\FloatTok{0.1}\NormalTok{, }\FloatTok{0.9}\NormalTok{, }\DataTypeTok{by=}\FloatTok{0.1}\NormalTok{), }\FloatTok{0.95}\NormalTok{), }
              \DataTypeTok{funlabel=}\StringTok{"Pr(died)"}\NormalTok{))}
\end{Highlighting}
\end{Shaded}

\includegraphics{bookdown-demo_files/figure-latex/unnamed-chunk-192-1.pdf}

Again, we see that the effect of intubation is enormous, compared to the
effect of resection. Another way to see this is to plot the effect sizes
directly.

\subsection{Effect Sizes from Model E}\label{effect-sizes-from-model-e}

\begin{Shaded}
\begin{Highlighting}[]
\KeywordTok{plot}\NormalTok{(}\KeywordTok{summary}\NormalTok{(res_modE))}
\end{Highlighting}
\end{Shaded}

\includegraphics{bookdown-demo_files/figure-latex/model c effect plot-1.pdf}

\begin{Shaded}
\begin{Highlighting}[]
\KeywordTok{summary}\NormalTok{(res_modE)}
\end{Highlighting}
\end{Shaded}

\begin{verbatim}
             Effects              Response : died 

 Factor      Low High Diff. Effect  S.E.    Lower 0.95 Upper 0.95
 intubated   0   1    1      2.8640 0.64790 1.59410     4.1338   
  Odds Ratio 0   1    1     17.5310      NA 4.92390    62.4160   
 resection   2   4    2      1.0949 0.53783 0.04082     2.1491   
  Odds Ratio 2   4    2      2.9890      NA 1.04170     8.5769   
\end{verbatim}

\subsection{ANOVA for Model E}\label{anova-for-model-e}

\begin{Shaded}
\begin{Highlighting}[]
\KeywordTok{anova}\NormalTok{(res_modE)}
\end{Highlighting}
\end{Shaded}

\begin{verbatim}
                Wald Statistics          Response: died 

 Factor     Chi-Square d.f. P     
 intubated  19.54      1    <.0001
 resection   4.14      1    0.0418
 TOTAL      25.47      2    <.0001
\end{verbatim}

\subsection{Validation of Model E}\label{validation-of-model-e}

\begin{Shaded}
\begin{Highlighting}[]
\KeywordTok{validate}\NormalTok{(res_modE, }\DataTypeTok{method=}\StringTok{"boot"}\NormalTok{, }\DataTypeTok{B=}\DecValTok{40}\NormalTok{)}
\end{Highlighting}
\end{Shaded}

\begin{verbatim}
          index.orig training   test optimism index.corrected  n
Dxy           0.7340   0.7286 0.7309  -0.0022          0.7363 40
R2            0.4128   0.4222 0.3996   0.0227          0.3901 40
Intercept     0.0000   0.0000 0.0062  -0.0062          0.0062 40
Slope         1.0000   1.0000 0.9604   0.0396          0.9604 40
Emax          0.0000   0.0000 0.0100   0.0100          0.0100 40
D             0.2408   0.2445 0.2319   0.0126          0.2282 40
U            -0.0149  -0.0149 0.0091  -0.0241          0.0091 40
Q             0.2558   0.2594 0.2228   0.0367          0.2191 40
B             0.0727   0.0687 0.0776  -0.0089          0.0816 40
g             1.3970   1.4728 1.3697   0.1030          1.2939 40
gp            0.1597   0.1539 0.1561  -0.0023          0.1620 40
\end{verbatim}

Our bootstrap validated assessments of discrimination and goodness of
fit look somewhat more reasonable now.

\subsection{Do any points seem particularly
influential?}\label{do-any-points-seem-particularly-influential}

As a last step, I'll look at influence, and residuals, associated with
model E.

\begin{Shaded}
\begin{Highlighting}[]
\NormalTok{inf.E <-}\StringTok{ }\KeywordTok{which.influence}\NormalTok{(res_modE, }\DataTypeTok{cutoff=}\FloatTok{0.3}\NormalTok{)}

\NormalTok{inf.E}
\end{Highlighting}
\end{Shaded}

\begin{verbatim}
$Intercept
[1] "84" "94"

$resection
[1] "84" "94"
\end{verbatim}

\begin{Shaded}
\begin{Highlighting}[]
\KeywordTok{show.influence}\NormalTok{(inf.E, }\DataTypeTok{dframe =} \KeywordTok{data.frame}\NormalTok{(resect))}
\end{Highlighting}
\end{Shaded}

\begin{verbatim}
   Count resection
84     2        *2
94     2        *6
\end{verbatim}

\subsection{\texorpdfstring{Fitting Model E using \texttt{glm} to get
plots about
influence}{Fitting Model E using glm to get plots about influence}}\label{fitting-model-e-using-glm-to-get-plots-about-influence}

\begin{Shaded}
\begin{Highlighting}[]
\NormalTok{res_modEglm <-}\StringTok{ }\KeywordTok{glm}\NormalTok{(died }\OperatorTok{~}\StringTok{ }\NormalTok{intubated }\OperatorTok{+}\StringTok{ }\NormalTok{resection, }
                  \DataTypeTok{data=}\NormalTok{resect, }\DataTypeTok{family=}\StringTok{"binomial"}\NormalTok{)}
\KeywordTok{par}\NormalTok{(}\DataTypeTok{mfrow=}\KeywordTok{c}\NormalTok{(}\DecValTok{1}\NormalTok{,}\DecValTok{2}\NormalTok{))}
\KeywordTok{plot}\NormalTok{(res_modEglm, }\DataTypeTok{which=}\KeywordTok{c}\NormalTok{(}\DecValTok{4}\OperatorTok{:}\DecValTok{5}\NormalTok{))}
\end{Highlighting}
\end{Shaded}

\includegraphics{bookdown-demo_files/figure-latex/unnamed-chunk-196-1.pdf}

Using this \texttt{glm} residuals approach, we again see that points 84
and 94 have the largest influence on our model E.

\section{Conclusions}\label{conclusions}

It appears that \texttt{intubated} status and, to a lesser degree, the
extent of the \texttt{resection} both play a meaningful role in
predicting death associated with tracheal carina resection surgery.
Patients who are intubated are associated with worse outcomes (greater
risk of death) and more extensive resections are also associated with
worse outcomes.

\bibliography{text.bib}


\end{document}
